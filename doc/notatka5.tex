\documentclass[a4paper,10pt]{article}
\usepackage{polski}
\usepackage[utf8]{inputenc}
\usepackage{array}
%custom margins
\usepackage[left=2.5cm, top=2.5cm, bottom=3cm, right=2cm, foot=2cm, head=0.5cm]{geometry}
\usepackage{fancyhdr}
\usepackage[bookmarks=true, pdftex]{hyperref}

%styl nagłowków
\pagestyle{fancy} 
\parindent 2cm 

%opening
\title{Notatka ze spotkania}
\author{3@KASK}

\begin{document}
\bibliographystyle{plain}


\maketitle


\begin{center}
%budowanie tabeli
\begin{tabular}{|p{7cm}|p{7cm}|}
\hline
Symbol projektu: & Opiekun projektu:   \tabularnewline 
3@KASK & mgr inż. Tomasz Boiński    \tabularnewline \hline
\multicolumn{2}{|l|}{Nazwa Projektu: } \tabularnewline
\multicolumn{2}{|l|}{Wizualizacja grafów za pomocą biblioteki Prefuse } \tabularnewline 
\hline
\multicolumn{2}{l}{ } \tabularnewline %pusta linijka
\hline 
Nazwa Dokumentu: & Data spotkania:   \tabularnewline 
Notatka ze spotkania & 21 kwietnia 2009 \tabularnewline \hline
Odpowiedzialny za dokument: & Obecni na spotkaniu:   \tabularnewline 
Anna Jaworska & Zespół projektowy \tabularnewline \hline
Przeznaczenie: & Data ostatniej aktualizacji:   \tabularnewline 
WEWNĘTRZNE & \today \tabularnewline \hline
\end{tabular}
\end{center}



\section{Temat spotkania}

SWs i przygotowania do konsultacji z klientem.

\section{Poruszone zagadnienia}

\begin{itemize}
 	\item Dalsze uzupełnianie SWS
	\item Jak konkretnie wizualizaować rózne elementy ontologii
	\item Problem okienka właściwości
	\item Klasy i obiekty przedstawiamy przy pomocy figur geometrycznych
	\item Zwiazki (property) za pomocą strzałek/linii
	\item zalezności miedzy klasami musza sie roznic od zwiazków predykatowych
	
\end{itemize}


\section{Podjęte ustalenia}
\begin{itemize}
 \item naley sprawdzic jakie kształy i jakie rodzaje kresek mozna rysować w Prefuse - Piotr K.
\item Należy zdefiniować co ma w wizualizacji ontologii różnić sie od siebie - Piotr O.
\item Przejrzeć jak róznice są wizualizowane w innych programach - Radek
\item Ania poszuka danych testowych	
 \item całość na poniedziałek godz. 11
\end{itemize}


\end{document}
