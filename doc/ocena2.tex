
\section{Podsumowanie projektu}



\begin{center}
%budowanie tabeli
\begin{tabular}{|p{7cm}|p{7cm}|}
\hline
Symbol projektu: & Opiekun projektu:   \tabularnewline 
3@KASK & mgr inż. Tomasz Boiński    \tabularnewline \hline
\multicolumn{2}{|l|}{Nazwa Projektu: } \tabularnewline
\multicolumn{2}{|l|}{Wizualizacja grafów za pomocą biblioteki Prefuse } \tabularnewline 
\hline
\multicolumn{2}{l}{ } \tabularnewline %pusta linijka
\hline 
Nazwa Dokumentu: & Nr wersji:   \tabularnewline 
Podsumowanie projektu & 1.0 \tabularnewline \hline
Odpowiedzialny za dokument: & Data pierwszego sporządzenia:   \tabularnewline 
Anna Jaworska & 1 lutego 2010 \tabularnewline \hline
Przeznaczenie: & Data ostatniej aktualizacji:   \tabularnewline 
Dla klienta & \today \tabularnewline \hline
\end{tabular}
\end{center}

\begin{center}
\begin{tabular}{|c|p{4cm}|c|c|c|}
\multicolumn{5}{c}{\textbf{Historia dokumentu}} \tabularnewline \hline
\textbf{Wersja} & \textbf{Opis modyfikacji} & \textbf{Rozdział/strona} & \textbf{Autor modyfikacji} & \textbf{Data} \tabularnewline \hline 
1 & Stworzenie & wszystkie & Anna Jaworska & 01.02.10 \tabularnewline \hline
& & & &\tabularnewline \hline
\end{tabular}
 

\end{center}


\newpage




\subsection{Ocena realizacji celów projektu}
\paragraph{} Celem projektu było utworzenie biblioteki umożliwiającej wizualizację ontologii zapisanych w OWL API. Do tego celu wykorzystano język Java oraz bibliotekę graficzną Prefuse.  W szczególności wyróżniono cele:
\begin{itemize}
 \item Wizualizację elementów niejawnych (np. klasy anonimowe wyrażone
poprzez unie, przecięcie itp. oraz dziedziczenie po tych klasach,
łączenie wielu odwzorowań niejawnych)  -  cel został zrealizowany 
\item  Wizualizację powiązań między klasami oraz innymi elementami grafu - cel został zrealizowany 
\item  Udokumentowanie stworzonej biblioteki za pomoca JavaDoc - cel został zrealizowany
\item  Zapewnienie możliowości integracji uzyskanej biblioteki z istniejącą aplikacją OCS - cel został zrealizowany połowicznie, biblioteka jest zdolna do integracji z aplikacją OCS, sama integracja została dopiero rozpoczęta.
\end{itemize}


\subsubsection{ Ocena realizacji założonych celów biznesowych}




\begin{center}
\begin{tabular}{|m{3cm}|m{9cm}|} \hline

CB001 & Ułatwienie pracy programistom tworzącym aplikacje wizualizujące ontologie  \\ \hline
Priorytet: & bardzo ważne \\ \hline
\textbf{Ocena:} & Brak możliwości weryfikacji czy cel został zrealizowany. Niestety biblioteka nie może zostać udostępniona ze względów formalnych większej publiczności. Podjete zostały kroki w kierunki realizacji pluginu do aplikacji Protege, wykorzystującym uzyskaną bibliotekę, więc istnieje mozliwość oceny czy cel został zrealizowany w późniejszym terminie.    \\ \hline

\multicolumn{2}{c}{} \\

%priorytety  00 01 10 11
%priorytety bardzo wazne, wazne, średnio ważne, mało wazne

 \hline
CB002 & Ułatwienie zakończenia projektu OCS   \\ \hline
Priorytet: & bardzo ważne \\ \hline
\textbf{Ocena:} & W ocenie klienta cel został zrealizowany, uzyskana biblioteka istotnie wspomoże projekt OCS \\ \hline
\multicolumn{2}{c}{} \\
 \hline
CB003 & Zwiększenie aktrakcyjności portalu OCS   \\ \hline
Priorytet: & mało ważne \\ \hline
\textbf{Ocena:} & W ocenie klienta cel został zrealizowany. \\ \hline
\multicolumn{2}{c}{} \\
\end{tabular}

%\begin{tabular}{|m{3cm}|m{9cm}|} \hline
%CB004 &    \\ \hline
%Opis: &   \\ \hline
%Źródło: &  \\ \hline
%Priorytet: & \\ \hline
%\end{tabular}

\end{center}

\subsubsection{Ocena realizacji celów funkcjonalnych}

%Cele funkcjonalne wymieniają główne funkcje, które ma spełniać system.

\begin{center}

\begin{tabular}{|m{3cm}|m{9cm}|} \hline

CF001 & Intuicyjne API \\ \hline
Priorytet: & średnio ważne  \\ \hline
\textbf{Ocena:} & W ocenie klienta i programistów cel został zrealizowany. \\ \hline
\multicolumn{2}{c}{} \\

 \hline
CF002 & Dobra dokumentacja \\ \hline
Priorytet: & bardzo ważne \\ \hline
\textbf{Ocena:} & Cel został zrealizowany, choć brak weryfikacji, czy dokumentacja jest wystarczająca. \\ \hline
\multicolumn{2}{c}{} \\
 \hline
CF003 & Wizualizacja ontologii \\ \hline
Priorytet: & bardzo ważne \\ \hline
\textbf{Ocena:} & Cel został zrealizowany \\ \hline
\multicolumn{2}{c}{} \\
 \hline
CF004 & Umożliwienie graficznej edycji i dodawania obiektów OWL API \\ \hline
Priorytet: & średnio ważne \\ \hline
\textbf{Ocena:} & Cel okazał się leżeć poza zakresem projektu. Zakładana funkcjonalność musi być zaimplementowana w aplikacji korzystajacej z biblioteki (np. w OCS) \\ \hline
\multicolumn{2}{c}{} \\


 \hline

CF005 & Udostępnienie informacji do debuggowania  \\ \hline
Priorytet: & średnio ważne \\ \hline
\textbf{Ocena:} & Cel zrealizowany \\ \hline
\end{tabular}

\end{center}




\subsection{Ocena realizacji wymagań}

\paragraph{} W trakcie realizacji projektu wymagania nie zmieniały się drastycznie, wprowadzone zostały tylko pomniejsze zmiany. Zmienione zostały tylko wymagania dotyczące samej wizualizacji ontologii, ze względu na: prośby klienta i ograniczenia implementacyjne.




\subsubsection{Wymagania funkcjonalne}

%Wymagania funkcjonalne stanowią mocno rozbudowaną część specyfikacji. Można je podzielić na grupy dotyczące różnych zadań, różnych użytkowników (systemów zewnętrznych) albo różnych komponentów.

\begin{center}

\begin{tabular}{|m{3cm}|m{9cm}|} \hline

WF001 & Udostępnienie kilku algorytmów wizualizacji \\ \hline
Priorytet: & średnio ważny\\ \hline
\textbf{Ocena:} & Klient zmienił interpretacje wymagania, zakładając, że zapewniony widok grafu musi byc modyfikowalny poprzez róznego rodzaju filtry, co zostało zrealizowane. \\ \hline
\multicolumn{2}{c}{} \\
 \hline

WF002 & Parametryzacja trybów wizualizacyjnych \\ \hline
Priorytet: & średnio ważny \\ \hline
\textbf{Ocena:} & Wymaganie zrealizowane \\ \hline
\multicolumn{2}{c}{} \\
 \hline

WF003 & Udostępnienie strumienia błędów \\ \hline
Priorytet: & ważne \\ \hline
\textbf{Ocena:} & Wymaganie zrealizowane \\ \hline
\multicolumn{2}{c}{} \\
 \hline

WF010 & Dodatkowe informacje \\ \hline
Priorytet: & średnio ważne \\ \hline
\textbf{Ocena:} & Funkcjonalnośc nie zaimplementowana. Klient obniżył jej priorytet. \\ \hline
\end{tabular}

\end{center}

\subsubsection{Wymagania wizualizacji ontologii}

\begin{center}
\begin{tabular}{|m{3cm}|m{9cm}|} \hline

WF004 & Rozróżnialność podstawowych symboli  \\ \hline
Priorytet: & bardzo ważne \\ \hline
\textbf{Ocena:} & Wymaganie zrealizowane \\ \hline
%Projekt: & \includegraphics{myimage.png}

\multicolumn{2}{c}{} \\
 \hline

WF005 &   Rozróżnialność szczególnych typów Class\\ \hline
Priorytet: &  ważne \\ \hline
\textbf{Ocena:} & Wymaganie zrealizowane \\ \hline
\multicolumn{2}{c}{} \\
 \hline

WF006 &  Rozróżnialność związków między klasami (Class), instancjami (Individual) oraz predykatami (Property)\\ \hline
Priorytet: & ważne \\ \hline
\textbf{Ocena:} & Wymaganie zrealizowane \\ \hline
\multicolumn{2}{c}{} \\
 \hline

WF007 & Rozróżnialność ograniczeń predykatów (Restrictions) \\ \hline
Priorytet: & ważne \\ \hline
\textbf{Ocena:} & Wymaganie zrealizowane \\ \hline
\multicolumn{2}{c}{} \\
 \hline

WF008 &  Podświetlanie wybranych związków i powiazań.\\ \hline
Priorytet: & mało ważne \\ \hline
\textbf{Ocena:} & Wymaganie nie zrealizowane, ze względu na trudności implementacyjne \\ \hline
\multicolumn{2}{c}{} \\
 \hline

WF009 & Możliwość definiowania zdarzeń. \\ \hline
Priorytet: & mało ważne \\ \hline
\textbf{Ocena:} & Wymaganie nie zrealizowane. Wymaganie powinno być poza zakresem projektu, ponieważ funkcjonalność należy zaimplementować w OCS. \\ \hline
\end{tabular}

\end{center}

\subsubsection{Wykorzystanie projektu wizualizacji}

\paragraph{} Wizualizacja została zrealizowana dokładnie tak, jak ja zaprojektowano. Dodatkowo, ze względu na sugestie uzyskane po semestralnej ocenie projektu wprowadzono możliwość definiowania własnych schematów kolorów. Wszystkie kolory zawarte w projekcie wizualizacji zostały umieszczone w plikach properties i mogą zostać przedefiniowane przez użytkownika biblioteki, a nawet użytkownika aplikacji korzystającej z biblioteki.


\subsubsection{Wymagania na dane}

%Wymagania na dane pomagają w określeniu, jakie dane będą przetwarzane w systemie. Nie trzeba precyzować wszystkich danych. Szczegóły znajdą się w projekcie bazy danych.

\begin{center}

\begin{tabular}{|m{3cm}|m{9cm}|} \hline

WD001 & Obsługa obiektów OWL API \\ \hline
Priorytet: &  bardzo ważne \\ \hline
\textbf{Ocena:} & Wymaganie zrealizowane\\ \hline
\end{tabular}

\end{center}

\subsubsection{Wymagania jakościowe}

%Określenie wymagań jakościowych ułatwia późniejsze uzyskanie wysokiej jakości systemu. Podział wymagań jakościowych na kategorie jest związany z drzewem jakości (dotyczy wszystkich gałęzi drzewa za wyjątkiem funkcjonalności).

\subsubsection{Wymagania w zakresie wiarygodności}

%Wymagania w zakresie wiarygodności będą rozszerzały wymagania funkcjonalne.


\begin{center}

\begin{tabular}{|m{3cm}|m{9cm}|} \hline

WJ001 & Poprawność wizualizacji \\ \hline
Priorytet: & bardzo ważne \\ \hline
\textbf{Ocena:} & Wymaganie zrealizowane, biblioteka wizualizuje dostarczone dane, bez ingerencji w nie.\\ \hline
\multicolumn{2}{c}{} \\
 \hline

WJ002 & Kompletność wizualizacji \\ \hline
Priorytet: & ważne \\ \hline
\textbf{Ocena:} & Wymaganie zrealizowane \\ \hline
\end{tabular}

\end{center}

\subsubsection{Wymagania w zakresie elastyczności}

%Wymagania w zakresie elastyczności będą miały zastosowanie w czasie wyboru koncepcji systemu.

\begin{center}

\begin{tabular}{|m{3cm}|m{9cm}|} \hline

WJ003 & Obsługiwane wersje Javy \\ \hline
Priorytet: & bardzo ważne \\ \hline
\textbf{Ocena:} & Wymaganie zrealizowane \\ \hline
\multicolumn{2}{c}{} \\
 \hline

WJ004 & Obsługiwane wersje OWL API \\ \hline
Priorytet: & bardzo ważne \\ \hline
\textbf{Ocena:} & Wymaganie zrealizowane i zweryfikowane z wersjami biblioteki OWL API dostępnymi w trakcie realizacji projektu  \\ \hline
\end{tabular}

\end{center}




\subsubsection{Wymagania programowe}

%Trzeba odróżniać rzeczywiste wymagania programowe klienta od jego sugestii (np. przez podanie opcjonalnego priorytetu).

\begin{center}

\begin{tabular}{|m{3cm}|m{9cm}|} \hline

WD003 & JVM \\ \hline
Priorytet: & ważne \\ \hline
\textbf{Ocena:} & Wymaganie zrealizowane \\ \hline
\end{tabular}

\end{center}

\subsubsection{Inne wymagania}

\begin{center}

\begin{tabular}{|m{3cm}|m{9cm}|} \hline

WI001 & Dokumentacja w javadoc \\ \hline
Priorytet: & ważne \\ \hline
\textbf{Ocena:} & Wymaganie zrealizowane \\ \hline
\multicolumn{2}{c}{} \\
 \hline

WI002 & Dokumentacja w języku angielskim \\ \hline
Priorytet: & mało ważne \\ \hline
\textbf{Ocena:} & Wymaganie nie zrealizowane \\ \hline

\multicolumn{2}{c}{} \\
 \hline

WI003 & Dokumentacja w języku polskim \\ \hline
Priorytet: & ważne \\ \hline
\textbf{Ocena:} & Wymaganie zrealizowane \\ \hline

\multicolumn{2}{c}{} \\
 \hline
WI004 & Nazwy zmiennych i funkcji w języku angielskim \\ \hline
Priorytet: & ważne \\ \hline
\textbf{Ocena:} & Wymaganie zrealizowane \\ \hline

\end{tabular}

\end{center}

\subsection{Spełnienie kryterii akceptacyjnych}

%Tu podać kryteria, jakim zostanie poddany gotowy system przed ostatecznym jego przyjęciem.

\begin{center}

\begin{tabular}{|m{3cm}|m{9cm}|} \hline

KA001 & Spełnione są podstawowe wymagania wymienione w dokumencie SWS \\ \hline
Opis: & Spełnione są wszystkie wymagania ważne i bardzo ważne zdefiniowane w SWS. \\ \hline
Dotyczy: & wszystkie wymagania ważne i bardzo ważne \\ \hline
Źródło: & klient - mgr inż. Tomasz Boiński \\ \hline
Priorytet: & ważne  \\ \hline %skonsultować z klientem
\textbf{Ocena:} & Zrealizowane \\ \hline
\multicolumn{2}{c}{} \\
 \hline

KA002 & Biblioteka współpracuje z OWL API dostarczonym przez KASK \\ \hline
Priorytet: & ważne  \\ \hline %skonsultować z klientem
\textbf{Ocena:} & Zrealizowane \\ \hline
\end{tabular}

\end{center}


\subsection{Wnioski i uwagi końcowe}

\paragraph{} Uważamy, że projekt rokuje pozytywnie na przyszłość i dalszy jego rozwój. W ocenie zespołu praca włożona w przygotowanie dokumentacji zaprocentowała poprzez ułatwienie implementacji i zmniejszyła potrzebę dodatkowych spotkań zespołu. 
\paragraph{} W trakcie implementacji napotkaliśmy wiele problemów. Ich główną przyczyną był fakt, że dokumentacja dla biblioteki Prefuse i biblioteki OWL API praktycznie nie istnieje. Byliśmy zmuszeni poszukiwać przykładów wykorzystania róznych funkcji w kodzie innych publicznie dostępnych projektów. To bardzo opóźniło implementację i uniemożliwiło planową realizację harmonogramu w drugim semestrze. Niestety dalszy rozwój biblioteki może być z tego powodu utrudniony. 
\paragraph{} Integracja biblioteki z aplikacją OCS nie powinna przysporzyć istotnych problemów, jednakże bedzie bardzo czasochłonna ze względu na nie zdefiniowane jeszcze wymagania tego projektu.
\paragraph{} Największym wyzwaniem projektu było przygotowanie projektu wizualizacji. Jesteśmy bardzo zadowoleni ze spójności wizualizacji i faktu, że jest ona czytelna dla zwykłych użytkowników ontologii, nie znających przygotowanej przez nas specyfikacji.


