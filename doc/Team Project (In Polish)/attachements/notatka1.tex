\documentclass[a4paper,10pt]{article}
\usepackage{polski}
\usepackage[utf8]{inputenc}
\usepackage{array}
%custom margins
\usepackage[left=2.5cm, top=2.5cm, bottom=3cm, right=2cm, foot=2cm, head=0.5cm]{geometry}
\usepackage{fancyhdr}
\usepackage[bookmarks=true, pdftex]{hyperref}

%styl nagłowków
\pagestyle{fancy} 
\parindent 2cm 

%opening
\title{Notatka ze spotkania}
\author{3@KASK}

\begin{document}
\bibliographystyle{plain}


\maketitle


\begin{center}
%budowanie tabeli
\begin{tabular}{|p{7cm}|p{7cm}|}
\hline
Symbol projektu: & Opiekun projektu:   \tabularnewline 
3@KASK & mgr inż. Tomasz Boiński    \tabularnewline \hline
\multicolumn{2}{|l|}{Nazwa Projektu: } \tabularnewline
\multicolumn{2}{|l|}{Wizualizacja grafów za pomocą biblioteki Prefuse } \tabularnewline 
\hline
\multicolumn{2}{l}{ } \tabularnewline %pusta linijka
\hline 
Nazwa Dokumentu: & Data spotkania:   \tabularnewline 
Notatka ze spotkania & 30.03.09 \tabularnewline \hline
Odpowiedzialny za dokument: & Obecni na spotkaniu:   \tabularnewline 
Anna Jaworska  & Kompletny zespół \tabularnewline \hline
Przeznaczenie: & Data ostatniej aktualizacji:   \tabularnewline 
WEWNĘTRZNE & 31.03.09 \tabularnewline \hline
\end{tabular}
\end{center}

\section{Temat spotkania}
\paragraph{}Sporządzenie dokumentu zlecenia projektowego. Identyfikacja  zagadnień do poruszenia w studium wykonalności. 

\section{Poruszone zagadnienia}
\begin{itemize}
	\item Prezentacja przygotowanego w TeX przez Anię szablonu dokumentów 
	\item Krytyczne przejrzenie punktów w dostarczonym przez KASK szablonie \textit{Zlecenia projektowego}
	\item Zdefiniowanie celów projektu
	\item Sporządzenie pierwszej wersji \textit{Zlecenia projektowego}
	\item Identyfikacja zagadnień do rozważenia w \textit{Studium wykonalności}
\end{itemize}


\section{Podjęte ustalenia}
\paragraph{} Dalsze prace podzielono następująco:
	\begin{itemize}
 		\item Anna Jaworska - dopracowanie szablonów w TeXu, umieszczenie dotychczasowych wyników prac na SVN
		\item Radosław Kleczkowski  - zbadanie tematu istnienia standardów tworzenia bibliotek JAVA 
 		\item Piotr Kunowski - przygotowanie krótkiego przeglądu bibliotek alternatywnych do Prefuse
		\item Piotr Orłowski  - przygotowanie wylistowania elementów do wizualizacji używanych w OWL 
	\end{itemize}




%\clearpage
%\phantomsection
%\addcontentsline{toc}{section}{Literatura}
%\bibliography{biblio}

\end{document}
