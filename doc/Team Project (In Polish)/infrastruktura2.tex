


%\fancyhead{} % Clear all header fields
%\fancyhead[C]{\includegraphics{}}
%\fancyfoot{} % Clear all footer fields  
%styl nagłowków

%\fancyhead[C]{\includegraphics[width=3cm]{picture.jpg}}

%\parindent 2cm 



\section{Infrastruktura projektu}



\begin{center}
%budowanie tabeli
\begin{tabular}{|p{7cm}|p{7cm}|}
\hline
Symbol projektu: & Opiekun projektu:   \tabularnewline 
3@KASK & mgr inż. Tomasz Boiński    \tabularnewline \hline
\multicolumn{2}{|l|}{Nazwa Projektu: } \tabularnewline
\multicolumn{2}{|l|}{Wizualizacja grafów za pomocą biblioteki Prefuse } \tabularnewline 
\hline
\multicolumn{2}{l}{ } \tabularnewline %pusta linijka
\hline 
Nazwa Dokumentu: & Nr wersji:   \tabularnewline 
Infrastruktura projektu & 1.0 \tabularnewline \hline
Odpowiedzialny za dokument: & Data pierwszego sporządzenia:   \tabularnewline 
Anna Jaworska  & 31.03.09 \tabularnewline \hline
Przeznaczenie: & Data ostatniej aktualizacji:   \tabularnewline 
WEWNĘTRZNE & 20.04.09 \tabularnewline \hline
\end{tabular}
\end{center}

\begin{center}
\begin{tabular}{|c|p{4cm}|c|c|c|}
\multicolumn{5}{c}{\textbf{Historia dokumentu}} \tabularnewline \hline
\textbf{Wersja} & \textbf{Opis modyfikacji} & \textbf{Rozdział/strona} & \textbf{Autor modyfikacji} & \textbf{Data} \tabularnewline \hline 
0.0 & Stworzenie & wszystkie & Anna Jaworska & 31.03.09 \tabularnewline \hline
1.0 & Wpisanie używanych narzędzi & wszystkie & Anna Jaworska & 20.04.09 \tabularnewline \hline
& & & &\tabularnewline \hline
\end{tabular}
 

\end{center}


\newpage


\subsection{Organizacja zespołu projektu}
\begin{center}
\begin{tabular}{|l|l|} \hline
	Nazwa roli & Osoba(y) \tabularnewline \hline
	Kierownik projektu & Piotr Kunowski \tabularnewline \hline
	Specjalista ds. testów & Radosław Kleczkowski \tabularnewline \hline
	Analityk ds. ontologii & Piotr Orłowski \tabularnewline \hline
	Analityk ds. Prefuse & Piotr Kunowski \tabularnewline \hline
	Analityk główny & Anna Jaworska \tabularnewline \hline
	Programiści & cały zespół \tabularnewline \hline
\end{tabular}
\end{center}

%\subsection{Komunikacja  w zespole}

\subsection{Dokumentacja}

\paragraph{} Dokumenty tworzone sa na podstawie następujących szablonów składownych na SVN:
\begin{itemize}
 \item szablon.tex
\item notatka\_szablon.tex
\end{itemize}


%\subsection{Współdzielenie dokumentów i kodu}

\subsection{Narzędzia i wymiana informacji}
\subsubsection{Narzędzia programistyczne}
\begin{itemize}
 \item Netbeans 6.5
 
\end{itemize}
\subsubsection{Biblioteki i środowisko}
\begin{itemize}
	\item JAVA ver 6
  	\item Prefuse ver prefuse-beta20071021
	\item OWL API ver 2.1.1
\end{itemize}

\subsubsection{Komunikacja w zespole}
\begin{itemize}
 	\item Gadu-gadu
	\item Email
	\item Telefonicznie
	\item Wymiana dokumentacji przez SVN, materiałów dodatkowych przez email
\end{itemize}

\subsubsection{Tworzenie dokumentacji}
\begin{itemize}
 	\item Dokumenty w LateX
	\item na SVN wrzucamy pliki tex i ich wersje pdf
\end{itemize}


\subsubsection{Inne używane programy}
\begin{description}
 \item[Rysowanie notacji dla ontologii]  Inkspace i Dia
 \item[UML] Netbeans
 \item[Ontologie] Programy używane jako wzorcowe zarówno w kwestii wizualizacji jak i implementacji: Protege, GrOWL.
 \item[Harmonogramy] GanttProject
\end{description}

