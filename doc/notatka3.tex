\documentclass[a4paper,10pt]{article}
\usepackage{polski}
\usepackage[utf8]{inputenc}
\usepackage{array}
%custom margins
\usepackage[left=2.5cm, top=2.5cm, bottom=3cm, right=2cm, foot=2cm, head=0.5cm]{geometry}
\usepackage{fancyhdr}
\usepackage[bookmarks=true, pdftex]{hyperref}

%styl nagłowków
\pagestyle{fancy} 
\parindent 2cm 

%opening
\title{Notatka ze spotkania}
\author{3@KASK}

\begin{document}
\bibliographystyle{plain}


\maketitle


\begin{center}
%budowanie tabeli
\begin{tabular}{|p{7cm}|p{7cm}|}
\hline
Symbol projektu: & Opiekun projektu:   \tabularnewline 
3@KASK & mgr inż. Tomasz Boiński    \tabularnewline \hline
\multicolumn{2}{|l|}{Nazwa Projektu: } \tabularnewline
\multicolumn{2}{|l|}{Wizualizacja grafów za pomocą biblioteki Prefuse } \tabularnewline 
\hline
\multicolumn{2}{l}{ } \tabularnewline %pusta linijka
\hline 
Nazwa Dokumentu: & Data spotkania:   \tabularnewline 
Notatka ze spotkania & 07.04.09 \tabularnewline \hline
Odpowiedzialny za dokument: & Obecni na spotkaniu:   \tabularnewline 
Anna Jaworska & Cały zespół \tabularnewline \hline
Przeznaczenie: & Data ostatniej aktualizacji:   \tabularnewline 
WEWNĘTRZNE & WSTAW DATĘ \tabularnewline \hline
\end{tabular}
\end{center}



\section{Temat spotkania}
\paragraph{} Dokończenie dokumentu studium. 


\section{Poruszone zagadnienia}
\begin{itemize}
 \item przejrzenie dokuemntu studium i uzupełnienie braków
 \item nadal są problemy z uruchamianiem portalSubsystem, opiekun już deklarował, że postara się pomóc 
\end{itemize}



\section{Podjęte ustalenia}
\begin{itemize}
 \item Piotr O. uzupełni fragmenty o OWL i przeprowadzi korektę dokuemntu
\item Radek i i Piotr K. w późniejszym terminie (tj. już po świętach) dołączą do studium graficzna reprezentacje harmonogramu
\item ze wzgledu na święta nie przydzielamy dalszych prac, członkowie zespołu wedle uznania mogą zgłębiać zgadnienie zwiazane w projektem
\item kolejne spotkanie odbędzie się w czwartek po świętach o 9 rano 


\end{itemize}


\end{document}
