\documentclass[a4paper,10pt]{article}
\usepackage{polski}
\usepackage[utf8]{inputenc}
\usepackage{array}
%custom margins
\usepackage[left=2.5cm, top=2.5cm, bottom=3cm, right=2cm, foot=2cm, head=0.5cm]{geometry}
\usepackage{fancyhdr}
\usepackage[bookmarks=true, pdftex]{hyperref}

%styl nagłowków
\pagestyle{fancy} 
\parindent 2cm 

%opening
\title{Notatka ze spotkania}
\author{3@KASK}

\begin{document}
\bibliographystyle{plain}


\maketitle


\begin{center}
%budowanie tabeli
\begin{tabular}{|p{7cm}|p{7cm}|}
\hline
Symbol projektu: & Opiekun projektu:   \tabularnewline 
3@KASK & mgr inż. Tomasz Boiński    \tabularnewline \hline
\multicolumn{2}{|l|}{Nazwa Projektu: } \tabularnewline
\multicolumn{2}{|l|}{Wizualizacja grafów za pomocą biblioteki Prefuse } \tabularnewline 
\hline
\multicolumn{2}{l}{ } \tabularnewline %pusta linijka
\hline 
Nazwa Dokumentu: & Data spotkania:   \tabularnewline 
Notatka ze spotkania & 12 czerwca 09 \tabularnewline \hline
Odpowiedzialny za dokument: & Obecni na spotkaniu:   \tabularnewline 
Anna Jaworska & Anna Jaworska, Piotr Kunowski, Radosław Kleczkowski  \tabularnewline \hline
Przeznaczenie: & Data ostatniej aktualizacji:   \tabularnewline 
WEWNĘTRZNE & \today \tabularnewline \hline
\end{tabular}
\end{center}



\section{Temat spotkania}

Przygotowania do oddania dokumentacji

\section{Poruszone zagadnienia}
\begin{itemize}
 \item Poprawki na diagramie klas - uwzględnienie decyzji podejętych podczas implementacji prototypu
\item Dokończenie plakatu
\end{itemize}


\section{Podjęte ustalenia}
\begin{itemize}
 \item Przejrzeć dokumenty pod względem poprawności: klasy, SWS -PK, notatki -AJ, Słownik -PO, studium - RK
\item 
\end{itemize}


Na następnym spotkaniu
\begin{itemize}
 \item zrobienie prezentacji końcowej
 \item ustalenie prac na drugi semestr i sporządzenie harmonogramu
\end{itemize}



\newpage
\tableofcontents
\newpage


\clearpage
\phantomsection
\addcontentsline{toc}{section}{Literatura}
\bibliography{biblio}

\end{document}
