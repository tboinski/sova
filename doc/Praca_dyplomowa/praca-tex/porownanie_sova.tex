\chapter{Podsumowanie}

Podstawowym celem pracy było stworzenie modułu wizualizującego ontologie zapisane w języku OWL.  Cel ten został osiągnięty dzięki dobrze wykonanemu projektowi 
architektury systemu oraz gruntownej analizie rekomendacji W3C dotyczących ontologii i języka OWL.  Efektem pracy jest biblioteka SOVA, nowy moduł wizualizacji 
w~edytorze OCS oraz plugin do aplikacji \protege. 
% Biblioteka pozwala na wizualizację ontologii i może zostać wykorzystana w dowolnym projekcie informatycznym. 
\par
Wdrożenie rozwiązania w systemie OCS znacznie zwiększyło jego atrakcyjność. W tabeli \ref{t:porownanie2} jeszcze raz przedstawiono wyniki przeprowadzonego porównania 
aplikacji pozwalających na wizualizację ontologii. Tabela została rozszerzona o testy dla edytora OCS z nowym modułem wizualizacji SOVA. Możemy zauważyć, że ocena 
wizualizacji dla OCS wzrosła z 0.34 do 0.83. Co klasyfikuje go na pierwszym miejscu. 
\par 
Bardzo dobrym pomysłem okazało się  stworzenie pluginu do edytora \protege. Wydany na licencji LGPL plugin, został umieszczony i opisany na stronie \textit{http://protegewiki.stanford.edu/wiki/SOVA}.
Ciągle pobierane ze strony projektu rozwiązanie dobrze promuje uczelnię oraz katedralny system OCS. 
\par
Prace nad stworzeniem rozwiązania do wizualizacji ontologii trwały dziewięć miesięcy. W tym czasie powstało ok. 8 tysięcy linii kodu. 
Bardzo trudnym elementem okazała się realizacja zadań zgodnie z zaplanowanym harmonogramem. Nie wszystkie elementy pracy zostały zrealizowane w wyznaczonym terminie, 
a całość prac wydłużyła się o miesiąc. Przyczyną opóźnień m.in. jest słaba dokumentacja bibliotek OWL API i Prefuse.  
\par
Ograniczenia czasowe spowodowały, iż nie udało się zrealizować wszystkich wymagań i założeń pracy. Nie zostały wykonane następujące elementy:
\begin{enumerate}
 \item Brak wizualizacji DataType, co zostało wykryte podczas testowania aplikacji.
 \item Brak opcji wyszukiwania elementów ontologii na grafie. 
\end{enumerate}

Poza elementami, które nie zostały zrealizowane podczas pracy, autor zauważa możliwości dalszego rozwoju oraz rozszerzenia aplikacji o~następujące funkcje:
\begin{enumerate}
 \item Zmiana biblioteki OWLAPI na najnowszą wersję 3.0. Zmiana powinna być dokonana zarówno w bibliotece SOVA, jak i w kodzie OCS.
 \item W module wizualizacji OCS oraz pluginie dodanie możliwość edycji ontologii z poziomu wizualizacji. Biblioteka SOVA daje możliwość podpięcia dodatkowej obsługi 
zdarzeń, dzięki temu możemy np. na zdarzenia kliknięcia prawym przyciskiem myszy na element ontologii, wywołać dodatkowe okno lub nemu. 
 \item Zwiększenie przejrzystość wizualizacji poprzez dynamiczny dobór optymalnych ustawień wizualizacji (np. długość krawędzi) w zależności od liczby obrazowanych 
elementów wczytanej ontologii.
\end{enumerate}


\newpage


\begin{longtable}{|m{6cm}|m{1.6cm}|m{1.6cm}|m{1.6cm}|m{1.6cm}|m{1.6cm}|}
\caption{Porównanie rozwiązań do wizualizacji ontologii}
\label{t:porownanie2}
 \\
\hline


 Kryterium  &  OntoVis  &  Jambalaya &  Growl & OCS & OCS + SOVA \\ \hline

 \multicolumn{6}{|c|}{{\bf Kompletność wizualizacji} (max 30 punktów)}  \\ \hline
Sposób wizualizacji klas	&  2  &  2  &  2  & 2 & 2 \\ \hline
Sposób wizualizacji bytów	&  2  &  2  &  2  & 0 & 2 \\ \hline
Wizualizacja klas ananimowych	&  0  &  0  &  2  & 0 & 2 \\ \hline
Wizualizacja relacji		&  4  &  3  &  10 & 2 & 11 \\ \hline
Wizualizacja właściwości	&  2  &  2  &  2  & 0 & 2 \\ \hline
Wizualizacja związków odnoszącymi
	   się do właściwościami&  0  &  0  &  4 & 0 & 7 \\ \hline
Wizualizacja pozostałych 
	  elementów ontologii   &  0  &  3  &  1 &  0 & 4 \\ \hline
{\bf Łącznie dla kategorii}	&{\bf 0,33 }   &{\bf 0,4  }   & {\bf 0,77 }  & {\bf 0,13 } & {\bf 1,00}\\ \hline %max 30 punktów
 \multicolumn{6}{|c|}{{\bf Przejrzystość wizualizacji} (max 8 punktów)}\\ \hline
Czytelność ontologii		& 3  &  4 &  1 &  2 & 3 \\ \hline
Dostępność filtrów		& 2  &  3 &  0 &  0 & 3 \\ \hline
{\bf Łącznie dla kategorii}	&{\bf 0,625 }   &{\bf 0,875 }   & {\bf 0,125 }  & {\bf 0,25 } & {\bf 0,75} \\ \hline %max 8 punktów
 \multicolumn{6}{|c|}{{\bf Sposoby wizualizacji} (max 4 punkty)} \\ \hline
Liczba sposobów wizualizacji		&  0 &  3 &  1 & 1 & 2  \\ \hline
Wizualizacja wywnioskowanej hierarchii	&  0 &  0 &  0 & 1 & 1 \\ \hline
{\bf Łącznie dla kategorii}	&{\bf 0 }   &{\bf 0,75 }   & {\bf 0,25 }  & {\bf 0,5 } & {\bf 0,75} \\ \hline %max 4 punkty
 \multicolumn{6}{|c|}{{\bf Użyteczność} (max 11 punktów)}\\ \hline
Użyteczność			&  4 &  5 &  5 & 5 & 5 \\ \hline
Dostępność pomocy		&  2 &  2 &  2 & 0 & 1 \\ \hline
Licencja			&  2 &  1 &  2 & 2 & 2 \\ \hline
{\bf Łącznie dla kategorii}	&{\bf 0,73 }   &{\bf 0,73 }   & {\bf 0,82 }  & {\bf 0,64 } & {\bf 0,73} \\ \hline %max 11 punktów
{\bf Ocena rozwiązania}		&{\bf 0,42 }   &{\bf 0,66 }   & {\bf 0,5}  & {\bf 0,34 }  & {\bf 0,83} \\ \hline
\end{longtable}



% \begin{table}[t]
% \caption{Porównanie rozwiązań do wizualizacji ontologii}
% \label{t:porownanie}
% 
% \begin{longtable}{|m{6cm}|m{2cm}|}\hline
% 
%  Kryterium  &  OCS z modułem SOVA\\ \hline
% 
%  \multicolumn{2}{|c|}{{\bf Kompletność wizualizacji} (max 30 punktów)}  \\ \hline
% Sposób wizualizacji klas	&  2   \\ \hline
% Sposób wizualizacji bytów	&  2  \\ \hline
% Wizualizacja klas ananimowych	&  2  \\ \hline
% Wizualizacja relacji		&  11 \\ \hline
% Wizualizacja właściwości	&  2 \\ \hline
% Wizualizacja związków odnoszącymi
% 	   się do właściwościami&  7 \\ \hline
% Wizualizacja pozostałych 
% 	  elementów ontologii   &  4 \\ \hline
% {\bf Łącznie dla kategorii}	&{\bf 1,00} \\ \hline %max 30 punktów
%  \multicolumn{2}{|c|}{{\bf Przejrzystość wizualizacji} (max 8 punktów)}\\ \hline
% Czytelność ontologii		& 3  \\ \hline
% Dostępność filtrów		& 3 \\ \hline
% {\bf Łącznie dla kategorii}	&{\bf 0,75 }  \\ \hline %max 8 punktów
%  \multicolumn{2}{|c|}{{\bf Sposoby wizualizacji} (max 4 punkty)} \\ \hline
% Liczba sposobów wizualizacji		&  2 \\ \hline
% Wizualizacja wywnioskowanej hierarchii	&  1 \\ \hline
% {\bf Łącznie dla kategorii}	&{\bf 0,75 }  \\ \hline %max 4 punkty
%  \multicolumn{2}{|c|}{{\bf Użyteczność} (max 11 punktów)}\\ \hline
% Użyteczność			&  5 \\ \hline
% Dostępność pomocy		&  1 \\ \hline
% Licencja			&  2 \\ \hline
% {\bf Łącznie dla kategorii}	&{\bf 0,73 }   \\ \hline %max 11 punktów
% {\bf Ocena rozwiązania}		&{\bf 0,83 } \\ \hline
% \end{longtable}
% 
% \end{table}




% \begin{enumerate}
%  \item{\bf Kompletność wizualizacji}
% 
%   \begin{itemize}
%    \item[-]{\bf Sposób wizualizacji klas}
% 
%   \nopagebreak  
% Główny widok wizualizacji obrazuje ontologię jak graf. Klasy na tym grafie są wyróżnione kolorem i kształtem wierzchołka, każda z nich posiada identyfikującą ją nazwę. 
% 2~punkty.
% 
%    \item[-]{\bf Sposób wizualizacji bytów}
% 
%   \nopagebreak 
% Byty są wierzchołkami grafu, które zostały wyróżnione szarym kolorem oraz prostokątnym kształtem. Każdy wierzchołek zawiera nazwę. 
% 
% 
%    \item[-]{\bf Wizualizacja klas ananimowych}
% 
%   \nopagebreak 
% Aplikacja umożliwia wizualizowanie klas anonimowych. Klasy te na grafie zostały oznaczone żółtym kółkiem z odpowiednim symbolem (np. symbol sumy zbiorów lub 
% iloczynu zbiorów). Taki forma zapisu klas anonimowych jest czytelna dla użytkownika. 2~punkty.
% 
% 
%    \item[-]{\bf Wizualizacja relacji}
% 
%   \nopagebreak 
% SOVA pozwala na wizualizację wszystkich zdefiniowanych w języku OWL DL relacji. Dlatego rozwiązanie w tej kategorii uzyskuje maksymalną notę - 11~punktów.
% 
%  \item[-]{\bf Wizualizacja właściwości} 
% 
%   \nopagebreak
% 
% 
%  \item[-]{\bf Wizualizacja związków odnoszącymi się do właściwościami} 
% 
%   \nopagebreak
% Brak wizualizacji. 0~punktów
%  \item[-]{\bf Wizualizacja pozostałych elementów ontologii} 
% 
%   \nopagebreak
% Brak wizualizacji. 0~punktów
%   \end{itemize}
% 
% \item{\bf Przejrzystość wizualizacji}
% \begin{itemize}
%  \item[-]{\bf Czytelność ontologii} 
% 
%   \nopagebreak
% Pomimo wizualizacji tylko klas wizualizacja dla dużej ontologii okazała się nieczytelna, klasy się nakładały. 2~punkty.
% 
% 
%  \item[-]{\bf Dostępność filtrów} 
% 
%   \nopagebreak
% Brak. 0~punktów. 
% 
% \end{itemize}
% 
% \item{\bf Sposoby wizualizacji}
% \begin{itemize}
%  \item[-]{\bf Liczba sposobów wizualizacji } 
% 
%   \nopagebreak
% OCS umożliwia 2 sposoby wizualizacji: ogólny oraz wywnioskowaną hierarchię . Aplikacja uzyskuje 1~punkt.
%  
% 
%  \item[-]{\bf  Wizualizacja wywnioskowanej hierarchii } 
% 
%   \nopagebreak
% OCS daje możliwość oglądanie wywnioskowanej hierarchii klas i~bytów. 1~punkt.
% 
% \end{itemize}
% 
% 
% \item{\bf Użyteczność}
% \begin{itemize}
%  \item[-]{\bf Użyteczność  } 
%   Aplikacja jest intuicyjna i prosta~w obsłudze. Posiada bardzo prosty, nierozbudowany interfejs. Jednym mankamentem, podobnie jak w~Growl, może 
% być brak opcji wyszukiwania. 
% Łącznie aplikacja uzyskuje 5~punktów.
%   \nopagebreak
% 
% 
%  \item[-]{\bf Dostępność pomocy  } 
% 
%   \nopagebreak
% OCS co prawda posiada stronę internetową, ale nie zawiera ona żądnych informacji związanych z~instalacją i~obsługą edytora. 0~punkty.
% 
% 
% 
%  \item[-]{\bf Licencja  } 
% 
%   \nopagebreak
% OCS posiada licencje LGPL i~tym samym uzyskuje 2~punkty.
% \end{itemize}
% 
% 
% \end{enumerate}
