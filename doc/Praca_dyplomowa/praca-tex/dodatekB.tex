
\chapter*{Dodatek B - Plik konfiguracyjny}

\begin{longtable}{|l|m{7cm}|} \hline
Klucz : & Opis  \\ \hline



node.color.thingNodeColor & Kolor węzła reprezentującego klasę "Thing" \\ \hline

node.color.classNodeColor & Kolor węzłów reprezentujących definicje klas  \\ \hline
node.color.nothingNodeColor & Kolor węzła reprezentującego klasę "Nothing"  \\ \hline
node.color.individualNodeColor & Kolor węzłów reprezentujących instancje klas (OWL Individual)   \\ \hline
node.color.differentNodeColor & Kolor węzłów oznaczających relację DifferentFrom lub AllDifferent między wystąpieniami klas (OWL Individual)   \\ \hline
node.color.sameAsNodeColor & Kolor węzłów oznaczających relację OWL SameAs między wystąpieniami klas (OWL Individual)   \\ \hline
node.color.propertyNodeColor & Kolor węzłów reprezentujących definicje predykatów (OWL Property)  \\ \hline
node.color.someValuesFromNodeColor & Kolor węzłów Property typu SomeValuesFrom  \\ \hline
node.color.allValuesFromNodeColor & Kolor węzłów Property typu AllValuesFrom   \\ \hline
node.color.dataTypeNodeColor & Kolor węzłów OWL DataType   \\ \hline
node.color.anonymousClassNodeColor & Kolor węzłów reprezentujących różnorakie klasy anonimowe   \\ \hline
node.color.cardinalityValueNodeColor & Kolor węzłów reprezentujących dokładne ograniczenie kardynalności  \\ \hline
node.color.minCardinalityValueNodeColor & Kolor węzłów reprezentujących minimalne ograniczenie kardynalności   \\ \hline
node.color.maxCardinalityValueNodeColor & Kolor węzłów reprezentujących maksymalne ograniczenie kardynalności   \\ \hline
node.color.informationNodeColor & Kolor węzłów oznaczających właściwości predykatów (OWL Property)   \\ \hline
edge.color.subClassEdgeColor & Kolor krawedzi subClassEdge   \\ \hline
edge.color.subPropertyEdgeColor & Kolor krawedzi subPropertyEdge   \\ \hline
edge.color.edgeColor & Kolor zwykłych krawędzi (bez grotów)   \\ \hline
edge.color.propertyEdgeColor & Kolor krawędzi oznaczających relacje między Property a klasą   \\ \hline
edge.color.domainEdgeColor & Kolor krawędzi łączących definicję property z jego domeną  \\ \hline
edge.color.rangeEdgeColor & Kolor krawędzi łączących definicję property z jego przestrzenią (OWL Range)   \\ \hline
edge.color.disjointEdgeColor & Kolor krawędzi łączących klasy rozłączne  \\ \hline
edge.color.equivalentEdgeColor & Kolor krawędzi łączących klasy równoważne (OWL Equivalent)   \\ \hline
edge.color.equivalentPropertyEdgeColor & Kolor krawędzi łączących predykaty (OWL Property) równoważne (OWL Equivalent)   \\ \hline
edge.color.functionalEdgeColor & Kolor krawędzi łączących definicję Property z jego właściwościami np. functional, symmetric  \\ \hline
edge.color.instancePropertyEdgeColor & Krawędz łącząca wystąpienie Property z jego definicją    \\ \hline
edge.color.inverseOfEdgeColor & Kolor krawędzi łączących predykat (OWL Property) odwrotny (OWL InverseOf) do zadanego   \\ \hline
edge.color.instanceOfEdgeColor & Kolor krawędzi instancji klasy (OWL InstanceOf)   \\ \hline 
edge.color.inverseOfMutualEdgeColor & Kolor krawędzi łączących predykaty (OWL Property) wzajemnie odwrotne (OWL InverseOf)   \\ \hline 
edge.color.operationEdgeColor & Kolor krawędzi oznaczających operacje, w wyniku których powstają klasy anonimowe np. unia, przecięcie   \\ \hline

\end{longtable}

