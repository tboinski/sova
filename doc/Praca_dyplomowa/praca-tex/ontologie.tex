\chapter{Wstęp}
\section{Motywacje}
% W tym rozdziale zostaną przedstawione zagadania teoretyczne związane z ontologiami. Podana zostanie definicja ontologii, jej zastosowanie oraz sposoby zapisu. 
% Zostaną opisane różne sposoby reprezentacji wiedzy, m.in. logika opisowa, język RDF oraz OWL. Każdy sposób reprezentacji wiedzy zostanie krótko opisany oraz, 
% dla lepszego zrozumienia, zobrazowany przykładem. 

Szybki rozwój techniki komputerowej spowodował, że dyski komputerów na całym świecie przechowują ogromną liczbę danych. Internet jest wszechstronną składnicą informacji. Obecnie
nie mamy problemu z brakiem wiedzy, ale z jej efektywnym wyszukiwaniem. Wiedza w Internecie jest w znacznym stopniu ukierunkowana na człowieka. Składowane dokumenty często nie posiadają 
kontekstu, są zapisane w postaci zrozumiałej tylko dla nas, np. graficznie. Komputery nie potrafią przetwarzać tych dokumentów. Konieczne jest więc tworzenie autonomicznych 
systemów klasyfikacji pojęć. Podkreślany jest też fakt, iż najlepszym sposobem przechowywania wiedzy są systemy zapisujące wiedzę w sposób zrozumiały zarówno dla człowieka, jak i~dla
komputera.   
\par
Jednym z rozwiązań, spełniających powyższe założenie, jest zaproponowana przez Bernersa Lee sieć semantyczna, w której dane powiązane są ze sobą znaczeniowo. Sieci semantyczne 
definiowane są jako rozszerzenie istniejących sieci, posiadających dobrze zdefiniowane informacje. Właściwa informacja została rozszerzona o~znacznik semantyczny (kontekst), 
który maszyny mogą w łatwy sposób przetwarzać. Ponieważ znacznik semantyczny ma być rozumiany przez maszyny, musi posiadać formalnie zdefiniowany zapis. W3Cache Consortium
 zaproponowało ontologię, czyli w wielkim uproszczeniu reprezentację wiedzy poprzez zdefiniowanie zbioru pojęć i~relacji pomiędzy nimi.
\par 
Ontologie bardzo szybko zyskały uznanie wśród inżynierów i badaczy z zakresu bezpieczeństwa i medycyny. Jednak okazało się, że ich budowa nie jest łatwym zadaniem. 
Nawet dla niewielkich rozwiązań liczba elementów ontologii jest bardzo duża. Na rynku wraz z zainteresowaniem ontologiami pojawiły się narzędzia służące do ich tworzenia,
 edycji i zapisu. Autorzy edytorów ontologii, chcąc ułatwić pracę nad ontologiami, nieustannie ulepszają swoje rozwiązania. Aby móc sprawnie konstruować 
poprawne ontologie, konieczna jest ich wizualizacja, która znacznie ułatwia percepcję całego rozwiązania oraz umożliwia jego prawidłową ocenę. 

\subsection{Zakres prac}
Celem tej pracy było stworzenie aplikacji pozwalającej na wizualizację ontologii zapisanych w języku OWL. W ramach pracy zostały wykonane:
\begin{enumerate}
 \item {\bf Projekt wizualizacji} - każdemu elementowi  zdefiniowanemu  w rekomendacji OWL DL został przypisany symbol graficzny. 
 \item {\bf Projekt systemu} - została zaproponowana architektura aplikacji oraz dokonano wyboru pomocniczych bibliotek.
 \item {\bf Integracja z OCS} - biblioteka została wdrożona do systemu OCS.
 \item {\bf Plugin do \protege} - po zapoznaniu się z API aplikacji \protege, powstał plugin wizualizujący. 
\end{enumerate}
 Efektem pracy są:
\begin{enumerate}
 \item {\bf Biblioteka SOVA}, która pozwalana na wizualizację dostarczonych obiektów OWL API. Może zostać wykorzystana w dowolnym rozwiązaniu informatycznym. Posiada 
dobry i intuicyjny interfejs oraz dokumentację w javadocu.
 \item {\bf Moduł wizualizacje OCS} jest rozszerzeniem edytora OCS o nowe możliwość wizualizacji, które dostarcza biblioteka SOVA. 
 \item {\bf Plugin do \protege} pozwolił na pokazanie możliwość biblioteki SOVA w zewnętrznym rozwiązaniu oraz dostarczenie nowych opcji wizualizacji dla \protege.

Powyższe rozwiązania będą przydatne dla inżynierów ontologów, ułatwią im pracę nad ontologiami oraz pozwolą na łatwiejsze wykrywanie błędów.  
\end{enumerate}
\section{Ontologie}
\subsection*{Definicja ontologii}
Jako pierwszy definicję ontologii w 1993 roku podał T. Gruber \cite{gruber}. Określił on ontologię jako {\it specyfikację konceptualizacji}, czyli reprezentację pewnego wycinka wiedzy leżącego 
w~zasięgu zainteresowania inżynierów. Zastosowanie ontologi w różnych dziedzinach i~różnych społecznościach spowodowało powstanie wielu jej definicji. W~sieciach 
semantycznych przyjęto poniższą definicję \cite{knowledge}: 
\begin{quote}
\textit{{\bf ontologia} jest formalną, jawną specyfikacją wspólnej konceptualizacji}
\end{quote}
%  dziedziny badań. 
Aby w~pełni zrozumieć definicję, wymagane jest wyjaśnienie użytych w~niej słów:

\begin{itemize}
 \item {\bf formalna} 

Ontologia jest wyrażona  w~języku reprezentacji wiedzy posiadającym formalną semantykę. Gwarantuje to, że ontologia może być przetwarzana przez maszyny i~jednoznacznie 
interpretowana.

\item {\bf jawna}

Rodzaje użytych konceptów i~ograniczenia i~ich używanie są jawnie wyspecyfikowane. Niewyspecyfikowane elementy, pomimo iż są zrozumiałe dla człowieka, mogą nie być zrozumiałe 
dla przetwarzającej je maszyny. 

\item {\bf wspólna }

Ontologia obejmuję wiedzą ogólnie uznaną, nie wiedzę prywatną jednostki.  Aby wiedza zawarta w ontologii nie była 
subiektywnym spojrzeniem na jakieś zjawisko, często przy budowie ontologii bierze udział kilka osób.

\item {\bf konceptualizacja }

Ontologie określają wiedzę w sposób koncepcyjny, poprzez symbole reprezentujące koncepcje i~relacje między nimi. Ponadto konceptualizacja ontologii pozwana na opis danej wiedzy
w~sposób ogólny, a~nie tylko szczegółowy na poziomie konkretnych bytów. 

%   \item {\bf dziedzina badań }
% 
% Ontologie obejmują pewną dziedzinę nauki. Im dziedzina jest bardziej ograniczona, tym bardziej inżynierowie mogą szczegółowo opisywać dane zjawisko. Przy badaniu szerszej dziedziny 
% można użyć kilku ontologii o~różnych domenach badań. 
 
\end{itemize}

\par
 Ontologie możemy klasyfikować wedle dwóch wymiarów \cite{knowledge}: poziomu szczegółowości lub poziomu zależności od zadania lub punktu widzenia. W pierwszym 
przypadku możemy mieć szczegółowo opisane ontologie oraz ontologie skierowane na możliwości wnioskowania, które charakteryzują się mniejszą szczegółowością. Drugi 
wymiar klasyfikacji pozwala wyróżnić: ontologię wysokiego poziomu, ontologię określającą dziedzinę, ontologię określającą zadanie oraz ontologię zastosowaną. 


\begin{itemize}
  \item \textit{Ontologia wysokiego poziomu }określa podstawową, ogólną, niezależną  od dziedziny wiedzę, np. opisują przestrzeń, czas, materię, wydarzenie  itp.
  \item \textit{Ontologia określająca dziedzinę lub zadanie }jest ontologią opisującą daną dziedzinę (np.: samochody, leki) lub zadania i działalności (np.: diagnozowanie, sprzedaż). 
Rozszerza ona pojęcia zdefiniowane w ontologii wysokiego poziomu o ich specyfikę w danej dziedzinie lub działalności. 
  \item \textit{Ontologia zastosowana} przeważnie bazuje na ontologii określającej dziedzinę i~ontologii określającej zadanie. Często zdarza się tak, że np. konkretna ontologia 
dziedziny może być wykorzystywana z różnymi ontologiami określającymi zadania, np. ontologia samochodowych części zamiennych może być wykorzystana razem  z ontologią
 procesu składania nowego samochodu, jak i z ontologią diagnozowania samochodu.
\end{itemize}

Klasyfikacja ontologii oraz zależności pomiędzy kolejnymi jej grupami zostały przedstawione na rys. \ref{fig:klasyfikacja}.
\insertscaledimage{0.4}{images/teoria/klasyfikacja.png}{Klasyfikacja ontologii \cite{knowledge}}{fig:klasyfikacja}

\section{Zastosowania ontologii}

Ontologie szybko zyskały uznanie inżynierów wiedzy oraz badaczy rożnych dziedzin nauki. Poniżej opisano kilka przykładowych ontologii z różnych dziedzin, aby pokazać 
różnorodność branży, w których używa się takiego zapisu wiedzy. Na poniższe przykłady zwrócili również uwagę autorzy książki \cite{ebook2}:

\subsubsection*{Przykład 1 - The Gene Ontology Project (http://www.geneontology.org/)}

Zadaniem projektu jest stworzenie słownika pojęć związanych z genetyką dla każdego organizmu. Budowana ontologia składa się z trzech dziedzin: budowa komórki 
(opisuje elementy występujące w komórce oraz jej otoczenie pozakomórkowe), funkcje molekularne (opis aktywności genów na poziomie molekularnym, np.: procesy wiązania 
lub katalizy) oraz procesy biologiczne (czynności oraz zestawy wydarzeń na poziomie molekularnym opisane od początku do końca). Ontologa ta jest często aktualizowana 
i~dynamicznie rozwijana. Można ją pobrać ze strony domowej w~różnych formatach. 


\subsubsection*{Przykład 2 - The Object-Oriented Software Design Ontology(ODOL)}
Ontologia opisuje wzorce projektowe w programowaniu obiektowym. Posiada zbiór pojęć i ich relacji dotyczących programowania obiektowego. Ontologia nie jest zależna 
od żadnego z języków programowania. Projekt jest próbą przeniesienia wzorców projektowych, które najczęściej są zapisane w nieformalnym języku opisowym lub jako diagramy
UML, na język ontologii. W tym przypadku językiem ontologii jest OWL i właśnie w formacie *.owl możemy pobrać ontologię ze strony projektu. 


\subsubsection*{Przykład 3 - The Friend Of A Friend Ontology (FOAF)} 
Obecnie jedna z najbardziej popularnych ontologii. Ułatwia wymianę informacji o osobach, ich zdjęciach, blogach itp. pomiędzy witrynami sieci web. Powstała przy 
projekcie FOAF, który zakłada stworzenie sieci gromadzącej strony użytkowników (tzw. sieci społecznej). Strony w tej sieci, dzięki użytej ontologii, mogą być przetwarzane
przez komputer. 

\section{Reprezentacja wiedzy}
\subsection*{Ramki Minsky'ego}
Za jedną z pierwszych ontologicznych metod reprezentacji wiedzy uważa się ramki zaproponowane przez M. Minsky'ego w 1975 roku. Rozwiązanie to opisuje ontologię przy użyciu 
następujących elementów\cite{goczyla}:
\begin{itemize}
\item { \bf klasy } - pojęcia (koncepty),
\item { \bf klatki } - własności klas opisujące ich cechy i atrybuty (właściwości, ang. slot). Możemy wyróżnić dwa rodzaje klatek: własne (prywatne własności danej ramki) 
oraz szablonowe (służą do tworzenia innych ramek, będących wystąpieniami tych pierwszych), 
\item { \bf fasety } - własności klatek (np. ograniczenia nałożone na klatki),
\item { \bf dodatkowe } -  ograniczenia wyrażone w~języku logiki.
\end{itemize}
Wedle Minsky'ego elementy otaczającego nas świata możemy opisać za pomocą ramek. Najważniejszymi ramkami są klasy. Klasy posiadają klatki, które mogą być  ograniczone przez fasety.
 Klatki też są ramkami. Każda z klas może posiadać klatki własne i~szablonowe. Klasy mogą dziedziczyć od siebie klatki własne, jak i~szablonowe, tworząc hierarchię klas. W~przypadku 
wystąpień danej klasy klatki szablonowe stają się klatkami własnymi. Gdy wystąpienie danej klasy ma klatki szablonowe, to klasę tę nazywamy metaklasą. Jeśli zaś klasa 
nie ma klatek szablonowych, ani nie może posiadać swoich wystąpień, nazywamy ją obiektem. Przykładowa ontologia zapisana w formie ramek została przedstawiona na rys.~\ref{fig:ramki}.

\insertscaledimage{0.4}{images/teoria/ramki.png}{Przykładowa ontologia zapisana w formie ramek}{fig:ramki}

Okazało się jednak, że ramki mają kilka wad. Przede wszystkim są zbyt ogólne,  nie posiadają ścisłej, formalnej definicji, co sprawiała, że mogą doprowadzić do niejednoznaczności.
 Wprowadzenie metaklas prowadzi nawet do nierozstrzygalności pewnych problemów wnioskowania. 

\subsection*{Logika opisowa}

Innym językiem reprezentacji wiedzy jest, rozwijana już blisko 30 lat, logika opisowa (ang. Description Logic, DL). Logika opisowa to nie tylko formalny język opisu wiedzy, 
ale cała rodzina języków przeznaczonych do opisu ontologii, cechujących się różnym stopniem ekspresji oraz możliwości wnioskowania. DL jest formalizmem, opartym na rachunku predykatów 
pierwszego rzędu, przez co wnioskowanie tak zapisanej wiedzy jest w pełni rozstrzygalne. 


Logika opisowa opiera się na prostych i~zarazem ogólnych założeniach \cite{goczyla,goczyla2}:

\begin{itemize}
 \item  istnieje pewne {\it uniwersum}, które chcemy opisać w~formie ontologii (dziedzina zainteresowań),
  \item elementy uniwersum nazywane są osobnikami i są wystąpieniami konceptów,
\item elementy ontologii powiązane są binarnymi relacjami. 
\end{itemize}

\insertscaledimage{0.4}{images/teoria/AboxTbox.png}{System zarządzania wiedzą z ontologią DL \cite{goczyla2}}{fig:atBox}

Na podstawie powyższych założeń ontologie wyrażone w języku logiki opisowej składają się z dwóch części: TBox (ang. terminological box) i ABox (ang. assertional box).
TBox jest zbiorem konceptów, relacji zachodzących pomiędzy nimi oraz zbiorem aksjomatów, które nakładają ograniczenia na koncepty i~ich relacje. ABox zawiera wystąpienia 
pojęć (konceptów) i~ról (relacji). Na rysunku \ref{fig:atBox} przedstawiono ogólny model reprezentacji wiedzy w DL. 



Pomimo licznej rodziny języków logiki opisowej istnieją wspólne elementy, które posiada każdy z tych języków:
\begin{itemize}
 \item   pojęcia (koncepty) atomowe  - koncept uniwersalny  $ \top $, który przedstawia naszą dziedzinę zainteresowań oraz koncept pusty $ \bot $, który nie 
posiada żadnych wystąpień,
\item role atomowe,
\item konstruktory do tworzenia bardziej złożonych konceptów i ról. 


\end{itemize}
\pagebreak[3]
\begin{longtable}{|m{3cm}|m{9cm}|} 
\caption{Konstruktory języka ALC \cite{goczyla,goczyla2}}
\label{t:alc} \\
\hline
\bf{Konstruktor} &  \bf{Znaczenie}  \\ \hline
$ \neg C$ & Negacja konceptu \\ \hline
$ C \sqcap D$ & Cześć wspólna konceptów \\ \hline
$ C \sqcup D $ & Suma konceptów \\ \hline
$ \exists R.C $ & Kwantyfikacja egzystencjalna  - zbiór osobników powiązanych przynajmniej raz relacją R z osobnikami z~konceptu C. \\ \hline
$\forall R.C $ &Kwantyfikacja ogólna - zbiór osobników, dla których wszystkie powiązania relacją R odnoszą się do osobników z konceptu C.  \\ \hline
\end{longtable}

W celu dokładniejszego zrozumienia logiki opisowej w tabeli \ref{t:ontology}  została przedstawiona ontologia opisująca rodzinę. Jako język dla zaprezentowanej ontologii wybrano 
dość prosty, ale pozwalający na definiowanie nietrywialnych ontologii język   $ ALC $. Konstruktory tego języka zostały przedstawione w tabeli \ref{t:alc}.

\begin{longtable}{|m{7cm}|m{5cm}|} 
\caption{Przykładowa ontologia DL \cite{goczyla2}}
\label{t:ontology} \\
\hline
\bf{TBox} &  \bf{ABox}  \\ \hline
Mężczyzna $ \sqsubseteq $ Osoba & Mężczyzna(Karol) \\ 
Kobieta $ \equiv $ Osoba $\sqcap \neg$ Mężczyzna & Kobieta(Anna) \\ 
Rodzic $\equiv$ Osoba $\sqcap \exists$ maDziecko.$\top$ & Kobieta(Joanna) \\ 
Ojciec $\equiv$ Mężczyzna $\sqcap$ Rodzic & maDziecko(Anna, Karol)  \\ 
Matka $\equiv$ Kobieta $\sqcap$ Rodzic  & maDziecko(Anna,Joanna)   \\ 
& maDziecko(Anna,Maria) \\ \hline
\end{longtable}   

Ontologia rodziny (tabela \ref{t:ontology}) zawiera koncepty: Osoba, Mężczyzna, Kobieta, Rodzic, Matka, Ojciec oraz jedną role: maDziecko. TBox przedstawionej ontologii zawiera
następujące typy aksjomatów: aksjomat zawierania (oznaczony symbolem $ \sqsubseteq$) oraz aksjomat równoważności (oznaczony symbolem $\equiv$). Pierwszy aksjomat, będący
aksjomatem zawierania, mówi nam, że Mężczyzna jest Osobą. Kolejny przedstawia Kobietę jako Osobę nie będącą Mężczyzną. Następny mówi o tym, że rodzic to osoba, która ma choć jedno
dziecko. Ostanie dwa aksjomaty definiują Ojca i Matkę jak Rodzica będącego odpowiednio Mężczyzną i Kobietą. 
ABox przedstawionej ontologii zawiera informacje o osobach: Karolu, Annie, Joannie i Marii. Wiemy, że Karol jest mężczyzną, a Anna i Joanna to kobiety. Anna ma troje dzieci:
Karola, Joanne i Marię. Płeć Marii nie jest nam znana. 


\subsection*{RDF}
RDF (ang. Resource Description Framework) jest językiem do opisu zasobów w sieci WWW. Powstał w 1999 roku jak rekomendacja W3C.
Język ten opiera się na ``trójkowym'' modelu danych. Wiedza zapisywana jest za pomocą trójek (triples), które posiadają następujące elementy:


\begin{verbatim}
  podmiot - predykat - obiekt  
  podmiot - orzeczenie - dopełnienie  
  obiekt - właściwość - wartość
\end{verbatim}

 Każda trójka może zostać przedstawiona na grafie skierowanym (Rys. \ref{fig:trojka}). 

\insertscaledimage{0.3}{images/teoria/trojka.png}{Graf przedstawiający trójkę RDF}{fig:trojka}


Dowolny dokument RDF może być przedstawiony za pomocą grafu skierowanego, w którym krawędzie jak i niektóre węzły posiadają URI, czyli ich jednoznaczny identyfikator. 
W~grafie mogą wystąpić węzły puste, które są pomocniczymi węzłami służącymi do opisu wielowartościowych zależności. Węzły puste nie posiadają URI.
Podstawowy model zawiera poniższe obiekty\cite{jedruch}:

\begin{itemize}
 \item \textit{Zasoby} są to  dokumenty, części dokumentów, elementy i przedmioty świata rzeczywistego oraz zasoby abstrakcyjne. Na grafie zasoby opisywane
 są przez węzły identyfikowane przez URI.
\item \textit{Właściwości} są zasobami, które opisują właściwości innych zasobów. Na grafie są przedstawiane jako krawędzie skierowane opisane własnym URI. 
\item \textit{Wyrażenia} są klasycznymi trójkami składającymi się z zasobów, właściwości oraz wartości właściwości. Element trójki będący dopełnieniem może być pewnym zasobem
lub wartością literalną. Wartość literalna na grafie jest znaczona jako węzeł oznaczony tą wartością.  
\end{itemize}

Na rysunku \ref{fig:rdf} został zaprezentowany przykładowy graf RDF, który jest prezentacją pliku RDF/XML przedstawionego na listingu  \ref{lst:teory:rdf}

\lstset{language=xml, caption={Listing dokumentu RDF opisującego Erica Millera \cite{RDFprimer}}, stepnumber=0,  captionpos=b, label={lst:teory:rdf}}
\lstinputlisting{code/rdf.rdf}


\insertscaledimage{0.6}{images/teoria/rdf.png}{Graf RDF opisujący Erica Millera \cite{RDFprimer}}{fig:rdf}



Bardzo ważne jest, aby istniała możliwość definiowania typów własnych. Np. typ osoba, który może przechowywać informacje o nazwisku, adresie danej osoby czy jej numerze
telefonu. Do tego celu służy RDF Schema.  

\subsection*{OWL}

\insertscaledimage{0.55}{images/teoria/owl-tree.png}{Drzewo genealogiczne rodziny języków OWL \cite{jankowski}}{fig:owl:tree}

Użycie języków RDF i RDF Schema pozawala na tworzenie rozbudowanych ontologii, jednak posiadają one pewne ograniczenia. Nie możemy np. zdefiniować liczby wystąpień danego
obiektu. A więc nie będziemy mogli zapisać informacji o tym, że kwartet smyczkowy składa się dokładnie z 4 muzyków. 
\par
Powstało jeszcze kilka innych języków opisu ontologii. Jednymi z bardziej dojrzałych są opierające się na wcześniejszych rozwiązaniach języki: DAML i OIL. 
Język DAML powstał w ramach badań prowadzonych przez wojskową agencję rządową w USA w roku 2000. OIL jest językiem, który powstał przy wsparciu Unii Europejskiej. Pozwala on 
na definiowanie ontologii, jednak nie posiada wsparcia dla XML.  W roku 2003 wyłonił się język będący połączeniem tych dwóch języków: DAML+OIL. Na podstawie tego  
ostatniego języka W3C stworzyła język OWL (ang. Web Ontology Language). Na rysunku \ref{fig:owl:tree} przedstawiono historię rozwoju języków zapisu ontologii.

Język OWL stał się standardem zapisu ontologii w sieciach semantycznych. Częściowo bazuje on na trójkach RDF. Semantyka języka OWL opiera się na logice opisowej,
a dokładnie na dialekcie SHIOQ(D), który rozszerza konstruktory ALC o np.: ograniczenia liczebności (kardynalność), role przechodnie, symetryczne czy odwrotne. 

 
W celu dostosowania języka OWL do wymogów każdego z użytkowników, posiada on 3 dialekty \cite{owl_rek,owl_rek1}:
\begin{itemize}
 \item {\bf OWL Lite} 
został stworzony z myślą o użytkownikach, którzy chcą zbudować hierarchię i nadać jej niewielkie ograniczenia. OWL Lite pozwala na nadanie kardynalności 0 lub 1. 
Dialekt ten nie dopuszcza konstrukcji posiadających złożone opisy klas oraz wymaga separacji typów.  
  \item {\bf OWL DL}
jest dialektem  o dużej sile ekspresji. 
W szczególności nałożona w nim nacisk na możliwości wnioskowania. OWL~DL może być mapowany na logikę opisową SHOIN, przez co mamy pewność, że czas obliczeń wnioskowania 
jest skończony. Dialekt ten posiada bardzo dużo ograniczeń, np. ograniczenie rozdzielności  typów. 
 

  \item {\bf OWL Full}
jest najbardziej rozbudowanym dialektem, w porównaniu do pozostałych dostarczą największej siły ekspresji. Jednak poprzez mniejszą liczbę ograniczeń, w szczególności 
dotyczących właściwości przechodnich, jest językiem nierozstrzygalnym. Co oznacza, iż nie mamy pewności, że wnioskowanie zakończy się w skończonym czasie. 
\end{itemize}


Każdy z powyżej opisanych języków jest rozszerzeniem swojego poprzednika. Zachodzi niżej opisana kompatybilność pomiędzy dialektami. 
\begin{itemize}
 \item  Każda poprawna ontologia OWL Lite jest poprawną ontologią OWL~DL,
 \item  Każda poprawna ontologia OWL~DL jest poprawną ontologią OWL Full,
 \item  Każdy poprawny wniosek w OWL Lite jest poprawnym wnioskiem w OWL~DL,
 \item  Każdy poprawny wniosek w OWL~DL jest poprawnym wnioskiem w OWL Full.
\end{itemize}


A oto przykład ontologii rodziny zapisany w języku OWL\cite{jankowski}: 

\subsubsection*{Mężczyzna to osoba nie będąca kobietą}
\begin{verbatim}
<owl:Class rdf:ID="Mezczyzna">
  <rdfs:subClassOf rdf:resource="#Osoba" />
  <owl:disjointWith rdf:resource="#Kobieta" />
</owl:Class>
\end{verbatim}

\subsubsection*{Rodzic to taka Osoba, która ma przynajmniej jedno dziecko}
\begin{verbatim}
 <!-- definicja klasy rodzic jako klasy rownowaznej
do klasy anonimowej, ktorej wystapienia podlegaja
ograniczeniu liczebnosciowemu (tutaj przynajmniej
1 dziecko) zwiazanemu z wlasciwoscia maDziecko -->
<owl:Class rdf:ID="Rodzic">
  <owl:equivalentClass>
  <!-- klasa rownowazna jest klasa anonimowa o ponizszej
   definicji -->
    <owl:Class>
      <owl:intersectionOf rdf:parseType="Collection">
        <owl:Restriction>
           <owl:onProperty rdf:resource="#maDziecko">
           <owl:minCardinality
                rdf:datatype="http://www.w3.org/2001/XMLSchema#int">
                1
           </owl:minCardinality>
         </owl:Restriction>
         <owl:Class rdf:ID="Osoba"/>
       </owl:intersectionOf>
     </owl:Class>
   </owl:equivalentClass>
</owl:Class>
<!-- definicja wlasciwosci maDziecko,
ktorej dziedzina i zakresem jest klasa osoba,
wlasciwosc maDziecko jest wlasciwoscia odwrotna
w stosunku do wlasciwosci maRodzica -->
<owl:ObjectProperty rdf:about="#maDziecko">
   <rdfs:domain rdf:resource="#Osoba"/>
   <rdfs:range rdf:resource="#Osoba"/>
   <owl:inverseOf rdf:resource="#maRodzica" />
</owl:ObjectProperty>
<!-- definicja wlasciwosci maRodzica -->
<owl:ObjectProperty rdf:about="#maRodzica">
   <!-- zdefiniowanie dziedziny i zakresu wlasciwosci,
     dla objectProperty dziedzina i zakres sa opisami klas -->
   <rdfs:domain rdf:resource="#Osoba"/>
   <rdfs:range rdf:resource="#Osoba"/>
   <owl:inverseOf rdf:resource="#maDziecko"/>
</owl:ObjectProperty>

\end{verbatim}


\subsubsection*{Ojciec to Mężczyzna, będący Rodzicem}
\begin{verbatim}
<!-- definicja klasy ojciec jako klasy rownowaznej
do przeciecia (czesci wspolnej) klas rodzic
i mezczyzna -->
<owl:Class rdf:ID="Ojciec">
  <owl:equivalentClass>
     <owl:Class>
       <owl:intersectionOf rdf:parseType="Collection">
         <owl:Class rdf:about="#Rodzic"/>
         <owl:Class rdf:about="#Mezczyzna"/>
       </owl:intersectionOf>
     </owl:Class>
   </owl:equivalentClass>
</owl:Class>
\end{verbatim}



\subsubsection*{Dziecko to Osoba, która ma dwoje Rodziców}
\begin{verbatim}
<owl:Class rdf:about="#Dziecko">
  <owl:equivalentClass>
    <owl:Class>
      <owl:intersectionOf rdf:parseType="Collection">
        <owl:Restriction>
          <owl:onProperty>
            <owl:ObjectProperty rdf:ID="maRodzica"/>
          </owl:onProperty>
          <owl:cardinality
              rdf:datatype="http://www.w3.org/2001/XMLSchema#int">
              2
          </owl:cardinality>
        </owl:Restriction>
        <owl:Class rdf:about="#Osoba"/>
      </owl:intersectionOf>
     </owl:Class>
   </owl:equivalentClass>
</owl:Class>
\end{verbatim}
  
Język OWL jest obecnie najlepszym sposobem zapisu ontologii, o czym może świadczyć jego popularność wśród inżynierów ontologów. Chociaż umożliwia on zapisanie dowolnej informacji, 
w październiku 2009 W3C zaproponowało jego ulepszoną wersję OWL2 \cite{OWL2}. Jednak ze względu na wcześniejsze prace grupy projektowej oraz ustalenia z opiekunem projektu, 
w tej pracy zostanie użyty język OWL. 
