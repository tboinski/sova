
\chapter{Przegląd dostępnych rozwiązań do wizualizacji ontologii}
\section{Wstęp}
W tej części pracy przedstawiona zostanie szczegółowa analiza rozwiązań obrazujących ontologie, która umożliwi wyznaczenie najlepszej aplikacji
 do wizualizacji.
\par   W pierwszej części rozdziału zostaną opisane sposoby wizualizacji ontologii oraz cechy dobrej wizualizacji. Następnie zostaną przedstawione 
wybrane aplikacje oraz sposób ich analizy. W ostatniej części rozdziału przedstawione zostaną wyniki testów oraz zostanie wyznaczone najlepsze rozwiązanie. 
\section{Sposoby wizualizacji ontologii}
Budowanie ontologii nie jest łatwą pracą. Nawet w~niewielkich rozwiązaniach często wymaga analizy dużej liczny elementów. 
Aby poprawnie tworzyć ontologie, jest wymagana  ich wizualizacja. Upraszcza ona postrzeganie całego rozwiązania oraz konkretnych jego części.
 Umożliwia również łatwą ocenę i poprawę błędów. 
\par
Istnieją różne sposoby graficznej prezentacji ontologii. Większość 
z nich opiera się na modelu 2D, w którym  ontologie są wizualizowane są za~pomocą grafu. Wierzchołki tego grafu 
oznaczają elementy ontologii, a~krawędzie odzwierciedlają związki pomiędzy tymi elementami. Istnieją również rozwiązania 3D np.: OntoSphere \cite{OntoSphere}. W~tym przypadku klasy zostały odzwierciedlone
 jako sfery (kule), a ich podklasy znajdują się pod nimi w trójwymiarowym drzewie przypominającym stożek. Klasy są połączone trójwymiarową krawędzią 
o~odpowiednim kolorze i kształcie grotu. 
\section{Elementy wizualizacji}
Wizualizacja ontologii o~dużej liczbie elementów może stać się nieprzejrzysta i~nieczytelna, dlatego należy zwrócić uwagę na niżej wymienione elementy \cite{vizComp}.
\begin{enumerate}
\item{\bf Sposób całościowej wizualizacji}

\nopagebreak

Dobra aplikacja do wizualizacji ontologii powinna wyświetlić wszystkie jej elementy. Użytkownik powinien mieć możliwość podglądu całej edytowanej
ontologii, jak również wybranej jej części. Aby wizualizacja była przejrzysta i~czytelna musi posiadać możliwość wyłączenia wizualizacji niektórych elementów, 
np. ukrycie  rysowania związków danego typu.
\item{\bf Sposób wizualizacji klas i~bytów}

\nopagebreak

Klasy oraz ich instancje są najważniejszymi elementami ontologii, dlatego jest konieczne ich poprawne wizualizowanie. Zły sposób obrazowania tych elementów 
może zniechęcić użytkownika do korzystania z rozwiązania. Wizualizacja powinna pokazywać wszystkie klasy lub tylko wybrane przez użytkownika. 
Każda klasa powinna zawierać przynajmniej nazwę, zapisaną w~zrozumiały sposób. Klasy mogą być obrazowane jako wierzchołki grafu. Powinny one jednoznacznie
 wyróżniać się od instancji klas. Np. poprzez kolor lub kształt wierzchołka. To podejście może się jednak nie sprawdzić przy wizualizacji ontologii o~znacznej
liczbie elementów. W takim przypadku może istnieć wiele połączonych z wybranym węzłem klas, obrazowanie takiej sytuacji na grafie może okazać się nieczytelne. 
Alternatywnym rozwiązaniem jest wyświetlenie wszystkich powiązanych klas z zadanym elementem w oddzielnym oknie aplikacji.   

\item{\bf Wizualizacja taksonomii}

\nopagebreak

Klasy tworzą hierarchię (taksonomię) klas, poprzez relację nadklasa - podklasa (związek isa). Prezentacja taksonomii ma kluczowe znaczenia dla
 zrozumienia relacji dziedziczenia pomiędzy klasami. Wizualizacja powinna dać możliwość całościowego bądź częściowego przeglądu hierarchii dziedziczenia.

\item{\bf Sposób wizualizacji relacji}

\nopagebreak

Relacje występujące pomiędzy elementami są najczęściej obrazowane jako związek łączący te elementy. Różne typy relacji można wyróżnić poprzez nadanie 
etykiety krawędzi łączącej, poprzez zmianę koloru lub kształtu linii. Ważne jest, aby można było opcjonalnie wyłączyć wizualizację zadanych typów
 relacji. 

\item{\bf Wizualizacja właściwości}

\nopagebreak

Ontologie byłyby bardzo ubogie, gdyby nie posiadały właściwości. Właściwości pozwalają na zdefiniowanie ogólnych informacji dotyczących zarówno
 klas jak i~instancji klas. Wyróżniamy dwa typy właściwości: właściwości dla których zakresem są obiekty (owl:objectProperty) bądź wartości 
(owl:dataProperty). Właściwości powinny być zaznaczone na grafie wizualizacji lub w oddzielnym oknie do tego przeznaczonym. 

\item{\bf Wyszukiwanie}

\nopagebreak

Podczas wizualizacji dużej ontologii możemy napotkać na problem szybkiego wyszukania interesującego nas elementu. Dlatego aplikacji wizualizacji powinna 
dać możliwość szukania elementów w grafie ontologii.    

\end{enumerate}


\section{Przedstawienie wybranych rozwiązań do wizualizacji ontologii}
Na rynku dostępnych jest wiele programów do tworzenia i~edycji ontologii. Większość z~nich ułatwia budowanie ontologii poprzez ich graficzną reprezentację.
 Niektóre rozwiązanie, takie jak rozwijany na~Uniwersytecie Stanforda edytor \protege , poprzez pluginy dostarczają kilku sposobów wizualizacji. Inne poza
ogólnym mechanizmem wizualizacji posiadają dodatkowy tryb do prezentacji drzewa wywnioskowanej hierarchii klas i bytów. Poniżej przedstawiono kilka rozwiązań 
pozwalających obrazować ontologię. 

\begin{enumerate}

 \item{\bf OntoViz}

  \nopagebreak 
 
 OntoViz \cite{OntoViz}, rozwijany na Uniwersytecie Stanforda, jest najczęściej używanym pluginem \proteges do wizualizacji ontologii. Wykorzystuje bibliotekę GraphViz do 
tworzenia prostych grafów 2D (\figurename \space \ref{fig:viz:ontoviz} ). Na grafie klasy są reprezentowane jako prostokąty zawierające informację
 o~nazwie klasy jak i~dodatkowe informacje o~relacjach i~właściwościach klasy. Istnieje możliwość okrojenia i~ukrycia części wyświetlanych komponentów przez 
panel konfiguracyjny po prawej stronie.


\insertimage{images/wizualizacja/ontoviz.png}{Przykład wizualizacji ontologii za pomocą OntoViz}{fig:viz:ontoviz}

 \item{\bf Jambalaya}

  \nopagebreak 
 Jambalaya \cite{Jambalaya,JambalayaProtege} jest pluginem do \proteges rozwijanym na~Uniwersytecie w~Wiktorii (Kanada). Opiera się na graficznym zestawie
 narzędzi Piccolo do tworzenia interaktywnych grafów 2D. Jambalaya charakteryzuje się kilkoma rodzajami widoku oraz specyficznym sposobem obrazowania związku~ISA.

 \item{\bf Growl} \cite{growl}

  \nopagebreak 
Rozwiązanie powstało na Uniwersytecie w Vermont. Posiada ono ciekawy sposób wizualizacji ontologii, w którym autorzy postawili na kompletność wizualizacji. 
Edytor Growl pozwala na wyświetlenie wszystkich elementów ontologii na grafie, co daje możliwość łatwego zrozumienia edytowanej ontologii. 
 \item{\bf OCS}

  \nopagebreak 
  OCS (ang. ONTOLOGY CREATION SYSTEM) \cite{jankowski,boinski} jest systemem do tworzenia i~edycji ontologii rozwijanym na Wydziale Elektroniki, Telekomunikacji i~Informatyki Politechniki
 Gdańskiej. Edytor ontologii posiada 2 sposoby wizualizacji ontologii. Pierwszym jest hierarchia klas i~bytów, drugi zaś to ogólny obraz ontologii. System wizualizuje
 tylko podstawowe elementy ontologii, przedstawia tylko klasy, instancje i~podstawowe relacje pomiędzy nimi. 
 
\end{enumerate}


\section{Opis metody porównania}
\subsection*{Sposób wyznaczenia oceny rozwiązań}
Porównanie rozwiązań do wizualizacji ontologii będzie podzielone na kilka kategorii. Każda z kategorii będzie posiadała zestaw zagadnień lub pytań. 
Odpowiedzi na pytania pozwolą wyznaczyć ocenę dla zadanej kategorii. Każde z zagadnień, poza treścią, posiada również opis badanej cechy, liczbę 
punktów, które mogą być przyznane za pytanie oraz sposób przydzielania punktów. Ocena dla danej kategorii będzie liczbą zmiennoprzecinkową 
z~przedziału $ <0,1> $  i~będzie wyznaczana na podstawie wzoru:
\begin{equation}
   {O_{K}} =\frac{\sum_{k=1}^N  {p_{u}(i)}}{\sum_{k=1}^N  {P_{max}(i)}} 
\end{equation}
gdzie: \\
$ N $ -- liczba pytań w kategorii, \newline
$ O_{K} $ -- ocena danej kategorii, wartość ta będzie z~przedziału $ <0,1> $,\newline
$ p_{u}(i) $ -- liczna punktów uzyskanych w i-tym pytaniu, \\
$ P_{max}(i) $ -- maksymalna liczba punktów możliwa do uzyskania na i-te pytanie. \\


Całkowita ocena będzie również liczną rzeczywistą z przedziału $ <1,0> $, gdzie $1$ jest najlepszą notą do zdobycia. Ocena rozwiązania będzie średnią 
ważoną ocen zdobytych dla poszczególnych kategorii. Wagi zostaną podane poniżej wraz z opisem kategorii. 

\subsection*{Opis kategorii}

\begin{enumerate}
 \item{\bf Kompletność wizualizacji} 
  \nopagebreak   \newline
Wiele rozwiązań obrazujących ontologie wyświetla tylko podstawowe informacje o~klasach i~związkach pomiędzy nimi. Aby ułatwić pracę nad ontologiami, 
wizualizacja musi być kompletna, wszystkie jej elementy muszą być przedstawione na grafie lub w dodatkowych oknach aplikacji.  Kryterium to pozwoli
ocenić kompletności wizualizacji. 
  \nopagebreak  \newline
{\bf Waga} : 5

 \item{\bf Przejrzystość wizualizacji}  \newline
  \nopagebreak                  
Kolejnym kryterium jest przejrzystość wizualizacji. Aby wizualizacja była zrozumiała musi być czytelna, szczególnie gdy wczytamy dużą ontologię. 
Kryterium to pozwoli sprawdzić, czy aplikacja posiada opcję filtrowania danych i wizualizuje tylko zadane elementy lub związki. 
  \nopagebreak \newline
{\bf Waga} : 4

 \item{\bf Sposoby wizualizacji} \newline
  \nopagebreak                  
  Ontologia może być przedstawiona na drzewie bądź grafie, jej elementy mogą być rozmieszczone w~różnych odległościach. Kryterium pozwoli sprawdzić
czy poza podstawową wizualizacją rozwiązania posiadają różne algorytmy i sposoby obrazowania danych. Ocenione zostanie również, czy wizualizacja 
opiera się o~dane jawnie pobrane z ontologii, czy może jest wywnioskowaną hierarchią klas i~bytów.   
  \nopagebreak \newline
{\bf Waga} : 3


 \item{\bf Użyteczność} \newline
  \nopagebreak                  
  Kryterium to pozwoli ocenić w jakim stopniu rozwiązanie jest przyjazne dla użytkownika. Sprawdzi również stopień dostępności pomocy poprzez 
istnienie strony Internetowej, samouczków czy forów poświęconych danemu rozwiązaniu. Zwrócona zostanie też uwaga na licencję oprogramowania. Wyróżnione 
zostaną aplikacje wydane na darmowych licencjach i posiadające otwarty kod. \newline
  \nopagebreak 
{\bf Waga} : 3

\end{enumerate}




\subsection*{Kryteria oceny dla kategorii kompletność}

Poniżej znajduje się lista kryteriów, które pozwolą na ocenę kompletności wizualizacji. Kryteria zostały wyróżnione i~pogrupowane na podstawie
elementów składowych ontologii. 

\begin{longtable}{|m{3cm}|m{10cm}|}\hline
Kryterium:                   &  \bf{Sposób wizualizacji klas}\\ \hline
Opis kryterium:              & Klasy są najważniejszymi elementami ontologii, dlatego wizualizacja powinna zawierać przynajmniej nazwę klasy. 
                               Klasy poprzez kolor lub kształt powinny wyróżniać się na tle innych elementów ontologii.\\ \hline
Liczna punktów do uzyskania: & 2 \\ \hline
Sposób oceny:                &  1 punkt za wizualizację i jednoznaczne oznaczenie klas, \newline
                                1 punkt za wyróżnienie klas od innych elementów ontologii.\\ \hline

\end{longtable}



\begin{longtable}{|m{3cm}|m{10cm}|}\hline
Kryterium:                   & \bf{Sposób wizualizacji bytów}\\ \hline
Opis kryterium:              & Byty, będące instancjami klas, powinny odróżniać się od klas, np.~poprzez kolor lub kształt wierzchołka. 
                               Każdy z nich powinien identyfikować się nazwą.  \\ \hline
Liczna punktów do uzyskania: & 2\\ \hline
Sposób oceny:                &  1 punkt za wizualizację i odpowiednie oznaczenie bytów,  \newline
                                1 punkt za wyróżnienie bytów od innych elementów np. klas.\\ \hline

\end{longtable}




\begin{longtable}{|m{3cm}|m{10cm}|}\hline
Kryterium:                   & \bf{ Sposób wizualizacji klas anonimowych }\\ \hline
Opis kryterium:              &  Klasy anonimowe są rezultatem relacji innych klas. Klasą anonimową może być np.~suma dwóch klas. 
				Elementy te nie posiadają nazwy, dlatego wizualizacja powinna pokazać z~jakiej relacji i~z~jakich klas one powstały.     \\ \hline
Liczna punktów do uzyskania: &  2  \\ \hline
Sposób oceny:                &  1 punkt za wizualizację klas anonimowych,  \newline 
                                1 punkt za przejrzyste ukazanie relacji i
				klas, których owa klasa jest rezultatem
                                 \\ \hline
                               
\end{longtable}



\begin{longtable}{|m{3cm}|m{10cm}|}\hline
Kryterium:                   & \bf{ Wizualizacja relacji dotyczących klas i bytów }\\ \hline
Opis kryterium:              &  Kryterium pozwoli ocenić stopień pokrycia związków pomiędzy klasami i bytami  \\ \hline
Liczna punktów do uzyskania: &  11   \\ \hline
Sposób oceny:                &  1  punkt za wizualizację związku ``subclass'',\newline
                                1  punkt za wizualizację związku ``instanceOf'', \newline	      
                                1  punkt za wizualizację związku ``equivalentClass'',\newline
                                1  punkt za wizualizację związku ``disjointWith'',\newline
                                1  punkt za wizualizację związku ``differentFrom'',\newline
                                1  punkt za wizualizację związku ``allDifferent'',\newline
                                1  punkt za wizualizację związku ``sameAs'',\newline
                                1  punkt za wizualizację związku ``oneOf'',\newline
                                1  punkt za wizualizację związku ``unionOf'',\newline
                                1  punkt za wizualizację związku ``intersectionOf'',\newline
                                1  punkt za wizualizację związku ``complementOf''

  \\ \hline
\end{longtable}
\pagebreak[4]
\begin{longtable}{|m{3cm}|m{10cm}|}\hline
Kryterium:                   & \bf{ Wizualizacja właściwości }\\ \hline
Opis kryterium:              &  Właściwości pozwalają na zdefiniowanie ogólnych informacji dotyczących zarówno klas jak i instancji klas. \\ \hline
Liczna punktów do uzyskania: &  2   \\ \hline
Sposób oceny:                &     1  punkt za wizualizację DataTypeProperty,\newline
                                   1  punkt za wizualizację ObjectProperty, \
			      \\ \hline
\end{longtable}

\begin{longtable}{|m{3cm}|m{10cm}|}\hline
Kryterium:                   & \bf{ Wizualizacja związków odnoszących się do właściwości}\\ \hline
Opis kryterium:              &    Kryterium pozwoli ocenić stopień pokrycia związków dotyczących właściwości  \\ \hline
Liczna punktów do uzyskania: &  7   \\ \hline
Sposób oceny:                &  1  punkt za wizualizację związku ``subProperty'',\newline
                                1  punkt za wizualizację związku ``equivalentProperty'', \newline	      
                                1  punkt za wizualizację związku ``functionalProperty'',\newline
                                1  punkt za wizualizację związku ``inversFunctionalProperty'',\newline
                                1  punkt za wizualizację związku ``symmetricProperty'',\newline
                                1  punkt za wizualizację związku ``transitiveProperty'',\newline
                                1  punkt za wizualizację związku ``inverseOf(property)'',
 \\ \hline
\end{longtable}



\begin{longtable}{|m{3cm}|m{10cm}|}\hline
Kryterium:                   & \bf{ Wizualizacja pozostałych elementów ontologii }\\ \hline
Opis kryterium:              &  Kryterium pozwoli ocenić stopień pokrycia pozostałych elementów ontologii, takich jak np. kardynalność.  \\ \hline
Liczna punktów do uzyskania: &   4  \\ \hline
Sposób oceny:                &  1  punkt za wizualizację  ``sameValuesFrom/allValuesForm'',\newline
                                1  punkt za wizualizację   ``Cardinality'', \newline	      
                                1 punkt za wizualizację  ``Domain'',\newline
                                1 punkt za wizualizację  ``Range'',\newline 
  \\ \hline
\end{longtable}


\subsection*{Kryteria oceny dla kategorii przejrzystość wizualizacji}
Poniżej znajduje się lista kryteriów i pytań, które pozwolą na ocenę czytelności i zrozumiałości wizualizacji. 


\begin{longtable}{|m{3cm}|m{10cm}|}\hline
Kryterium:                   & \bf{ Czytelność ontologii }\\ \hline
Opis kryterium:              &  Kryterium pozwoli ocenić jakość wizualizacji dużej, liczącej ponad 100 klas ontologii. Wizualizacja powinna dać możliwość 
                                wyświetlenia całej ontologii.  Oceniony zostanie sposób rozmieszczenia 
                                 danych. Należy sprawdzić, czy elementy nie nachodzą na siebie i czy nie zakrywają innych elementów ontologii. 
                                \\ \hline
Liczna punktów do uzyskania: &  5   \\ \hline
Sposób oceny:                &  0-3 punktów za ogólną czytelność dużej ontologii.\newline
                                0-2 punktów za nienachodzenie na siebie elementów. \\ \hline
\end{longtable}

\begin{longtable}{|m{3cm}|m{10cm}|}\hline
Kryterium:                   & \bf{ Dostępność filtrów }\\ \hline
Opis kryterium:              &  Wprowadzenie możliwości wizualizacji tylko wybranych typów elementów ontologii znacznie ułatwia pracę z tą ontologią.
                                Wizualizacja powinna dać możliwość filtrowania danych. Kryterium to sprawdzi jakie filtry posiadają wybrane 
                                rozwiązania i w jakim stopniu filtry pokrywają liczbę typów wyświetlanych elementów.\\ \hline
Liczna punktów do uzyskania: &  3   \\ \hline
Sposób oceny:                &  0-3 punkty za ilość filtrów.    \\ \hline
\end{longtable}


\subsection*{Kryteria oceny dla kategorii sposoby wizualizacji}
Poniżej zamieszczono listę kryteriów, które pozwolą ocenić różnorodność sposobów wizualizacji zaproponowanych w testowanych rozwiązaniach. 


\begin{longtable}{|m{3cm}|m{10cm}|}\hline
Kryterium:                   & \bf{ Liczba sposobów wizualizacji }\\ \hline
Opis kryterium:              &  Kryterium pozwoli sprawdzić,  ile różnych sposobów rozmieszczenia danych i wizualizacji znajduje się w danym rozwiązaniu. 
                               \\ \hline
Liczna punktów do uzyskania: &  3   \\ \hline
Sposób oceny:                &  0  punktów, jeśli rozwiązanie posiada tylko jeden sposób wizualizacji \newline	
                                1  punkt, jeśli wizualizacja posiada 2 różne sposoby obrazowania danych \newline
                                2  punkty dla 3 różnych sposobów wizualizacji \newline
                                3 punkty, jeśli rozwiązanie pozwala na wizualizację za pomocą więcej niż 3 różnych algorytmów obrazowania danych\\ \hline
\end{longtable}

\begin{longtable}{|m{3cm}|m{10cm}|}\hline
Kryterium:                   & \bf{ Wizualizacja wywnioskowanej hierarchii  }\\ \hline
Opis kryterium:              &  Wywnioskowana hierarchia bytów i klas pozwala na łatwiejsze zrozumienie ontologii. Może być też przydatna przy kontroli 
                                jakości ontologii, np. przy sprawdzaniu spójności ontologii.     \\ \hline
Liczna punktów do uzyskania: &  1   \\ \hline
Sposób oceny:                &  1 punkt, jeśli rozwiązanie umożliwia wizualizację hierarchii wygenerowanej przez narzędzie wnioskujące   \\ \hline
\end{longtable}


\subsection*{Kryteria oceny dla kategorii użyteczność}
Poniżej znajduje się lista kryteriów które pozwolą ocenić użyteczność rozwiązań. 

\begin{longtable}{|m{3cm}|m{10cm}|}\hline
Kryterium:                   & \bf{ Użyteczność  }\\ \hline
Opis kryterium:              &   Oceniona zostanie intuicyjność  łatwość  korzystania i rozwiązania. \\ \hline
Liczna punktów do uzyskania: &   6  \\ \hline
Sposób oceny:                &  0-2 punktów za łatwość instalacji \newline
                                0-3 punktów za intuicyjność interfejsu \newline
                                1 punkt za możliwość wyszukiwania  \\ \hline
\end{longtable}


\begin{longtable}{|m{3cm}|m{10cm}|}\hline
Kryterium:                   & \bf{ Dostępność pomocy }\\ \hline
Opis kryterium:              &  Pozwali określić stopień dostępności pomocy dla rozwiązania   \\ \hline
Liczna punktów do uzyskania: &  3   \\ \hline
Sposób oceny:                &  1 punkt dla rozwiązań posiadających własną stronę internetową \newline
                                1 punkt za istnienie samouczków lub instrukcji użytkowania dla rozwiązania  \newline
                                1 punkt za dostępność forów  związanych z rozwiązaniem. \\ \hline
\end{longtable}

\begin{longtable}{|m{3cm}|m{10cm}|}\hline
Kryterium:                   & \bf{ Licencja }\\ \hline
Opis kryterium:              &  Zostanie sprawdzona licencja, na której zostało wydane oprogramowanie.  \\ \hline
Liczna punktów do uzyskania: &   2  \\ \hline
Sposób oceny:                &   1 punkt, jeśli oprogramowanie jest na darmowej licencji\newline
                                 1 punkt, jeśli posiada otwarty kod. \\ \hline
\end{longtable}


\section{Test rozwiązań do wizualizacji ontologii}
% 
% \subsection*{Test pluginu OntoViz}
% 
% Poniżej znajdują się wyniki test pluginu OntoViz w wersji 1.0. Testy zostały przeprowadzone z użyciem \proteges~3.4.4.
% 
% \begin{enumerate}
%  \item{\bf Kompletność wizualizacji}
% 
%   \begin{itemize}
%    \item[-]{\bf Sposób wizualizacji klas}
% 
%   \nopagebreak  
%     W rozwiązaniu tym klasy wizualizowane są jako prostokąty, posiadają jednoznaczną nazwę identyfikującą dany obiekt (+1~punkt).
% Klasy wyróżnione zostały za pomocą koloru obramowania wierzchołka od innych elementów (+1 punkt).
%   Rozwiązanie dostaje  2~punkty.
% 
%    \item[-]{\bf Sposób wizualizacji bytów}
% 
%   \nopagebreak 
% Byty wizualizowane są identycznie jak klasy, posiadają swoją  nazwę, wyróżniają się kolorem od klas. Za wizualizację bytów rozwiązanie 
% dostaje 2 punkty.
%    \item[-]{\bf Wizualizacja klas ananimowych}
% 
%   \nopagebreak 
% Klasy anonimowe nie są wizualizowane na grafie - 0~punktów.  
% 
%    \item[-]{\bf Wizualizacja relacji}
% 
%   \nopagebreak 
% OntoViz obrazuje poniżej wymienione relacja: subClass, instanceOf, differentFrom, sameAs. 
% Rozwiązanie w tej kategorii uzyskuje 4 punkty. 
% 
% 
%  \item[-]{\bf Wizualizacja właściwości} 
% 
%   \nopagebreak
% Właściwości wizualizowane są związkami łączącymi klasy. 2~punkty
%  \item[-]{\bf Wizualizacja związków odnoszącymi się do właściwościami} 
% 
%   \nopagebreak
% Brak funkcjonalności. 0~punktów.
%  \item[-]{\bf Wizualizacja pozostałych elementów ontologii} 
% 
%   \nopagebreak
% Żadne z wymienionych związków w tym kryterium nie są wizualizowane. 0~punktów.
%   \end{itemize}
% 
% \item{\bf Przejrzystość wizualizacji}
% \begin{itemize}
%  \item[-]{\bf Czytelność ontologii} 
% 
%   \nopagebreak
% Wczytana ontologia została wyświetlona jako bardzo rozciągnięty graf. Aby zobaczyć ją w całości, trzeba było zmniejszyć obrazek do 
% tego stopnia, iż stawał on się kompletnie nieczytelny. Dalego za ogólną wizualizację rozwiązanie dostaje +1~punt. Jednak żadna z wyświetlonych 
% klas nie nakrywała innej klasy (+2~punkty). Włącznie 3~punkty.
% 
%  \item[-]{\bf Dostępność filtrów} 
% 
%   \nopagebreak
% OntoViz posiada panel z filtrami. Umożliwia on wizualizację tylko wybranych klas i bytów. Daje możliwość wyłączenia
%  wizualizacji niektórych elementów. Nie posiada on jednak możliwości wyłączania konkretnych typów elementów ontologii, 
% dlatego rozwiązanie w tej dziedzinie uzyskuje 2 punkty.
% 
% 
% \end{itemize}
% 
% \item{\bf Sposoby wizualizacji}
% \begin{itemize}
%  \item[-]{\bf Liczba sposobów wizualizacji } 
% 
%   \nopagebreak
% 
% Plugin OntoViz opiera się tylko na  jednym sposobie prezentowanie danych. 0~punktów. 
% 
%  \item[-]{\bf  Wizualizacja wywnioskowanej hierarchii } 
% 
%   \nopagebreak
%  Rozwiązanie nie umożliwia wizualizacji wywnioskowanej hierarchii klas i bytów. 0~punktów.
% \end{itemize}
% 
% 
% \item{\bf Użyteczność}
% \begin{itemize}
%  \item[-]{\bf Użyteczność  } 
% 
%   \nopagebreak
% Instalacja \proteges nie sprawiła żadnych problemów. Instalowanie dodatkowych pluginów również jest proste i polega na skopiowaniu plugin
%  do odpowiedniego katalogu (+2~punkty). 
% 
% Interfejs ogólne jest przyjemny, jednak słabo opisany. Podczas pierwszego uruchomienia trzeba poświęcić trochę czasu 
% na zrozumienie jak dodawać klasy do wizualizacji i jak obrazować różne elementy. (dlatego +2~punkty). 
% OntoViz posiada możliwość wyszukiwania, jednak okazało się, że funkcja ta nie działa. (0~punktów). Łącznie w tej kategorii +4~punkty.
% 
% 
%  \item[-]{\bf Dostępność pomocy  } 
% 
%   \nopagebreak
%   Projekt posiada swoją stronę internetową, na której możemy odnaleźć informacje o~sposobie instalacji jak i~również instrukcję użycia. 
% Aplikacja nie posiada forum dla użytkowników. Łącznie otrzymuje 2~punkty.
% 
%  \item[-]{\bf Licencja  } 
% 
%   \nopagebreak
%   OntoViz został wydany na licencji Mozilla Public License. Licencja ta umożliwia korzystanie z oprogramowania w~celach osobistych jak i~komercyjnych. 
% Udostępniony jest też kod aplikacji. Za otwartą licencję OntoViz uzyskuje 2~punkty.
% \end{itemize}
% 
% 
% \end{enumerate}
% 
% 
% \subsection*{Test pluginu Jambalaya}
% 
% Poniżej znajdują się wyniki test pluginu Jambalaya w wersji 2.7.0. Testy zostały przeprowadzone z użyciem \proteges~3.4.4.
% 
% \begin{enumerate}
%  \item{\bf Kompletność wizualizacji}
% 
%   \begin{itemize}
%    \item[-]{\bf Sposób wizualizacji klas}
% 
%   \nopagebreak  
% Jambalaya wizualizacje klasy za pomocą prostokątów, wewnątrz których znajdują się mniejsze prostokąty ich podklas. Każda z klas identyfikowana jest nazwą,
%  nawet gdy najedziemy na bardzo mały zagnieżdżony element, aplikacja wyświetli etykietę jego nazwy. Rozwiązanie otrzymuje 2~punkty.
% 
% 
%    \item[-]{\bf Sposób wizualizacji bytów}
% 
%   \nopagebreak 
% Wizualizacja bytów jest bardzo podobna do wizualizacji klas. Byty zostały wyróżnione innym kolorem niż klasy i znajdują się we wnętrzu prostokąta 
% klasy której są instancjami. Rozwiązanie uzyskuje 2~punkty.
% 
% 
%    \item[-]{\bf Wizualizacja klas ananimowych}
% 
%   \nopagebreak 
% Klasy anonimowe nie są wizualizowane na grafie - 0~punktów. 
% 
% 
%    \item[-]{\bf Wizualizacja relacji}
% 
%   \nopagebreak 
% Plugin obrazuje poniżej wymienione relacja:  subClass, disjointWith, instanceOf. Rozwiązanie w tej kategorii uzyskuje 3.
% 
% 
%  \item[-]{\bf Wizualizacja właściwości} 
% 
%   \nopagebreak
% Właściwości klasy opisane są w formatce wewnątrz pola klasy. 2~punkty.
%  \item[-]{\bf Wizualizacja związków odnoszącymi się do właściwościami} 
% 
%   \nopagebreak
% Związki pomiędzy właściwościami nie są wizualizowane. 0~punktów.
%  \item[-]{\bf Wizualizacja pozostałych elementów ontologii} 
% 
%   \nopagebreak
% Poza kardynalnością wszystkie wymienione w tej kategorii elementy są obrazowane. Rozwiązanie uzyskuje 3~punkty.
%   \end{itemize}
% 
% \item{\bf Przejrzystość wizualizacji}
% \begin{itemize}
%  \item[-]{\bf Czytelność ontologii} 
% 
%   \nopagebreak
% Podczas wizualizacji klasy zostały rozmieszczone wewnątrz elementu thing w bardzo czytelny sposób. Klasy na sienie na nachodziły (+2~punkty),
%  jednak zostały ona zakryte przez dużą liczbę związków obrazujących właściwości. Dlatego za ogólną wizualizację +2~punkty. Łącznie 4~punkty.
% 
% 
%  \item[-]{\bf Dostępność filtrów} 
% 
%   \nopagebreak
% Jambalaya posiada wiele różnych filtrów pozwalających na ukrycie zadanych typów elementów jak i~konkretnych elementów. Rozwiązanie otrzymuje 3~punkty.
% 
% 
% 
% \end{itemize}
% 
% \item{\bf Sposoby wizualizacji}
% \begin{itemize}
%  \item[-]{\bf Liczba sposobów wizualizacji } 
% 
%   \nopagebreak
% Rozwiązanie to posiada wiele różnych typów wizualizacji elementów. Wszystkie opierają się o~przyjęty model wizualizacji potomków wewnątrz
%  wierzchołków rodziców. Rozwiązanie otrzymuje 3~punkty.
% 
%  
% 
%  \item[-]{\bf  Wizualizacja wywnioskowanej hierarchii } 
% 
%   \nopagebreak
% Jambalaya nie posiada tej funkcji - 0~punktów. 
% 
% \end{itemize}
% 
% 
% \item{\bf Użyteczność}
% \begin{itemize}
%  \item[-]{\bf Użyteczność  } 
% 
%   \nopagebreak
% Instalacja \proteges nie sprawiła żadnych problemów. Instalowanie dodatkowych pluginów również jest proste i polega na skopiowaniu plugiu do 
% odpowiedniego katalogu (+2~punkty). Interfejs pomimo znacznej funkcjonalności jest w marę intuicyjny. Łatwo można zmieniać typy 
% wizualizacji jak i dodawać filtry(+3~punkty). Istnieje również możliwość wyszukiwania elementów (+1~punkt)  Łącznie w tej kategorii +5~punkty.
% 
%  \item[-]{\bf Dostępność pomocy  } 
% 
%   \nopagebreak
% Jambalaya posiada opis instalacji oraz instrukcje użytkowania dla początkujących osób zamieszczona na stronie autorów.
%  Podobnie jak w poprzednim rozwiązaniu nie istnieje forum poświęcone temu rozwiązaniu. Łącznie w tej kategorii 2~punkty.
% 
% 
% 
%  \item[-]{\bf Licencja  } 
% 
%   \nopagebreak
% Autorzy udostępniają kod źródłowy aplikacji, jednak nic nie wspominają o licencji. 1~punkt.
% \end{itemize}
% 
% 
% \end{enumerate}
% 
% 
% \subsection*{Test aplikacji Growl}
% 
% Poniżej znajdują się wyniki testu przeprowadzonego dla aplikacji Growl w~wersji~0.02
% 
% \begin{enumerate}
%  \item{\bf Kompletność wizualizacji}
% 
%   \begin{itemize}
%    \item[-]{\bf Sposób wizualizacji klas}
% 
%   \nopagebreak  
% 
% Klasy, będące wierzchołkami grafu, wizualizowane są za pomocą prostokątów posiadających nazwę klasy w środku. Klasy wyróżniają się kształtem 
% i kolorem od pozostałych elementów klas.  Rozwiązanie otrzymuje 2~punkty.
% 
% 
%    \item[-]{\bf Sposób wizualizacji bytów}
% 
%   \nopagebreak 
% Wizualizacja bytów jest bardzo podobna do wizualizacji klas, aby je odróżnić od klas ich wierzchołki są wypełnione białym kolorem.
% Rozwiązanie otrzymuje 2~punkty. 
% 
% 
%    \item[-]{\bf Wizualizacja klas ananimowych}
% 
%   \nopagebreak 
% Klasy anonimowe są reprezentowane przez wierzchołki w~kształcie koła z~odpowiednim znakiem mówiącym o~rodzaju klasy anonimowej. 2~punkty.
% 
% 
%    \item[-]{\bf Wizualizacja relacji}
% 
%   \nopagebreak 
% Aplikacja obrazuje poniżej wymienione relacja:  subClass, equivalentClass, disjointWith, intersectionOf, complementOf, instanceOf, differentFrom,  
% oneOf, unionOf, sameAs. Rozwiązanie w~tej kategorii uzyskuje 10~punktów.
% 
% 
% 
%  \item[-]{\bf Wizualizacja właściwości} 
% 
%   \nopagebreak
% Właściwości wyświetlane jak wierzchołki grafu wyróżniają się kształtem i~kolorem. 2~punkty
% 
%  \item[-]{\bf Wizualizacja związków odnoszącymi się do właściwościami} 
% 
%   \nopagebreak
% Wizualizowane są następujące związki: subProperty, inverseOf, equivalentProperty. Aplikacja uzyskuje 4 punkty.
%  \item[-]{\bf Wizualizacja pozostałych elementów ontologii} 
% 
%   \nopagebreak
% Wizualizowany jest element związany z właściwościami sameValuesFrom/allValuesForm. Rozwiązanie uzyskuje 1~punkt.
%   \end{itemize}
% 
% \item{\bf Przejrzystość wizualizacji}
% \begin{itemize}
%  \item[-]{\bf Czytelność ontologii} 
% 
%   \nopagebreak
% Wczytanie dużej ontologii powoduje spadek jakości wizualizacji (+1~punkt). Elementy zachodzą na siebie, a~nawet się pokrywają (0~punktów). 
% Łącznie aplikacja uzyskuje 1 punkt.
% 
% 
%  \item[-]{\bf Dostępność filtrów} 
% 
%   \nopagebreak
% Growl nie posiada możliwości filtrowania danych, wszystkie elementy są wyświetlane naraz na grafie. 0~punktów. 
% 
% 
% 
% 
% \end{itemize}
% 
% \item{\bf Sposoby wizualizacji}
% \begin{itemize}
%  \item[-]{\bf Liczba sposobów wizualizacji } 
% 
%   \nopagebreak
% Growl umożliwia wizualizacje z~włączoną bądź wyłączoną animacją. Posiada dodatkowy widok związany tylko z~właściwościami. Aplikacja uzyskuje 1~punkt.
%  
% 
%  \item[-]{\bf  Wizualizacja wywnioskowanej hierarchii } 
% 
%   \nopagebreak
% Growl nie posiada tej funkcji - 0~punktów. 
% 
% \end{itemize}
% 
% 
% \item{\bf Użyteczność}
% \begin{itemize}
%  \item[-]{\bf Użyteczność  } 
%   Aplikacja jest intuicyjna i prosta~w obsłudze. Posiada bardzo prosty, nierozbudowany interfejs. Jednym mankamentem może być brak opcji wyszukiwania. 
% Łącznie aplikacja uzyskuje 5~punktów.
%   \nopagebreak
% 
% 
%  \item[-]{\bf Dostępność pomocy  } 
% 
%   \nopagebreak
% Growl podobnie jak poprzednie rozwiązania posiada własną stronę, na której znajdziemy informacje o~sposobie instalacji jak i~krótką 
% instrukcję obsługi. 2~punkty.
% 
% 
% 
%  \item[-]{\bf Licencja  } 
% 
%   \nopagebreak
% Informacja o~licencji nie została umieszczona na stronie projektu. Jednak drogą mailową uzyskałem informację, iż Growl wydany jest na licencji  GNU~GPL.
% Tym samym rozwiązanie uzyskuje 2~punkty.
% \end{itemize}
% 
% 
% \end{enumerate}
% 
% 
% 
% \subsection*{Test OCS}
% 
% Poniżej znajdują się wyniki testu przeprowadzonego dla aplikacji OCS. 
% 
% \begin{enumerate}
%  \item{\bf Kompletność wizualizacji}
% 
%   \begin{itemize}
%    \item[-]{\bf Sposób wizualizacji klas}
% 
%   \nopagebreak  
% Podstawowy widok ukazuje tylko klasy. Są one wierzchołkami grafu i identyfikują się nazwą klasy. 2~punkty.
% 
%    \item[-]{\bf Sposób wizualizacji bytów}
% 
%   \nopagebreak 
% Byty wizualizowane w~hierarchii klas i~bytów, jednak ponieważ nie ma ich na głównej wizualizacji rozwiązanie uzyskuje 0~punktów.
% 
% 
%    \item[-]{\bf Wizualizacja klas ananimowych}
% 
%   \nopagebreak 
% Brak funkcjonalności. 0~punktów.
% 
% 
%    \item[-]{\bf Wizualizacja relacji}
% 
%   \nopagebreak 
% Aplikacja obrazuje tylko dwie, wymienione relacja:  subClass,  disjointWith. Rozwiązanie w~tej kategorii uzyskuje 2~punktów.
% 
%  \item[-]{\bf Wizualizacja właściwości} 
% 
%   \nopagebreak
% Właściwości nie są obrazowane. 0~punktów
% 
%  \item[-]{\bf Wizualizacja związków odnoszącymi się do właściwościami} 
% 
%   \nopagebreak
% Brak wizualizacji. 0~punktów
%  \item[-]{\bf Wizualizacja pozostałych elementów ontologii} 
% 
%   \nopagebreak
% Brak wizualizacji. 0~punktów
%   \end{itemize}
% 
% \item{\bf Przejrzystość wizualizacji}
% \begin{itemize}
%  \item[-]{\bf Czytelność ontologii} 
% 
%   \nopagebreak
% Pomimo wizualizacji tylko klas wizualizacja dla dużej ontologii okazała się nieczytelna, klasy się nakładały. 2~punkty.
% 
% 
%  \item[-]{\bf Dostępność filtrów} 
% 
%   \nopagebreak
% Brak. 0~punktów. 
% 
% \end{itemize}
% 
% \item{\bf Sposoby wizualizacji}
% \begin{itemize}
%  \item[-]{\bf Liczba sposobów wizualizacji } 
% 
%   \nopagebreak
% OCS umożliwia 2 sposoby wizualizacji: ogólny oraz wywnioskowaną hierarchię . Aplikacja uzyskuje 1~punkt.
%  
% 
%  \item[-]{\bf  Wizualizacja wywnioskowanej hierarchii } 
% 
%   \nopagebreak
% OCS daje możliwość oglądanie wywnioskowanej hierarchii klas i~bytów. 1~punkt.
% 
% \end{itemize}
% 
% 
% \item{\bf Użyteczność}
% \begin{itemize}
%  \item[-]{\bf Użyteczność  } 
%   Aplikacja jest intuicyjna i prosta~w obsłudze. Posiada bardzo prosty, nierozbudowany interfejs. Jednym mankamentem, podobnie jak w~Growl, może 
% być brak opcji wyszukiwania. 
% Łącznie aplikacja uzyskuje 5~punktów.
%   \nopagebreak
% 
% 
%  \item[-]{\bf Dostępność pomocy  } 
% 
%   \nopagebreak
% OCS co prawda posiada stronę internetową, ale nie zawiera ona żądnych informacji związanych z~instalacją i~obsługą edytora. 0~punkty.
% 
% 
% 
%  \item[-]{\bf Licencja  } 
% 
%   \nopagebreak
% OCS posiada licencje LGPL i~tym samym uzyskuje 2~punkty.
% \end{itemize}
% 
% 
% \end{enumerate}



% \section{Podsumowanie}

Wyniki przeprowadzonych badań (zawarte w  \tablename~\ref{t:porownanie}) ukazują, iż na rynku nie ma idealnego rozwiązania do wizualizacji ontologii. 
Żadna z badanych aplikacji nie uzyskała wyniku większego niż 70\% maksymalnej oceny.  Najlepszym rozwiązaniem okazała się Jambalaya, która uzyskała 
notę~0.66. Jednak poza oceną całościową należy zwrócić uwagę na oceny w~poszczególnych kategoriach. Stopień pokrycia 
wizualizowanych elementów dla Jambalaya'i wynosi tylko 40\%. W~tej kategorii bardzo dobrze wypadł Growl, który wizualizuje większość, bo aż 77\% elementów.
\par Ogólnie niski poziom narzędzi do wizualizacji ontologii jest motywacją na stworzenia nowego rozwiązania, które łączyło by wszystkie najlepsze cechy
 badanych edytorów. Takie narzędzie mogłoby zastąpić obecny moduł wizualizacji ontologii używany w~katedralnym systemie OCS. 
%\begin{table}[t]

\newpage
\begin{longtable}{|m{6cm}|m{2cm}|m{2cm}|m{2cm}|m{2cm}|}
\caption{Porównanie rozwiązań do wizualizacji ontologii}
\label{t:porownanie} \\
\hline
 Kryterium  &  OntoVis  &  Jambalaya &  Growl & OCS\\ \hline

 \multicolumn{5}{|c|}{{\bf Kompletność wizualizacji} (max 30 punktów)}  \\ \hline
Sposób wizualizacji klas	&  2  &  2  &  2  & 2 \\ \hline
Sposób wizualizacji bytów	&  2  &  2  &  2  & 0 \\ \hline
Wizualizacja klas ananimowych	&  0  &  0  &  2  & 0 \\ \hline
Wizualizacja relacji		&  4  &  3  &  10 & 2 \\ \hline
Wizualizacja właściwości	&  2  &  2  &  2  & 0 \\ \hline
Wizualizacja związków odnoszącymi
	   się do właściwościami&  0  &  0  &  4 & 0 \\ \hline
Wizualizacja pozostałych 
	  elementów ontologii   &  0  &  3  &  1 &  0\\ \hline
{\bf Łącznie dla kategorii}	&{\bf 0,33 }   &{\bf 0,4  }   & {\bf 0,77 }  & {\bf 0,13 } \\ \hline %max 30 punktów
 \multicolumn{5}{|c|}{{\bf Przejrzystość wizualizacji} (max 8 punktów)}\\ \hline
Czytelność ontologii		& 3  &  4 &  1 &  2\\ \hline
Dostępność filtrów		& 2  &  3 &  0 &  0\\ \hline
{\bf Łącznie dla kategorii}	&{\bf 0,625 }   &{\bf 0,875 }   & {\bf 0,125 }  & {\bf 0,25 } \\ \hline %max 8 punktów
 \multicolumn{5}{|c|}{{\bf Sposoby wizualizacji} (max 4 punkty)} \\ \hline
Liczba sposobów wizualizacji		&  0 &  3 &  1 & 1 \\ \hline
Wizualizacja wywnioskowanej hierarchii	&  0 &  0 &  0 & 1 \\ \hline
{\bf Łącznie dla kategorii}	&{\bf 0 }   &{\bf 0,75 }   & {\bf 0,25 }  & {\bf 0,5 } \\ \hline %max 4 punkty
 \multicolumn{5}{|c|}{{\bf Użyteczność} (max 11 punktów)}\\ \hline
Użyteczność			&  4 &  5 &  5 & 5 \\ \hline
Dostępność pomocy		&  2 &  2 &  2 & 0\\ \hline
Licencja			&  2 &  1 &  2 & 2\\ \hline
{\bf Łącznie dla kategorii}	&{\bf 0,73 }   &{\bf 0,73 }   & {\bf 0,82 }  & {\bf 0,64 } \\ \hline %max 11 punktów
{\bf Ocena rozwiązania}		&{\bf 0,42 }   &{\bf 0,66 }   & {\bf 0,5}  & {\bf 0,34 } \\ \hline
\end{longtable}

%\end{table}