\chapter{Wymagania stawiane przed projektem}

Wstęp do rozdziału pierwszego. 

Przykładowa treść z podrozdziałami, cytatem i wyliczeniem.

\section{Podrozdział pierwszy}

Licencja Apache License stworzona została przez Apache Software
Foundation w~celu licencjonowania oprogramowania, którym~opiekuje~się fundacja.
Aktualna wersja oznaczona numerem 2.0 zezwala na~dowolne wykorzystanie,
modyfikowanie i~redystrybucję objętego nią kodu lub~jego wersji binarnych. 
Nie~jest wymagane, by~dzieła pochodne korzystały z~tej samej licencji, ani~nawet
by~były wolnym oprogramowaniem. Jedynym warunkiem wykorzystania kodu jest
zamieszczenie informacji o~tym, że~dany program korzysta z~kodu licencjonowanego
z~wykorzystaniem Apache License. Dzięki niewielkiej ilości obostrzeń licencja ta
jest kompatybilna praktycznie ze~wszystkimi licencjami Open Source (poza GPL
w~wersjach wcześniejszych niż 3.0 i~im podobnych)\cite{Apache}.

\section{Podrozdział drugi}

\begin{enumerate}

\item{\bf Swoboda Redystrybucji (Free Redistribution)}

\nopagebreak

Licencja nie~może żadnej ze~stron ograniczać prawa do dystrybucji (odpłatnej 
bądź~nie) oprogramowania, ani przewidywać honorariów czy~innych opłat od~takiej
sprzedaży. Oprogramowanie może~być swobodnie włączane do~większych dystrybucji 
składających~się z~programów z~różnych źródeł. Ma~to głównie na~celu
zapobieżenie naciskom na~programistów wartościowego oprogramowania, by~zmienili
licencję na~komercyjną.

\item {\bf Kod źródłowy (Source code)}

\nopagebreak

Program musi~być rozprowadzany wraz z~kodem źródłowym, a~licencja musi
umożliwiać dystrybucję nie~tylko kodu źródłowego, ale~także postaci
skompilowanej lub~pośredniej(np. wynik działania preprocesora). Jeśli 
oprogramowanie z~różnych przyczyn nie~jest dystrybuowane z~załączonym kodem 
źródłowym, musi~zawierać jasno sformułowany opis sposobu uzyskania tego~kodu 
w~cenie nie~przekraczającej kosztów wykonania kopii. Sugeruje~się udostępnienie 
źródeł w~Internecie i~umożliwienie bezpłatnego ich~pobrania. Dołączany kod źródłowy
musi~mieć postać umożliwiającą prostą modyfikację. Niedopuszczalne jest celowe
zmniejszanie czytelności kodu lub~dołączanie go jedynie w~formie pośredniej.
Te~wymagania w~swym założeniu mają uprościć, a~co za~tym idzie, również
przyspieszyć, modyfikację kodu i~ewolucję oprogramowania.

\item {\bf Dzieła pochodne (Derived works)}

\nopagebreak

Licencja nie~może w~żaden sposób zabraniać wprowadzania zmian i~tworzenia na~ich
podstawie dzieł pochodnych. Musi także umożliwiać rozprowadzanie ich na~tych 
samych warunkach co~produkt bazowy. Daje to możliwość
szybszego rozwoju oprogramowania. Umożliwia eksperymentowanie z~istniejącymi
rozwiązaniami i~ułatwia dzielenie~się usprawnieniami.

\end{enumerate}