\begin{thebibliography}{99}
  \addcontentsline{toc}{chapter}{Bibliografia}
  \bibliographystyle{plainnat}

 
  \bibitem{owl_rek1} Bechhofer S.,  Harmelen F., Hendler J.,  Horrocks I,  McGuinness D. L.,  Patel-Schneider P. F., Stein L. A., \textit{OWL Web Ontology Language Reference W3C Recommendation}, luty 2004, http://www.w3.org/TR/owl-ref/.

  \bibitem{boinski} Boiński T., \textit{Architektura portalu dziedzinowego}, KASKBOOK 2007, Politechnika Gdańska, Gdańsk, Białogóra, 2007.

  \bibitem{nasz} Boiński T., Jaworska A., Kleczkowski R., Kunowski P., \textit{Ontology Visualization},  Politechnika Gdańska, 2nd International Conference on Information Technology, Gdańsk, czerwiec 2010

  \bibitem{JambalayaProtege} Ernst N., Lintern R., Perrin D., Storey M., \textit{Visualization and \protege }, University of Victoria, Kanada, lipiec 2005.

  \bibitem{ebook2} Gasević D., Djurić D., Devedzić V., \textit{Model Driven Engineering and Ontology Development}, Springer, 2009.

  \bibitem{goczyla2} Goczyła K., \textit{Wnioskowanie w ontologiach opartych na logice opisowej}, KASKBOOK 2007, Politechnika Gdańska, Gdańsk, Białogóra, 2007.

  \bibitem{goczyla} Goczyła K., Zawadzka T., \textit{Ontologie w Sieci Semantycznej}, Wydział Elektroniki, Telekomunikacji i Informatyki, Politechnika Gdańska, 2006

  \bibitem{knowledge} Grimm S., Hitzler P., Abecker A., Knowledge Representation and Ontologies, Niemcy, 2007.

  \bibitem{gruber} Gruber T. R., \textit{Toward Principles for the Design of Ontologies Used for Knowledge Sharing}, Stanford Knowledge Systems Laboratory, Stanford University, 1993. 

  \bibitem{prefuse} Heer J, Card S. K., Landay J. A., \textit{prefuse: a toolkit for interactive information visualization}. CHI 2005, 2005.

  \bibitem{owlapi} Horridge M., Bechhofer S., \textit{The OWL API: A Java API for Working with OWL 2 Ontologies}, University of Manchester, sierpień 2009.

  \bibitem{owlapi1} Horridge M., Bechhofer S., Noppens O., \textit{Igniting the OWL 1.1 Touch Paper: The OWL API}, University of Manchester, czerwiec 2007.

  \bibitem{jankowski} Jankowski A., \textit{Praca dyplomowa magisterska - Platforma do edycji ontologii z dostępem przez WWW}, Politechnika Gdańska, Gdańsk, grudzień 2008. 
 
  \bibitem{jedruch} Jędruch A., \textit{Rozwój języków ontologii}, KASKBOOK 2007, Politechnika Gdańska, Gdańsk, Białogóra, 2007.

  \bibitem{vizComp} Katifori A., Halatsis C., Lepouras G., Vassilakis C., Giannopoulou E. G., \textit{ Ontology visualization methods  - a survey} ,  ACM Comput. Surv, 2007.

  \bibitem{growl} Krivov1 S., Villa F., Williams R., Wu X., \textit{On Visualization of OWL Ontologies}, lipiec 2006.

  \bibitem{Jambalaya} Lintern R., Storey M., \textit{Jambalaya express: on demand knowledge visualization}, University of Victoria, Kanada, lipiec 2005.

  \bibitem{RDFprimer} Manola F., Miller E., \textit{RDF Primer W3C Recommendation}, 2004, http://www.w3.org/TR/rdf-primer/.

  \bibitem{prefuse_sdj} Nazimek P., \textit{Prefuse}, Software Developer's Journal 3/2008.

  \bibitem{OntoSphere} \textit{OntoSphere3D}, 2004, http://ontosphere3d.sourceforge.net/ .

  \bibitem{OWL2} \textit{OWL 2 Web Ontology Language Profiles W3C Recommendation}, 27 października 2009, http://www.w3.org/TR/owl2-profiles/.

  \bibitem{OntoViz} Sintek M., \textit{OntoViz} , 2007, http://protegewiki.stanford.edu/wiki/OntoViz.

  \bibitem{pellet} Sirin E., Parsia D., Grau B. C., Kalyanpur A., Katz Y., \textit{Pellet: A Practical OWL-DL Reasoner},  University of Maryland, 2005.

  \bibitem{owl_rek} Smith M. K., Welty C.,  McGuinness D. L., \textit{OWL Web Ontology Language Guide W3C Recommendation}, luty 2004, http://www.w3.org/TR/owl-guide/. 

  \bibitem{biblioteka_standard} Travis G., \textit{Build your own Java library}, maj 2001, http://www.digilife.be/quickreferences/PT/Build your own Java library.pdf.
 


\end{thebibliography}

%   \bibitem{} \textit{}