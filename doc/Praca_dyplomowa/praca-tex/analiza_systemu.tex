
\chapter{Analiza systemu}
\section{Wstęp}
W tej części pracy zostanie przeprowadzona analiza tworzonego systemu. Na wstępie zostaną przedstawione cele systemu, 
a następnie wymagania stawiana nowemu rozwiązaniu. Każdemu z tych elementów, tzn celom i wymaganiom, zostanie nadany jeden z następujących priorytetów: 
bardzo ważny, ważny, średnio ważny, mało ważny. Priorytety zostały wymienione od najwyższego do najniższego.  
\section{Cele systemu}


Głównym celem pracy jest stworzenie biblioteki pozwalającej na wizualizację ontologii zapisanej w~języku OWL.
 Aby przetestować rozwiązanie zostanie ono wdrożone do rozwijanego przez Katedrę Architektury Systemów Komputerowych systemu OCS. Zostanie również 
utworzony plugin do aplikacji \protege, dzięki niemu biblioteka trafi do większego grona użytkowników. Przedstawione poniżej cele oraz wymagania dotyczą 
samej biblioteki. Jednak pod względem funkcjonalności pokrywają się one z~wymaganiami stawianymi przez użytkowników aplikacji wykorzystujących bibliotekę. 
Aplikacje te, aby mogły posłużyć testowaniu biblioteki  muszą implementować wszystkie dostarczane przez nią funkcje.  
\par 
 
Nie istnieją żadne formalne zalecenia dotyczące tworzenia bibliotek JAVA. Są jednak pewne zalecenia co do stosowanych praktyk \cite{biblioteka_standard}:
\begin{enumerate}
\item \textbf{Hermetyzacja kodu.} Publiczne powinny być jedynie te klasy i~metody, które są istotne dla użytkownika i~z~których będzie on~bezpośrednio korzystał.
\item \textbf{Możliwość debugowania.} Użytkownik powinien mieć możliwość debugowania kodu biblioteki, bez konieczności znajomości każdego jej szczegółu.
\item  \textbf{Przejrzystość.} Kod biblioteki powinien być odpowiednio udokumentowany za pomocą javadoc. W~szczególności, bardzo dokładnie należy opisać klasy oraz metody publiczne.
\item \textbf{Łatwość użycia.} Biblioteka powinna zawierać klasy, pokazujące przykłady wykorzystania jej klas i~metod.
\item \textbf{ Rozszerzalność.} Struktura wewnętrzna biblioteki powinna być odpowiednio podzielona na klasy (wykorzystując klasy abstrakcyjne i~interfejsy). Dzięki temu użytkownik będzie miał możliwość stworzenia własnych klas, rozszerzających funkcjonalność biblioteki.
\item \textbf{Uniwersalność.} Biblioteka powinna mieć jasno określony problem, który rozwiązuje. Wyniki powinny być podane użytkownikowi w~wygodny dla niego sposób (lub na kilka sposobów), 
który będzie umożliwiał wykorzystanie biblioteki w~różnych aplikacjach. Innymi słowy, biblioteka powinna udostępniać łatwy i~przejrzysty dla użytkownika interfejs.
% \item Biblioteka powinna być napisana w taki sposób, aby użytkownik spojrzał na nią i mógł powiedzieć: "Wow, to jest dokładnie to, czego potrzebuję i dokładnie tak samo bym to napisał!".

\end{enumerate}
\subsection{Cele biznesowe}
Poniżej zostają opisane cele biznesowe systemu. Pozwolą one na lepsze zrozumienie problemu od strony biznesowej. Ujawniają one korzyści jak będzie posiadała 
organizacja korzystająca z zamówionego oprogramowania. W tym projekcie za odbiorcę i zleceniodawcę projektu uważa się Katedrę Architektury Systemów Komputerowych.   
\begin{center}
\begin{tabular}{|m{3cm}|m{9cm}|} \hline

CB001 & Ułatwienie pracy programistom tworzącym aplikacje wizualizujące ontologie  \\ \hline
Opis: & Istnieje zapotrzebowanie na bibliotekę tłumaczącą OWL bezpośrednio na elementy graficzne. Programiści aplikacji związanych z ontologiami będą mogli 
wykorzystać gotową bibliotekę do wizualizacja. Dzięki temu więcej czasu będą mogli poświęcić innym zagadnieniom tworzonej przez nich aplikacji.   \\ \hline
% Źródło: & Wstępna specyfikacja projektu \\ \hline
Priorytet: & bardzo ważne \\ \hline
\multicolumn{2}{c}{} \\

%priorytety  00 01 10 11
%priorytety bardzo ważne, ważne, średnio ważne, mało ważne

 \hline
CB002 & Ułatwienie zakończenia projektu OCS   \\ \hline
Opis: & Moduł wizualizujący ontologie w OCS wymaga modernizacji i~rozbudowy funkcjonalności. Zapewnienie biblioteki wizualizującej ontologie ułatwi 
i~przyspieszy ten proces.  \\ \hline
% Źródło:& Klient - mgr inż. Tomasz Boiński   \\ \hline
Priorytet: & bardzo ważne \\ \hline
\multicolumn{2}{c}{} \\
 \hline
CB003 & Zwiększenie atrakcyjności portalu OCS   \\ \hline
Opis: & Poprawa estetyki modułu wizualizującego ontologię może przyczynić się do sukcesu portalu po jego wdrożeniu.  \\ \hline
% Źródło: & Klient - mgr inż. Tomasz Boiński \\ \hline
Priorytet: & mało ważne \\ \hline
\multicolumn{2}{c}{} \\
 \hline
CB004 & Ułatwienie pracy twórcom ontologii poprzez dodanie pluginu  do \protege   \\ \hline
Opis: & Z związku z~zmianą architektury dokonaną w~\proteges istnieje zapotrzebowanie na plugin do wizualizacji ontologii. 
Plugin ten pozwoli na zwiększenie efektywności tworzenia ontologii.  \\ \hline
% Źródło: & Klient - mgr inż. Tomasz Boiński \\ \hline
Priorytet: & ważne \\ \hline
\multicolumn{2}{c}{} \\
\end{tabular}



%\begin{tabular}{|m{3cm}|m{9cm}|} \hline
%CB004 &    \\ \hline
%Opis: &   \\ \hline
%Źródło: &  \\ \hline
%Priorytet: & \\ \hline
%\end{tabular}

\end{center}

\subsection{Cele funkcjonalne}

Poniżej przedstawione zostały cele funkcjonalne stawiana systemowi. 

\begin{center}

\begin{longtable}{|m{3cm}|m{9cm}|} \hline

CF001 & Intuicyjne API \\ \hline
Opis: & API powinno być uznane za intuicyjne w~opinii członków zespołu i~klienta. \\ \hline
% Źródło: & Klient - mgr inż. Tomasz Boiński \\ \hline
Priorytet: & średnio ważne  \\ \hline
\multicolumn{2}{c}{} \\

 \hline
CF002 & Dobra dokumentacja \\ \hline
Opis: & Przygotowanie dokumentacji w~Javadoc ułatwi pracę użytkownikom biblioteki. \\ \hline
% Źródło: & Klient - mgr inż. Tomasz Boiński  \\ \hline
Priorytet: & bardzo ważne \\ \hline

\multicolumn{2}{c}{} \\
 \hline
CF003 & Wizualizacja ontologii \\ \hline
Opis: & Stworzenie biblioteki, która pozwoli na wizualizacje obiektów OWL API przy użyciu odpowiedniej biblioteki graficznej. \\ \hline
% Źródło:  & Specyfikacja projektu  \\ \hline
Priorytet: & bardzo ważne \\ \hline

\multicolumn{2}{c}{} \\
 \hline
CF004 & Umożliwienie graficznej edycji i~dodawania obiektów OWL API \\ \hline
Opis: &  Dostarczenie tej funkcjonalności ułatwi tworzenie programów z~interfejsem pozwalającym na edycję ontologii zapisanych w OWL API. \\ \hline
% Źródło: & Klient - mgr inż. Tomasz Boiński \\ \hline
Priorytet: & średnio ważne \\ \hline
\multicolumn{2}{c}{} \\


 \hline

CF005 & Udostępnienie informacji do debugowania  \\ \hline
Opis: &  Biblioteka powinna wysyłać komunikaty informacyjne, ostrzegawcze oraz informujące o~błędach na strumień udostępniony użytkownikowi.  \\ \hline
% Źródło: & Standard tworzenia biblioteki 
%\footnote{Code conventions for the javatm programming language. publikacja elektroniczna, kwiecień 1999.
%\url{http://java.sun.com/docs/codeconv/html/CodeConventions.doc8.html}
%}
%  \\ \hline
Priorytet: & średnio ważne \\ \hline

\end{longtable}

\end{center}

\section{Otoczenie systemu}

%Zespół projektowy musi poznać otoczenie, w jakim ma pracować system. Z rozmów z klientem powinno dać się wyszczególnić użytkowników oraz systemy zewnętrzne. Jeśli się nie da, to otoczenie systemu trzeba będzie zdefiniować w trakcie analizy funkcjonalnej.

\subsection{Użytkownicy}



\begin{tabular}{|m{3cm}|m{9cm}|} \hline

U001 & Programista \\ \hline
Opis: &  Użytkownik, który wykorzysta napisaną bibliotekę w swoim rozwiązaniu. \\ \hline
% Potrzeby: &  \\ \hline
% Zadania: &  \\ \hline
% Źródło: &  \\ \hline
% Priorytet: &  \\ \hline

\end{tabular}

\begin{tabular}{|m{3cm}|m{9cm}|} \hline

U002 & Twórca ontologii w~systemie OCS \\ \hline
Opis: &  Użytkownik systemu OCS wykorzystujący edytor OCS. \\ \hline
% Potrzeby: &  \\ \hline
% Zadania: &  \\ \hline
% Źródło: &  \\ \hline
% Priorytet: &  \\ \hline

\end{tabular}


\begin{tabular}{|m{3cm}|m{9cm}|} \hline

U002 & Twórca ontologii w~programie \protege \\ \hline
Opis: &  Użytkownik \proteges z~zainstalowanym pluginem. \\ \hline
% Potrzeby: &  \\ \hline
% Zadania: &  \\ \hline
% Źródło: &  \\ \hline
% Priorytet: &  \\ \hline

\end{tabular}

% \subsection{Systemy zewnętrzne}

% Specyfika systemu nie wymaga definiowania systemów zewnętrznych.

%\begin{center}
%\begin{tabular}{|m{3cm}|m{9cm}|} \hline

%Tutaj jest ID & A tutaj nazwa \\ \hline
%Opis: &  \\ \hline
%Interfejsy: &  \\ \hline
%Źródło: &  \\ \hline
%Priorytet: &  \\ \hline

%\end{tabular}
%\end{center}

% \section{Przewidywane komponenty systemu}

%Wyszczególnienie komponentów systemu powinno pomóc w uzyskaniu kompletności wymagań. Trzeba wówczas sprawdzić, czy każdy komponent ma jakieś wymagania (zwłaszcza funkcjonalne). W przypadku bardziej złożonego systemu może być konieczne wyszczególnienie podsystemów.

% \subsection{Podsystemy}
% Specyfika projektu sprawia, że podsystemy nie będą rozpatrywane.

% \subsection{Komponenty sprzętowe}
% Specyfika projektu sprawia, że komponenty sprzętowe nie będą rozpatrywane.

\subsection{Komponenty programowe}

\begin{center}
\begin{longtable}{|m{3cm}|m{9cm}|} \hline

KS001 & Prefuse [\cite{prefuse} \cite{prefuse_sdj}]\\ \hline
Opis: &  
 Elastyczny pakiet dostarczający programiście narzędzia do przechowywania danych, manipulowania nimi oraz ich interaktywnej wizualizacji.
 Biblioteka jest rozwijana w~całości w~języku Java. Może być wykorzystana do budowania niezależnych aplikacji, wizualnych komponentów rozbudowanych 
aplikacji oraz tworzenia apletów.
\\ \hline
% Powiązania: &  \\ \hline
% Źródło: & Specyfikacja projektu \\ \hline
% Priorytet: & bardzo ważne \\ \hline

\multicolumn{2}{c}{} \\
 \hline

KS002 & OWL API \\ \hline
Opis: &  Biblioteka do przetwarzania ontologii zapisanych w~języku OWL. Napisana w~języku Java.\\ \hline
% Powiązania: &  \\ \hline
% Źródło: & Specyfikacja projektu \\ \hline
% Priorytet: & bardzo ważne \\ \hline


\end{longtable}
\end{center}

\section{Wymagania funkcjonalne}

%Wymagania funkcjonalne stanowią mocno rozbudowaną część specyfikacji. Można je podzielić na grupy dotyczące różnych zadań, różnych użytkowników (systemów zewnętrznych) albo różnych komponentów.

\begin{center}

\begin{longtable}{|m{3cm}|m{9cm}|} \hline

WF001 & Udostępnienie kilku algorytmów wizualizacji \\ \hline
Opis: & Biblioteka powinna udostępniać kilka trybów prezentacji grafów (np. w~formie drzewa, w~formie gwiazdy i~innych).    \\ \hline
Dotyczy: & CF003 \\ \hline
% Źródło: & klient - mgr Tomasz Boiński \\ \hline
Powiązania: &WF003 \\ \hline
Priorytet: & średnio ważny\\ \hline

\multicolumn{2}{c}{} \\
 \hline

WF002 & Wizualizacja wywnioskowanej hierarchii \\ \hline
Opis: & Biblioteka powinna dać możliwość wizualizacji wywnioskowanej hierarchii klas i~bytów. \\ \hline
Dotyczy: & CF003 \\ \hline
% Źródło: & klient - mgr Tomasz Boiński \\ \hline
Powiązania: &WF003 \\ \hline
Priorytet: & średnio ważny\\ \hline

\multicolumn{2}{c}{} \\
 \hline

WF003 & Parametryzacja trybów wizualizacyjnych \\ \hline
Opis: & Domyślne parametry w trybach wizualizacji (takie jak długość krawędzi grafu, automatyczne układanie) powinny zostać dobrane w~taki sposób,
 by obraz był przejrzysty, stabilny i~czytelny.    \\ \hline
Dotyczy: & CF003 \\ \hline
% Źródło: &  klient - mgr Tomasz Boiński\\ \hline
Powiązania: & WF001 \\ \hline
Priorytet: & średnio ważny \\ \hline

\multicolumn{2}{c}{} \\
 \hline

WF004 & Udostępnienie strumienia błędów \\ \hline
Opis: &   Biblioteka będzie udostępniać strumień danych, w którym znajdą się komunikaty o błędach. Strumień ten będzie mógł zostać wykorzystany
 przez użytkownika. \\ \hline
Dotyczy: &  CF005  \\ \hline
% Źródło: & klient - mgr inż. Tomasz Boiński \\ \hline
Powiązania: & \\ \hline
Priorytet: & ważne \\ \hline
\end{longtable}

\end{center}

\subsection{Wymagania wizualizacji ontologii}

\begin{center}
\begin{longtable}{|m{3cm}|m{9cm}|} \hline

WF005 & Rozróżnialność podstawowych symboli  \\ \hline
Opis: &  Class, Individual, Property powinny mieć rozróżnialne symbole   \\ \hline
Dotyczy: &  CF003 \\ \hline
% Źródło: &  klient - mgr inż. Tomasz Boiński\\ \hline
Powiązania: & \\ \hline
Priorytet: & bardzo ważne \\ \hline
%Projekt: & \includegraphics{myimage.png}

\multicolumn{2}{c}{} \\
 \hline

WF006 &   Rozróżnialność szczególnych typów Class\\ \hline
Opis: &   Klasa anonimowa, datatype, Thing i Nothing powinny być łatwo rozpoznawalne.  \\ \hline
Dotyczy: &  CF003 \\ \hline
% Źródło: &  klient - mgr inż. Tomasz Boiński\\ \hline
Powiązania: & WF005 \\ \hline
Priorytet: &  ważne \\ \hline

\multicolumn{2}{c}{} \\
 \hline

WF007 &  Rozróżnialność związków między klasami (Class), instancjami (Individual) oraz predykatami (Property)\\ \hline
Opis: & Różne symbole dla equivalentClass, disjointWith, subClassOf, sameAs, differentFrom, allDifferent, oneOf, unionOf, intersectionOf,
 complementOf, subProperty, equivalentProperty, hasProperty.   \\ \hline
Dotyczy: &  CF003\\ \hline
% Źródło: &  klient - mgr inż. Tomasz Boiński \\ \hline
Powiązania: & WF006, WF005 \\ \hline
Priorytet: & ważne \\ \hline

\multicolumn{2}{c}{} \\
 \hline

WF008 & Rozróżnialność ograniczeń predykatów (Restrictions) \\ \hline
Opis: & Wyróżnić kardynalność (cardinality), domeny (domains) predykatów, inverseOf, właściwości predykatów (transitive, symmetric, 
functional, inverseFunctional). \\ \hline
Dotyczy: &  CF003\\ \hline
% Źródło: &  klient - mgr inż. Tomasz Boiński \\ \hline
Powiązania: & WF005\\ \hline
Priorytet: & ważne \\ \hline

\multicolumn{2}{c}{} \\
 \hline

WF009 &  Podświetlanie wybranych związków i powiazań.\\ \hline
Opis: &   Podświetlać subklasy danej klasy po ich wybraniu myszką. \\ \hline
Dotyczy: &  CF003\\ \hline
% Źródło: &  klient - mgr inż. Tomasz Boiński \\ \hline
Powiązania: & WF007\\ \hline
Priorytet: & mało ważne \\ \hline


\multicolumn{2}{c}{} \\
 \hline

WF010 &  Wyszukiwanie elementów.\\ \hline
Opis: &   Powinna istnieć możliwość wyszukiwania elementów ontologii na wyświetlanym grafie. Spełniające kryterium wyszukiwania elementy powinny 
zostać wyróżnione.  \\ \hline
Dotyczy: &  CF003\\ \hline
% Źródło: &  klient - mgr inż. Tomasz Boiński \\ \hline
Powiązania: & \\ \hline
Priorytet: & mało ważne \\ \hline

\multicolumn{2}{c}{} \\
 \hline

WF011 &  Zmiana kolorów wizualizowanych elementów.\\ \hline
Opis: &  Każdy z obrazowanych elementów powinien mieć możliwość zmiany jego koloru, tak aby użytkownik mógł dostosować wizualizację do swoich potrzeb.  \\ \hline
Dotyczy: &  CF003\\ \hline
% Źródło: &  klient - mgr inż. Tomasz Boiński \\ \hline
Powiązania: &\\ \hline
Priorytet: & mało ważne \\ \hline


\end{longtable}

\end{center}






\subsection{Wymagania na dane}

%Wymagania na dane pomagają w określeniu, jakie dane będą przetwarzane w systemie. Nie trzeba precyzować wszystkich danych. Szczegóły znajdą się w projekcie bazy danych.

\begin{center}

\begin{tabular}{|m{3cm}|m{9cm}|} \hline

WD001 & Obsługa obiektów OWL API \\ \hline
Opis: & Biblioteka będzie przystosowana do pobierania i obróbki obiekty OWL API. \\ \hline
Powiązania: &  \\ \hline
% Źródło: & Klient - mgr inż. Tomasz Boiński  \\ \hline
Priorytet: &  bardzo ważne \\ \hline

\end{tabular}

\end{center}

\subsection{Wymagania jakościowe}

%Określenie wymagań jakościowych ułatwia późniejsze uzyskanie wysokiej jakości systemu. Podział wymagań jakościowych na kategorie jest związany z drzewem jakości (dotyczy wszystkich gałęzi drzewa za wyjątkiem funkcjonalności).

\subsubsection{Wymagania w zakresie wiarygodności}

%Wymagania w zakresie wiarygodności będą rozszerzały wymagania funkcjonalne.


\begin{center}

\begin{tabular}{|m{3cm}|m{9cm}|} \hline

WJ001 & Poprawność wizualizacji \\ \hline
Opis: & Wszystkie wizualizowane elementy powinny pochodzić z ontologii otrzymanej na wejściu programu. Program nie powinien dodawać własnych elementów (np. wywnioskowanych). Wyjątkowo dla klas, które nie mają zdefioniowany nadklas zostanie utworzony związek z klasą Thing. \\ \hline
Powiązania: & WJ002 \\ \hline
% Źródło: &  klient - mgr inż. Tomasz Boiński \\ \hline
Priorytet: & bardzo ważne \\ \hline

\multicolumn{2}{c}{} \\
 \hline

WJ002 & Kompletność wizualizacji \\ \hline
Opis: & Jeżeli biblioteka nie wizualizuje danej funkcji OWL API informacja o tym powinna znaleźć się w strumieniu błędów. \\ \hline
Powiązania: & CF005, WJ001, WD001 \\ \hline
% Źródło: & klient - mgr inż. Tomasz Boiński \\ \hline
Priorytet: & ważne \\ \hline

\end{tabular}

\end{center}

% \subsubsection{Wymagania w zakresie wydajności}

%Wymagania w zakresie wydajności będą miały zastosowanie w czasie projektowania architektury systemu.

% Brak wymogów wydajnościowych ze względu na specyfikę projektu.

%\begin{tabular}{|m{3cm}|m{9cm}|} \hline

%Tutaj jest ID & A tutaj nazwa \\ \hline
%Opis: &  \\ \hline
%Powiązania: &  \\ \hline
%Źródło: klient - mgr inż. Tomasz Boiński &  \\ \hline
%Priorytet: &  \\ \hline

%\end{tabular}


\subsubsection{Wymagania w zakresie elastyczności}

%Wymagania w zakresie elastyczności będą miały zastosowanie w czasie wyboru koncepcji systemu.

\begin{center}

\begin{tabular}{|m{3cm}|m{9cm}|} \hline

WJ003 & Obsługiwane wersje Javy \\ \hline
Opis: & Biblioteka powinna wspierać wersje Javy 1.5 i nowsze.\\ \hline
Powiązania: &  \\ \hline
% Źródło: & klient - mgr inż. Tomasz Boiński \\ \hline
Priorytet: & bardzo ważne \\ \hline

\multicolumn{2}{c}{} \\
 \hline

WJ004 & Obsługiwane wersje OWL API \\ \hline
Opis: & Powinna istnieć możliwość podpięcia zewnętrznego OWL API (wybranego przez użytkownika/programistę).\\ \hline
Powiązania: &  \\ \hline
% Źródło: & klient - mgr inż. Tomasz Boiński \\ \hline
Priorytet: & bardzo ważne \\ \hline

\end{tabular}

\end{center}






%\begin{tabular}{|m{3cm}|m{9cm}|} \hline

%Tutaj jest ID & A tutaj nazwa \\ \hline
%Opis: &  \\ \hline
%Powiązania: &  \\ \hline
%Źródło: &  \\ \hline
%Priorytet: &  \\ \hline

%\end{tabular}


\subsection{Dodatkowe wymagania}

%W tym miejscu podaje się te wymagania, które nie mieszczą się w zakresie poprzednich kategorii wymagań.

% \subsubsection{Wymagania sprzętowe}

%Wymagania sprzętowe można by umieścić w ramach specyfikacji komponentów sprzętowych, ale jeśli jest wiele komponentów sprzętowych różnych z punktu widzenia funkcjonalnego, ale o wspólnych wymaganiach sprzętowych, to można te wymagania umieścić właśnie tutaj.

% Ze względu na specyfikę projektu wymagania sprzętowe nie będą rozpatrywane.

%\begin{tabular}{|m{3cm}|m{9cm}|} \hline

%Tutaj jest ID & A tutaj nazwa \\ \hline
%Opis: &  \\ \hline
%Dotyczy: &  \\ \hline
%Źródło: &  \\ \hline
%Priorytet: &  \\ \hline

%\end{tabular}


\subsubsection{Wymagania programowe}

%Trzeba odróżniać rzeczywiste wymagania programowe klienta od jego sugestii (np. przez podanie opcjonalnego priorytetu).

\begin{center}

\begin{tabular}{|m{3cm}|m{9cm}|} \hline

WD003 & JVM \\ \hline
Opis: & Do skorzystania z biblioteki niezbędna jest JVM.\\ \hline
Dotyczy: & CF001, CF002 \\ \hline
% Źródło: & klient - mgr inż. Tomasz Boiński \\ \hline
Priorytet: & ważne \\ \hline

\end{tabular}

\end{center}

\subsubsection{Inne wymagania}

\begin{center}

\begin{tabular}{|m{3cm}|m{9cm}|} \hline

WI001 & Dokumentacja w javadoc \\ \hline
Opis: & Wszystkie ważne klasy i funkcje powinny mieć odpowiednią dokumentację w formacie javadoc.\\ \hline
Dotyczy: & CF001, CF002 \\ \hline
% Źródło: & klient - mgr inż. Tomasz Boiński \\ \hline
Priorytet: & ważne \\ \hline


\multicolumn{2}{c}{} \\
 \hline

WI003 & Dokumentacja w języku polskim \\ \hline
Opis: & Dokumentacja wszystkich funkcji i klas powinna posiadać polską wersję językową. \\ \hline
Dotyczy: & CF001, CF002 \\ \hline
% Źródło: & klient - mgr inż. Tomasz Boiński \\ \hline
Priorytet: & ważne \\ \hline

\multicolumn{2}{c}{} \\
 \hline
WI004 & Nazwy zmiennych i funkcji w języku angielskim \\ \hline
Opis: & Nazwy zmiennych i funkcji powinny zostać dobrane w języku angielskim i zgodnie ze standardami programowania w javie %\footnote{Greg    Travis.          Build    your   own     java  library.          publikacja elektroniczna.
%\url{http://www.digilife.be/quickreferences/PT/Build your own Java library.pdf}
%}.
\\ \hline
Dotyczy: & CF001, CF002 \\ \hline
% Źródło: & klient - mgr inż. Tomasz Boiński \\ \hline
Priorytet: & ważne \\ \hline

\end{tabular}

\end{center}

\subsection{Kryteria akceptacyjne}

%Tu podać kryteria, jakim zostanie poddany gotowy system przed ostatecznym jego przyjęciem.

\begin{center}

\begin{tabular}{|m{3cm}|m{9cm}|} \hline

KA001 & Spełnione są podstawowe wymagania wymienione w dokumencie SWS \\ \hline
Opis: & Spełnione są wszystkie wymagania ważne i bardzo ważne zdefiniowane w SWS. \\ \hline
Dotyczy: & wszystkie wymagania ważne i bardzo ważne \\ \hline
% Źródło: & klient - mgr inż. Tomasz Boiński \\ \hline
Priorytet: & ważne  \\ \hline %skonsultować z klientem

\end{tabular}

\end{center}

