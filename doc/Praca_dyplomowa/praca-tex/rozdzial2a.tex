
\chapter{Przegląd dostępnych rozwiązań do wizualizacji ontologii}
\section{Wstęp}

\section{Sposoby wizualizacji ontologii}
Budowanie ontologii nie jest łatwą pracą. Nawet w~niewielkich rozwiązaniach wymaga często analizy dużej liczny elementów. 
Aby poprawnie tworzyć ontologie wymagana jest ich wizualizacja. Upraszcza ona postrzeganie całego rozwiązania oraz konkretnych jego części.
 Umożliwia również łatwą ocenę i poprawę błędów. 
\newline
Istnieje kilka podejść do graficznej prezentacji ontologii. Większość 
z nich opiera się na modelu 2D. Istnieją również rozwiązania 3D np.: OntoSphere \cite{VisOnto} \cite{OntoSphere}. W tym przypadku klasy zostały odzwierciedlone
 jako sfery (kule), a ich podklasy znajdują się, w trójwymiarowym drzewie przypominającym stożek, pod nimi. Klasy są połączone trójwymiarową krawędzią 
o~odpowiednim kolorze i kształcie grotu. 
\section{Elementy wizualizacji}
Najczęściej ontologie są wizualizowane są za~pomocą grafu, w~którym wierzchołki 
oznaczają elementy ontologii, a~krawędzie odzwierciedlają związki pomiędzy tymi elementami. Jednak podczas wizualizacji ontologii o~dużej liczbie 
elementów taki sposób obrazowania może być nieprzejrzysty i~nieczytelny. Dlatego należy zwrócić uwagę na następujące elementy:
\begin{enumerate}
\item{\bf Sposób całościowej wizualizacji}

\nopagebreak

Dobra aplikacja do wizualizacji ontologii powinna wyświetlić wszystkie jej elementy. Użytkownik powinien mieć możliwość podglądu całej edytowanej
ontologii, jak i również wybranej jej części. Aby wizualizacja była przejrzysta i czytelna musi posiadać możliwość ukrycia wizualizacji niektórych elementów. 
Np.: wyłączenie rysowania związków danego typu.
\item{\bf Sposób wizualizacji klas i~bytów}

\nopagebreak

Klasy oraz ich instancji są najważniejszymi elementami ontologii, dlatego poprawne ich wizualizowanie jest konieczne. Zły sposób obrazowania tych elementów 
może zniechęcić użytkownika do korzystania z rozwiązania. Wizualizacja powinna pokazywać wszystkie klasy lub tylko wybrane przez użytkownika. 
Każda klasa powinna zawierać przynajmniej nazwę, zapisaną w~zrozumiały sposób. Klasy mogą być obrazowane jako wierzchołki grafu. Powinny one jednoznacznie
 wyróżniać się od instancji klas. Np.: poprzez kolor lub kształt wierzchołka. To podejście może się jednak nie sprawdzić przy wizualizacji ontologi o~znacznej
liczbie elementów. W tym przypadku może istnieć wiele połączonych klas z wybranym węzłem, obrazowanie takiej sytuacji na grafie może okazać się nieczytelne. 
Alternatywnym rozwiązaniem jest wyświetlenie wszystkich powiązanych klas z zadanym elementem w oddzielnym oknie aplikacji.   

\item{\bf Wizualizacja taksonomii}

\nopagebreak

Klasy tworzą hierarchię (taksonomię) klas, poprzez relację nadklasa - podklasa (związek isa). Prezentacja taksonomii mam kluczowe znaczenia dla
 zrozumienia relacji dziedziczenia pomiędzy klasami. Wizualizacja powinna dać możliwość całościowego bądź częściowego przeglądu hierarchii dziedziczenia.

\item{\bf Sposób wizualizacji relacji}

\nopagebreak

Relacje występujące pomiędzy elementami najczęściej obrazowane są jako związek łączący te elementy. Różne typy relacji można wyróżnić poprzez nadanie 
etykiety krawędzi łączącej, poprzez zmianę koloru lub kształtu linii. Ważne jest, aby można było opcjonalnie wyłączyć wizualizację zadanych typów
 relacji. 

\item{\bf Wizualizacja właściwości}

\nopagebreak

Ontologie byłyby bardzo ubogie, gdyby nie posiadały właściwości. Właściwości pozwalają na zdefiniowanie ogólnych informacji dotyczących zarówno
 klas jak i~instancji klas. Właściwości powinny być zaznaczone na grafie wizualizacji bądź w oddzielnym oknie do tego przeznaczonym. 

\item{\bf Wyszukiwanie}

\nopagebreak

Podczas wizualizacji dużej ontologi możemy napotkać na problem szybkiego wyszukania interesującego nas elementu. Dlatego aplikacji wizualizacji powinna 
dać możliwość szukania elementów w grafie ontologii.    

\end{enumerate}


\section{Przedstawienie wybranych rozwiązań do wizualizacji ontologii}
Na rynku dostępnych jest wiele programów do tworzenia i~edycji ontologii. Większość z~nich ułatwia budowanie ontologii poprzez ich graficzną reprezentację.
 Niektóre rozwiązanie, takie jak rozwijany na~Uniwersytecie Stanforda edytor \protege , poprzez pluginy dostarczają kilku sposobów wizualizacji. Inne poza
ogólnym mechanizmem wizualizacji posiadają dodatkowy do prezentacji drzewa wywnioskowanej hierarchii klas i bytów.

\begin{enumerate}

 \item{\bf OntoViz}

  \nopagebreak 
 
 OntoViz \cite{OntoViz} jest najczęściej używany pluginem \proteges do wizualizacji ontologii. Wykorzystuje on bibliotekę GraphViz do 
tworzenia prostych grafów 2D (\figurename \space \ref{fig:viz:ontoviz} ). Na grafie klasy są reprezentowane jako prostokąty zawierające informację o
nazwie klasy jak i dodatkowe informacje o relacjach i właściwościach klasy. Istnieje możliwość okrojenia i ukrycia części wyświetlanych komponentów przez 
panel konfiguracyjny po prawej stronie. Instancje klas zostały wyróżnione od klas innym kolorem. 


\insertimage{images/wizualizacja/ontoviz.png}{Przykład wizualizacji ontologii za pomocą OntoViz}{fig:viz:ontoviz}

 \item{\bf Jambalaya}

  \nopagebreak 
 Jambalaya \cite{Jambalaya} \cite{JambalayaProtege} jest pluginem do \proteges rozwijanym na Uniwersytecie Wiktorii (Kanada). Opiera się na graficznym zestawie
 narzędzi Piccolo do tworzenia interaktywnych grafów 2D. Jambalaya charakteryzuje się kilkoma rodzajami widoku. 

 \item{\bf Growl}

  \nopagebreak 

 \item{\bf OCS}

  \nopagebreak 

  OCS (ang. ONTOLOGY CREATION SYSTEM) jest systemem do tworzenia i edycji ontologii rozwijanym na Wydziale Elektroniki, Telekomunikacji i Informatyki Politechniki
 Gdańskiej. Edytor ontologi posiada 2 sposoby wizualizacji ontologii. Pierwszym jest hierarchia klas i bytów, drugi zaś to ogólny obraz ontologii. System wizualizuje
 tylko podstawowe elementy ontologii, przedstawia tylko klasy, instancje i podstawowe relacji pomiędzy nimi. 
 
\end{enumerate}


\section{Opis metody porównania}
\subsection*{Sposób wyznaczenia oceny rozwiązań}
Porównanie rozwiązań do wizualizacji ontologi będzie podzielone na kilka kategorii. Każda z kategorii będzie posiadała zestaw zagadnień lub pytań. 
Odpowiedzi na pytania pozwolą wyznaczyć ocenę dla zadanej kategorii. Każde z zagadnień, poza treścią, posiada również opis badanej cechy, liczbę 
punktów, które mogą być przyznane za pytanie oraz sposób przydzielania punktów. Ocena dla danej kategorii będzie liczbą zmiennoprzecinkową 
z~przedziału $ <0,1> $  i~będzie ona wyznaczana na podstawie wzoru:
\begin{equation}
   {P_{K}} =\frac{\sum_{k=1}^N  {p_{u}(i)}}{\sum_{k=1}^N  {P_{max}(i)}} 
\end{equation}
gdzie: \\
$ N $ -- liczba pytań w kategorii \newline
$ P_{K} $ -- liczba punktów uzyskanych w~danej kategorii, wartość ta będzie z~przedziału $ <0,1> $,\newline
$ p_{u}(i) $ -- liczna punktów uzyskanych w i-tym pytaniu \\
$ P_{max}(i) $ -- maksymalna liczba punktów możliwa do uzyskania na i-te pytanie. \\


Całkowita ocena będzie również liczną rzeczywistą z przedziału $ <1,0> $, gdzie $1$ jest najlepszą notą do zdobycia. Ocena rozwiązania będzie średnią 
ważoną ocen zdobytych dla poszczególnych kategorii. Wagi kategorii zostaną podane poniżej, wraz z~ich opisem. 

\subsection*{Opis kategorii}

\begin{enumerate}
 \item{\bf Kompletność wizualizacji} 

  \nopagebreak  
Wiele rozwiązań obrazujących ontologie wyświetla tylko podstawowe informacje o~klasach i~związkach pomiędzy nimi. Aby ułatwić pracę nad ontologiami 
wizualizacja musi być kompletna, wszystkie jej elementy muszą być przedstawione na grafie lub w dodatkowych oknach aplikacji.  Kryterium to pozwoli
ocenić kompletności wizualizacji. 
  \nopagebreak 

{\bf Waga} : 5

 \item{\bf Przejrzystość wizualizacji} 

  \nopagebreak                  
Kolejnym kryterium jest przejrzystość wizualizacji. Aby wizualizacja była zrozumiała musi być czytelna, szczególnie gdy wczytamy dużą ontologię. 
Kryterium to m.in. odpowie na pytanie czy aplikacja posiada opcję filtrowania danych, wyświetlania tylko zadanych elementów i~związków.  
  \nopagebreak 

{\bf Waga} : 5

 \item{\bf Sposoby wizualizacji} 

  \nopagebreak                  
  Ontologia może być przedstawiona na drzewie bądź grafie, jej elementy mogą być rozmieszczone w~różnych odległościach. Kryterium to sprawdzi,
czy poza podstawową wizualizacją rozwiązania posiadają różne algorytmy i sposoby obrazowania danych. Ocenione zostanie również, czy wizualizacja 
opiera się o~dane jawnie pobrane z ontologii, czy może jest wywnioskowaną hierarchią klas i~bytów.   
  \nopagebreak 

{\bf Waga} : 5


 \item{\bf Użyteczność} 

  \nopagebreak                  
  Kryterium to pozwoli ocenić czy dane rozwiązanie jest ``przyjazne'' dla użytkownika. 
  \nopagebreak 

{\bf Waga} : 5

\end{enumerate}




\subsection*{Kryteria oceny dla kategorii kompletność}

Poniżej znajduje się lista kryteriów, które pozwolą na ocenę kompletności wizualizacji. Kryteria zostały wyróżnione i~pogrupowane na podstawie
elementów składowych ontologii. 

\begin{longtable}{|m{3cm}|m{10cm}|}\hline
Kryterium:                   &  \bf{Sposób wizualizacji klas}\\ \hline
Opis kryterium:              & Klasy są najważniejszymi elementami ontologii. I wizualizacja powinna być powinna zawierać przynajmniej nazwę klasy. 
                               Klasy poprzez kolor lub kształt powinny wyróżniać się na tle innych elementów ontologii.\\ \hline
Liczna punktów do uzyskania: & 2 \\ \hline
Sposób oceny:                &  1 punkt za wizualizację i jednoznaczne oznaczenie klas, \newline
                                1 punkt za wyróżnienie klas od innych elementów ontologii.\\ \hline

\end{longtable}



\begin{longtable}{|m{3cm}|m{10cm}|}\hline
Kryterium:                   & \bf{Sposób wizualizacji bytów}\\ \hline
Opis kryterium:              & Byty, będące instancjami klas powinny odróżniać od klas, np.~poprzez kolor lub kształt wierzchołka. 
                               Każdy z nich powinien identyfikować się nazwą.  \\ \hline
Liczna punktów do uzyskania: & 2\\ \hline
Sposób oceny:                &  1 punkt za wizualizację i odpowiednie oznaczenie bytów,  \newline
                                1 punkt za wyróżnienie bytów od innych elementów np. klas.\\ \hline

\end{longtable}




\begin{longtable}{|m{3cm}|m{10cm}|}\hline
Kryterium:                   & \bf{ Sposób wizualizacji klas anonimowych }\\ \hline
Opis kryterium:              &  Klasy anonimowe są rezultatem relacji innych klas. Klasą anonimową może być np.~suma dwóch klas. 
				Elementy te nie posiadają nazwy, dlatego wizualizacja powinna pokazać z~jakiej relacji i~z~jakich klas one powstały.     \\ \hline
Liczna punktów do uzyskania: &  2  \\ \hline
Sposób oceny:                &  1 punkt za wizualizację klas anonimowych,  \newline 
                                1 punkt za przejrzyste ukazanie relacji i
				klas, których owa klasa jest rezultatem
                                 \\ \hline
                               
\end{longtable}



\begin{longtable}{|m{3cm}|m{10cm}|}\hline
Kryterium:                   & \bf{ Wizualizacja relacji dotyczących klasami i bytami }\\ \hline
Opis kryterium:              &  Kryterium pozwoli ocenić stopień pokrycia związków pomiędzy klasami i bytami  \\ \hline
Liczna punktów do uzyskania: &  11   \\ \hline
Sposób oceny:                &  1  punkt za wizualizację związku ``subclass'' ,\newline
                                1  punkt za wizualizację związku ``instanceOf'', \newline	      
                                1  punkt za wizualizację związku ``equivalentClass'',\newline
                                1  punkt za wizualizację związku ``disjointWith'',\newline
                                1  punkt za wizualizację związku ``differentFrom'',\newline
                                1  punkt za wizualizację związku ``allDifferent'',\newline
                                1  punkt za wizualizację związku ``sameAs'',\newline
                                1  punkt za wizualizację związku ``oneOf'',\newline
                                1  punkt za wizualizację związku ``unionOf'',\newline
                                1  punkt za wizualizację związku ``intersectionOf'',\newline
                                1  punkt za wizualizację związku ``complementOf''

  \\ \hline
\end{longtable}

\begin{longtable}{|m{3cm}|m{10cm}|}\hline
Kryterium:                   & \bf{ Wizualizacja właściwości }\\ \hline
Opis kryterium:              &  Właściwości pozwalają na zdefiniowanie ogólnych informacji dotyczących zarówno klas jak i instancji klas. \\ \hline
Liczna punktów do uzyskania: &  2   \\ \hline
Sposób oceny:                &     1  punkt za wizualizację DataTypeProperty ,\newline
                                   1  punkt za wizualizację ObjectProperty, \
			      \\ \hline
\end{longtable}

\begin{longtable}{|m{3cm}|m{10cm}|}\hline
Kryterium:                   & \bf{ Wizualizacja związków odnoszącymi się do właściwościami }\\ \hline
Opis kryterium:              &    Kryterium pozwoli ocenić stopień pokrycia związków dotyczących właściwości  \\ \hline
Liczna punktów do uzyskania: &  7   \\ \hline
Sposób oceny:                &  1  punkt za wizualizację związku ``subProperty'' ,\newline
                                1  punkt za wizualizację związku ``equivalentProperty'', \newline	      
                                1  punkt za wizualizację związku ``functionalProperty'',\newline
                                1  punkt za wizualizację związku ``inversFunctionalProperty'',\newline
                                1  punkt za wizualizację związku ``symmetricProperty'',\newline
                                1  punkt za wizualizację związku ``transitiveProperty'',\newline
                                1  punkt za wizualizację związku ``inverseOf(property)'',
 \\ \hline
\end{longtable}



\begin{longtable}{|m{3cm}|m{10cm}|}\hline
Kryterium:                   & \bf{ Wizualizacja pozostałych elementów ontologii }\\ \hline
Opis kryterium:              &  Kryterium pozwoli ocenić stopień pokrycia pozostałych elementów ontologii, takich jak np. kardynalność.  \\ \hline
Liczna punktów do uzyskania: &   4  \\ \hline
Sposób oceny:                &  1  punkt za wizualizację  ``sameValuesFrom/allValuesForm'' ,\newline
                                1  punkt za wizualizację   ``Cardinality'', \newline	      
                                1 punkt za wizualizację  ``Domain'',\newline
                                1 punkt za wizualizację  ``Range'',\newline 
  \\ \hline
\end{longtable}


\subsection*{Kryteria oceny dla kategorii przejrzystość wizualizacji}
Poniżej znajduje się lista kryteriów i pytań, które pozwolą na ocenę czytelności i zrozumiałości wizualizacji. 


\begin{longtable}{|m{3cm}|m{10cm}|}\hline
Kryterium:                   & \bf{ Czytelność ontologii }\\ \hline
Opis kryterium:              &  Kryterium oceni jakość wizualizacji dużej liczącej ponad 100 klas ontologii. Wizualizacja powinna dać możliwość 
                                wyświetlenia całej ontologii.  Oceniony zostanie sposób rozłożenia 
                                 danych. Należy sprawdzić, czy elementy nie nachodzą na siebie i czy nie zakrywają innych elementów ontologii. 
                                \\ \hline
Liczna punktów do uzyskania: &  5   \\ \hline
Sposób oceny:                &  0-3 punktów za ogólną czytelność dużej ontologii.\newline
                                0-2 punktów za nienachodzenie na siebie elementów. \\ \hline
\end{longtable}

\begin{longtable}{|m{3cm}|m{10cm}|}\hline
Kryterium:                   & \bf{ Dostępność filtrów }\\ \hline
Opis kryterium:              &  Wprowadzenie możliwości wizualizacji tylko wybranych typów elementów ontologii znacznie ułatwia pracę z tą ontologią.
                                Dlatego wizualizacja powinna dać możliwość filtrowania danych. Kryterium to sprawdzi jakie filtry posiadają wybrane 
                                rozwiązania i w jakim stopniu filtry pokrywają ilość wyświetlanych elementów.\\ \hline
Liczna punktów do uzyskania: &  3   \\ \hline
Sposób oceny:                &  0-3 punkty za ilość filtrów.    \\ \hline
\end{longtable}

\begin{longtable}{|m{3cm}|m{10cm}|}\hline
Kryterium:                   & \bf{ Skalowalność rozwiązania }\\ \hline
Opis kryterium:              & Kryterium skalowalności rozwiązań pozwoli ocenić jak zmienia się jakość wizualizacji wraz ze wzrostem.  \\ \hline
Liczna punktów do uzyskania: &     \\ \hline
Sposób oceny:                &     \\ \hline
\end{longtable}


\subsection*{Kryteria oceny dla kategorii sposoby wizualizacji}
Poniżej znajduje się lista kryteriów które pozwolą ocenić różnorodność sposobów wizualizacji zaproponowanych w testowanych rozwiązaniach. 


\begin{longtable}{|m{3cm}|m{10cm}|}\hline
Kryterium:                   & \bf{ Liczba sposobów wizualizacji }\\ \hline
Opis kryterium:              &  Kryterium sprawdzi ile różnych sposobów rozmieszczenia danych i wizualizacji znajduje się w danym rozwiązaniu. 
                               \\ \hline
Liczna punktów do uzyskania: &  3   \\ \hline
Sposób oceny:                &  0  punktów jeśli rozwiązanie posiada tylko jeden sposób wizualizacji \newline	
                                1  punkt jeśli wizualizacja posiada 2 różne sposoby obrazowania danych \newline
                                2  punkty dla 2-3 różnych sposobów wizualizacji \newline
                                3 punkty jeśli rozwiązanie pozwala na wizualizację za pomocą więcej niż 3 różnych algorytmów obrazowania danych\\ \hline
\end{longtable}

\begin{longtable}{|m{3cm}|m{10cm}|}\hline
Kryterium:                   & \bf{ Wizualizacja wywnioskowanej hierarchii  }\\ \hline
Opis kryterium:              &  Wywnioskowana hierarchia bytów i klas pozwala na łatwiejsze zrozumienie ontologii. Może być też przydatna przy kontroli 
                                jakości ontologii, np. przy sprawdzaniu spójności ontologii.     \\ \hline
Liczna punktów do uzyskania: &  1   \\ \hline
Sposób oceny:                &  1 punkt jeśli rozwiązanie umożliwia wizualizację hierarchii wygenerowanej przez narzędzie wnioskujące   \\ \hline
\end{longtable}


\subsection*{Kryteria oceny dla kategorii użyteczność}
Poniżej znajduje się lista kryteriów które pozwolą ocenić użyteczność rozwiązań. 

\begin{longtable}{|m{3cm}|m{10cm}|}\hline
Kryterium:                   & \bf{ Użyteczność  }\\ \hline
Opis kryterium:              &   Oceniona zostanie intuicyjność  łatwość  korzystania i rozwiązania. \\ \hline
Liczna punktów do uzyskania: &   6  \\ \hline
Sposób oceny:                &  0-2 punktów za łatwość instalacji \newline
                                0-3 punktów za intuicyjność interfejsu \newline
                                1 punkt za możliwość wyszukiwania  \\ \hline
\end{longtable}


\begin{longtable}{|m{3cm}|m{10cm}|}\hline
Kryterium:                   & \bf{ Dostępność pomocy }\\ \hline
Opis kryterium:              &  Pozwala określić stopień dostępności pomocy dla rozwiązania   \\ \hline
Liczna punktów do uzyskania: &  3   \\ \hline
Sposób oceny:                &  1 punkt za stronę z pomocą \newline
                                1 punkt za samouczek lub instrukcja użytkowania  \newline
                                1 punkt za dostępność forów  związanych z rozwiązaniem . \\ \hline
\end{longtable}

\begin{longtable}{|m{3cm}|m{10cm}|}\hline
Kryterium:                   & \bf{ Licencja }\\ \hline
Opis kryterium:              &   Zostanie tutaj sprawdzona licencja, na której zostało wydane oprogramowanie.    \\ \hline
Liczna punktów do uzyskania: &   2  \\ \hline
Sposób oceny:                &   1 punkt jeśli oprogramowanie jest na darmowej licencji\newline
                                 1 punkt jeśli posiada otwarty kod. \\ \hline
\end{longtable}



Kryteria porównawcze
\begin{enumerate}
 \item prezentacja całościowa
  \begin{itemize}
   \item prezentacja całości
  \item możliwość filtrów
  \item filtr na odległość
  \item możliwość wyszukiwania
  \item różne typy wizualizacji
  
  
  \end{itemize}
 \item dodatkowe metody wizualizacji
   \begin{itemize}
    \item dodatkowe sposoby wizualizacji
    \item wizualizacja wywnioskowanej hierarchii
  \end{itemize}

\item wizualizacja elementów
  \begin{itemize}
    \item Klasy nazwane
    \item Klasy anonimowe
    \item (Byty)Instancje
    \item Relacje
    \item Właściwości (ObjectProperty, DataTypeProperty)

  \end{itemize}
\item skalowalność
\item licencja



\end{enumerate}


