\documentclass{article}
\usepackage{polski}
\usepackage[utf8]{inputenc} % by użyć polskich znaków w systemach Linux używamy kodowania "latin2", dla Windows "cp1250" 
\usepackage{floatflt,graphics,textcomp}
\usepackage{hyperref}

\title{Zastosowanie aplikacji WIKI w przetwarzaniu zespołowym}
\author{Anna Jaworska, Piotr Orłowski}

\date{...}

\begin{document}
\maketitle
\tableofcontents
\newpage

\section{Wstęp}

% słowa wstępne...  Niniejszy referat ma na celu opisanie samego mechanizmu WIKI, 
% zarówno przybliżenie czytelnikowi samego mechanizmu WIKI, który często mylnie kojarzony jest wyłącznie z Wikipedią
% jak również, na co wskazuje sam tytuł, przedstawić możliwości zastosowania go w pracy zespołowej.
% W rozdziale tym a tym przedstawimy to a to...
% do zrobienia: napisać to
% do nauczenia się: inteligentne odnośniki do numerów sekcji


	\subsection {WIKI na tle WEB 2.0}
% WIKI jest przedstawicielem nurtu zwanego WEB 2.0
% - Co to jest WEB 2.0
% - dlaczego WIKI jest jego przedstawicielem?
% - WEB 3.0: i co dalej?

	%we are moving from passive readers to active participants.

%       The reality is, Wikipedia is quite different from organizational wiki sites
%both because of its primary use — encyclopedia — and the way its community
%is structured. I often say it’s the most extreme instance of wiki in existence
%because everyone can see its entire contents, anyone can contribute, and people
%can do so anonymously.
%                                                                     there are
%two types of wiki communities — the all-virtual versus the wiki that mirrors a
%physical community

%Back-office to Front-office str 95


	\subsection{Rys historyczny}
	%   The first wiki, WikiWikiweb (http://c2.com/cgi/wiki?WelcomeVisitors),
%was created in 1995 by Ward Cunningham to document and collaboratively
%update information on software design patterns.
%wiki(z hawajskiego) szybki 

% - skąd Cunningham wziął ten pomysł?
% - historyjka o autobusie...
% - jak to ewoluowało i jak zmieniało się podejście innych ludzi
% 

	\subsection{WIKI jako CMS}
%              Because Wikipedia resides on the open Web, people assume
%that if they used a wiki for internal collaboration anyone could change the
%information on any page, even if their edits result in inaccurate or completely
%erroneous information. The reality is quite different when a wiki is used inside
%an organization, a

%                                          most organizations don’t really know
%what they know, and are poor at transmitting new ideas and new plans in a
%way that’s understandable. Organizations are mostly organized around their
%current goals. Some organizations have a part that tries to improve the process
%for attaining current goals. But very few organizations improve the process of
%figuring out what the goals should be
%   As I read this, I realized that it’s a brilliant argument for why the wiki can be
%a vital tool for organizations. Because it doesn’t define the terms of interaction
%and collaboration from the outset, and allows structure to be created, modified
%and removed as needed, the wiki quickly becomes a desirable tool because
%it ‘‘learns’’ how people work as they work, not after the fact. This means it
%captures more of the actual process, giving them an opportunity to regularly
%look at how they collaborate, even during a current project.

%   In the wiki of a private organization, this kind of vandalism or erroneous
%editing is extremely unlikely to happen because the wiki is being used within
%an already established social and organizational structure. The fact is people
%just don’t abuse tools that are important to their professional work.
			
%Wiki versus Intranet Powered
%by Content Management System  str 98
	
% przetłumaczyt=ć to co powyżej - bo w sumie WIKI zamienia się w takie gie bo pisać w tym strony da się...

\newpage
\section{Stosowane języki programowania}
Pierwsza WIKI, WikiWikiWeb, powstała na autorskim silniku napisanym w języku Pearl. Wkrótce po zaakceptowaniu przez programistów formy tego portalu zaczęły stopniowo pojawiać się jego klony pisane w różnych językach. Te popularniejsze wkrótce wewoluowały w potężne silniki z własną społecznością, które do dzisiaj stopniowo się rozwijają. 
	\subsection{PHP} 
%wymienić wiki z tym jezykiem, skrytykować że wersja jezyka jest stara
	\subsection{JEE}
%napisać coś
	\subsection{Perl}
	\subsection{Inne}
% wspomnieć, że wiki można napisać nawet w Javascripcie....

\newpage
\section{Technologie pomocnicze}
	\subsection{Subversion}
	% o tym, że większość WIKI z niego korzysta do przechowywania danych...  
	\subsection{LDAP}
	%SSO , poziomy dostępu, 
	\subsection{RSS}
	%o tym, że WIKI stosują takie technologie jak RSS i Atom...
\newpage
\section{Bazy danych}
	\subsection{Wiki bez bazy danych}
	% DokuWiki nie ma nawet Subversion... dlaczego co i jak jest w tym dobre
	\subsection{Bazy Opensource}
	\subsection{Bazy komercyjne}

\newpage
\section{Edycja danych}
	\subsection{WYSIW(M)G}
		%TinyMCE
	Odkąd wiki stały się popularne wśród użytkowników nieobeznanych z metajęzykami, 
w większości pojawiły się edytory WYSWIG. Edycja stała się bardziej przyjazna użytkownikowi
%   As wikis have grown more popular and entered the mainstream, wiki
%vendors have added WYSIWYG (what you see is what you get) editors to
%make the editing experience look and feel more like the Microsoft Word-style
%interface many people are used to. The primary benefit of a WYSIWYG editor
%is that it can ease the transition when people first start using a wiki, but here’s
%the rub: if it’s the only editing tool people use, the WYSIWYG editor offers a
%much more limited editing experience and prevents people from making the
%leap necessary to fully understand the power of a wiki.

	\subsection{WYMIWYG}
What You Mean Is What You Get

	\subsection{Edycja w programach zewnętrznych}


	\subsection{Nazywanie artykułow}
%różne podejścia...	
%CamelCase 

\newpage
\section{Sposoby organizacji informacji}	
	\subsection{Przestrzenie nazw}
	\subsection{Tag clouds}
%                                                                      Tagging
%has emerged as one of the most effective ways to organize information because
%the organization is the result of a collaborative process where people create
%tags as needed to describe content, and instead of putting content in folders
%which is restrictive because a file can only reside in one folder, multiple tags
%can be assigned to a document by multiple people, which ultimately results in
%a better categorization of its content.


	
	%\subsection{Hierarchiczna organizacja)

\newpage
\section{Zastosowania w pracy zespołowej}
Mechanizm WIKI, odpowiednio użyty, może znacząco ułatwić pracę realizowaną zespołowo. W niniejszym rozdziale zostaną przedstawione podstawowe zastosowania oraz problemy, które WIKI pozwala rozwiązywać dużo łatwiej. 
	\subsection{Budowa portali}  
	\subsection{Gromadzanie wiedzy}
		%Błędy w oprogramowanie - testowanie
		%Dokumentacja
		%Wiedza wewnętrzna organizacji
%                Unlike KM systems, wikis focus completely on letting people
%work together online the same way they’d work in person, and approach
%knowledge as the product of that organic, nonlinear human connection and
%collaboration. 

	\subsection{Integracja z innymi narzędziami}
	\subsection{Mniej emaili}
%                             By reducing the number of general emails, people
%will be more likely to pay attention to the few that really do affect them, and
%by putting more specific communication on the wiki, people can identify and
%keep up with what’s most relevant to them.


%    eep in mind that I’ve just given the simplest example with only two
%people. If it can be this difficult for only two people to ‘‘collaborate’’ over
%email, imagine how these problems can balloon exponentially when more
%people are involved! A new wiki user recently told me a story of how it took
%three days to reconcile the edits that had gotten out of control when twelve
%people tried to use email to collaborate on a press release. The edits themselves
%didn’t take that long — within a few hours after the file had been emailed
%to everyone, twelve distinct files were returned with twelve different sets
%of edits!

	\subsection{Pisanie publikacji}
	%ksiązki, artykuły
	%wydawaca/redaktor ma staly wglad w postepy prac

	\subsection{Wiki jako współdzielony dysk}
%str 99 beware of svn


	\subsection{Żywiolowa wymiana wiedzy i pomysłów}
%tego pragnie Pan Samolot najbardziej

%                                                               That’s what makes
%a wiki so special; it enables the natural patterns of interaction that previously
%could only happen in a physical meeting — fast-paced discussion, overlapping
%ideas, quick decisions on changes, quick error correction, introducing different
%viewpoints and collaboratively working to reach agreement — but it removes
 %                                                     Back-office to Front-office  57
%the need for everyone to be in the same place at the same time, and it documents
%the interaction better than a traditional meeting could.

	\subsection{Szybsza edycja}
%tego też pragnie Pan Samolot :)

%   By contrast, if we were using a wiki, you might see the wiki page at 2:55 P.M.
%waiting for your input, and decide to give it a quick read. If you get a great
%idea, want to make some minor changes, or even just notice a typo, all you
%have to do is click ‘‘Edit’’, make some quick changes, click ‘‘Save’’, and head
%off to your meeting. Now that the document has your input, collaboration
%can keep moving forward because I can revise based on your edits, send it to
%others for further input, and so on.

\newpage
\section{Zanim wdrożysz}
Kiedy już zdecydujemy, że WIKI jest dokładnie tym mechanizmem, który chcemy zastosować np. w naszej firmie warto zastanowić się jeszcze nad kilkoma związanymi z tym oprogramowaniem problemami.
	\subsection{Problem licencjonowania}
	Zarówno przy wyborze WIKI komercyjnej jak i tej darmowej np. z otwartym kodem warto się zastanowić nad licencją, na której otrzymujemy to oprogramowanie.
	\subsubsection{Licencje komercyjne}
	% - zastanowić się, które WIKI mają restrykcyjne ograniczenia, np. licencję IBM albo coś w ten deseń, o ograniczeniach na ilość użytkowników nie wspominając...
	\subsubsection{Licencje OpenSource}
	% - wspomnieć o zagrożeniach związanych z licnejcją GPL
	% - wspomnieć o innych licencjach, np. tej M$ itp...
	\subsection{Sposoby instalacji}
		Proste
		\begin{itemize}
			\item "Cegła na Enterze"
			\item Menadżer pakietów
			\item make install
		\end{itemize}
		Konfiguracja po instalacji
		\begin{itemize}
			\item Ustawienia lokalne
			\item Podłączenie bazy danych
			\item Inne
		\end{itemize}
	\subsection{Bezpieczeństwo}
	%kombinacje technologii z przkładami
		SSL
		\begin{itemize}
			\item Twoja Wiki może korzystać z HTTPS
		\end{itemize}
		Bezpieczeństwo wewnątrz Wiki
		\begin{itemize}
			\item Poziomy dostępu
			\item Role użytkowników
		\end{itemize}
	\subsection{Internacjonalizacja}
	%kombinacje technologii z przkładami
		Na co warto zwrócić uwagę:
		\begin{itemize}
			\item Czy dana Wiki obsługuje utf8?
			\item Czy posiada wszystkie wymagane wersje językowe?
			\item Czy baza danych i SVN danej Wiki też obsługuje utf8?
		\end{itemize}
			
	\subsection{Wsparcie techniczne}
		Dla użytkowników indywidualnych
		\begin{itemize}
			\item Zwróć uwagę na istnienie forum, grupy dyskusyjnej, możliwość raportowania błędów 
			\item Nie wydawaj pieniędzy na nadmiarowe wsparcie komercyjne
		\end{itemize}
		
		Dla małych grup komercyjnych
		\begin{itemize}
			\item Mała firma lub dział/grupa robocza w dużej firmie
			\item Rozważ wybór Wiki w technologii, którą potrafisz sam wspierać
			\item Rozważ Wiki, którą możesz powiększyć, gdy znajdą się chętni z innych działów
			\item Nie kupuj wsparcia na zapas, zbadaj czy na pewno jest Ci potrzebne
		\end{itemize}
		Dla dużych organizacji i firm
		\begin{itemize}
			\item Wybierz dojrzałą Wiki w technologii, z którą firma ma największe doświadczenie 
			\item Wybierz Wiki wymagającą minimalnej pracy przy wdrażaniu (LDAP, baza danych, której serwer już masz... ) 
			\item Korzystaj ze wsparcia, w końcu za nie płacisz
		\end{itemize}
\newpage
\section{Wdrażanie}
%tego się nie dotykam - Piotrek
	\subsection{Problem zmiany polityki organizacji}
	\subsection{Zmiana sposobu obiegu dokumentów}
	%agneda -> meeting-> projects
	
	\subsection{Problem własności informacji}
	%Closed systems implicitly assume that people can’t be trusted and technology
%has to be relied on to control access to information. This encourages an
%organizational culture where people don’t trust each other and are concerned
%with maintaining control over information access. Some people do this because
%of the perception that with control comes power, and they don’t want to give
%up that power. Others operate on the perception that they should keep
%information close to the vest so they’ll be perceived as experts and others will
%need to come to them for information, thus safeguarding their position. The
%closed approach creates inefficiency, information redundancy, and reduced
%focus on common goals because it’s hard for people to know what others are
%doing so information ends up in restricted silos, and it is difficult to change
%this behavior once it becomes ingrained.


	\subsection{Change summary - Podsumowanie zmian i  jego przydatność w organizacji}
	\subsection{Istotnośc pluginów}
	%wazne jest aby organizacja mogła sama rozszerzać soft wiki poprzez pluginy
	\subsection{Historie wdrożeń} %jak w firmie/uczelni

\newpage
\section{Podsumowanie}

\newpage
\section{Bibliografia}
	\begin{itemize}
		\item 	Wikipatterns, Stewart Mader, Wiley Publishing, Inc., 2008
 		\item \url{http://www.oreillynet.com/pub/a/oreilly/tim/news/2005/09/30/what-is-web-20.html}
		\item \url{http://en.wikipedia.org/wiki/History_of_wikis}
		\item \url{http://www.wikimatrix.org}
		\item \url{http://en.wikipedia.org/wiki/Comparison_of_wiki_software}
	\end{itemize}
\end{document}