\documentclass[a4paper,10pt]{article}
\usepackage{polski}
\usepackage[utf8]{inputenc}
\usepackage{array}
%custom margins
\usepackage[left=2.5cm, top=2.5cm, bottom=3cm, right=2cm, foot=2cm, head=0.5cm]{geometry}
\usepackage{fancyhdr}
\usepackage[bookmarks=true, pdftex]{hyperref}

%styl nagłowków
\pagestyle{fancy} 
\parindent 2cm 

%opening
\title{Specyfikacja wymagań systemowych}
\author{3@KASK}

\begin{document}
\bibliographystyle{plain}


\maketitle


\begin{center}
%budowanie tabeli
\begin{tabular}{|p{7cm}|p{7cm}|}
\hline
Symbol projektu: & Opiekun projektu:   \tabularnewline 
3@KASK & Tomasz Boiński    \tabularnewline \hline
\multicolumn{2}{|l|}{Nazwa Projektu: } \tabularnewline
\multicolumn{2}{|l|}{Wizualizacja grafów za pomocą biblioteki Prefuse } \tabularnewline 
\hline
\multicolumn{2}{l}{ } \tabularnewline %pusta linijka
\hline 
Nazwa Dokumentu: & Nr wersji:   \tabularnewline 
Specyfikacja wymagań systemowych & 0.0 \tabularnewline \hline
Odpowiedzialny za dokument: & Data pierwszego sporządzenia:   \tabularnewline 
Piotr Orłowski & 15 kwietnia 2009 \tabularnewline \hline
Przeznaczenie: & Data ostatniej aktualizacji:   \tabularnewline 
??? & \today \tabularnewline \hline
\end{tabular}
\end{center}


\begin{center}
\begin{tabular}{|c|p{4cm}|c|c|c|}
\multicolumn{5}{c}{\textbf{Historia dokumentu}} \tabularnewline \hline
\textbf{Wersja} & \textbf{Opis modyfikacji} & \textbf{Rozdział/strona} & \textbf{Autor modyfikacji} & \textbf{Data} \tabularnewline \hline 
1 & Stworzenie & wszystkie & Grupa projektowa & 15.04.09 \tabularnewline \hline
2 & Wpisanie celów &  & &\tabularnewline \hline
\end{tabular}
 

\end{center}


\newpage
\tableofcontents
\newpage

\section{Cele systemu}

%Chociaż cele projektu zostały sformułowane w zleceniu projektowym, to w tym miejscu precyzuje się je z podziałem na cele biznesowe i cele funkcjonalne. Zdarza się tak, że zleceniodawca początkowo wymienia tylko cele funkcjonalne zachowując do swojej wiadomości cele biznesowe. Aby jednak właściwie rozumieć te funkcje w odpowiednim kontekście, zespół projektowy musi poznać motywacje biznesowe klienta.

\subsection{Cele biznesowe}

Cele biznesowe precyzują korzyści związane z wdrożeniem systemu.

\begin{center}
\begin{tabular}{|l|l|} \hline

CB001 & Ułatwienie pracy programistom tworzącym aplikacje wizualizujące ontologie  \\ \hline
Opis: & Istnieje zapotrzebowanie na bibliotekę tłumaczącą OWL bezpośrednio na elementy graficzne.  \\ \hline
Źródło: & Wstępna specyfikacja projektu \\ \hline
Priorytet: & bardzo ważne \\ \hline
\end{tabular}
\end{center}
%priorytety  00 01 10 11
%priorytety bardzo wazne, wazne, średnio ważne, mało wazne
\begin{center}
\begin{tabular}{|l|l|} \hline
CB002 & Ułatwienie zakończenia projektu OCS   \\ \hline
Opis: & Moduł wizualizujący ontolgie w OCS wymaga modernizacji i rozbudowy funkcjonalności. Zapewnienie biblioteki wizualizującej ontologie ułatwi i przyspieszy ten proces.  \\ \hline
Źródło: Klient - mgr inż. Tomasz Boiński &  \\ \hline
Priorytet: & bardzo ważne \\ \hline
\end{tabular}
\end{center}

\begin{center}
\begin{tabular}{|l|l|} \hline
CB003 & Zwiększenie aktrakcyjności portalu OCS   \\ \hline
Opis: & Poprawa estetyki modułu wizualizującego ontologię moze przyczynic się do sukcesu portalu po jego wdrożeniu.  \\ \hline
Źródło: & Klient - mgr inż. Tomasz Boiński \\ \hline
Priorytet: & mało ważne \\ \hline
\end{tabular}
\end{center}

\begin{center}
\begin{tabular}{|l|l|} \hline
CB004 &    \\ \hline
Opis: &   \\ \hline
Źródło: &  \\ \hline
Priorytet: & \\ \hline
\end{tabular}
\end{center}


\subsection{Cele funkcjonalne}

Cele funkcjonalne wymieniają główne funkcje, które ma spełniać system.

\begin{center}
\begin{tabular}{|l|l|} \hline

CF001 & Intuicyjne API \\ \hline
Opis: &  \\ \hline
Źródło: &  \\ \hline
Priorytet: & średnio ważne  \\ \hline

\end{tabular}
\end{center}

\begin{center}
\begin{tabular}{|l|l|} \hline

CF002 & Dobra dokumentacja \\ \hline
Opis: & Przygotowanie dokumentacji w Javadoc ułatwi pracę użytkownikom biblioteki. \\ \hline
Źródło: & Klient - mgr inż. Tomasz Boiński  \\ \hline
Priorytet: & bardzo ważne \\ \hline

\end{tabular}
\end{center}

\begin{center}
\begin{tabular}{|l|l|} \hline

CF003 & Wizualizacja ontologii \\ \hline
Opis: & Stworzenie biblioteki, która pozwoli na wizualizacje obiektów OWL API przy użyciu odpowiedniej biblioteki graficznej. \\ \hline
Źródło:  & Specyfikacja projektu  \\ \hline
Priorytet: & bardzo ważne \\ \hline

\end{tabular}
\end{center}

\begin{center}
\begin{tabular}{|l|l|} \hline

CF004 & Umożliwienie graficznej edycji i dodawania obiektów OWL API \\ \hline
Opis: &  Dostarczenie tej funkcjonalności ułatwi tworzenie programów z interfejsem pozwalającym na edycję ontologii zapisanych w OWL API. \\ \hline
Źródło: & Klient - mgr inż. Tomasz Boiński i Anna Jaworska \\ \hline
Priorytet: & średnio ważne \\ \hline

\end{tabular}
\end{center}


\begin{center}
\begin{tabular}{|l|l|} \hline

CF005 & Udostępnienie informacji do debuggowania  \\ \hline
Opis: &  Biblioteka powinna wysyłać komunikaty informacyjne, ostrzegawcze oraz informujace o błędach na strumień udostępniony użytkownikowi.  \\ \hline
Źródło: & Standard tworzenia biblioteki \cite{artOstandardach} \\ \hline
Priorytet: & średnio ważne \\ \hline

\end{tabular}
\end{center}


\section{Otoczenie systemu}

Zespół projektowy musi poznać otoczenie, w jakim ma pracować system. Z rozmów z klientem powinno dać się wyszczególnić użytkowników oraz systemy zewnętrzne. Jeśli się nie da, to otoczenie systemu trzeba będzie zdefiniować w trakcie analizy funkcjonalnej.

\subsection{Użytkownicy}

\begin{center}
\begin{tabular}{|l|l|} \hline

Tutaj jest ID & A tutaj nazwa \\ \hline
Opis: &  \\ \hline
Potrzeby: &  \\ \hline
Zadania: &  \\ \hline
Źródło: &  \\ \hline
Priorytet: &  \\ \hline

\end{tabular}
\end{center}

\subsection{Systemy zewnętrzne}

\begin{center}
\begin{tabular}{|l|l|} \hline

Tutaj jest ID & A tutaj nazwa \\ \hline
Opis: &  \\ \hline
Interfejsy: &  \\ \hline
Źródło: &  \\ \hline
Priorytet: &  \\ \hline

\end{tabular}
\end{center}

\section{Przewidywane komponenty systemu}

Wyszczególnienie komponentów systemu powinno pomóc w uzyskaniu kompletności wymagań. Trzeba wówczas sprawdzić, czy każdy komponent ma jakieś wymagania (zwłaszcza funkcjonalne). W przypadku bardziej złożonego systemu może być konieczne wyszczególnienie podsystemów.

\subsection{Podsystemy}
Specyfika projektu sprawia, że podsystemy nie będa rozpatrywane.

\subsection{Komponenty sprzętowe}
Specyfika projektu sprawia, że komponenty sprzętowe nie będa rozpatrywane. 

\subsection{Programowe}

\begin{center}
\begin{tabular}{|l|l|} \hline

Tutaj jest ID & A tutaj nazwa \\ \hline
Opis: &  \\ \hline
Powiązania: &  \\ \hline
Źródło: &  \\ \hline
Priorytet: &  \\ \hline

\end{tabular}
\end{center}

\section{Wymagania funkcjonalne}

Wymagania funkcjonalne stanowią mocno rozbudowaną część specyfikacji. Można je podzielić na grupy dotyczące różnych zadań, różnych użytkowników (systemów zewnętrznych) albo różnych komponentów.

\begin{center}
\begin{tabular}{|l|l|} \hline

WF001 & Udostępnienie kilku algorytmów wizualizacji \\ \hline
Opis: & Biblioteka powinna udostępniać kilka trybów prezentacji grafów (np. w formie drzewa, w formie gwiazdy i innych).    \\ \hline
Dotyczy: &  \\ \hline
Źródło: &  \\ \hline
Powiązania: & \\ \hline
Priorytet: &  \\ \hline

\end{tabular}
\end{center}
\begin{center}
\begin{tabular}{|l|l|} \hline

WF002 & Parametryzacja trybów wizualizacyjnych \\ \hline
Opis: & Domyślne parametry w trybach wizualizacji (takie jak długość krawędzi grafu, automatyczne układanie) powinny zostać dobrane w taki sposób, by obraz był przejrzysty, stabilny i czytelny.    \\ \hline
Dotyczy: &  \\ \hline
Źródło: &  \\ \hline
Powiązania: & \\ \hline
Priorytet: &  \\ \hline

\end{tabular}
\end{center}




\begin{tabular}{|l|l|} \hline

WF003 &   \\ \hline
Opis: &     \\ \hline
Dotyczy: &  \\ \hline
Źródło: &  \\ \hline
Powiązania: & \\ \hline
Priorytet: &  \\ \hline

\end{tabular}

\begin{tabular}{|l|l|} \hline

WF002 &  \\ \hline
Opis: &     \\ \hline
Dotyczy: &  \\ \hline
Źródło: &  \\ \hline
Powiązania: & \\ \hline
Priorytet: &  \\ \hline

\end{tabular}

\begin{tabular}{|l|l|} \hline

WF002 &  \\ \hline
Opis: &     \\ \hline
Dotyczy: &  \\ \hline
Źródło: &  \\ \hline
Powiązania: & \\ \hline
Priorytet: &  \\ \hline

\end{tabular}


\section{Wymagania na dane}

Wymagania na dane pomagają w określeniu, jakie dane będą przetwarzane w systemie. Nie trzeba precyzować wszystkich danych. Szczegóły znajdą się w projekcie bazy danych.

\begin{center}
\begin{tabular}{|l|l|} \hline

WD001 & Obsługa obiektów OWL API \\ \hline
Opis: & Bibliotek będzie przystosowana do pobierania, obróki i zwracania obiektów OWL API \\ \hline
Powiązania: &  \\ \hline
Źródło: & Klient - mgr inż. Tomasz Boiński  \\ \hline
Priorytet: &  bardzo ważne \\ \hline

\end{tabular}
\end{center}

\section{Wymagania jakościowe}

Określenie wymagań jakościowych ułatwia późniejsze uzyskanie wysokiej jakości systemu. Podział wymagań jakościowych na kategorie jest związany z drzewem jakości (dotyczy wszystkich gałęzi drzewa za wyjątkiem funkcjonalności).

\subsection{Wymagania w zakresie wiarygodności}

Wymagania w zakresie wiarygodności będą rozszerzały wymagania funkcjonalne. 

\begin{center}
\begin{tabular}{|l|l|} \hline

Tutaj jest ID & A tutaj nazwa \\ \hline
Opis: &  \\ \hline
Powiązania: &  \\ \hline
Źródło: &  \\ \hline
Priorytet: &  \\ \hline

\end{tabular}
\end{center}

\subsection{Wymagania w zakresie wydajności}

Wymagania w zakresie wydajności będą miały zastosowanie w czasie projektowania architektury systemu.

\begin{center}
\begin{tabular}{|l|l|} \hline

Tutaj jest ID & A tutaj nazwa \\ \hline
Opis: &  \\ \hline
Powiązania: &  \\ \hline
Źródło: &  \\ \hline
Priorytet: &  \\ \hline

\end{tabular}
\end{center}

\subsection{Wymagania w zakresie elastyczności}

Wymagania w zakresie elastyczności będą miały zastosowanie w czasie wyboru koncepcji systemu.

\begin{center}
\begin{tabular}{|l|l|} \hline

Tutaj jest ID & A tutaj nazwa \\ \hline
Opis: &  \\ \hline
Powiązania: &  \\ \hline
Źródło: &  \\ \hline
Priorytet: &  \\ \hline

\end{tabular}
\end{center}

\subsection{Wymagania w zakresie użyteczności}

Wymagania w zakresie użyteczności będą brane pod uwagę głównie w czasie projektowania interfejsu użytkownika.

\begin{center}
\begin{tabular}{|l|l|} \hline

Tutaj jest ID & A tutaj nazwa \\ \hline
Opis: &  \\ \hline
Powiązania: &  \\ \hline
Źródło: &  \\ \hline
Priorytet: &  \\ \hline

\end{tabular}
\end{center}

\section{Sytuacje wyjątkowe}

Sytuacje wyjątkowe stanowią dalsze rozszerzenie wymagań funkcjonalnych i wiarygodnościowych.

\begin{center}
\begin{tabular}{|l|l|} \hline

Tutaj jest ID & A tutaj nazwa \\ \hline
Opis: &  \\ \hline
Powiązania: &  \\ \hline
Źródło: &  \\ \hline
Priorytet: &  \\ \hline

\end{tabular}
\end{center}

\section{Dodatkowe wymagania}

W tym miejscu podaje się te wymagania, które nie mieszczą się w zakresie poprzednich kategorii wymagań.

\subsection{Wymagania sprzętowe}

Wymagania sprzętowe można by umieścić w ramach specyfikacji komponentów sprzętowych, ale jeśli jest wiele komponentów sprzętowych różnych z punktu widzenia funkcjonalnego, ale o wspólnych wymaganiach sprzętowych, to można te wymagania umieścić właśnie tutaj.

\begin{center}
\begin{tabular}{|l|l|} \hline

Tutaj jest ID & A tutaj nazwa \\ \hline
Opis: &  \\ \hline
Dotyczy: &  \\ \hline
Źródło: &  \\ \hline
Priorytet: &  \\ \hline

\end{tabular}
\end{center}

\subsection{Wymagania programowe}

Trzeba odróżniać rzeczywiste wymagania programowe klienta od jego sugestii (np. przez podanie opcjonalnego priorytetu).

\begin{center}
\begin{tabular}{|l|l|} \hline

Tutaj jest ID & A tutaj nazwa \\ \hline
Opis: &  \\ \hline
Dotyczy: &  \\ \hline
Źródło: &  \\ \hline
Priorytet: &  \\ \hline

\end{tabular}
\end{center}

\subsection{Inne wymagania}
 
\begin{center}
\begin{tabular}{|l|l|} \hline

Tutaj jest ID & A tutaj nazwa \\ \hline
Opis: &  \\ \hline
Dotyczy: &  \\ \hline
Źródło: &  \\ \hline
Priorytet: &  \\ \hline

\end{tabular}
\end{center} 
 
\section{Kryteria akceptacyjne}

Tu podać kryteria, jakim zostanie poddany gotowy system przed ostatecznym jego przyjęciem.

\begin{center}
\begin{tabular}{|l|l|} \hline

Tutaj jest ID & A tutaj nazwa \\ \hline
Opis: &  \\ \hline
Dotyczy: &  \\ \hline
Źródło: &  \\ \hline
Priorytet: &  \\ \hline

\end{tabular}
\end{center}


\clearpage
\phantomsection
\addcontentsline{toc}{section}{Literatura}
\bibliography{biblio}

\end{document}
