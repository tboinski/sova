\documentclass[a4paper,10pt]{article}
\usepackage{polski}
\usepackage[utf8]{inputenc}
\usepackage{array}
%custom margins
\usepackage[left=2.5cm, top=2.5cm, bottom=3cm, right=2cm, foot=2cm, head=0.5cm]{geometry}
\usepackage{fancyhdr}
\usepackage[bookmarks=true, pdftex]{hyperref}

%styl nagłowków
\pagestyle{fancy} 
\parindent 2cm 

%opening
\title{Infrastruktura projektu}
\author{3@KASK}

\begin{document}
\bibliographystyle{plain}


\maketitle


\begin{center}
%budowanie tabeli
\begin{tabular}{|p{7cm}|p{7cm}|}
\hline
Symbol projektu: & Opiekun projektu:   \tabularnewline 
3@KASK & mgr inż. Tomasz Boiński    \tabularnewline \hline
\multicolumn{2}{|l|}{Nazwa Projektu: } \tabularnewline
\multicolumn{2}{|l|}{Wizualizacja grafów za pomocą biblioteki Prefuse } \tabularnewline 
\hline
\multicolumn{2}{l}{ } \tabularnewline %pusta linijka
\hline 
Nazwa Dokumentu: & Nr wersji:   \tabularnewline 
Infrastruktura projektu & 1.0 \tabularnewline \hline
Odpowiedzialny za dokument: & Data pierwszego sporządzenia:   \tabularnewline 
Anna Jaworska  & 31.03.09 \tabularnewline \hline
Przeznaczenie: & Data ostatniej aktualizacji:   \tabularnewline 
WEWNĘTRZNE & 20.04.09 \tabularnewline \hline
\end{tabular}
\end{center}

\begin{center}
\begin{tabular}{|c|p{4cm}|c|c|c|}
\multicolumn{5}{c}{\textbf{Historia dokumentu}} \tabularnewline \hline
\textbf{Wersja} & \textbf{Opis modyfikacji} & \textbf{Rozdział/strona} & \textbf{Autor modyfikacji} & \textbf{Data} \tabularnewline \hline 
0.0 & Stworzenie & wszystkie & Anna Jaworska & 31.03.09 \tabularnewline \hline
1.0 & wWpisanie używanych narzędzi & wszystkie & Anna Jaworska & 20.04.09 \tabularnewline \hline
& & & &\tabularnewline \hline
\end{tabular}
 

\end{center}


\newpage
\tableofcontents
\newpage

\section{Organizacja zespołu projektu}
\begin{center}
\begin{tabular}{|l|l|} \hline
	Nazwa roli & Osoba(y) \tabularnewline \hline
	Kierownik projektu & Piotr Kunowski \tabularnewline \hline
	Specjalista ds. testów & Radosław Kleczkowski \tabularnewline \hline
	Analityk ds. ontologii & Piotr Orłowski \tabularnewline \hline
	Analityk ds. Prefuse & Piotr Kunowski \tabularnewline \hline
	Analityk główny & Anna Jaworska \tabularnewline \hline
	Programiści & cały zespół \tabularnewline \hline
\end{tabular}
\end{center}

%\section{Komunikacja  w zespole}

\section{Dokumentacja}

\paragraph{} Dokumenty tworzone sa na podstawie następujących szablonów składownych na SVN:
\begin{itemize}
 \item szablon.tex
\item notatka\_szablon.tex
\end{itemize}


%\section{Współdzielenie dokumentów i kodu}

\section{Narzędzia i wymiana informacji}
\subsection{Narzędzia programistyczne}
\begin{itemize}
 \item Netbeans 6.5
 
\end{itemize}
\subsection{Biblioteki i środowisko}
\begin{itemize}
	\item JAVA ver 6
  	\item Prefuse ver prefuse-beta20071021
	\item OWL API ver 2.1.1
\end{itemize}

\subsection{Komunikacja w zespole}
\begin{itemize}
 	\item Gadu-gadu
	\item Email
	\item Telefonicznie
	\item Wymiana dokumentacji przez SVN, materiałów dodatkowych przez email
\end{itemize}

\subsection{Tworzenie dokumentacji}
\begin{itemize}
 	\item Dokumenty w LateX
	\item na SVN wrzucamy pliki tex i ich wersje pdf
\end{itemize}


\subsection{Inne używane programy}

\begin{description}
 \item[Rysowanie notacji dla ontologii] Inkspace i Dia
 \item[UML] Netbeans
 \item[Ontologie] Programy używane jako wzorcowe zarówno w kwestii wizualizacji jak i implementacji: Protege, GrOWL.
 \item[Harmonogramy] GanttProject
\end{description}

\end{document}
