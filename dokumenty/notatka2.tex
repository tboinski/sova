\documentclass[a4paper,10pt]{article}
\usepackage{polski}
\usepackage[utf8]{inputenc}
\usepackage{array}
%custom margins
\usepackage[left=2.5cm, top=2.5cm, bottom=3cm, right=2cm, foot=2cm, head=0.5cm]{geometry}
\usepackage{fancyhdr}
\usepackage[bookmarks=true, pdftex]{hyperref}

%styl nagłowków
\pagestyle{fancy} 
\parindent 2cm 

%opening
\title{Notatka ze spotkania}
\author{3@KASK}

\begin{document}
\bibliographystyle{plain}


\maketitle


\begin{center}
%budowanie tabeli
\begin{tabular}{|p{7cm}|p{7cm}|}
\hline
Symbol projektu: & Opiekun projektu:   \tabularnewline 
3@KASK & mgr inż. Tomasz Boiński    \tabularnewline \hline
\multicolumn{2}{|l|}{Nazwa Projektu: } \tabularnewline
\multicolumn{2}{|l|}{Wizualizacja grafów za pomocą biblioteki Prefuse } \tabularnewline 
\hline
\multicolumn{2}{l}{ } \tabularnewline %pusta linijka
\hline 
Nazwa Dokumentu: & Data spotkania:   \tabularnewline 
Notatka ze spotkania & 31.03.09 \tabularnewline \hline
Odpowiedzialny za dokument: & Obecni na spotkaniu:   \tabularnewline 
Anna Jaworska & Kompletny zespół \tabularnewline \hline
Przeznaczenie: & Data ostatniej aktualizacji:   \tabularnewline 
WEWNĘTRZNE & 31.03.09 \tabularnewline \hline
\end{tabular}
\end{center}

%\newpage
%\tableofcontents
%\newpage

\section{Temat spotkania}
\paragraph{}Spotkanie ma na celu omówienie zagadnień do poruszenia w \textit{Studium wykonalności} i przygotowanie pierwszej wersji tego dokumentu. Ponadto powinny zostać podjęte decyzje odnośnie dalszego przebiegu prac. 

\section{Poruszone zagadnienia}
\begin{itemize}
 	\item Przegląd proponowanych dokumentów
	\item Utwardzenie decyzji o tworzeniu notatek
	\item Prezentacja przez Piotra K. bibliotek do wizualizacji grafów: JGraph, Piccolo Toolkit, JUNG, Prefuse
	\item Prezentacj przez Piotra K. programu do wizualizacji ontologii Protege (plugin Jambalaya) 
	\item Piotr K. dostarczył ciekawy artykul o Prefuse \cite{art}
	\item Raport Radka o tworzeniu bibliotek na podstawie  \cite{pdfik}
	\item Podjęcie decyzji aby w przysłości przygotować szablon do implementacji klas
	\item Jak testować - JUINT a TestNG, \cite{linczek}  sprawdzanie pokrycia by EMMA \cite{innylinczke}
	\item  Przegląd elmentów OWL prezentowany przez Piotra O. \cite{linkW3c}	
	\item Przegląd informacji o OWL zawartych w pracy magisterskiej Andrzeja Jakowskiego \cite{pracka}
	\item Wybranie własności property jako głównego elementu nad zdefiniowanie którego należy skupić się definiując reprezentację graficzną
	   
\end{itemize}


\section{Podjęte ustalenia}
\begin{itemize}
 	\item Należy zapytać opiekuna na jakiej licencji ma być wytworzona przez nas biblioteka; propozycje: BSD, LGPL. 
	\item Należy zapytać opiekuna gdzie są biblioteki i pliki np. ikonek do uruchomienia portalSubsystem.
	\item Piotr K. opisze biblioteki JAVA do wizualizacji grafów/danych ze szczególnym naciskiem na Prefuse.
	\item Ania dokończy Studium.
	\item Radek opisze wymogi technologiczne (standardy tworzenia biblioteki).
	\item Radek poszuka tutoriala do TestNG.
	\item Wszyscy przeczytają informacje o Prefuse.
	\item Piotr O. zidentyfikuje fragmenty kodu portalSubsystem odpowiedzialne za tłumaczenie OWL na wizualizację w Prefuse.
	\item Piotr O. zaktualizuje słownik.
\end{itemize}





\clearpage
\phantomsection
\addcontentsline{toc}{section}{Literatura}
\bibliography{biblio}

\end{document}
