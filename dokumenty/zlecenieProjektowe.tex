\documentclass[a4paper,10pt]{article}
\usepackage{polski}
\usepackage[utf8]{inputenc}
\usepackage{array}
%custom margins
\usepackage[left=2.5cm, top=2.5cm, bottom=3cm, right=2cm, foot=2cm, head=0.5cm]{geometry}
\usepackage{fancyhdr}
\usepackage[bookmarks=true, pdftex]{hyperref}

%styl nagłowków
\pagestyle{fancy} 
\parindent 2cm 

%opening
\title{Zlecenie projektowe }
\author{3@KASK}

\begin{document}
\bibliographystyle{plain}


\maketitle


\begin{center}
%budowanie tabeli
\begin{tabular}{|p{7cm}|p{7cm}|}
\hline
Symbol projektu: & Opiekun projektu:   \tabularnewline 
3@KASK & mgr inż. Tomasz Boiński    \tabularnewline \hline
\multicolumn{2}{|l|}{Nazwa Projektu: } \tabularnewline
\multicolumn{2}{|l|}{Wizualizacja grafów za pomocą biblioteki Prefuse} \tabularnewline 
\hline
\multicolumn{2}{l}{ } \tabularnewline %pusta linijka
\hline 
Nazwa Dokumentu: & Nr wersji:   \tabularnewline 
Zlecenie projektowe & 0.0 \tabularnewline \hline
Odpowiedzialny za dokument: & Data pierwszego sporządzenia:   \tabularnewline 
Piotr Kunowski & 30 marca 2009 \tabularnewline \hline
Przeznaczenie: & Data ostatniej aktualizacji:   \tabularnewline 
Wewnętrzne & \today \tabularnewline \hline
\end{tabular}
\end{center}

\begin{center}
\begin{tabular}{|c|p{4cm}|c|c|c|}
\multicolumn{5}{c}{\textbf{Historia dokumentu}} \tabularnewline \hline
\textbf{Wersja} & \textbf{Opis modyfikacji} & \textbf{Rozdział/strona} & \textbf{Autor modyfikacji} & \textbf{Data} \tabularnewline \hline 
1 & Stworzenie dokumentu & wszystkie & Grupa projektowa & 30.03.09 \tabularnewline \hline
2 & Dodanie forumułki o RUP   & 4   & Anna Jaworska & 15.04.09\tabularnewline \hline
\end{tabular}
 

\end{center}


\newpage
\tableofcontents
\newpage

\section{Cele i opis projektu}
\paragraph{} Celem projektu jest utworzenie biblioteki umożliwiającej wizualizację ontologii zapisanych w OWL API. Do tego celu należy wykorzystać język Java oraz bibliotekę Prefuse. Szczególny nacisk w projekcie należy położyć na:
\begin{itemize}
 \item Wizualizację elementów niejawnych (np. klasy anonimowe wyrażone
poprzez unie, przecięcie itp. oraz dziedziczenie po tych klasach,
łączenie wielu odwzorowań niejawnych) 
\item  Wizualizację powiązań między klasami oraz innymi elementami grafu 
\item  Udokumentowanie stworzonej biblioteki za pomoca JavaDoc
\item  Zapewnienie możliowości integracji uzyskanej biblioteki z istniejącą aplikacją OCS
\end{itemize}
	
\section{Zleceniodawca}
\paragraph{} mgr inż. Tomasz Boiński, Katedra Architektury Systemów Komputerowych, Wydział Elektorniki, Telekomunikacji i Informatyki, Politechnika Gdańska.


\section{Zleceniobiorca}
\paragraph{} Studenci wydziału Elektorniki, Telekomunikacji i Informatyki, Katedry Architektury Systemów Komputerowych.
\begin{center}
\begin{tabular}{|l|l|l|l|}
\hline
\textbf{Imię i nazwisko} & \textbf{Rola} & \textbf{E-mail} & \textbf{Telefon} \tabularnewline \hline
Piotr Kunowski & Kierownik projektu & p.kunos@gmail.com & 781-765-187 \tabularnewline \hline 
Anna Jaworska & Członek zespołu & valanthe86@gmail.com & 666-089-481 \tabularnewline \hline
Radosław Kleczkowski & Członek zespołu & radoslaw1201@gmail.com & brak \tabularnewline \hline
Piotr Orłowski & Członek zespołu & cmsptcp@gmail.com & brak \tabularnewline \hline
\end{tabular}
\end{center}

\section{Zakres prac}

\paragraph{} \textbf{Pierwszy etap projektu}

\begin{enumerate}
 \item Studium wykonalności - stworzenie następujących dokumentów:
	\begin{itemize}
 		\item  Zlecenie projektowe
		\item  Harmonogram
		\item Słownik
		\item Studium wykonalności
	\end{itemize}
 
\item Analiza wymagań - stworzenie następujących dokumentów:
	\begin{itemize}
 		\item Specyfikacja wymagań
		\item Specyfikacja przypadków użycia
	\end{itemize}


\item Analiza obiektowa - stworzenie następujących dokumentów:
	\begin{itemize}
 		\item Model klas
		\item Model dynamiki
		\item Specyfikacja przypadków testowych
	\end{itemize}

\item Prototyp - stworzenie kodu i dokumentów:
	\begin{itemize}
 		\item Prototyp klas
		\item Opis prototypu  
	\end{itemize}

\item Odbiór projektu - stworzenie następujących dokumentów:
	\begin{itemize}
 		\item Plakat
		\item Prezentacja
	\end{itemize}

\end{enumerate}

\paragraph{} \textbf{Drugi etap projektu}

Z uwagi na przyjętą metodykę wytwarzania oprogramowania (RUP) ta część dokumentu zostanie rozwinięta później.

%\clearpage
%\phantomsection
%\addcontentsline{toc}{section}{Literatura}
%\bibliography{biblio}

\end{document}
