\documentclass[a4paper,10pt]{article}
\usepackage{polski}
\usepackage[utf8]{inputenc}
\usepackage{array}
%custom margins
\usepackage[left=2.5cm, top=2.5cm, bottom=3cm, right=2cm, foot=2cm, head=0.5cm]{geometry}
\usepackage{fancyhdr}
\usepackage[bookmarks=true, pdftex]{hyperref}

%styl nagłowków
\pagestyle{fancy} 
\parindent 2cm 

%opening
\title{Notatka ze spotkania}
\author{3@KASK}

\begin{document}
\bibliographystyle{plain}


\maketitle


\begin{center}
%budowanie tabeli
\begin{tabular}{|p{7cm}|p{7cm}|}
\hline
Symbol projektu: & Opiekun projektu:   \tabularnewline 
3@KASK & mgr inż. Tomasz Boiński    \tabularnewline \hline
\multicolumn{2}{|l|}{Nazwa Projektu: } \tabularnewline
\multicolumn{2}{|l|}{Wizualizacja grafów za pomocą biblioteki Prefuse } \tabularnewline 
\hline
\multicolumn{2}{l}{ } \tabularnewline %pusta linijka
\hline 
Nazwa Dokumentu: & Data spotkania:   \tabularnewline 
Notatka ze spotkania & 18.05.09 \tabularnewline \hline
Odpowiedzialny za dokument: & Obecni na spotkaniu:   \tabularnewline 
Piotr Orłowski & Grupa projektowa \tabularnewline \hline
Przeznaczenie: & Data ostatniej aktualizacji:   \tabularnewline 
WEWNĘTRZNE & 18.05.09 \tabularnewline \hline
\end{tabular}
\end{center}



\section{Temat spotkania: rysowanie pizzy przy użyciu wybranej przez nas notacji cz. 2.}


\section{Poruszone zagadnienia}
1. Ćwiczenia 18-22, 31. Ćwiczenia z wnioskowaniem przy użyciu Protege zostały w większości ominięte..
2. Przygotowaia do tworzenia diagramu klas
3. Pierwsza wersja diagramu klas

\section{Podjęte ustalenia}
1. Zmiana podejścia do wizualizacji property. Property będziemy wizualizować w dwóch formach: jego ogólnej definicji i zastosowaniach. Definicja i zastosowanie będą różniły się tłem (wyróżnione kolory).
2. AllValuesFrom i SomeValuesFrom - zmieniamy sposób wizualizacji tych cech Property. Będą one różniły się odcieniem tła.  
3. Sprawdzić jakie metody potrzebne są do rysowania krawędzi i wierzchołków - Ania
4. Czy do prefuse należy napisać własny Reader i jakiego kontenera danych Prefuse użyć - Piotr K.
5. Jak trzymane są dane w OWL API, jak je przeglądać itp. - Radek
6. Jak w GrOWL wczytywane jest OWL API; zapoznać się z kodem w miarę szczegółowo - Piotr O.



\clearpage
\phantomsection
\addcontentsline{toc}{section}{Literatura}
\bibliography{biblio}

\end{document}
