\documentclass[a4paper,10pt]{article}
\usepackage{polski}
\usepackage[utf8]{inputenc}
\usepackage{array}
%custom margins
\usepackage[left=2.5cm, top=2.5cm, bottom=3cm, right=2cm, foot=2cm, head=0.5cm]{geometry}
\usepackage{fancyhdr}
\usepackage[bookmarks=true, pdftex]{hyperref}

%styl nagłowków
\pagestyle{fancy} 
\parindent 2cm 

%opening
\title{Słownik}
\author{3@KASK}

\begin{document}
\bibliographystyle{plain}


\maketitle


\begin{center}
%budowanie tabeli
\begin{tabular}{|p{7cm}|p{7cm}|}
\hline
Symbol projektu: & Opiekun projektu:   \tabularnewline 
3@KASK & mgr inż. Tomasz Boiński    \tabularnewline \hline
\multicolumn{2}{|l|}{Nazwa Projektu: } \tabularnewline
\multicolumn{2}{|l|}{Wizualizacja grafów za pomocą biblioteki Prefuse } \tabularnewline 
\hline
\multicolumn{2}{l}{ } \tabularnewline %pusta linijka
\hline 
Nazwa Dokumentu: & Nr wersji:   \tabularnewline 
Słownik pojęć & 0.03 \tabularnewline \hline
Odpowiedzialny za dokument: & Data pierwszego sporządzenia:   \tabularnewline 
Piotr Orłowski & 31.03.09 \tabularnewline \hline
Przeznaczenie: & Data ostatniej aktualizacji:   \tabularnewline 
WEWNĘTRZNE & 07.04.09 \tabularnewline \hline
\end{tabular}
\end{center}

\begin{center}
\begin{tabular}{|c|p{4cm}|c|c|c|}
\multicolumn{5}{c}{\textbf{Historia dokumentu}} \tabularnewline \hline
\textbf{Wersja} & \textbf{Opis modyfikacji} & \textbf{Rozdział/strona} & \textbf{Autor modyfikacji} & \textbf{Data} \tabularnewline \hline 
1 & Stworzenie zarysu słownika & wszystkie & Anna Jaworska & 31.03.09 \tabularnewline \hline
2 & Podstawowe pojęcia Semantic Web & Pojęcia ogólne & Piotr Orłowski & 31.03.09 \tabularnewline \hline
3 & Licencje wolnego oprogramwania & Pojęcia ogólne & Piotr Orłowski & 07.04.09 \tabularnewline \hline
& & & &\tabularnewline \hline
\end{tabular}
\end{center}

\newpage
\tableofcontents
\newpage

\section{Jak korzystać ze slownika}
Z uwagi na przyjętą metodykę wytwarzania oprogramowania (RUP) ta część dokumentu zostanie rozwinięta później.

\section{Pojęcia ogólne}
\begin{description}
	\item[WETI/ETI] Wydział Elektroniki, Telekomunikacji i Informatyki Politechniki Gdańskiej
	\item[KASK] Katedra Architektury Systemów Komputerowych WETI
 	\item[OCS] Ontology  Creation System - projekt realizowany w ramach grantu (tu id grantu) na KASK-u
	\item[JAVA] Obiektowy język programowania; pojęcie używane czasem w sensie maszyny wirtualnej jezyka JAVA  
	\item[Prefuse] Biblioteka języka JAVA, pozwalająca na estetyczna prezentacje danych, w szczególności grafów
	\item[API] ang. Application Programming Interface, interfejs dla programów, zestaw poleceń, funkcji, metod, formatów i danych, które służą do wymiany informacji pomiędzy aplikacją i systemem operacyjnym oraz innymi programami lub sterownikami.
	\item[W3C] World Wide Web Consorcium - organizacja odpowiedzialna za ustalanie standardów dla metajęzyków
	\item[ontologia] dział filozofii starający się badać strukturę rzeczywistości i zajmujący się problematyką związaną z pojęciami bytu, istoty, istnienia i jego sposobów, przedmiotu i jego własności, przyczynowości, czasu, przestrzeni, konieczności i możliwości.
	\item[SVN] SubVersioN - system kontroli wersji
	\item[XML] ang. Extensible Markup Language, uniwersalny język formalny przeznaczony do reprezentowania różnych danych w ustrukturalizowany sposób
	\item[RDF] Resource Description Framework, jest specyfikacją W3C stosowaną do modelowania metadanych w postaci wyrażeń zawierających predykaty, klasy i podmioty; wyrażenia te tworzą graf skierowany
	\item[RDFS] RDF Schema, język reprezentacji wiedzy oparty na RDF 
	\item[OWL] Web Ontology Language, jest to rozszerzenie RDFS
	\item[Sieć Semantyczna] ang. Semantic Web, projekt, który ma umożliwić łatwiejsze i bardziej logiczne wyszukiwanie przez maszyny i programy(agenty) danych w sieci Internet; znaczenie zasobów informacyjnych opisywane jest tu przy pomocy ontologii; do standardów rozwijanych wraz z Semantic Web należą m.in. OWL, RDF oraz RDFS 
	\item[metadane] są to dane opisujące inne dane, stosowane w celu ułatwienia korzystania z tych danych
	\item[agent] (lm. agenty) jednostka (np. program), działającą w pewnym środowisku, zdolna do komunikowania się, monitorowania swego otoczenia i podejmowania autonomicznych decyzji, aby osiągnąć cele określone podczas jej projektowania lub działania.
	\item[aplikacja standalone] to aplikacja, która do uruchomienia nie wymaga innych programów
	\item[GPL] GNU General Public License, jedna z licencji Wolnego Oprogramowania stworzona przez Richarda Stallmana i Ebena Moglena; zawiera zastrzeżenie, że wszystkie pochodne prace bazujące na kodzie wydanym na licencji GPL muszą być wydane na licencji GPL 
	\item[BSD] Berkeley Software Distribution License, jedna z licencji zgodnych z zasadami Wolnego Oprogramowania stworzona na Uniwersytecie Kalifornijskim w Berkeley.
	\item[debugowanie] znany także jako odpluskwianie, proces szukania i naprawiania błędów	w programach komputerowychza pomocą specjalnych narzędzi do teog przeznaczonych.
	\item[strumień błędów]
\end{description}



\section{Pojęcia specificzne dla projektu}
\begin{description}
 	\item[portalSubsystem] część projektu OCS, pozwala na wizualizację online plików OWL
 	\item[kardynalność] tutaj występująca w języku OWL liczność elementu 
\end{description}



\clearpage
\phantomsection
\addcontentsline{toc}{section}{Literatura}
%\bibliography{biblio}

\end{document}
