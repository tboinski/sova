\documentclass[a4paper,10pt]{article}
\usepackage{polski}
\usepackage[utf8]{inputenc}
\usepackage{array}
%custom margins
\usepackage[left=2.5cm, top=2.5cm, bottom=3cm, right=2cm, foot=2cm, head=0.5cm]{geometry}
\usepackage{fancyhdr}
\usepackage[bookmarks=true, pdftex]{hyperref}

%styl nagłowków
\pagestyle{fancy} 
\parindent 2cm 

%opening
\title{Tytuł dokumentu}
\author{3@KASK}

\begin{document}
\bibliographystyle{plain}


\maketitle


\begin{center}
%budowanie tabeli
\begin{tabular}{|p{7cm}|p{7cm}|}
\hline
Symbol projektu: & Opiekun projektu:   \tabularnewline 
3@KASK & mgr inż. Tomasz Boiński    \tabularnewline \hline
\multicolumn{2}{|l|}{Nazwa Projektu: } \tabularnewline
\multicolumn{2}{|l|}{Wizualizacja grafów za pomocą biblioteki Prefuse } \tabularnewline 
\hline
\multicolumn{2}{l}{ } \tabularnewline %pusta linijka
\hline 
Nazwa Dokumentu: & Nr wersji:   \tabularnewline 
Słownik pojęć & 0.0 \tabularnewline \hline
Odpowiedzialny za dokument: & Data pierwszego sporządzenia:   \tabularnewline 
Piotr Orłowski & 31.03.09 \tabularnewline \hline
Przeznaczenie: & Data ostatniej aktualizacji:   \tabularnewline 
WEWNĘTRZNE & 31.03.09 \tabularnewline \hline
\end{tabular}
\end{center}

\begin{center}
\begin{tabular}{|c|p{4cm}|c|c|c|}
\multicolumn{5}{c}{\textbf{Historia dokumentu}} \tabularnewline \hline
\textbf{Wersja} & \textbf{Opis modyfikacji} & \textbf{Rozdział/strona} & \textbf{Autor modyfikacji} & \textbf{Data} \tabularnewline \hline 
1 & Stworzenie zarysu słownika & wszystkie & Anna Jaworska & 31.03.09 \tabularnewline \hline
& & & &\tabularnewline \hline
\end{tabular}
\end{center}

\newpage
\tableofcontents
\newpage

\section{Jak korzystać ze slownika}

\section{Pojęcia ogólne}
\begin{description}
	\item[WETI/ETI] Wydział Elektroniki, Telekomunikacji i Informatyki Politechniki Gdańskiej
	\item[KASK] Katedra Architektury Systemów Komputerowych WETI
 	\item[OCS] Ontology  C.. System - projekt realizowany w ramach grantu (tu id grantu) na KASK-u
	\item[JAVA] Obiektowy język programowania; pojęcie używane czasem w sensie maszyny wirtualnej jezyka JAVA  
	\item[Prefuse] Biblioteka języka JAVA, pozwalająca na estetyczna prezentacje danych, w szczególności grafów
	\item[API] 
	\item[W3C] World Wide Web Consorcium - organizacja odpowiedzialna za ustalanie standardów dla metajęzyków
	\item[ontologia] 
	\item[SVN]
	\item[XML]
	\item[debugowanie]
	\item[Javadoc]  
\end{description}



\section{Pojęcia specificzne dla projektu}
\begin{description}
 \item[portalSubsystem]
\end{description}



\clearpage
\phantomsection
\addcontentsline{toc}{section}{Literatura}
\bibliography{biblio}

\end{document}
