\documentclass[a4paper,10pt]{article}
\usepackage{polski}
\usepackage[utf8]{inputenc}
\usepackage{array,floatflt,textcomp,graphicx,longtable}
%custom margins
\usepackage[left=3cm, top=2.5cm, bottom=3cm, right=2cm, foot=2cm, head=0.5cm]{geometry}
\usepackage{fancyhdr}
\usepackage[bookmarks=true, pdftex]{hyperref}

%styl nagłowków
\pagestyle{fancy} 
\parindent 2cm 

%opening
\title{Analiza obiektowa}
\author{3@KASK}

\begin{document}
\bibliographystyle{plain}


\maketitle


\begin{center}
%budowanie tabeli
\begin{longtable}{|p{7cm}|p{7cm}|}
\hline
Symbol projektu: & Opiekun projektu:   \tabularnewline 
3@KASK & mgr inż. Tomasz Boiński    \tabularnewline \hline
\multicolumn{2}{|l|}{Nazwa Projektu: } \tabularnewline
\multicolumn{2}{|l|}{Wizualizacja grafów za pomocą biblioteki Prefuse } \tabularnewline 
\hline
\multicolumn{2}{l}{ } \tabularnewline %pusta linijka
\hline 
Nazwa Dokumentu: & Nr wersji:   \tabularnewline 
Analiza obiektowa & 2.0 \tabularnewline \hline
Odpowiedzialny za dokument: & Data pierwszego sporządzenia:   \tabularnewline 
Piotr Kunowski & 23 maja 2009 \tabularnewline \hline
Przeznaczenie: & Data ostatniej aktualizacji:   \tabularnewline 
DLA KLIENTA & \today \tabularnewline \hline
\end{longtable}
\end{center}


\begin{center}
\begin{longtable}{|c|p{4cm}|c|c|c|}
\multicolumn{5}{c}{\textbf{Historia dokumentu}} \tabularnewline \hline
\textbf{Wersja} & \textbf{Opis modyfikacji} & \textbf{Rozdział/strona} & \textbf{Autor modyfikacji} & \textbf{Data} \tabularnewline \hline 
1 & Stworzenie & wszystkie & Grupa projektowa & 23.05.09 \tabularnewline \hline
1.1 & Dodano pakiet Utils & 1, 3 & Anna Jaworska & 2.06.09 \tabularnewline \hline
2 & Dodano zaktualizowane diagramy oraz opisy klas & wszystkie & Grupa projektowa & 16.06.09 \tabularnewline \hline

\end{longtable}
\end{center}


\newpage
\tableofcontents
\newpage

\section{Pakiety}

\subsection{Diagram}
 \includegraphics[width=\linewidth]{./modelowanie/OV_UML/PackageDiagram.png}


\subsection{Opis pakietów}


\begin{center}
\begin{longtable}{|m{3cm}|m{9cm}|} \hline

P001 & options\\ \hline
Opis: & Pakiet zawierający klasy z polami opisującymi różne (modyfikowalne) ustawienia wizualizacji takie jak: kolory, grubość linii itp.     \\ \hline
Interfejsy: &     \\ \hline
Realizowane wymagania: & WF002, WF001, WI004 \\ \hline
Priorytet: & średnio ważne \\ \hline

\multicolumn{2}{c}{} \\
 \hline

P002 & nodes\\ \hline
Opis: & Pakiet z klasami odpowiedzialnymi za wizualizację i przechowywanie danych o wierzchołkach.    \\ \hline
Interfejsy: &     \\ \hline
Realizowane wymagania: & WF004, WF005, WF006, WF007, WI004 \\ \hline
Priorytet: & bardzo ważne \\ \hline

\multicolumn{2}{c}{} \\
 \hline

P003 & edges\\ \hline
Opis: & Pakiet z klasami odpowiedzialnymi za wizualizację i przechowywanie danych o krawędziach.    \\ \hline
Interfejsy: &     \\ \hline
Realizowane wymagania: & WF006, WF007, WI004 \\ \hline
Priorytet: & bardzo ważne \\ \hline

\multicolumn{2}{c}{} \\
 \hline

P004 & visualization\\ \hline
Opis: & Zawiera dodatkowe klasy przydatne w wizualizacji.\\ \hline
Interfejsy: &     \\ \hline
Realizowane wymagania: & WF001, WF008, WI004 \\ \hline
Priorytet: & średnio ważne \\ \hline

\multicolumn{2}{c}{} \\
 \hline

P005 & graph\\ \hline
Opis: & Pakiet zawiera klasy, które zawierają podstawowe operacje na danych OwlApi oraz graph. \\ \hline
Interfejsy: &     \\ \hline
Realizowane wymagania: & WD001 \\ \hline
Priorytet: & bardzo ważne \\ \hline

%\multicolumn{2}{c}{} \\
% \hline
\multicolumn{2}{c}{} \\
 \hline

P006 & utils\\ \hline
Opis: & Pakiet zawiera klasy pomocnicze \\ \hline
Interfejsy: &     \\ \hline
Realizowane wymagania: & CF005 \\ \hline
Priorytet: & bardzo ważne \\ \hline

\end{longtable}

\end{center}

\section{Pakiet options}

\subsection{Diagram}

 \scalebox{0.50}{ \includegraphics{./modelowanie/OV_UML/OptionsClassDiagram.png}

}

\subsection{Opis klasy}

\begin{center}
 


\begin{longtable}{|m{3cm}|m{9cm}|} \hline

CO001 & EdgeColors \\ \hline
Opis: & Zawiera definicje kolorów dla poszczególnych rodzajów krawędzi.   \\ \hline
Klasy nadrzędne: &     \\ \hline
Atrybuty: & \begin{itemize}
 \item domainEdgeColor
 \item edgeColor
 \item equivalentEdgeColor
 \item equivalentPropertyEdgeColor
 \item functionalEdgeColor
 \item inverseOfEdgeColor
 \item propertyEdgeColor
 \item rangeEdgeColor
 \item subEdgeColor

\end{itemize}
 \\ \hline
Metody: & %\begin{itemize}
% \item 
%\end{itemize}
  \\ \hline
Realizowane wymagania: & WF002 \\ \hline
Priorytet: & średnio ważny \\ \hline

\multicolumn{2}{c}{} \\
 \hline

CO002 & NodeColors \\ \hline
Opis: & Zawiera definicje kolorów dla poszczególnych rodzajów krawędzi.   \\ \hline
Klasy nadrzędne: &     \\ \hline
Atrybuty: & \begin{itemize}
 \item allValuesFromNodeColor
 \item cardinalityNodeColor
 \item cardinalityValueNodeColor
 \item classNodeColor
 \item complementOfNodeColor
 \item dataTypeNodeColor
 \item differentNodeColor
 \item functionalPropertyNodeColor
 \item individualNodeColor
 \item informationNodeColor
 \item intersectionOfNodeColor
 \item inverseFunctionalNodeColor
 \item maxCardinalityValueNodeColor
 \item minCardinalityValueNodeColor
 \item nothingNodeColor
 \item oneOfNodeColor
 \item propertyNodeColor
 \item sameAsNodeColor
 \item someValuesFromNodeColor
 \item symmetricPropertNodeColor
 \item thingNodeColor
 \item transitivePropertyNodeColor
 \item unionOfNodeColor 
\end{itemize}
 \\ \hline
Metody: & %\begin{itemize}
 %\item 
%\end{itemize}
  \\ \hline
Realizowane wymagania: & WF002 \\ \hline
Priorytet: & średnio ważny \\ \hline

%\multicolumn{2}{c}{} \\
% \hline

\end{longtable}


\end{center}

\section{Pakiet nodes}

\subsection{Diagram}

 \includegraphics[width=\linewidth]{./modelowanie/OV_UML/NodesClassDiagram.png}

\subsection{Opis klasy}

\begin{center}
 


\begin{longtable}{|m{3cm}|m{9cm}|} \hline

CN001 & Node \\ \hline
Opis: & Klasa nadrzędna względem wszystkich klas obsługi wierzchołków. Zawiera definicje podstawowych atrybutów i metod.    \\ \hline
Klasy nadrzędne: &     \\ \hline
Atrybuty: & \begin{itemize}
 \item strokeWidth
 \item height
 \item width
 \item annotation
 \item comment
\item Color fillColor
\item String label
\end{itemize}
 \\ \hline
Metody: &% \begin{itemize}
 %\item renderShape - metoda wizualizująca dany typ wierzchołka
%\end{itemize}
  \\ \hline
Realizowane wymagania: & WF004, WF005, WF006, WF007, WI004 \\ \hline
Priorytet: & bardzo ważne  \\ \hline

\multicolumn{2}{c}{} \\
 \hline

CN002 & AllValuesFromPropertyNode \\ \hline
Opis: & Klasa reprezentuje wierzchołek, będący OWL Property typu AllValuesFrom.    \\ \hline
Klasy nadrzędne: & Node     \\ \hline
Atrybuty: & 
%\begin{itemize}
 %\item 
%\end{itemize}
 \\ \hline
Metody: & %\begin{itemize}
 %\item 
%\end{itemize}
  \\ \hline
Realizowane wymagania: & WF004, WF006, WF007, WI004 \\ \hline
Priorytet: & ważne \\ \hline

\multicolumn{2}{c}{} \\
 \hline

CN003 & AnonymousClassNode \\ \hline
Opis: & Klasa reprezentuje wierzchołek klas anonimowych OWL.    \\ \hline
Klasy nadrzędne: & Node     \\ \hline
Atrybuty: & %\begin{itemize}
 %\item 
%\end{itemize}
 \\ \hline
Metody: & %\begin{itemize}
 %\item 
%\end{itemize}
  \\ \hline
Realizowane wymagania: & WF005, WI004 \\ \hline
Priorytet: & ważne \\ \hline

\multicolumn{2}{c}{} \\
 \hline

CN004 & CardinalityNode \\ \hline
Opis: & Klasa reprezentuje wierzchołek klas anonimowych OWL będących wynikiem ograniczenia kardynalności.    \\ \hline
Klasy nadrzędne: & AnonymousNode     \\ \hline
Atrybuty: & %\begin{itemize}
 %\item 
%\end{itemize}
 \\ \hline
Metody: & %\begin{itemize}
 %\item 
%\end{itemize}
  \\ \hline
Realizowane wymagania: & WF007, WI004 \\ \hline
Priorytet: & ważne  \\ \hline

\multicolumn{2}{c}{} \\
 \hline

CN005 & CardinalityValueNode \\ \hline
Opis: & Klasa reprezentuje wierzchołek z dokładnym ograniczeniem kardynalności (OWL Cardinality). \\ \hline
Klasy nadrzędne: & Node     \\ \hline
Atrybuty: & %\begin{itemize}
 %\item 
%\end{itemize}
 \\ \hline
Metody: & %\begin{itemize}
 %\item 
%\end{itemize}
  \\ \hline
Realizowane wymagania: & WF007, WI004 \\ \hline
Priorytet: & ważne  \\ \hline

\multicolumn{2}{c}{} \\
 \hline

CN006 & ClassNode \\ \hline
Opis: & Klasa reprezentuje wierzchołek OWL Class.    \\ \hline
Klasy nadrzędne: & Node     \\ \hline
Atrybuty: & %\begin{itemize}
 %\item 
%\end{itemize}
 \\ \hline
Metody: & %\begin{itemize}
 %\item 
%\end{itemize}
  \\ \hline
Realizowane wymagania: & WF004, WF005, WI004 \\ \hline
Priorytet: & ważne  \\ \hline

\multicolumn{2}{c}{} \\
 \hline

CN007 & ComplementOfNode \\ \hline
Opis: & Klasa reprezentuje wierzchołek klas anonimowych OWL będących wynikiem dopełnienia (OWL ComplementOf).    \\ \hline
Klasy nadrzędne: & Node     \\ \hline
Atrybuty: & %\begin{itemize}
 %\item 
%\end{itemize}
 \\ \hline
Metody: & %\begin{itemize}
 %\item 
%\end{itemize}
  \\ \hline
Realizowane wymagania: & WF006, WF007, WI004 \\ \hline
Priorytet: & ważne  \\ \hline

\multicolumn{2}{c}{} \\
 \hline

CN008 & DataTypeNode \\ \hline
Opis: & Klasa reprezentuje wierzchołek OWL DataType.    \\ \hline
Klasy nadrzędne: & Node     \\ \hline
Atrybuty: & %\begin{itemize}
 %\item 
%\end{itemize}
 \\ \hline
Metody: & %\begin{itemize}
 %\item 
%\end{itemize}
  \\ \hline
Realizowane wymagania: & WF004, WI04 \\ \hline
Priorytet: & ważne  \\ \hline

\multicolumn{2}{c}{} \\
 \hline

CN009 & DifferentNode \\ \hline
Opis: & Klasa reprezentuje wierzchołek oznaczający relację DifferentFrom lub AllDifferent pomiędzy wystąpieniami klas (OWL Individual).    \\ \hline
Klasy nadrzędne: & Node     \\ \hline
Atrybuty: & %\begin{itemize}
 %\item 
%\end{itemize}
 \\ \hline
Metody: & %\begin{itemize}
 %\item 
%\end{itemize}
  \\ \hline
Realizowane wymagania: & WF006, WF007, WI004 \\ \hline
Priorytet: & ważne  \\ \hline

\multicolumn{2}{c}{} \\
 \hline

CN010 & FunctionalPropertyNode \\ \hline
Opis: & Klasa reprezentuje wierzchołek oznaczający, że dane OWL Property to FunctionalProperty.  \\ \hline
Klasy nadrzędne: & InformationNode     \\ \hline
Atrybuty: & %\begin{itemize}
 %\item 
%\end{itemize}
 \\ \hline
Metody: & %\begin{itemize}
 %\item 
%\end{itemize}
  \\ \hline
Realizowane wymagania: & WF006, WF007, WI004 \\ \hline
Priorytet: & ważne  \\ \hline

\multicolumn{2}{c}{} \\
 \hline

CN011 & IndividualNode \\ \hline
Opis: & Klasa reprezentuje wierzchołek instancji OWL Individual.  \\ \hline
Klasy nadrzędne: & Node     \\ \hline
Atrybuty: & %\begin{itemize}
 %\item 
%\end{itemize}
 \\ \hline
Metody: & %\begin{itemize}
 %\item 
%\end{itemize}
  \\ \hline
Realizowane wymagania: & WF004, WI004 \\ \hline
Priorytet: & ważne  \\ \hline

\multicolumn{2}{c}{} \\
 \hline

CN012 & InformationNode \\ \hline
Opis: & Klasa ta jest klasą nadrzędną, dla klas wierzchołków reprezentujących informacje o różnych właściwościach OWL Property.    \\ \hline
Klasy nadrzędne: & Node     \\ \hline
Atrybuty: & %\begin{itemize}
 %\item 
%\end{itemize}
 \\ \hline
Metody: & %\begin{itemize}
 %\item 
%\end{itemize}
  \\ \hline
Realizowane wymagania: & WF010, WI004 \\ \hline
Priorytet: & ważne  \\ \hline

\multicolumn{2}{c}{} \\
 \hline

CN013 & IntersectionOfNode \\ \hline
Opis: & Klasa reprezentuje wierzchołek klas anonimowych OWL będących wynikiem przecięcia (OWL IntersectionOf).    \\ \hline
Klasy nadrzędne: & AnonymousNode     \\ \hline
Atrybuty: & %\begin{itemize}
 %\item 
%\end{itemize}
 \\ \hline
Metody: & %\begin{itemize}
 %\item 
%\end{itemize}
  \\ \hline
Realizowane wymagania: & WF005, WI004 \\ \hline
Priorytet: & ważne  \\ \hline

\multicolumn{2}{c}{} \\
 \hline

CN014 & inverseFunciotnalPropertyNode \\ \hline
Opis: & Klasa reprezentuje wierzchołek oznaczający, że dane OWL Property to InverseFunctionalProperty.    \\ \hline
Klasy nadrzędne: & InformationNode     \\ \hline
Atrybuty: & %\begin{itemize}
 %\item 
%\end{itemize}
 \\ \hline
Metody: & %\begin{itemize}
 %\item 
%\end{itemize}
  \\ \hline
Realizowane wymagania: & WF007, WI004 \\ \hline
Priorytet: & ważne  \\ \hline

\multicolumn{2}{c}{} \\
 \hline

CN015 & MaxCardinalityValueNode \\ \hline
Opis: & Klasa reprezentuje wierzchołek ograniczenia kardynalności OWL MaxCardinality.    \\ \hline
Klasy nadrzędne: & CardinalityValueNode     \\ \hline
Atrybuty: & %\begin{itemize}
 %\item 
%\end{itemize}
 \\ \hline
Metody: & %\begin{itemize}
 %\item 
%\end{itemize}
  \\ \hline
Realizowane wymagania: & WF007, WI004 \\ \hline
Priorytet: & ważne  \\ \hline

\multicolumn{2}{c}{} \\
 \hline

CN016 & MinCardinalityValueNode \\ \hline
Opis: & Klasa reprezentuje wierzchołek ograniczenia kardynalności OWL MinCardinality.    \\ \hline
Klasy nadrzędne: & CardinalityValueNode     \\ \hline
Atrybuty: & %\begin{itemize}
 %\item 
%\end{itemize}
 \\ \hline
Metody: & %\begin{itemize}
 %\item 
%\end{itemize}
  \\ \hline
Realizowane wymagania: & WF007, WI004 \\ \hline
Priorytet: & ważne  \\ \hline

\multicolumn{2}{c}{} \\
 \hline

CN017 & NothingNode \\ \hline
Opis: & Klasa reprezentuje wierzchołek OWL Nothing.    \\ \hline
Klasy nadrzędne: & Node     \\ \hline
Atrybuty: & %\begin{itemize}
 %\item 
%\end{itemize}
 \\ \hline
Metody: & %\begin{itemize}
 %\item 
%\end{itemize}
  \\ \hline
Realizowane wymagania: & WF004, WF005, WI004 \\ \hline
Priorytet: & ważne  \\ \hline

\multicolumn{2}{c}{} \\
 \hline

CN018 & OneOfNode \\ \hline
Opis: & Klasa reprezentuje wierzchołek klas anonimowych OWL reprezentujących 1 z klas określonego zbioru (wynik OWL OneOf).    \\ \hline
Klasy nadrzędne: & AnonymousClassNode     \\ \hline
Atrybuty: & %\begin{itemize}
 %\item 
%\end{itemize}
 \\ \hline
Metody: & %\begin{itemize}
 %\item 
%\end{itemize}
  \\ \hline
Realizowane wymagania: & WF005, WF006, WI004 \\ \hline
Priorytet: & ważne  \\ \hline

\multicolumn{2}{c}{} \\
 \hline

CN019 & PropertyNode \\ \hline
Opis: & Klasa reprezentuje wierzchołek OWL Property.    \\ \hline
Klasy nadrzędne: & Node     \\ \hline
Atrybuty: & %\begin{itemize}
 %\item 
%\end{itemize}
 \\ \hline
Metody: & %\begin{itemize}
 %\item 
%\end{itemize}
  \\ \hline
Realizowane wymagania: & WF004, WF007, WI004 \\ \hline
Priorytet: & ważne  \\ \hline

\multicolumn{2}{c}{} \\
 \hline

CN020 & SameAsNode \\ \hline
Opis: & Klasa reprezentuje wierzchołek oznaczający relację OWL SameAs pomiędzy wystąpieniami klas (OWL Individual).    \\ \hline
Klasy nadrzędne: & InformationNode     \\ \hline
Atrybuty: & %\begin{itemize}
 %\item 
%\end{itemize}
 \\ \hline
Metody: & %\begin{itemize}
 %\item 
%\end{itemize}
  \\ \hline
Realizowane wymagania: & WF005, WF006, WI004 \\ \hline
Priorytet: & ważne  \\ \hline

\multicolumn{2}{c}{} \\
 \hline

CN021 & SomeValuesFromPropertyNode \\ \hline
Opis: & Klasa reprezentuje wierzchołek, będący OWL Property typu SomeValuesFrom. \\ \hline
Klasy nadrzędne: & PropertyNode \\ \hline
Atrybuty: & %\begin{itemize}
 %\item 
%\end{itemize}
 \\ \hline
Metody: & %\begin{itemize}
 %\item 
%\end{itemize}
  \\ \hline
Realizowane wymagania: & WF005, WF006, WI004 \\ \hline
Priorytet: & ważne  \\ \hline

\multicolumn{2}{c}{} \\
 \hline

CN022 & SymmetricPropertNode \\ \hline
Opis: & Klasa reprezentuje wierzchołek oznaczający, że dane OWL Property to SymmetricProperty.    \\ \hline
Klasy nadrzędne: & InformationNode     \\ \hline
Atrybuty: & %\begin{itemize}
 %\item 
%\end{itemize}
 \\ \hline
Metody: & %\begin{itemize}
 %\item 
%\end{itemize}
  \\ \hline
Realizowane wymagania: & WF007, WI004 \\ \hline
Priorytet: & ważne  \\ \hline

\multicolumn{2}{c}{} \\
 \hline

CN023 & ThingNode \\ \hline
Opis: & Klasa reprezentuje wierzchołek OWL Thing.    \\ \hline
Klasy nadrzędne: & Node     \\ \hline
Atrybuty: & %\begin{itemize}
 %\item 
%\end{itemize}
 \\ \hline
Metody: & %\begin{itemize}
 %\item 
%\end{itemize}
  \\ \hline
Realizowane wymagania: & WF004, WF005, WI004 \\ \hline
Priorytet: & ważne  \\ \hline

\multicolumn{2}{c}{} \\
 \hline

CN024 & TreansitivePropertyNode \\ \hline
Opis: & Klasa reprezentuje wierzchołek oznaczający, że dane OWL Property to TransitiveProperty.    \\ \hline
Klasy nadrzędne: & InformationNode     \\ \hline
Atrybuty: & %\begin{itemize}
 %\item 
%\end{itemize}
 \\ \hline
Metody: & %\begin{itemize}
 %\item 
%\end{itemize}
  \\ \hline
Realizowane wymagania: & WF006, WF007, WI004 \\ \hline
Priorytet: & ważne  \\ \hline

\multicolumn{2}{c}{} \\
 \hline

CN025 & UnionOfNode \\ \hline
Opis: & Klasa reprezentuje wierzchołek klas anonimowych OWL będących wynikiem unii (OWL UnionOf).    \\ \hline
Klasy nadrzędne: & AnonymousNode     \\ \hline
Atrybuty: & %\begin{itemize}
 %\item 
%\end{itemize}
 \\ \hline
Metody: & %\begin{itemize}
 %\item 
%\end{itemize}
  \\ \hline
Realizowane wymagania: & WF005, WF006, WI004 \\ \hline
Priorytet: & ważne  \\ \hline

%\multicolumn{2}{c}{} \\
% \hline


\end{longtable}

\end{center}

\section{Pakiet edges}

\subsection{Diagram}

\includegraphics[width=\linewidth]{./modelowanie/OV_UML/EdgeClassDiagram.png}

\subsection{Opis klasy}

\begin{center}

\begin{longtable}{|m{3cm}|m{9cm}|} \hline

CE001 & Edge \\ \hline
Opis: &  Klasa reprezentująca prostą krawędź na grafie. Jest nadklasą dla pozostałych klas krawędzi.  \\ \hline
Klasy nadrzędne: &     \\ \hline
Atrybuty: & \begin{itemize}
\item Color strokeColor
\item  int strokeWidth 
\item boolean hasArrow
\item    boolean hasInvertedArrow
  \item  Polygon arrowHead
  \item Color arrowHeadColor
\end{itemize}
 \\ \hline
Metody: & \begin{itemize}
 \item    getStrokeColor () 
\item setStrokeColor (Color val)
  \item  getStrokeWidth () 
\item setStrokeWidth (int val) 
\item getArrowHead()
\item setArrowHead(Polygon arrowHead)
\item isHasArrow()
 \item setHasArrow(boolean hasArrow)
  \item isHasInvertedArrow()
\item setHasInvertedArrow(boolean hasInvertedArrow)
\item getArrowHeadColor()
 \item setArrowHeadColor(Color arrowHeadColor)
\end{itemize}
  \\ \hline
Realizowane wymagania: & WF006, WF007, WI004 \\ \hline
Priorytet: & bardzo ważne  \\ \hline

\multicolumn{2}{c}{} \\
 \hline

CE002 & DisjointEdge \\ \hline
Opis: & Klasa reprezentująca krawędź oznaczającą rozłączność klas (OWL Disjoint). \\ \hline
Klasy nadrzędne: & Edge    \\ \hline
Atrybuty: & %\begin{itemize}
 %\item 
%\end{itemize}
 \\ \hline
Metody: & %\begin{itemize}
 %\item 
%\end{itemize}
  \\ \hline
Realizowane wymagania: & WF006, WF007, WI004 \\ \hline
Priorytet: & ważne  \\ \hline

\multicolumn{2}{c}{} \\
 \hline

CE003 & DomainEdge  \\ \hline
Opis: & Klasa reprezentująca krawędź łączącą Property z klasą właściwości OWL DomainOf.   \\ \hline
Klasy nadrzędne: & Edge \\ \hline
Atrybuty: & %\begin{itemize}
 %\item 
%\end{itemize}
 \\ \hline
Metody: & %\begin{itemize}
 %\item 
%\end{itemize}
  \\ \hline
Realizowane wymagania: & WF006, WF007, WI004 \\ \hline
Priorytet: & ważne  \\ \hline

\multicolumn{2}{c}{} \\
 \hline

CE004 & EquivalentEdge \\ \hline
Opis: & Klasa reprezentująca krawędź oznaczającą równoznaczność (OWL Equivalent).    \\ \hline
Klasy nadrzędne: & Edge    \\ \hline
Atrybuty: & %\begin{itemize}
 %\item 
%\end{itemize}
 \\ \hline
Metody: & %\begin{itemize}
 %\item 
%\end{itemize}
  \\ \hline
Realizowane wymagania: & WF006, WF007, WI004 \\ \hline
Priorytet: & ważne  \\ \hline

\multicolumn{2}{c}{} \\
 \hline

CE005 & EquivalentPropertyEdge \\ \hline
Opis: & Klasa reprezentująca krawędź oznaczającą równoznaczność OWL Property (OWL EquivalentProperty).    \\ \hline
Klasy nadrzędne: & EquivalentEdge    \\ \hline
Atrybuty: & %\begin{itemize}
 %\item 
%\end{itemize}
 \\ \hline
Metody: & %\begin{itemize}
 %\item 
%\end{itemize}
  \\ \hline
Realizowane wymagania: & WF006, WF007, WI004 \\ \hline
Priorytet: & ważne  \\ \hline

\multicolumn{2}{c}{} \\
 \hline

CE006 & FunctionaltEdge \\ \hline
Opis: & Klasa reprezentująca krawędź łączącą wierzchołki InformationNode(CN012) z OWL Property, którego dotyczy.   \\ \hline
Klasy nadrzędne: & Edge    \\ \hline
Atrybuty: & %\begin{itemize}
 %\item 
%\end{itemize}
 \\ \hline
Metody: & %\begin{itemize}
 %\item 
%\end{itemize}
  \\ \hline
Realizowane wymagania: & WF006, WF007, WI004 \\ \hline
Priorytet: & ważne  \\ \hline

\multicolumn{2}{c}{} \\
 \hline

CE007 & InverseOfEdge \\ \hline
Opis: & Klasa reprezentująca krawędź oznaczającą odwrotność (OWL InverseOf).    \\ \hline
Klasy nadrzędne: & Edge    \\ \hline
Atrybuty: & %\begin{itemize}
 %\item 
%\end{itemize}
 \\ \hline
Metody: & %\begin{itemize}
 %\item 
%\end{itemize}
  \\ \hline
Realizowane wymagania: & WF006, WF007, WI004 \\ \hline
Priorytet: & ważne  \\ \hline

\multicolumn{2}{c}{} \\
 \hline

CE008 & PropertyEdge \\ \hline
Opis: &  Klasa reprezentująca krawędź oznaczającą relację między Property a klasą.   \\ \hline
Klasy nadrzędne: & Edge    \\ \hline
Atrybuty: & %\begin{itemize}
 %\item 
%\end{itemize}
 \\ \hline
Metody: & %\begin{itemize}
 %\item 
%\end{itemize}
  \\ \hline
Realizowane wymagania: & WF006, WF007, WI004 \\ \hline
Priorytet: & ważne  \\ \hline

\multicolumn{2}{c}{} \\
 \hline

CE009 & RangeEdge \\ \hline
Opis: & Klasa reprezentująca na grafie krawędź łączącą Property z klasą właściwości OWL Range.     \\ \hline
Klasy nadrzędne: & Edge    \\ \hline
Atrybuty: & %\begin{itemize}
 %\item 
%\end{itemize}
 \\ \hline
Metody: & %\begin{itemize}
 %\item 
%\end{itemize}
  \\ \hline
Realizowane wymagania: & WF006, WF007, WI004 \\ \hline
Priorytet: & ważne  \\ \hline

\multicolumn{2}{c}{} \\
 \hline

CE010 & SubEdge \\ \hline
Opis: & Klasa reprezentująca krawędź związku OWL SubClass pomiędzy klasami.   \\ \hline
Klasy nadrzędne: & Edge    \\ \hline
Atrybuty: & %\begin{itemize}
 %\item 
%\end{itemize}
 \\ \hline
Metody: & %\begin{itemize}
 %\item 
%\end{itemize}
  \\ \hline
Realizowane wymagania: & WF006, WF007, WI004 \\ \hline
Priorytet: & ważne  \\ \hline

%\multicolumn{2}{c}{} \\
% \hline


\end{longtable}

\end{center}

\section{Pakiet visualization }

\subsection{Diagram}

\includegraphics[width=\linewidth]{./modelowanie/OV_UML/VisualizationClassDiagram.png}

\subsection{Opis klasy}

\begin{center}
 

\begin{longtable}{|m{3cm}|m{9cm}|} \hline

CV001 & EdgeRenderer \\ \hline
Opis: & Klasa przeciążająca metody renderowania krawędzi grafu z biblioteki prefuse. \\ \hline
Klasy nadrzędne: &  prefuse.render.EdgeRenderer   \\ \hline
Atrybuty: & %\begin{itemize}
 %\item 
%\end{itemize}
 \\ \hline
Metody: & \begin{itemize}
 \item render(Graphics2D g, VisualItem item) - metoda renderująca krawędź
\end{itemize}
  \\ \hline
Realizowane wymagania: & WF001, WF008, WI004 \\ \hline
Priorytet: & ważne  \\ \hline

\multicolumn{2}{c}{} \\
 \hline

CV002 & NodeRenderer \\ \hline
Opis: & Klasa przeciążająca metody renderowania wierzchołków grafu z biblioteki prefuse.    \\ \hline
Klasy nadrzędne: &  prefuse.render.LabelRenderer   \\ \hline
Atrybuty: & %\begin{itemize}
 %\item 
%\end{itemize}
 \\ \hline
Metody: & \begin{itemize}
 \item render (Graphics2D g, VisualItem item) - metoda renderująca wierzchołek
\item drawString(Graphics2D g, FontMetrics fm, String text,
            boolean useInt, double x, double y, double w) - metoda wypisujaca na wierzchołku String
\end{itemize}
  \\ \hline
Realizowane wymagania: & WF001, WF008, WI004 \\ \hline
Priorytet: & ważne  \\ \hline

\multicolumn{2}{c}{} \\
 \hline

CV003 & OVDisplay \\ \hline
Opis: &  Klasa tworząca obiekt JComponent do umieszczenia na okienku JAVA zawierający wygenerowany graf z wizualizacją   \\ \hline
Klasy nadrzędne: &  prefuse.Display   \\ \hline
Atrybuty: & \begin{itemize}
 \item Graph graph - obiekt typu prefuse.data.graph zawierajacy dane o grafie do wyświetlenia. 
\end{itemize}
 \\ \hline
Metody: & \begin{itemize}
 \item getGraph() - zwarca graf z wyśiwetlanymi danymi
 \item setGraph(Graph graph) - nadpisuje obecny graf podanym
 \item generateGraphFromOWl(OWLOntology ont) - wpisuje do klasy obiekt Grpah wygenrowany na podstawie ontologii
	 
\end{itemize}
  \\ \hline
Realizowane wymagania: & WF001, WF002, WF008, WI004 \\ \hline
Priorytet: & ważne  \\ \hline

\multicolumn{2}{c}{} \\
 \hline

CV004 & OVFilter \\ \hline
Opis: & Klasa zawierająca filtry służace do wyświetlania danych w różnych zakresach    \\ \hline
Klasy nadrzędne: &     \\ \hline
Atrybuty: & %\begin{itemize}
 %\item 
%\end{itemize}
 \\ \hline
Metody: & %\begin{itemize}
 %\item 
%\end{itemize}
  \\ \hline
Realizowane wymagania: & WF001, WF008, WI004 \\ \hline
Priorytet: & ważne  \\ \hline

%\multicolumn{2}{c}{} \\
% \hline


\end{longtable}

\end{center}

\section{Pakiet graph}

\subsection{Diagram}

\includegraphics[width=\linewidth]{./modelowanie/OV_UML/GraphClassDiagram.png}

\subsection{Opis klasy}

\begin{center}
 


\begin{longtable}{|m{3cm}|m{9cm}|} \hline

CG001 & GraphToOWLConverter \\ \hline
Opis: & Klasa zawierająca metody pozwalające na przetwarzanie obiektów grafów z prefuse na obiekty OWL API. Klasa jest singletonem. \\ \hline
Klasy nadrzędne: &     \\ \hline
Atrybuty: & \begin{itemize}
 \item INSTANCE - instancja klasy GraphToOWLConverter 
\end{itemize}
 \\ \hline
Metody: & \begin{itemize}
	\item getInstance() - zwraca instancję klasy
	\item GraphToOWL(OWLOntology ontology) -Zamienia graf z biblioteki prefuse na ontologię zapisana w OWL API.
\end{itemize}
  \\ \hline
Realizowane wymagania: & WD001, WI004 \\ \hline
Priorytet: & ważne  \\ \hline

\multicolumn{2}{c}{} \\
 \hline

CG002 & OWLtoGraphConverter \\ \hline
Opis: & Klasa zawierająca metody pozwalające na przetwarzanie obiektów OWL API na obiekty prefuse. Klasa jest singletonem.\\ \hline
Klasy nadrzędne: &     \\ \hline
Atrybuty: & \begin{itemize}
 \item INSTANCE - instancja klasy GraphToOWLConverter 
\end{itemize}
 \\ \hline
Metody: & \begin{itemize}
	\item getInstance() - zwraca instancję klasy
	\item recursiveSubClassReader(Node parent, OWLClass cls,OWLOntology ontology ) - wczytuje do grafu OWL wszystkie klasy wraz z ich podklasami.
 	\item OWLToGraph(OWLOntology ontology) -Zamienia ontologię w OWL API na graf z biblioteki prefuse.
\end{itemize}
  \\ \hline
Realizowane wymagania: & WD001, WI004 \\ \hline
Priorytet: & ważne  \\ \hline

%\multicolumn{2}{c}{} \\
% \hline


\end{longtable}

\end{center}



\section{Pakiet utils}

\subsection{Diagram}

\includegraphics[width=0.5\linewidth]{./modelowanie/OV_UML/UtilsDiagram.png}

\subsection{Opis klasy}

\begin{center}

\begin{tabular}{|m{3cm}|m{9cm}|} \hline

CU001 & Debug \\ \hline
Opis: & Klasa do użycia przy debugowaniu, zapewnia strumien z błędami zwracanymi przez bibliotekę. Klasa jest singletonem.\\ \hline
Klasy nadrzędne: &     \\ \hline
Atrybuty: & \begin{itemize}
 \item INSTANCE - instacja klasy Debug
 \item Debug - Strumień do którego wpisywane są informacje potrzebne do debugowania
\end{itemize}
 \\ \hline
Metody: & \begin{itemize}
 \item getInstance() - zwraca instację klasy
 \item setStream(PrintStream ps) - ustawia podany strumień jako strumień na który zwracane będa błędy
 \item sendMessage(String s) - wysyła wiadomość na strumień do debugowania, jeżeli został wcześniej podpięty za pomocą funkcji setStream 	
\end{itemize}
  \\ \hline
Realizowane wymagania: & WF006, WF007, WI004 \\ \hline
Priorytet: & bardzo ważne  \\ \hline

%\multicolumn{2}{c}{} \\
% \hline
\end{tabular}
\end{center}

%\clearpage
%\phantomsection
%\addcontentsline{toc}{section}{Literatura}
%\bibliography{biblio}

\end{document}
