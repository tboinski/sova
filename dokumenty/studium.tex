\documentclass[a4paper,10pt]{article}
\usepackage{polski}
\usepackage[utf8]{inputenc}
\usepackage{array}
%custom margins
\usepackage[left=2.5cm, top=2.5cm, bottom=3cm, right=2cm, foot=2cm, head=0.5cm]{geometry}
\usepackage{fancyhdr}
\usepackage[bookmarks=true, pdftex]{hyperref}

%styl nagłowków
\pagestyle{fancy} 
\parindent 2cm 

%opening
\title{Studium wykonalności}
\author{3@KASK}

\begin{document}
\bibliographystyle{plain}


\maketitle


\begin{center}
%budowanie tabeli
\begin{tabular}{|p{7cm}|p{7cm}|}
\hline
Symbol projektu: & Opiekun projektu:   \tabularnewline 
3@KASK & Tomasz Boiński    \tabularnewline \hline
\multicolumn{2}{|l|}{Nazwa Projektu: } \tabularnewline
\multicolumn{2}{|l|}{Wizualizacja grafów za pomocą biblioteki Prefuse } \tabularnewline 
\hline
\multicolumn{2}{l}{ } \tabularnewline %pusta linijka
\hline 
Nazwa Dokumentu: & Nr wersji:   \tabularnewline 
Studium wykonalności & 0.0 \tabularnewline \hline
Odpowiedzialny za dokument: & Data pierwszego sporządzenia:   \tabularnewline 
Anna Jaworska & 31.03.09 \tabularnewline \hline
Przeznaczenie: & Data ostatniej aktualizacji:   \tabularnewline 
WEWNĘTRZNE & 31.03.09 \tabularnewline \hline
\end{tabular}
\end{center}

\begin{center}
\begin{tabular}{|c|p{4cm}|c|c|c|}
\multicolumn{5}{c}{\textbf{Historia dokumentu}} \tabularnewline \hline
\textbf{Wersja} & \textbf{Opis modyfikacji} & \textbf{Rozdział/strona} & \textbf{Autor modyfikacji} & \textbf{Data} \tabularnewline \hline 
1 & Przygotowanie zarys dokumnetu i określenie zakresu badań & wszystkie & Anna Jaworska & 31.03.09 \tabularnewline \hline
& & & &\tabularnewline \hline
\end{tabular}
 

\end{center}


\newpage
\tableofcontents
\newpage

\section{Założenia realizacji studium}

\subsection{Podstawa wykonania i temat studium}
\paragraph{} Studium wykonywane jest przede wszystkim aby określic możliwe sposoby realizacji projektu. Ma także za zadanie zebranie i podsumowanie informacji potrzebnych zespołowi do realizacji projektu. 

\subsection{Cel studium}
\paragraph{} Celem studium jest zbadanie na potrzeby projektu \textit{Wizualizacja grafów za pomocą biblioteki Prefuse}: 
\begin{itemize}
 	\item jak należy tworzyć biblioteki w technologii JAVA
 	\item jakich mechnizmów wizualizacji grafów dostarczają biblioteki JAVA 
	\item czy realizacja projektu za pomocą Prefuse jest odpowiednim rozwiązaniem
	\item jaki standard OWL powinien być wspierany przez wytworzony produkt
\end{itemize}

\subsection{Ograniczenia}
\paragraph{} Do podstawowych ograniczeń należą:
\begin{itemize}
 	\item konieczność realizacji projektu w języku JAVA
	\item 
\end{itemize}



\section{Stan istniejący}

\subsection{Inne systemy i zasoby mające wpływ lub będące pod wypływem planowanego produktu}

\begin{itemize}
 	\item OCS
	\item OWL API 
\end{itemize}


\subsection{Istniejące na rynku podobne rozwiązania}

\begin{itemize}
 \item 
\end{itemize}

\subsection{Problem i motywacja wdrożenia nowego produktu}
\paragraph{} 


\section{Ogólne wymagania stawiane produktowi i ich priorytety}

\subsection{Architektura}

\subsection{Użytkownicy}

\subsection{Dane}

\subsection{Funkcjonalność}

\subsection{Wymogi techniczno - technologiczne}

\section{Ogólna ocena ryzyka i planowany sposób zarzadzania nim}

\section{Uwarunkowania prawne i inne}
%na jakiej to ma byc licencji - tylko dla KASK-u ?


\section{Proponowane rozwiązania}

%wybór w dizedzinach - biblioteki i Java - wersje itp, wersja OWL(RDF?), wersja standardu tworzenia biblioteki, wzorować się na cyzms istniejącym (?)

\section{Porównanie wariantów i rekomendacja}

\section{Strategia i wstępny harmonogram}


\clearpage
\phantomsection
\addcontentsline{toc}{section}{Literatura}
\bibliography{biblio}

\end{document}
