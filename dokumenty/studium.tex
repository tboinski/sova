\documentclass[a4paper,10pt]{article}
\usepackage{polski}
\usepackage[utf8]{inputenc}
\usepackage{array}
%custom margins
\usepackage[left=2.5cm, top=2.5cm, bottom=3cm, right=2cm, foot=2cm, head=0.5cm]{geometry}
\usepackage{fancyhdr}
\usepackage[bookmarks=true, pdftex]{hyperref}

%styl nagłowków
\pagestyle{fancy} 
\parindent 2cm 

%opening
\title{Studium wykonalności}
\author{3@KASK}

\begin{document}
\bibliographystyle{plain}


\maketitle


\begin{center}
%budowanie tabeli
\begin{tabular}{|p{7cm}|p{7cm}|}
\hline
Symbol projektu: & Opiekun projektu:   \tabularnewline 
3@KASK & mgr inż. Tomasz Boiński    \tabularnewline \hline
\multicolumn{2}{|l|}{Nazwa Projektu: } \tabularnewline
\multicolumn{2}{|l|}{Wizualizacja grafów za pomocą biblioteki Prefuse } \tabularnewline 
\hline
\multicolumn{2}{l}{ } \tabularnewline %pusta linijka
\hline 
Nazwa Dokumentu: & Nr wersji:   \tabularnewline 
Studium wykonalności & 0.0 \tabularnewline \hline
Odpowiedzialny za dokument: & Data pierwszego sporządzenia:   \tabularnewline 
Anna Jaworska & 31.03.09 \tabularnewline \hline
Przeznaczenie: & Data ostatniej aktualizacji:   \tabularnewline 
WEWNĘTRZNE & 31.03.09 \tabularnewline \hline
\end{tabular}
\end{center}

\begin{center}
\begin{tabular}{|c|p{4cm}|c|c|c|}
\multicolumn{5}{c}{\textbf{Historia dokumentu}} \tabularnewline \hline
\textbf{Wersja} & \textbf{Opis modyfikacji} & \textbf{Rozdział/strona} & \textbf{Autor modyfikacji} & \textbf{Data} \tabularnewline \hline 
1 & Przygotowanie zarysu dokumnetu i określenie zakresu badań & wszystkie & Anna Jaworska & 31.03.09 \tabularnewline \hline
& & & &\tabularnewline \hline
\end{tabular}
 

\end{center}


\newpage
\tableofcontents
\newpage

\section{Założenia realizacji studium}

\subsection{Podstawa wykonania i temat studium}
\paragraph{} Studium wykonywane jest przede wszystkim aby określic możliwe sposoby realizacji projektu. Ma także za zadanie zebranie i podsumowanie informacji potrzebnych zespołowi do realizacji projektu. 

\subsection{Cel studium}
\paragraph{} Celem studium jest zbadanie na potrzeby projektu \textit{Wizualizacja grafów za pomocą biblioteki Prefuse}: 
\begin{itemize}
 	\item jak należy tworzyć biblioteki w technologii JAVA
 	\item jakich mechnizmów wizualizacji grafów dostarczają biblioteki JAVA 
	\item czy realizacja projektu za pomocą Prefuse jest odpowiednim rozwiązaniem
	\item jaki standard OWL powinien być wspierany przez wytworzony produkt
\end{itemize}

\subsection{Ograniczenia}
\paragraph{} Do podstawowych ograniczeń należą:
\begin{itemize}
 	\item konieczność realizacji projektu w języku JAVA
	\item konieczność wykorzystania wersji bibliotek zgodnych z użytymi w OCS
	\item limit czasowy projektu 
\end{itemize}



\section{Stan istniejący}

\subsection{Inne systemy i zasoby mające wpływ lub będące pod wypływem planowanego produktu}

\begin{itemize}
 	\item OCS - Ontology C.. System 
	\item OWL API ver 2.1.1 - API do przetwarzania plików w formacie OWL zgodnych ze standardem W3C; ta wersja API została użyta w projekcie OCS
	\item 
\end{itemize}


\subsection{Istniejące na rynku podobne rozwiązania}

\begin{itemize}
 \item 
\end{itemize}

\subsection{Problem i motywacja wdrożenia nowego produktu}
\paragraph{} Nowa biblioteka powinna powstać aby:
\begin{itemize}
 \item ułatwić programistą wizualizację ontologii
\end{itemize}



\section{Ogólne wymagania stawiane produktowi i ich priorytety}
\paragraph{} Wymienione wymagania mają charakter orientacyjny, pozwalajacy nakreślic zakres problemu jaki ma pokrywać projekt. Szczegółową definicję wymagań zawiera dokument \textit{Specyfikacji wymagań}

\subsection{Architektura}

\subsection{Użytkownicy}
\paragraph{} Użytkownikami biblioteki będą programiści tworzący aplikacje wizualizujące ontologie. Inicjalnie będą to programiści związani z projektem OCS, później mogą to być dowolni inni programiści chętni do korzystania z biblioteki.  
\subsection{Dane}
\paragraph{} Biblioteka powinna obsługiwać te same formaty danych co OWL API (zgodne ze specyfikacją W3C):
\begin{itemize}
 	\item RDF
	\item OWL Lite
	\item OWL DL
	\item OWL Full
\end{itemize}

\paragraph{}Ponadto dane te powinny być wczytywane poprzez:
\begin{itemize}
 	\item podanie ścieżki do pliku OWL na dysku 
	\item podanie adresu sieciowego zasobu z plikiem OWL
	\item podanie strumienia/kontenera z XML 
\end{itemize}


\subsection{Funkcjonalność}
% tu mozna o OWL napisać i pare innych
\paragraph{} Zakładamy, że biblioteka będzie zawierać następujace funkcjonalności:
\begin{itemize}
 	\item wizualizacja elmentów OWL 
	\item pozwalać użytkownikowi na definiowanie akcji dla zdarzeń okna
	\item zawierać standardowe definicje zdarzeń
	\item  
\end{itemize}



\subsection{Wymogi techniczno - technologiczne}
\paragraph{} Produkt projektu, jako biblioteka w języku Java, powinien posiadać następujące cechy:
\begin{enumerate}
 \item \textbf{Odpowiednie kapsułkowanie.} Publiczne powinny być jedynie te klasy i metody, które są istotne dla użytkownika i z których będzie on bezpośrednio korzystał.
 \item \textbf{Możliwość debugowania.} Użytkownik powinien mieć możliwość debugowania kodu biblioteki, bez konieczności znajomości każdego jej szczegółu.
 \item \textbf{Przejrzystość.} Kod biblioteki powinien być odpowiednio udokumentowany za pomocą javadoc. W szczególności, bardzo dokładnie należy opisać klasy oraz metody publiczne.
 \item \textbf{Łatwość użycia.} Biblioteka powinna zawierać klasy, pokazujące przykłady wykorzystania jej klas i metod.
 \item \textbf{Rozszerzalność.} Struktura wewnętrzna biblioteki powinna być odpowiednio podzielona na klasy (wykorzystując klasy abstrakcyjne i interfejsy). Dzięki temu użytkownik będzie miał możliwość stworzenia własnych klas, rozszerzających funkcjonalność biblioteki.
 \item \textbf{Uniwersalność.} Biblioteka powinna mieć jasno określony problem, który rozwiązuje. Wyniki powinny być podane użytkownikowi w wygodny dla niego sposób (lub na kilka sposobów), który będzie umożliwiał wykorzystanie biblioteki w różnych aplikacjach. Innymi słowy, biblioteka powinna udostępniać łatwy i przejrzysty dla użytkownika interfejs.
 \item Biblioteka powinna być napisana w taki sposób, aby użytkownik spojrzał na nią i mógł powiedzieć: \textit{Wow, to jest dokładnie to, czego potrzebuję i dokładnie tak samo bym to napisał!} ;).
\end{enumerate}


\section{Ogólna ocena ryzyka i planowany sposób zarzadzania nim}


\paragraph{}Schemat opisu czynnika ryzyka
\begin{center}
\begin{tabular}{|l|p{12cm}|}
\hline
\textbf{ID czynnika} &  RISKXX \tabularnewline \hline
\textbf{Nazwa czynnika} & Nazwa \tabularnewline \hline
\textbf{Opis czynnika} & Opis... \tabularnewline \hline
\textbf{Sposób zarządzania} & Opis.. \tabularnewline \hline
\end{tabular}
\end{center}

\begin{center}
\begin{tabular}{|l|p{12cm}|}
\hline
\textbf{ID czynnika} &  RISK01 \tabularnewline \hline
\textbf{Nazwa czynnika} & Problemy logistyczne zespołu  \tabularnewline \hline
\textbf{Opis czynnika} & Uwzględniamy możliwość wystąpienia problemów osobistych członków zespołu powodujących ich wyłączenie z prac. \tabularnewline \hline
\textbf{Sposób zarządzania} & Jeśli ktoś zostanie wyłaczony z prac, reszta zespołu musi podzielić między siebie jego obowiązki i informować osobę wyłaczaną o postempach, tak aby miała wgląd w archiwa.  \tabularnewline \hline
\end{tabular}
\end{center}


\begin{center}
\begin{tabular}{|l|p{12cm}|}
\hline
\textbf{ID czynnika} &  RISK02 \tabularnewline \hline
\textbf{Nazwa czynnika} & Problemy członków zespołu na uczelni \tabularnewline \hline
\textbf{Opis czynnika} & Możliwe jest powstanie zaległości związanych z innymi uczelnianymi obowiązkami    \tabularnewline \hline
\textbf{Sposób zarządzania} & Członek zespołu musi zgłosić swoje problemy reszcie zespołu. W zależnosci od sytuacji termin wykonania jego zadań zostanie przedłużony lub zadania te przejmie ktoś inny. \tabularnewline \hline
\end{tabular}
\end{center}

\begin{center}
\begin{tabular}{|l|p{12cm}|}
\hline
\textbf{ID czynnika} &  RISK03 \tabularnewline \hline
\textbf{Nazwa czynnika} & Niedostępność opiekuna/klienta \tabularnewline \hline
\textbf{Opis czynnika} & Z różnych przyczyn niezależnych od zespołu opiekun może stać sie niedostępny.  \tabularnewline \hline
\textbf{Sposób zarządzania} & Wszelkie problemy wymagjace według zespołu poznania opinii opiekuna będą musiały zostać rozwiązanie poprzez podjęcie decyzji przez zespół bez wsparcia. Wszelkie problemy organizacyjne zwiazane z projektem grupowym powinny pod nieobecność zgłaszane do katedralnego koordynatora projektów grupowych.     \tabularnewline \hline
\end{tabular}
\end{center}


\begin{center}
\begin{tabular}{|l|p{12cm}|}
\hline
\textbf{ID czynnika} &  RISKXX \tabularnewline \hline
\textbf{Nazwa czynnika} & Nazwa \tabularnewline \hline
\textbf{Opis czynnika} & Opis... \tabularnewline \hline
\textbf{Sposób zarządzania} & Opis.. \tabularnewline \hline
\end{tabular}
\end{center}

\begin{center}
\begin{tabular}{|l|p{12cm}|}
\hline
\textbf{ID czynnika} &  RISKXX \tabularnewline \hline
\textbf{Nazwa czynnika} & Nazwa \tabularnewline \hline
\textbf{Opis czynnika} & Opis... \tabularnewline \hline
\textbf{Sposób zarządzania} & Opis.. \tabularnewline \hline
\end{tabular}
\end{center}

\begin{center}
\begin{tabular}{|l|p{12cm}|}
\hline
\textbf{ID czynnika} &  RISKXX \tabularnewline \hline
\textbf{Nazwa czynnika} & Nazwa \tabularnewline \hline
\textbf{Opis czynnika} & Opis... \tabularnewline \hline
\textbf{Sposób zarządzania} & Opis.. \tabularnewline \hline
\end{tabular}
\end{center}

\section{Uwarunkowania prawne i inne}
%na jakiej to ma byc licencji - tylko dla KASK-u ?



\section{Proponowane rozwiązania}

%wariant z DL 
%wariant z Full
%wybór w dizedzinach - biblioteki i Java - wersje itp, wersja OWL(RDF?), wersja standardu tworzenia biblioteki, wzorować się na cyzms istniejącym (?)

\section{Porównanie wariantów i rekomendacja}

\section{Strategia i wstępny harmonogram}


\clearpage
\phantomsection
\addcontentsline{toc}{section}{Literatura}
\bibliography{biblio}

\end{document}
