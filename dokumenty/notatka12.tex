\documentclass[a4paper,10pt]{article}
\usepackage{polski}
\usepackage[utf8]{inputenc}
\usepackage{array}
%custom margins
\usepackage[left=2.5cm, top=2.5cm, bottom=3cm, right=2cm, foot=2cm, head=0.5cm]{geometry}
\usepackage{fancyhdr}
\usepackage[bookmarks=true, pdftex]{hyperref}

%styl nagłowków
\pagestyle{fancy} 
\parindent 2cm 

%opening
\title{Notatka ze spotkania}
\author{3@KASK}

\begin{document}
\bibliographystyle{plain}


\maketitle


\begin{center}
%budowanie tabeli
\begin{tabular}{|p{7cm}|p{7cm}|}
\hline
Symbol projektu: & Opiekun projektu:   \tabularnewline 
3@KASK & mgr inż. Tomasz Boiński    \tabularnewline \hline
\multicolumn{2}{|l|}{Nazwa Projektu: } \tabularnewline
\multicolumn{2}{|l|}{Wizualizacja grafów za pomocą biblioteki Prefuse } \tabularnewline 
\hline
\multicolumn{2}{l}{ } \tabularnewline %pusta linijka
\hline 
Nazwa Dokumentu: & Data spotkania:   \tabularnewline 
Notatka ze spotkania & 23.05.09 \tabularnewline \hline
Odpowiedzialny za dokument: & Obecni na spotkaniu:   \tabularnewline 
Piotr Orłowski & Grupa projektowa \tabularnewline \hline
Przeznaczenie: & Data ostatniej aktualizacji:   \tabularnewline 
WEWNĘTRZNE & 23.05.09 \tabularnewline \hline
\end{tabular}
\end{center}



\section{Temat spotkania}
Diagram klas

\section{Poruszone zagadnienia}
\begin{enumerate}
\item Rozwijanie diagramu klas.
\item Dziedziczenie przez tworzone przez nas krawędzie i wierzchołki z klas prefuse.
\item Widoki, które mają być udostępnione.
\item Widok w formie drzewa - jak to zrobić? 
\end{enumerate}


\section{Podjęte ustalenia}
\begin{enumerate}
\item Kolory będą przechowywane w klasie statycznej opcji, natomiast każda klasa wierzchołka i krawędzi będzie posiadała odpowiednią funkcję get, która wywoła funkcję getColor z klasy opcji w taki sposób, że zostanie zwrócony kolor odpowiadające danej właściwości danej klasy. 
\item Udostępnimy widoki: standardowy (widoczne jest wszystko), drzewiasty, pudełkowy (map traybox?).
\item Ustalamy, że każdy node można zwinąć i rozwinąć.
\item Definiujemy klasę OVFilter - rozszerzającą klasę Filter z Prefuse; klasy dziedziczące z niej zostaną zdefiniowane później.
\item Napiszemy własną klasę layoutów - będzie ona zawierać pewne filtry, które odpowiednio sprawią, że niektóre dane nie zostaną wyświetlone (jako zbędne) natomiast pozwoli uwydatnić inne cechy.
\item W początkowym etapie skupiamy się na zaprojektowaniu, implementacji i wizualizacji podstawowych zależności między klasami.
\end{enumerate}



\clearpage
\phantomsection
\addcontentsline{toc}{section}{Literatura}
\bibliography{biblio}

\end{document}

