%język polski
\usepackage[utf8]{inputenc}	%kodowanie
\usepackage[OT4]{polski}	%polskie zasady łamania słów i przenoszenia wyrazów

%obsługa pdf
\usepackage[pdftex,usenames,dvipsnames]{color}	%obsługa kolorów
%\usepackage[pdftex,draft=false,pdfpagelabels=false,pdfstartview=FitH,pdfstartpage=1,bookmarks=false,pdfauthor={Piotr Kunowski},pdftitle={Praca Dyplomowa},pdfsubject={...},pdfkeywords={...},unicode=true]{hyperref}   %pagebackref=true pokazuje w spisie literatury numery stron, na których są odwołania do danej pozycji w bibliografi ..

%bibliografia
% \usepackage{packages/hypernat}	%łączy natbib z hyperref !!! wszystkie trzy muszą być definiowane w takiej jak teraz kolejności

%style
\usepackage{extsizes}	%wiecej rozmiarów czcionek
\usepackage[a4paper,left=3.5cm,right=2.5cm,top=2.5cm,bottom=2.5cm]{geometry}
\usepackage[sf,bf,outermarks]{titlesec}	%wygląd tytułów rozdziałów (arial bold, oddzielna linia dla nagłówków \paragraph)
\usepackage{tocloft}	%format spisu treści
\usepackage{array}	%ładniejsze tabelki
\usepackage[format=hang,labelsep=period,labelfont={bf,small},textfont=small]{caption}	%formatuje podpisy pod rysunkami i tabelami, format=hang powoduje, że kolejne linie podpisu będą wcięte aż do odległości nazwy podpisu np. "Rysunek 1."
\usepackage{floatflt}	%ładne opływanie obrazków tekstem
\usepackage{verbatim}
\usepackage{graphicx}
\usepackage{multirow}
%listingi
\usepackage{listings}
\usepackage{mathrsfs}
\usepackage[mathcal]{eucal}
%inne
\usepackage{packages/strona_tytulowa}
\usepackage{longtable}

\usepackage{indentfirst}
