

\section{Diagram klas i pakietów}
\insertimage{images/projekt/PackageDiagram.png}{Diagram pakietów}{fig:viz:package}
Na rysunku \figurename \space \ref{fig:viz:package} przedstawiono diagram pakietów i relacje zachodzące pomiędzy nimi. Wyróżniono sześć głównych
 pakietów, które zostaną opisane niżej.  Klasy zostały umieszczone w odpowiednich pakietach zgodnie z zachowaniem wzorca MVC (ang. Model View Controller).
Do nazw pakietów został dodany przedrostek "org.pg.eti.kask.sova". \\

\noindent 
{\bf Symbol pakietu :} P001 \newline
{\bf Nazwa pakietu :} options \newline
{\bf Opis :}  Pakiet zawierający klasy z polami opisującymi różne (modyfikowalne) ustawienia wizualizacji takie jak: kolory, grubość linii itp.  \newline

\noindent 
{\bf Symbol pakietu :} P002 \newline
{\bf Nazwa pakietu :} nodes \newline
{\bf Opis :} Pakiet z klasami odpowiedzialnymi za wizualizację i przechowywanie danych o wierzchołkach. \newline

\noindent 
{\bf Symbol pakietu :} P003 \newline
{\bf Nazwa pakietu :} edges \newline
{\bf Opis :} Pakiet z klasami odpowiedzialnymi za wizualizację i przechowywanie danych o krawędziach. \newline

\noindent 
{\bf Symbol pakietu :} P004 \newline
{\bf Nazwa pakietu :} visualization \newline
{\bf Opis :} Zawiera klasy obsługi wizualizacji min. klasę zwracającą display, klasy różnych trybów wizualizacji oraz klasy filtrów. \newline
\newline
\noindent 
{\bf Symbol pakietu :} P005 \newline
{\bf Nazwa pakietu :} graph \newline
{\bf Opis :} Pakiet zawiera klasy, które zawierają podstawowe operacje na danych OwlApi oraz graph. \newline
\newline
\noindent 
{\bf Symbol pakietu :} P006 \newline
{\bf Nazwa pakietu :} utils \newline
{\bf Opis :}  Pakiet zawiera klasy pomocnicze i dodatkowa narzędzia. \newline


\section{Pakiet visualization}

\insertimage{images/projekt/visualization_class.png}{Diagram klas dla pakietu wisualization}{fig:viz:pack_visualization}

Na rysunku  \figurename \space \ref{fig:viz:pack_visualization}  predstawiono diagram klas dla pakietu visualization. Klasy są odpowiedziale za obrazowanie danych. Spora ich część jest rozszerzeniem klas 
z~biblioteki Prefuse. 


% \begin{center}
 

\begin{longtable}{|m{3.5cm}|m{8.5cm}|} \hline

CV001 & EdgeRenderer \\ \hline
Klasy nadrzędne: &  prefuse.render.EdgeRenderer   \\ \hline
Opis: & Klasa przeciążająca metody renderowania krawędzi grafu z biblioteki prefuse. Umożliwia rysowanie własnych, wcześniej zaprojektowanych krawędzi. \\ \hline

\end{longtable}







\begin{longtable}{|m{3.5cm}|m{8.5cm}|} \hline

CV002 & NodeRenderer \\ \hline
Klasy nadrzędne: &  prefuse.render.LabelRenderer   \\ \hline
Opis: & Klasa przeciążająca metody renderowania wierzchołków grafu z biblioteki prefuse.  Umożliwia rysowanie własnych, 
wcześniej zdefiniowanych elementów wizualizacji (wierzchołków w grafie wizualizacji).  \\ \hline

\end{longtable}




\begin{longtable}{|m{3.5cm}|m{8.5cm}|} \hline

CV003 & OVDisplay \\ \hline
Klasy nadrzędne: &  prefuse.Display   \\ \hline
Opis: &  Klasa tworząca obiekt JComponent do umieszczenia na okienku JAVA zawierający wygenerowany graf z wizualizacją. Jest najważniejszą klasą 
z~punktu widzenia programisty wykorzystującego bibliotekę. Posiada metody pobrania wizualizacji oraz zmiany trybu wizualizacji.    \\ \hline

\end{longtable}

\begin{longtable}{|m{3.5cm}|m{8.5cm}|} \hline

CV004 & OVVisualization \\ \hline
Klasy nadrzędne: &  prefuse.Visualization   \\ \hline
Opis: &  Abstrakcyjna klasa obsługi wizualizacji rozszerzająca klasę wizualizacji biblioteki prefuse. Posiada metody ustawień wizualizacji
 oraz filtrów związane z wizualizacją ontologii. \\ \hline

\end{longtable}

\begin{longtable}{|m{3.5cm}|m{8.5cm}|} \hline

CV005 & ForceDirectedVis \\ \hline
Klasy nadrzędne: & OVVisualization (CV004)  \\ \hline
Opis: &  Klasa wizualizujące grafy w oparciu o algorytm ForceDirected.   \\ \hline

\end{longtable}

\begin{longtable}{|m{3.5cm}|m{8.5cm}|} \hline

CV006 & RadialGraphVis \\ \hline
Klasy nadrzędne: & OVVisualization (CV004) \\ \hline
Opis: &  Klasa wizualizująca graf w oparciu o algorytm RadialGraph. \\ \hline

\end{longtable}

\begin{longtable}{|m{3.5cm}|m{8.5cm}|} \hline

CV007 & OVNodeLinkTreeLayout \\ \hline
Klasy nadrzędne: & OVVisualization (CV004) \\ \hline
Opis: &  Klasa wizualizująca graf w oparciu o algorytm NodeLinkTree. Umożliwia wizualizację wywnioskowanego drzewa klas i bytów \\ \hline

\end{longtable}

\begin{longtable}{|m{3.5cm}|m{8.5cm}|} \hline

CV008 & OVItemFilter \\ \hline
Klasy nadrzędne: & prefuse.action.GroupAction \\ \hline
Opis: &  Klasa pozwalająca na odfiltrowanie niechcianych podczas wizualizacji elementów. \\ \hline

\end{longtable}

\begin{longtable}{|m{3.5cm}|m{8.5cm}|} \hline

CV009 & FilterOptions \\ \hline
Klasy nadrzędne: &   \\ \hline
Opis: &  Klasa zawierająca statyczne informacje o włączonych filtrach wizualizacji. \\ \hline


\end{longtable}

% \end{center}

\section{Pakiet graph}
\insertscaledimage{0.5}{images/projekt/graph_class.png}{Diagram klas dla pakietu graph}{fig:viz:pack_graph}

Na rysunku  \figurename \space \ref{fig:viz:pack_graph}  przedstawiono diagram klas dla pakietu graph.

\begin{center}
 


\begin{longtable}{|m{3.5cm}|m{8.5cm}|} \hline

CG001 & OWLtoGraphConverter \\ \hline
Klasy nadrzędne: &     \\ \hline
Opis: & Klasa zawierająca metody pozwalające na przetwarzanie obiektów OWL API na obiekty prefuse. Pobiera ona wszystkie elementy i~ich zależności
z~obiektu OWLAPI i konwertuje na krotki danych grafu. \\ \hline

\end{longtable}

\begin{longtable}{|m{3.5cm}|m{8.5cm}|} \hline


CG002 & OWLtoHierarchyTreeConverter \\ \hline
Klasy nadrzędne: &     \\ \hline
Opis: & Klasa zawierająca metody pozwalające na przetwarzanie obiektów OWL API na obiekty prefuse. Klasa poddaje podany obiekt OWLAPI wnioskowaniu, 
uzyskując w ten sposób drzewo klas i ich zależności.  \\ \hline

\end{longtable}

\begin{longtable}{|m{3.5cm}|m{8.5cm}|} \hline

CG003 & Constants \\ \hline
Klasy nadrzędne: &     \\ \hline
Opis: & Klasa zawierająca statyczne informacje o nazwach tabel i kolumn danych przechowywanych w kontenerach biblioteki prefuse. \\ \hline

% \multicolumn{2}{c}{} \\
% \hline

\end{longtable}

\end{center}
\newpage
\section{Pakiet options}


Na rysunku  \figurename \space \ref{fig:viz:pack_options}  przedstawiono diagram klas dla pakietu options.





\insertimage{images/projekt/options_class.png}{Diagram klas dla pakietu options}{fig:viz:pack_options}


\begin{center}

\begin{longtable}{|m{3.5cm}|m{8.5cm}|} \hline

CO001 & EdgeColors \\ \hline
Klasy nadrzędne: &     \\ \hline
Opis: & Zawiera definicje kolorów dla poszczególnych rodzajów wierzchołków.   \\ \hline

\end{longtable}

\begin{longtable}{|m{3.5cm}|m{8.5cm}|} \hline

CO002 & NodeColors \\ \hline
Klasy nadrzędne: &     \\ \hline
Opis: & Zawiera definicje kolorów dla poszczególnych rodzajów krawędzi.   \\ \hline



\end{longtable}

\begin{longtable}{|m{3.5cm}|m{8.5cm}|} \hline

CO003 & ArrowShapes \\ \hline
Klasy nadrzędne: &     \\ \hline
Opis: & Singleton przechowujący kształty grotów dla strzałek.   \\ \hline


\end{longtable}

\begin{longtable}{|m{3.5cm}|m{8.5cm}|} \hline

CO004 & NodeShapes \\ \hline
Klasy nadrzędne: &     \\ \hline
Opis: & Klasa przechowująca informacje o kształtach poszczególnych węzłów.   \\ \hline

\end{longtable}

\begin{longtable}{|m{3.5cm}|m{8.5cm}|} \hline

CO005 & NodeShapeType \\ \hline
Klasy nadrzędne: &     \\ \hline
Opis: &  Enum - rodzaje kształtów dla węzłów grafu.   \\ \hline

\end{longtable}

\end{center}


\section{Pakiet utils}


Na rysunku  \figurename \space \ref{fig:viz:pack_utils}  przedstawiono diagram klas dla pakietu utils.

\insertimage{images/projekt/utils_class.png}{Diagram klas dla pakietu utils}{fig:viz:pack_utils}

\begin{center}

\begin{longtable}{|m{3.5cm}|m{8.5cm}|} \hline

CU001 & Debug \\ \hline
Klasy nadrzędne: &     \\ \hline
Opis: & Klasa do użycia przy debugowaniu, zapewnia strumień z~błędami zwracanymi przez bibliotekę. Klasa ułatwia pracę programiście informując go
o~błędach i wykonywanych krokach wizualizacji.  Klasa jest singletonem.\\ \hline


\end{longtable}

\begin{longtable}{|m{3.5cm}|m{8.5cm}|} \hline
 
CU002 & VisualizationProperties \\ \hline
Klasy nadrzędne: &     \\ \hline
Opis: & Klasa odpowiada za wczytywanie ustawień kolorów dla węzłów oraz krawędzi z~wybranego lub domyślnego pliku właściwości. \\ \hline

\end{longtable}
\end{center}

\section{Pakiet edges}
\insertimage{images/projekt/edges_class.png}{Diagram klas dla pakietu edges}{fig:viz:pack_edges}

Na rysunku  \figurename \space \ref{fig:viz:pack_edges}  przedstawiony został diagram klas dla pakietu edges.


\begin{longtable}{|m{3.5cm}|m{8.5cm}|} \hline

CE001 & Edge \\ \hline
Klasy nadrzędne: &     \\ \hline
Opis: & Klasa reprezentująca prostą krawędź na grafie, zawiera podstawowe informacje o~jej kształcie i~kolerze. 
Jest nadklasą dla pozostałych klas krawędzi.\\ \hline


\end{longtable}

Klasy z~pakietu edges różnią się tylko tym, że każda z~nich odpowiada za wizualizację denej, wcześniej zaprojektowanej krawędzi na~grafie ontologii.
Dlatego poniżej zostaną wymienione klasy tego pakietu.

\begin{longtable}{|m{4cm}|m{8cm}|} \hline
CE001  & Edge  \\ \hline
CE002  & DisjointEdge \\ \hline
CE003  & DomainEdge \\ \hline
CE004  & EquivalentEdge \\ \hline
CE005  & EquivalentPropertyEdge \\ \hline
CE006  & FunctionaltEdge \\ \hline
CE007  & InstanceOfEdge \\ \hline
CE008  & InstancePropertyEdge \\ \hline 
CE009  & InverseOfEdge \\ \hline
CE010  & InverseOfMutualEdge \\ \hline 
CE011  & OperationEdge \\ \hline 
CE012  & PropertyEdge \\ \hline
CE013  & RangeEdge \\ \hline
CE014  & SubPropertyEdge \\ \hline
CE015  & SubClassEdge \\ \hline


\end{longtable}

 \newpage

\section{Pakiet nodes}
\insertimage{images/projekt/nodes_class.png}{Diagram klas dla pakietu nodes}{fig:viz:pack_nodes}

Na rysunku  \figurename \space \ref{fig:viz:pack_nodes}  przedstawiono diagram klas dla pakietu nodes.


\begin{longtable}{|m{3.5cm}|m{8.5cm}|} \hline

CN001 & Node \\ \hline
Klasy nadrzędne: &     \\ \hline
Opis: & Klasa abstrakcyjna, dziedziczą po niej wszystkie klasy z pakietu nodes,  zawiera podstawowe informacje o~jej kształcie i~kolorze. 
\\ \hline


\end{longtable}


Pakiet nodes zawiera najwięcej klas. Podobnie jak w pakiecie edges, klasy z pakietu nodes są podobne. Każda z niż odzwierciedla jakiś element ontologi.
 Ze względu na podobieństwo klas zostaną one tylko wymienione wraz z nadanym im identyfikatorem. 

\begin{longtable}{|m{4cm}|m{8cm}|} \hline
CN001  & Node  \\ \hline
CN002  & AllValuesFromPropertyNode \\ \hline
CN003  & AnonymousClassNode \\ \hline
CN004  & CardinalityNode \\ \hline
CN005  & CardinalityValueNode \\ \hline
CN006  & ClassNode \\ \hline
CN007  & ComplementOfNode \\ \hline
CN008  & DataTypeNode \\ \hline
CN009  & DifferentNode \\ \hline
CN010  & FunctionalPropertyNode \\ \hline
CN011  & IndividualNode \\ \hline
CN012  & InformationNode \\ \hline
CN013  & IntersectionOfNode \\ \hline 
CN014  & InverseFunciotnalPropertyNode \\ \hline 
CN015  & MaxCardinalityValueNode \\ \hline 
CN016  & MinCardinalityValueNode \\ \hline 
CN017  & NothingNode \\ \hline 
CN018  & OneOfNode \\ \hline 
CN019  & PropertyNode \\ \hline 
CN020  & SameAsNode \\ \hline 
CN021  & SomeValuesFromPropertyNode \\ \hline 
CN022  & SymmetricPropertNode \\ \hline 
CN023  & ThingNode \\ \hline 
CN024  & TreansitivePropertyNode \\ \hline 
CN025  & UnionOfNode \\ \hline 




\end{longtable}