\chapter{Testy aplikacji}
\vspace*{1 cm}
\section{Wstęp}
W ostatnim etapie pracy zostaną przeprowadzone testy aplikacji. Testy zostaną przeprowadzone dla następujących cech:
\begin{enumerate}
 \item Funkcjonalność/Functionality – dopasowanie systemu do potrzeb funkcjonalnych, stopień pokrycia wymaganych funkcji, łatwość orientowania się w~sposobie działania 
systemu, łatwość sprawdzenia poprawności działania systemu,
 \item Wydajność/Performance – zbiór cech związanych z osiągami systemu (szybkość działania systemu, szybkość komunikacji z~użytkownikiem, odporność systemu 
na zmiany środowiska),
 \item Wiarygodność/Dependability - stopień zaufania do systemu, niezawodność, stopień tolerancji błędów, bezpieczeństwo, stopień kontroli dostępu do systemu,
 zdolność do wykrywania i~identyfikacji błędów w systemie,
 \item Użyteczność/Usability - wysiłek, który musi być włożony w~nauczenie się programu, w~jego użycie, przygotowanie danych wejściowych i~interpretację danych
 wyjściowych.
 
\end{enumerate}

\section{Testy funkcjonalne}
Testy funkcjonalne pozwalają na sprawdzenie zgodności oprogramowania z~dokumentem specyfikacji wymagań. Sprawdzają realizację poszczególnych funkcji oprogramowania. 
Ten typ testów jest testem czarnej skrzynki. Osoba testująca nie analizuje kodu, czy architektury oprogramowania, interesują ją tylko dostarczona funkcjonalność  
testowanej aplikacji. 
\par 
Poniżej przedstawiono testy funkcjonalne biblioteki SOVA. Testy zostały podzielone na dwie kategorie (zależne do użytkownika systemu): wymagania funkcjonalne
 związane z wizualizacją oraz wymagania funkcjonalne stawiana przez programistę używającego biblioteki. Dla każdej kategorii stworzono zestaw przypadków 
testowych. W tabeli \ref{t:funk:api} oraz \ref{t:funk:wiz} zamieszczono tematykę każdego z przypadków i wyniki lub odpowiedzi na pytania zawarte w przypadkach testowych. 

\subsection{Testy funkcjonalne API dla programisty}

Przypadki testowe tej części dotyczą funkcji wymaganych przez programistów implementujących bibliotekę SOVA w swoich rozwiązaniach. Zostały ona przeprowadzone 
podczas implementacji pluginu do \proteges oraz wdrożenia do systemu OCS. 

\begin{longtable}{|m{7cm}|m{7cm}|} 
\caption{Przeprowadzone testy API programistycznego}
\label{t:funk:api} \\
\hline
\bf{Opis testu} 	&  \bf{Wynik} \\ \hline

Pełna wizualizacja ontologii. Test polega na napisaniu kodu pozwalającego na wizualizację obiektu OWL.
& 
Test zakończony powodzeniem. Napisany krótki kod pozwolił na obrazowanie zadanego obiektu OWLOntology. 
\\ \hline


Wizualizacja wywnioskowanej hierarchii klas i bytów. Test polega na napisaniu kodu pozwalającego na wizualizację taksonomii.
& 
Test zakończony powodzeniem. Napisany krótki kod pozwolił obrazować ontologię w oczekiwany sposób. 
\\ \hline

Intuicyjne API biblioteki pozwalające na szybką wizualizację ontologii.

& Dzięki zachowaniu norm nazewnictwa w języku Java, API biblioteki jest intuicyjne. Aby dokonać wizualizacji ontologi w tworzonym przez programistę środowisku, należy
stworzyć tylko obiekt OVDisplay i wywołać na nim metodę generateGraphFromOWL lub generateTreeFromOWL dla wywnioskowanej hierarchii. 
\\ \hline

Zapis informacji z strumienia błędów do pliku.
& Test zakończony powodzeniem. Wszystkie komunikaty debugiera pojawiły się w pliku. 
\\ \hline


Zmiana koloru wizualizowanych elementów poprzez użycie pliku properties  
& Test zakończony powodzeniem. Zostały wczytane wszystkie kolory zawarte w pliku properties. 
\\ \hline



\end{longtable}

\subsection{Testy funkcjonalne związane z wizualizacją}

Testy związane z wizualizacją zostały przeprowadzone przy użyciu edytora \proteges 4.0.1 oraz napisanego pluginu do wizualizacji SOVA. Przeprowadzone 
testy pozwoliły sprawdzić poprawności funkcji wizualizujących biblioteki SOVA zawartych w specyfikacji wymagań funkcjonalnych. 


\begin{longtable}{|m{7cm}|m{7cm}|} 
\caption{Przeprowadzone testy funkcjonalne}
\label{t:funk:wiz} \\
\hline
\bf{Opis testu} 	&  \bf{Wynik} \\ \hline

Zmiana trybu wizualizacji na RadialTree. 
& Test zakończony powodzeniem 
\\ \hline

Ustawienie odległości od zaznaczonego wierzchołka na 1. 
& Test zakończony powodzeniem. Zostały wyświetlone elementy oddalone o 1 od zaznaczonego.
\\ \hline

Wyłączenie wizualizacji klas i elementów bezpośrednio powiązanych z klasami. 
& Test zakończony powodzeniem. Klasy i związki pomiędzy nimi nie pojawiły się na wizualizacji.
\\ \hline


Wyłączenie wizualizacji bytów i elementów bezpośrednio powiązanych z nimi. 
& Test zakończony powodzeniem. Byty i związki pomiędzy nimi nie pojawiły się na wizualizacji.
\\ \hline


Wyłączenie wizualizacji właściwości i elementów bezpośrednio powiązanych z właściwościami. 
& Test zakończony powodzeniem. Właściwości i związki pomiędzy nimi nie pojawiły się na wizualizacji.
\\ \hline

\end{longtable}
\section{Testy wydajnościowe}
Wizualizacja ontologii jest skomplikowanym procesem. Algorytm przedstawiony w poprzednim rozdziale przegląda ontologię kilkukrotnie, wyszukując elementy 
zapisane w~aksjomatach. Następnie elementy te są  rozmieszczane w strukturach danych, filtrowane i przekazywane algorytmowi rozmieszczenia danych. Algorytm 
RadialTreeLayout, rozmieszczający dane, posiada złożoność obliczeniową $O(N\log N)$ oraz $O(E) $, gdzie N to liczba wierzchołków, E to liczba krawędzi. Złożoność obliczeniowa całego procesu
 wizualizacji może być duża. Dlatego należy wykonać testy wydajnościowe, aby upewnić się, że aplikacja jest skalowalna i~potrafi wizualizować duże ontologie w~przystępnym 
czasie. 

Testy zostały przeprowadzone dla 6 ontologii posiadających różne liczby elementów. Ontologie te zostały opisane w tabeli \ref{t:ontologie} . Tabela ta przedstawia 
nazwę ontologii, 
objętość pliku owl, z którego została wczytana. Liczbę wizualizowanych elementów ontologi, czyli liczbę wierzchołków i~krawędzi użytą podczas obrazowania ontologii przy 
użyciu zaprojektowanych elementów graficznych. Ostatnia kolumna tabeli przedstawia czas wizualizowania ontologii, potrzebny na przetworzenie zadanego obiektu
 wejściowego OWLOntology na odpowiednie struktury danych, nadanie wartości koloru i kształtu  elementom, rozmieszczenie ich algorytmem RadialTree i~narysowanie grafu na 
obiekcie display. Czas wczytania ontologii nie jest liczony, ponieważ biblioteka jako dane wejściowe otrzymuje wczytany już obiekt owl. Obiekt ten może pochodzić nie tylko z~pliku, ale 
np. z~bazy danych. 


\begin{longtable}{|m{4cm}|m{3cm}|m{3cm}|m{3cm}|} 
\caption{Pliki ontologii użyte podczas testowania}
\label{t:ontologie} \\
\hline
\bf{Nazwa ontologii} 	&  \bf{Rozmiar pliku [KB]} & \bf{Liczba \newline elementów} & \bf{Czas [ms]} \\ \hline

Pizza2.owl & 7,24 & 122 & 170\\ \hline
Pizza.owl & 121 & 1 398 & 203 \\ \hline
securityontology.owl & 517 & 4 883 & 353 \\ \hline
CL.owl & 1 171 & 9 093 & 900 \\ \hline
FBsp.owl& 3 159 & 23 917 & 1 947 \\ \hline
interpro2go.owl& 9 121 & 55 640 &  6 013\\ \hline
\end{longtable}

Na wykresie (Rys. \ref{fig:viz:czas}) został przedstawiony czas wizualizacji ontologi w zależności od liczby wizualizowanych elementów. Z wykresu możemy odczytać, że 
czas ten zależy liniowo od liczby wizualizowanych elementów. Liczba wizualizowanych elementów nie jest równa liczbie elementów zawartych w pliku ontologii, ponieważ niektóre
aksjomaty obrazowane są za pomocą kilku elementów graficznych. Np. związek sameAs łączy dwa byty, używając trzech elementów wizualizacji: dwóch linii i jednego wierzchołka. 
Należy jednak zauważyć, że liczba elementów potrzebna do zobrazowanie danego aksjomatu jest stała, więc liczba elementów i~liczna wizualizowanych elementów zależą
 od siebie w~sposób stały. Powyższe analiza udowadnia, że czas potrzebny na wizualizację zależy liniowo od liczby elementów zawartych w ontologii. 
\insertimage{images/wykresy/wyk_czas.png}{Czas wizualizacji ontologi}{fig:viz:czas}

\section{Testy wiarygodności}
Ten rodzaj testów pozwolił sprawdzić kompletność i poprawność wizualizacji.  Testowanie pozwoliło wykryć  czy wszystkie elementy ontologi są poprawnie wizualizowane. 
\par 
Podczas testowania zostały wczytane ontologie o niewielkiej liczbie elementów. Dla każdej z nich została sprawdzona poprawność wizualizacje poprzez porównanie obrazu 
ontologii z jej zapisem w pliku *.owl. Poniżej opisane zostały testy oraz przedstawiono ich wyniki.

\subsection*{Test 1 - ontologia win}
W teście została użyta mała ontologia posiadająca kilka klas i bytów. Między klasami występowały relacje m.in. subClass, disjointClass oraz oneOf. Ontologia została 
poprawnie zobrazowana. Żaden z elementów nie zostało pominięty, nie pojawiła się tez żadna informacja w logu. Wynik testu jest pozytywny. 

\subsection*{Test 2 - zmodyfikowana ontologia pizzy}
Ontologia pizzy posiada większość elementów zdefiniowanych w języku OWL DL: klasy, właściwości, byty, klasy anonimowe, kardynalność oraz rożnego rodzaju relacje i związki. 
Ontologia zostało poprawnie zobrazowana, wszystkie elementy pojawiły się na ekranie. W logu pojawiła się informacja o pominięciu kilku adnotacji dotyczących aksjomatów. 
Adnotacje te nie mają wpływu na wygląd wizualizacji, są tylko dodatkowymi informacjami. Wynik testu został uznany za pozytywny. 

\subsection*{Test 3 - ontologia bezpieczeństwa}
Ontologia bezpieczeństwa posiada wszystkie elementy języka OWL DL. Wizualizacja tej ontologii nie zawiera, zdefiniowanych w pliku *.owl, elementów DataType. Informacja 
ta pojawiła się w logu. Ponieważ elementy DataType nie są bardzo istotne przy wizualizacji, a w logu pojawiła się odpowiednia informacja, wynik testu jest pozytywny. 



\section{Testy użyteczności}
Testy użyteczności zostały przeprowadzone z pomocą dwóch użytkowników posiadających ogólną wiedzę o ontologiach i znających  język OWL. 
Do testów został użyty edytor OCS z opcją wizualizacji SOVA. Każdy użytkownik miał już styczność z tą aplikacją i potrafił ją obsługiwać. Podczas testu testerzy 
dostali arkusz z pytaniami, które pozwalają ocenić intuicyjność interfejsu oraz jakość wizualizacji dla dwóch, różniących się rozmiarem ontologii. Wyniki testu oraz
pytanie skierowane do testerów zostały zaprezentowane w tabeli \ref{t:testy:uzytkownika}.  



\begin{longtable}{|m{9cm}|m{2cm}|m{2cm}|} 
\caption{Wyniki testów użyteczności}
\label{t:testy:uzytkownika} \\
\hline

Pytanie  & Tester 1 &  Tester 2 \\ \hline
Oceń intuicyjność interfejsu pełnej wizualizacji w skali 0-5. &  4 & 5 \\ \hline
Oceń intuicyjność interfejsy wywnioskowanej hierarchii w skali 0-5.& 5  & 5  \\ \hline
Oceń intuicyjność opcji filtrowania w skali 0-5.& 4,5 & 4 \\ \hline
Oceń dobór kolorów w skali 0-5.& 4 &  4 \\ \hline
Oceń czytelność i przejrzystość pełnej wizualizacji dla ontologii pizza2.owl w skali 0-5.& 4 & 5 \\ \hline
Oceń czytelność i przejrzystość pełnej wizualizacji dla ontologii pizza.owl w skali 0-5.& 3 & 3 \\ \hline
Oceń czytelność wizualizacji wywnioskowanej hierarchii dla ontologii pizza.owl w skali 0-5.& 5 & 5 \\ \hline

\end{longtable}

\section{Podsumowanie testów}
Testy pokazały, że w aplikacji poprawnie zostały zaimplementowane podstawowe funkcje. Bardzo dobrze wypadły testy wydajnościowe, podczas których udowodniono liniową 
złożoność obliczeniową względem liczby elementów ontologii. Przychylną opinię uzyskał interfejs aplikacji, który został pozytywnie oceniony przez użytkowników. 
Martwić może fakt nieobrazowania wszystkich elementów ontologii. Testy wykazały, iż element DataType nie jest wizualizowany, czyli nie zostało spełnione jedno z wymagań
odnoszące się do kompletności wizualizacji. Na szczęście informacja o braku wizualizacji tego elementu pojawiła się w logu. Niską notę uzyskała również czytelność 
i~przejrzystość wizualizacji dużych ontologii. 
 


