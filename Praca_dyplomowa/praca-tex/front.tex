% *************** Strona tytułowa ***************
%\pagestyle{empty}
\sffamily

\begin{figure}
\begin{center}
\begin{tabular}[t]{l c r}
	\multirow{4}{*}{
		\ifpdf
			\includegraphics[width=0.22\textwidth]{images/logo_pg.png}
		\fi
	}
	& \Large\textbf{Politechnika Gdańska} &
	\multirow{4}{*}{
		\ifpdf
			\includegraphics[width=0.16\textwidth]{images//logo_eti.png}
		\fi
	} \\
&  & \\
& \textsc {\bf WYDZIAŁ ELEKTRONIKI,} & \\ 
& \textsc {\bf TELEKOMUNIKACJI I INFORMATYKI} & \\ 
\end{tabular}
\end{center}
\end{figure}

\begin{center}
\begin{tabular}[t]{r p{12cm}}
& \\
				& \\
Katedra:			&  Katedra Architektury Systemów Komputerowych \\
				& \\
Imię i nazwisko dyplomanta:	&  \textbf{Piotr Kunowski} \\
				& \\
Nr albumu:			&  106345 \\
				& \\
Forma i poziom studiów:		&  Stacjonarne jednolite studia magisterskie \\
				& \\
Kierunek studiów:		&  Informatyka, \\
				&  Aplikacje rozproszone i systemy internetowe \\
				& \\
				& \\
\multicolumn{2}{c}{\huge \textbf{Praca dyplomowa magisterska}} \\
& \\
& \\
Temat pracy:			&  \large \textbf{Wizualizacja ontologii zapisanych w języku OWL} \\
& \\
Kierujący pracą:		&  dr inż. Piotr Szpryngier \\
Konsultant pracy:		&  mgr inż. Tomasz Boiński \\
& \\
Zakres pracy:			&   \\
& \\
 \multicolumn{2}{c}{

\begin{right}
\begin{tabular}{p{12cm}} \\
{~
\space
\space
\space
W niniejszej pracy dyplomowej skupiono się na stworzeniu rozwiązania pozwalającego wizualizować ontologie. 
\par
~
\space
\space
\space
W części teoretycznej pracy przedstawiono języki i sposoby zapisu ontologii oraz wyjaśniono zagadnienia związane z sieciami semantycznymi. 
\par
~
\space
\space
\space
W części praktycznej pracy porównane zostały istniejące rozwiązania pozwalające na wizualizację ontologii. W oparciu o specyfikację wymagań stworzono projekt 
i~zaimplementowano własną bibliotekę wizualizującą ontologie. Biblioteka zastała wdrożona do edytora OCS oraz \proteges i~w~tych 
rozwiązaniach została przetestowana.

 }
\\

 \end{tabular}
\end{right}


} \\ 
 
& \\
\end{tabular}
\end{center}

\begin{center}
\small Gdańsk, 2010
\end{center}


% *************** Spis treœci ***************
% \pagenumbering{roman}
% \pagestyle{headings}
% \tableofcontents

% *************** Dedykacja ***************
%\vspace*{\fill}
%{\hfill\sffamily\itshape Pracę tš mojemu laptopowi poœwięcam ;)... }
%\clearpage
\linespread{1.25}  
\rmfamily
%\normalfont


