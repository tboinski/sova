\chapter{Projekt systemu}
\section{Wstęp}
W rozdziale tym zostaną zaprojektowane graficzne reprezentacje elementów ontologii oraz sposoby wizualizacji związków pomiędzy nimi. Każdy z elementów 
oraz każdy związek posiadał będzie jednoznacznie określający go symbol graficzny. 
\par 
Następnie zostanie dokonana analiza obiektowa. Zostaną zaprezentowane klasy wymagane do wizualizacji oraz ich przynależność do pakietów.   
\section{Nazwa i logo tworzonej biblioteki}

Po chwili burzliwych przemyśleń postanowiono, iż biblioteka będzie posiadać nazwę: SOVA, z ang. Simple Ontology Visualization API (co można przetłumaczyć na polski
jako: proste API do wizualizacji ontologii). Fonetyczna wymowa nazwy biblioteki oznacza ptaka - sowę. Sowa w języku angielskim to owl, co oczywiście kojarzy się 
z~językiem OWL i~tu koło znaczeń się zamyka. 
\par Jako logo biblioteki wybrano właśnie sowę. Logo zostało zaprezentowane na rysunku \ref{fig:sova}. Rysunek został pobrany z portalu Clker.com [www.clker.com], gdzie został 
upubliczniony na licencji: creative common public domain license. Licencja umożliwia darmowe kopiowanie, modyfikowanie i upublicznianie pobranej grafiki.

  \insertscaledimage{0.3}{images/SOVA.png}{Logo tworzonej biblioteki}{fig:sova}

\section{Projekt wizualizacji ontologii}

Specyfikacja języka OWL pozwala na dużą dowolność definiowania ograniczeń czy opisów elementów ontologii. Dlatego wizualizacja powinna 
odzwierciedlać elementy ontologii w sposób opisany przez autora w~języku OWL. Ważne jest, aby graficzna reprezentacja ontologii 
była jednoznaczna i~zarazem łatwa do zrozumienia. Aby spełnić te kryteria należy, zdefiniować sposób obrazowania każdego z~elementów języka OWL~DL.
Źle zaprojektowana wizualizacja może być przyczyną niepowodzenia projektu. 
\par Najważniejszymi elementami ontologii są klasy, właściwości, typy danych oraz bytu. Trzy pierwsze elementy biorą udział w podobnych związkach, dlatego
 będą obrazowane w postaci zaokrąglonego prostokąta o~kolorze zależnym od rodzaju elementu. Reprezentacją bytów, aby wyróżnić je od pozostałych 
elementów, będzie prostokąt. Byty będę posiadały też inny kolor niż pozostałe elementy. (\figurename \space \ref{fig:viz:projekt1}).

\insertimage{images/projekt_wiz/wiz_elementy_1.png}{Symbole reprezentujące klasę (a), właściwość (b), typ danych (c) oraz byt (d)}{fig:viz:projekt1}

\par
Największym wyzwaniem wizualizacji jest zrozumiałe przedstawienie klas anonimowych. Zaproponowano symbole przedstawiające złożone relacje, 
w postaci kółek z wpisanymi w nie znaczącymi symbolami. Przykładowe symbole zaprezentowano na \figurename \space \ref{fig:viz:projekt2}. Wykorzystano symbolikę matematyczną  
w~przypadku intersekcji, komplementarności, unii oraz kardynalności. Matematyczną symbolikę otrzymały także relacje , zachodzące zgodnie z~OWL~DL, 
pomiędzy individuals  - „sameAs” i~„allDifferent”, przy czym symbol relacji „sameAs” jest  nadmiarowy, aby zachować spójność wizualizacji
 (relacja ta jest przeciwna do differentFrom/allDifferent). Kardynalność również reprezentowana jest za pomocą anonimowego wierzchołka, 
przy czym dodatkowy wierzchołek z~ograniczeniem liczby jest wyróżniony kolorem w~zależności od typu ograniczenia (min, max, equal)

\insertimage{images/projekt_wiz/wiz_elementy_2.png}{Przykładowe symbole reprezentujące klasę anonimową (a), intersekcję (b), kardynalność typu max i min (c), 
relację sameAss (d) oraz relację allDifferent (e)}{fig:viz:projekt2}
\par


Zaprezentowanie relacji „allValuesFrom” i~ „someValuesFrom”  odbywa się poprzez wprowadzenie klasy anonimowej reprezentującej wynik podanego
 ograniczenia (\figurename \space \ref{fig:viz:projekt3}). Jest ona połączona z~symbolem właściwości, przy czym nazwa właściwości, występująca w~aksjomacie, 
 poprzedzona jest kwantyfikatorem
 ogólnym dla  „allValuesFrom” oraz szczególnym dla „someValuesFrom”. Następnie strzałka wskazuje klasę (warto zwrócić uwagę, że może to być także 
dowolnie złożona klasa anonimowa), określającą przeciwdziedzinę przedstawionej relacji. Dodatkowo użyty wierzchołek property połączony jest 
z~wierzchołkiem przedstawiającym jego definicję.

\insertimage{images/projekt_wiz/wiz_elementy_3.png}{Reprezentacja relacji someValuesFrom oraz allValuesFrom}{fig:viz:projekt3}

Różne proste relacje reprezentowane są przez strzałki o~różnych grotach. Relację „SubClass” i~„SubProperty” prezentuje zaczerpnięta ze specyfikacji 
UML strzałka o~pustym grocie.  Krawędzie reprezentujące związki „equivalent” i~„disjoint” mają odwrotne groty, które podkreślają odwrotność tych relacji.

\insertimage{images/projekt_wiz/wiz_elementy_4.png}{ Przykładowe symbole reprezentujące relacje proste: rdfs:subclassOf (a), instanceOf (b), owl:equivalentClass (c), owl:disjointWith (d), rdfs:domain (e) oraz rdfs:range (f)}{fig:viz:projekt4}

 W~przypadku definicji właściwości (Property), inwersja jest wyróżniona kolorem czerwonym i~oznaczana dwojako, ze względu na asymetryczność tej relacji. Z~kolei 
równoważność właściwości odróżniona jest od równoważności klas kolorem. Zastosowane zostały dodatkowe rodzaje grotów, umożliwiające łatwe rozróżnienie 
range i~domain dla property. Do oznaczenia konkretnych property wykorzystane zostały tradycyjne strzałki, które wskazują kierunek czytania  aksjomatu.
 Aksjomaty, które tworzą kolejne klasy anonimowe, są czytelne dzięki specjalnemu grotowi, który (podobnie jak przy właściwościach) był konieczny, aby wskazać 
kierunek ich czytania. Tam, gdzie nie jest to potrzebne, krawędzie nie posiadają grotów, co zwiększa czytelność wizualizacji. Przykładowe relacje 
zaprezentowano na \figurename \space \ref{fig:viz:projekt4}.



\section{Diagram klas i pakietów}
\insertimage{images/projekt/PackageDiagram.png}{Diagram pakietów}{fig:viz:package}
Na rysunku \figurename \space \ref{fig:viz:package} przedstawiono diagram pakietów i relacje zachodzące pomiędzy nimi. Wyróżniono sześć głównych
 pakietów, które zostaną opisane niżej.  Klasy zostały umieszczone w odpowiednich pakietach zgodnie z zachowaniem wzorca MVC (ang. Model View Controller).
Do nazw pakietów został dodany przedrostek "org.pg.eti.kask.sova". \\

\noindent 
{\bf Symbol pakietu :} P001 \newline
{\bf Nazwa pakietu :} options \newline
{\bf Opis :}  Pakiet zawierający klasy z polami opisującymi różne (modyfikowalne) ustawienia wizualizacji takie jak: kolory, grubość linii itp.  \newline

\noindent 
{\bf Symbol pakietu :} P002 \newline
{\bf Nazwa pakietu :} nodes \newline
{\bf Opis :} Pakiet z klasami odpowiedzialnymi za wizualizację i przechowywanie danych o wierzchołkach. \newline

\noindent 
{\bf Symbol pakietu :} P003 \newline
{\bf Nazwa pakietu :} edges \newline
{\bf Opis :} Pakiet z klasami odpowiedzialnymi za wizualizację i przechowywanie danych o krawędziach. \newline

\noindent 
{\bf Symbol pakietu :} P004 \newline
{\bf Nazwa pakietu :} visualization \newline
{\bf Opis :} Zawiera klasy obsługi wizualizacji min. klasę zwracającą display, klasy różnych trybów wizualizacji oraz klasy filtrów. \newline
\newline
\noindent 
{\bf Symbol pakietu :} P005 \newline
{\bf Nazwa pakietu :} graph \newline
{\bf Opis :} Pakiet zawiera klasy, które zawierają podstawowe operacje na danych OwlApi oraz graph. \newline
\newline
\noindent 
{\bf Symbol pakietu :} P006 \newline
{\bf Nazwa pakietu :} utils \newline
{\bf Opis :}  Pakiet zawiera klasy pomocnicze i dodatkowa narzędzia. \newline


\subsection*{Pakiet visualization}

\insertimage{images/projekt/visualization_class.png}{Diagram klas dla pakietu wisualization}{fig:viz:pack_visualization}

Na rysunku  \figurename \space \ref{fig:viz:pack_visualization}  predstawiono diagram klas dla pakietu visualization. Klasy są odpowiedziale za obrazowanie danych. Spora ich część jest rozszerzeniem klas 
z~biblioteki Prefuse. 


% \begin{center}
 

\begin{longtable}{|m{3.5cm}|m{8.5cm}|} \hline

CV001 & EdgeRenderer \\ \hline
Klasy nadrzędne: &  prefuse.render.EdgeRenderer   \\ \hline
Opis: & Klasa przeciążająca metody renderowania krawędzi grafu z biblioteki prefuse. Umożliwia rysowanie własnych, wcześniej zaprojektowanych krawędzi. \\ \hline

\end{longtable}


\newpage   %<-------------------------




\begin{longtable}{|m{3.5cm}|m{8.5cm}|} \hline

CV002 & NodeRenderer \\ \hline
Klasy nadrzędne: &  prefuse.render.LabelRenderer   \\ \hline
Opis: & Klasa przeciążająca metody renderowania wierzchołków grafu z biblioteki prefuse.  Umożliwia rysowanie własnych, 
wcześniej zdefiniowanych elementów wizualizacji (wierzchołków w grafie wizualizacji).  \\ \hline

\end{longtable}

\newpage %<-------------------------


\begin{longtable}{|m{3.5cm}|m{8.5cm}|} \hline

CV003 & OVDisplay \\ \hline
Klasy nadrzędne: &  prefuse.Display   \\ \hline
Opis: &  Klasa tworząca obiekt JComponent do umieszczenia na okienku JAVA zawierający wygenerowany graf z wizualizacją. Jest najważniejszą klasą 
z~punktu widzenia programisty wykorzystującego bibliotekę. Posiada metody pobrania wizualizacji oraz zmiany trybu wizualizacji.    \\ \hline

\end{longtable}

\begin{longtable}{|m{3.5cm}|m{8.5cm}|} \hline

CV004 & OVVisualization \\ \hline
Klasy nadrzędne: &  prefuse.Visualization   \\ \hline
Opis: &  Abstrakcyjna klasa obsługi wizualizacji rozszerzająca klasę wizualizacji biblioteki prefuse. Posiada metody ustawień wizualizacji
 oraz filtrów związane z wizualizacją ontologii. \\ \hline

\end{longtable}

\begin{longtable}{|m{3.5cm}|m{8.5cm}|} \hline

CV005 & ForceDirectedVis \\ \hline
Klasy nadrzędne: & OVVisualization (CV004)  \\ \hline
Opis: &  Klasa wizualizujące grafy w oparciu o algorytm ForceDirected.   \\ \hline

\end{longtable}

\begin{longtable}{|m{3.5cm}|m{8.5cm}|} \hline

CV006 & RadialGraphVis \\ \hline
Klasy nadrzędne: & OVVisualization (CV004) \\ \hline
Opis: &  Klasa wizualizująca graf w oparciu o algorytm RadialGraph. \\ \hline

\end{longtable}

\begin{longtable}{|m{3.5cm}|m{8.5cm}|} \hline

CV007 & OVNodeLinkTreeLayout \\ \hline
Klasy nadrzędne: & OVVisualization (CV004) \\ \hline
Opis: &  Klasa wizualizująca graf w oparciu o algorytm NodeLinkTree. Umożliwia wizualizację wywnioskowanego drzewa klas i bytów \\ \hline

\end{longtable}

\begin{longtable}{|m{3.5cm}|m{8.5cm}|} \hline

CV008 & OVItemFilter \\ \hline
Klasy nadrzędne: & prefuse.action.GroupAction \\ \hline
Opis: &  Klasa pozwalająca na odfiltrowanie niechcianych podczas wizualizacji elementów. \\ \hline

\end{longtable}

\begin{longtable}{|m{3.5cm}|m{8.5cm}|} \hline

CV009 & FilterOptions \\ \hline
Klasy nadrzędne: &   \\ \hline
Opis: &  Klasa zawierająca statyczne informacje o włączonych filtrach wizualizacji. \\ \hline


\end{longtable}

% \end{center}

\subsection*{Pakiet graph}
\insertscaledimage{0.5}{images/projekt/graph_class.png}{Diagram klas dla pakietu graph}{fig:viz:pack_graph}

Na rysunku  \figurename \space \ref{fig:viz:pack_graph}  przedstawiono diagram klas dla pakietu graph.

\begin{center}
 


\begin{longtable}{|m{3.5cm}|m{8.5cm}|} \hline

CG001 & OWLtoGraphConverter \\ \hline
Klasy nadrzędne: &     \\ \hline
Opis: & Klasa zawierająca metody pozwalające na przetwarzanie obiektów OWL API na obiekty prefuse. Pobiera ona wszystkie elementy i~ich zależności
z~obiektu OWLAPI i konwertuje na krotki danych grafu. \\ \hline

\end{longtable}

\begin{longtable}{|m{3.5cm}|m{8.5cm}|} \hline


CG002 & OWLtoHierarchyTreeConverter \\ \hline
Klasy nadrzędne: &     \\ \hline
Opis: & Klasa zawierająca metody pozwalające na przetwarzanie obiektów OWL API na obiekty prefuse. Klasa poddaje podany obiekt OWLAPI wnioskowaniu, 
uzyskując w ten sposób drzewo klas i ich zależności.  \\ \hline

\end{longtable}

\begin{longtable}{|m{3.5cm}|m{8.5cm}|} \hline

CG003 & Constants \\ \hline
Klasy nadrzędne: &     \\ \hline
Opis: & Klasa zawierająca statyczne informacje o nazwach tabel i kolumn danych przechowywanych w kontenerach biblioteki prefuse. \\ \hline

% \multicolumn{2}{c}{} \\
% \hline

\end{longtable}

\end{center}
\newpage
\subsection*{Pakiet options}


Na rysunku  \figurename \space \ref{fig:viz:pack_options}  przedstawiono diagram klas dla pakietu options.





\insertimage{images/projekt/options_class.png}{Diagram klas dla pakietu options}{fig:viz:pack_options}


\begin{center}

\begin{longtable}{|m{3.5cm}|m{8.5cm}|} \hline

CO001 & EdgeColors \\ \hline
Klasy nadrzędne: &     \\ \hline
Opis: & Zawiera definicje kolorów dla poszczególnych rodzajów wierzchołków.   \\ \hline

\end{longtable}

\begin{longtable}{|m{3.5cm}|m{8.5cm}|} \hline

CO002 & NodeColors \\ \hline
Klasy nadrzędne: &     \\ \hline
Opis: & Zawiera definicje kolorów dla poszczególnych rodzajów krawędzi.   \\ \hline



\end{longtable}

\begin{longtable}{|m{3.5cm}|m{8.5cm}|} \hline

CO003 & ArrowShapes \\ \hline
Klasy nadrzędne: &     \\ \hline
Opis: & Singleton przechowujący kształty grotów dla strzałek.   \\ \hline


\end{longtable}

\begin{longtable}{|m{3.5cm}|m{8.5cm}|} \hline

CO004 & NodeShapes \\ \hline
Klasy nadrzędne: &     \\ \hline
Opis: & Klasa przechowująca informacje o kształtach poszczególnych węzłów.   \\ \hline

\end{longtable}

\begin{longtable}{|m{3.5cm}|m{8.5cm}|} \hline

CO005 & NodeShapeType \\ \hline
Klasy nadrzędne: &     \\ \hline
Opis: &  Enum - rodzaje kształtów dla węzłów grafu.   \\ \hline

\end{longtable}

\end{center}


\subsection*{Pakiet utils}


Na rysunku  \figurename \space \ref{fig:viz:pack_utils}  przedstawiono diagram klas dla pakietu utils.

\insertimage{images/projekt/utils_class.png}{Diagram klas dla pakietu utils}{fig:viz:pack_utils}

\begin{center}

\begin{longtable}{|m{3.5cm}|m{8.5cm}|} \hline

CU001 & Debug \\ \hline
Klasy nadrzędne: &     \\ \hline
Opis: & Klasa do użycia przy debugowaniu, zapewnia strumień z~błędami zwracanymi przez bibliotekę. Klasa ułatwia pracę programiście informując go
o~błędach i wykonywanych krokach wizualizacji.  Klasa jest singletonem.\\ \hline


\end{longtable}

\begin{longtable}{|m{3.5cm}|m{8.5cm}|} \hline
 
CU002 & VisualizationProperties \\ \hline
Klasy nadrzędne: &     \\ \hline
Opis: & Klasa odpowiada za wczytywanie ustawień kolorów dla węzłów oraz krawędzi z~wybranego lub domyślnego pliku właściwości. \\ \hline

\end{longtable}
\end{center}

\subsection*{Pakiet edges}
\insertimage{images/projekt/edges_class.png}{Diagram klas dla pakietu edges}{fig:viz:pack_edges}

Na rysunku  \figurename \space \ref{fig:viz:pack_edges}  przedstawiony został diagram klas dla pakietu edges.


\begin{longtable}{|m{3.5cm}|m{8.5cm}|} \hline

CE001 & Edge \\ \hline
Klasy nadrzędne: &     \\ \hline
Opis: & Klasa reprezentująca prostą krawędź na grafie, zawiera podstawowe informacje o~jej kształcie i~kolerze. 
Jest nadklasą dla pozostałych klas krawędzi.\\ \hline


\end{longtable}

Klasy z~pakietu edges różnią się tylko tym, że każda z~nich odpowiada za wizualizację denej, wcześniej zaprojektowanej krawędzi na~grafie ontologii.
Dlatego poniżej zostaną wymienione klasy tego pakietu.

\begin{longtable}{|m{4cm}|m{8cm}|} \hline
CE001  & Edge  \\ \hline
CE002  & DisjointEdge \\ \hline
CE003  & DomainEdge \\ \hline
CE004  & EquivalentEdge \\ \hline
CE005  & EquivalentPropertyEdge \\ \hline
CE006  & FunctionaltEdge \\ \hline
CE007  & InstanceOfEdge \\ \hline
CE008  & InstancePropertyEdge \\ \hline 
CE009  & InverseOfEdge \\ \hline
CE010  & InverseOfMutualEdge \\ \hline 
CE011  & OperationEdge \\ \hline 
CE012  & PropertyEdge \\ \hline
CE013  & RangeEdge \\ \hline
CE014  & SubPropertyEdge \\ \hline
CE015  & SubClassEdge \\ \hline


\end{longtable}

 

\subsection*{Pakiet nodes}
\insertimage{images/projekt/nodes_class.png}{Diagram klas dla pakietu nodes}{fig:viz:pack_nodes}

Na rysunku  \figurename \space \ref{fig:viz:pack_nodes}  przedstawiono diagram klas dla pakietu nodes.


\begin{longtable}{|m{3.5cm}|m{8.5cm}|} \hline

CN001 & Node \\ \hline
Klasy nadrzędne: &     \\ \hline
Opis: & Klasa abstrakcyjna, dziedziczą po niej wszystkie klasy z pakietu nodes,  zawiera podstawowe informacje o~jej kształcie i~kolorze. 
\\ \hline


\end{longtable}


Pakiet nodes zawiera najwięcej klas. Podobnie jak w pakiecie edges, klasy z pakietu nodes są podobne. Każda z niż odzwierciedla jakiś element ontologi.
 Ze względu na podobieństwo klas zostaną one tylko wymienione wraz z nadanym im identyfikatorem. 

\begin{longtable}{|m{4cm}|m{8cm}|} \hline
CN001  & Node  \\ \hline
CN002  & AllValuesFromPropertyNode \\ \hline
CN003  & AnonymousClassNode \\ \hline
CN004  & CardinalityNode \\ \hline
CN005  & CardinalityValueNode \\ \hline
CN006  & ClassNode \\ \hline
CN007  & ComplementOfNode \\ \hline
CN008  & DataTypeNode \\ \hline
CN009  & DifferentNode \\ \hline
CN010  & FunctionalPropertyNode \\ \hline
CN011  & IndividualNode \\ \hline
CN012  & InformationNode \\ \hline
CN013  & IntersectionOfNode \\ \hline 
CN014  & InverseFunciotnalPropertyNode \\ \hline 
CN015  & MaxCardinalityValueNode \\ \hline 
CN016  & MinCardinalityValueNode \\ \hline 
CN017  & NothingNode \\ \hline 
CN018  & OneOfNode \\ \hline 
CN019  & PropertyNode \\ \hline 
CN020  & SameAsNode \\ \hline 
CN021  & SomeValuesFromPropertyNode \\ \hline 
CN022  & SymmetricPropertNode \\ \hline 
CN023  & ThingNode \\ \hline 
CN024  & TreansitivePropertyNode \\ \hline 
CN025  & UnionOfNode \\ \hline 




\end{longtable}


\section {Plik właściwości}

Zaproponowane przez autora kolory wierzchołków i krawędzi wizualizacji zostały dobrane w taki sposób, aby wizualizacja była czytelna i~``miła dla oka''.
Niektórzy użytkownicy mogą posiadać inną percepcję wzrokową niż autor. Dlatego należy dać możliwość ustawienia przez użytkownika własnej palety
 kolorów. Aby zmienić kolory wizualizacji należy dostarczyć plik konfiguracyjny z własnymi ustawieniami kolorów. Plik ten składa się z~kluczy przypisanych 
zadanym elementom wizualizacji oraz wartości nowo nadanego koloru. Kompletny spis dopuszczanych wartości (kluczy) został opisany w dodatku nr 2. 
\par Przykładowy wpis w~pliku właściwości wygląda następująco:
\begin{quote}
  \verb+ node.color.thingNodeColor=#00FF00+
\end{quote}
W powyższym przykładzie jest ustawiany kolor dla wierzchołka thing. Kolor zapisywany jest jako RGB reprezentowany szesnastkowo. 
  