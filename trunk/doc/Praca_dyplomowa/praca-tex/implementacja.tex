\chapter{Elementy implementacji}
\section{Wstęp}
Rozdział ten jest poświęcony szczegółom implementacyjnym tworzonej biblioteki SOVA.
Na początku zostaną przedstawione, poprzez opisanie i pokazanie ich architektur, dwie biblioteki użyte w projekcie OWL API oraz Prefuse.

 %Na wstępnie zostaną przedstawione dwie użyte w projekcie biblioteki OWL API oraz Prefuse. Każda z~nich zostanie opisana oraz pokazana zostanie jej architektura. 
\par 
W kolejnej części pokazany zostanie sposób połączenia bibliotek z aplikacją SOVA. Zostanie zaprezentowana architektura  nowej biblioteki oraz opisane zostaną algorytmy
 konwersji obiektów ontologii na wymagane do wizualizacji struktury.
\par Ostatnią część rozdziału poświęcono wdrożeniu biblioteki SOVA do konkretnych rozwiązań. Jednym z nich jest plugin do aplikacji \protege. 

\section{Biblioteka OWL API}
Podstawowym założeniem projektu jest to, że formatem wejściowym dla biblioteki wizualizującej ontologie będą obiekty OWL API, a dokładnie obiekt OWLOntology. 
OWL API jest biblioteką napisaną w javie, pozwalającą na obsługę ontologii zapisanych w języku OWL. Umożliwia tworzenie, manipulowanie oraz zapisywanie obiektów OWL.
O jej użyteczności może świadczyć fakt wdrożenia jej w \protege~4.
\par Najważniejszym obiektem biblioteki OWL API jest OWLOntologyManager. Dostarcza on metody  pozwalające m.in. na tworzenie, wczytanie i zapisanie ontologii. Przechowuje również zbiór 
ontologii zapisywanych w obiektach OWLOntology. Dla wszystkich ontologii przypisanych danemu managerowi zmiany w ontologii stosowane są również do tego managera. 
Sposób zarządzania ontologią w OWL API został przedstawiony na rysunku~\ref{fig:imp:owlapi}

\insertimage{images/implementacja/OWLAPIclass.png}{Diagram UML pokazujący związki pomiędzy obiektami biblioteki OWL API }{fig:imp:owlapi}

\section{Biblioteka Prefuse}
Głównym celem tej pracy jest wizualizacja ontologii. Musi być ona przejrzysta, a zarazem atrakcyjna dla użytkownika. Jednym z rozwiązań ułatwiających spełnienie tego wymagania 
jest używana w OCS biblioteka Prefuse. 
Prefuse jest pakietem, który dostarcza szybkich i nieskomplikowanych narzędzi do pobierania danych, manipulacji nimi oraz ich interaktywnej wizualizacji. 
W jego skład wchodzi:

\begin{itemize}
 \item {\bf prefuse.data} zawierający klasy przeznaczone do tworzenia struktur danych, zapisu i ich odczytu. Podstawową strukturą danych jest krotka. 
Wszystkie pozostałe struktury, np. tabele, składają się z krotek. Jednymi z bardziej rozbudowanych struktur są drzewa i grafy. Użytkownik może również 
definiować swoje struktury, aby w łatwy sposób edytować dane w swoim rozwiązaniu. 
\item{ \bf prefuse action} celem tego modułu jest obliczenie parametrów wizualizacji. Parametrami tymi mogą być m.in. rozmieszczenie elementów grafu, jak i~również 
nadanie kolorów. Wierzchołkowi można nadać kolor wypełnienia oraz kolor, na który ma zmienić się wypełnienie wierzchołka, w~efekcie wywołania akcji, np.~wskazanie elementu. 
\item{ \bf prefuse.visual } zawiera klasy przechowujące obiekty bezpośrednio związanie z wizualizacją. 

\item{ \bf prefuse.controls} w skład tego pakietu wchodzą klasy związanie z interaktywną wizualizacją danych,  m.in. klasy  i metody pozwalające na zmianę rozmiaru
elementów wizualizacji w reakcji na ruch myszą lub też klasy wywołujące odpowiednie akacje po najechaniu bądź zaznaczeniu elementu myszą. 


\end{itemize}


\insertimage{images/implementacja/prefuse.png}{Architektura logiczna biblioteki Prefuse }{fig:imp:prefuse}

Proces wizualizacji danych z wykorzystaniem biblioteki Prefuse został przedstawiany na rysunku\space \ref{fig:imp:prefuse}. 

Kolejnymi etapami wizualizacją są:

\begin{itemize}
 \item {\bf Abstract Data } \\
  Wizualizacja zaczyna się od mapowania danych na struktury dostępne w bibliotece, np. grafy. Biblioteka posiada klasy wejścia/wyjścia pozwalające, m.in. na wczytanie 
danych bezpośrednio z bazy danych lub pliku XML. 
\item {\bf Filtering} \\
  Filtrowanie jest procesem mapowania danych na obiekty przygotowane do wizualizacji (VisualItem). Obiekty te, poza źródłowymi danymi, przechowują informacje 
o~położeniu elementu, jego kolorze i rozmiarze, a filtrowane VisualItem tworzą struktury danych. Często struktury te są odwzorowaniem struktur z Abstract Data, 
jednak ze względu na ich elastyczność mogą ulec różnego rodzaju przekształceniom. Wprowadzenie filtrów może być rozumiane jako próba implementacji wzorca MVC. 
W konsekwencji czego dane z tymi samymi filtrami mogą być wizualizowane na różne sposoby.   
\par Obiekty VisualItem są tworzone i przechowywane w specjalnym rejestrze ItemRegistry, który zawiera wszystkie stany i~wartości dla danej wizualizacji. Rejestr
 ten poprzez mechanizm buforowania danych zapewnia skalowalność rozwiązania. 
\item {\bf Action}\\
Akcje uaktualniają obiekty VisualItem w rejestrze ItemRegistry. Mechanizm ten odpowiada za pobieranie danych, nadawanie im ustawień wizualizacji, wyrównania, 
przypisywaniu  kolorów oraz interpolacji.  
\item{\bf Rendering and Display}\\
 Obiekty VisualItem są obrazowane z wykorzystaniem obiektów renderujących. Wyróżnić można rendery  wizualizujące, m.in. wierzchołki, krawędzie lub tekst. Prezentacja 
renderowanych elementów odbywa się na specjalnym obiekcie Display. Display jest rozszerzeniem obiektu komponentu Javy z pakietu Swing, przyjmuje on wszystkie 
akcje pochodzące od użytkownika. 

\end{itemize}


\section{Integracja z OWL API i Prefuse}
Biblioteka Prefuse \cite{prefuse}\cite{prefuse_sdj}, pomimo jej bogatej funkcjonalności, nie spełnia wszystkich wymagań tworzonego rozwiązania. Jednak jej architektura pozwoliła w łatwy 
sposób dostosować ją do wymagań tworzonej biblioteki. Proces wizualizacji danych w bibliotece SOVA został przedstawiony na rysunku \ref{fig:imp:sova}. 

\insertimage{images/implementacja/sova.png}{Architektura logiczna biblioteki SOVA }{fig:imp:sova}

Poniżej przedstawiono kroki wizualizacji, opisano sposób wykorzystania elementów biblioteki Prefuse oraz wymagane rozszerzenia biblioteki graficznej


\begin{itemize}
 \item{\bf Converting} 
Pierwszym krokiem wizualizacji ontologii jest przekształcenie podanego obiektu OWL API na struktury danych biblioteki Prefuse. Każdy sposób wizualizacji (pełna wizualizacja oraz
wywnioskowana hierarchia) posiada inny algorytm konwersji ontologii na struktury biblioteki Prefuse.  Algorytm (zaprezentowany na listingu nr \ref{lst:code:algorytm}) konwertuje
obiekt OWLOntology na struktury typu graf i~jest używany przy pełnej wizualizacji. Działanie algorytmu zostało szczegółowo opisane w p.~\ref{konwersja}. W~przypadku
wywnioskowanej hierarchii klas i~bytów ontologia jest przekształcana na strukturę drzewiastą. Przekształcenie to jednak nie polega na zwykłym przeglądaniu ontologii 
i~przepisywaniu jej na odpowiednią strukturę danych. Wymagane jest użycie biblioteki wnioskującej Pellet. Rekurencyjny sposób budowy drzewa hierarchii  został opisany 
w~p.~\ref{konwersja-tree}.




 \item {\bf Abstract Data } \\
Dane przechowywane są w dwóch strukturach danych: graf i drzewo. Podstawową strukturą wykorzystywaną podczas wizualizacji jest graf. Posiada on zbiór wierzchołków 
i~zbiór krawędzi. Struktura ta jest wykorzystywana w podstawowym sposobie wizualizacji nazywanym również pełną wizualizacją, zaś struktura drzewa przy 
przechowywaniu wywnioskowanej hierarchii klas i~bytów. 

\item {\bf Filtering} \\
 Aplikacja korzysta z dostarczonego w bibliotece graficznej filtru na odległość. Filtr ten pozwala na obrazowanie wierzchołków oddalonych o nie więcej niż zadana
liczna kroków. Filtrowanie zostało rozszerzone o filtr umożliwiający ukrywanie zadanych elementów ontologii. Filtr sprawdza jakie opcje wizualizacji zostały
 ustawione w statycznej klasie  FilterOptions i na podstawie tych danych nadaje odpowiednie wartości obiektom VisualItem.
\item {\bf Action}\\
W rozwiązaniu można wyróżnić dwa rodzaje akcji: jedne związane ze zmianą położenia elementów wizualizacji, drugie odpowiedzialne za statyczne zmiany. Akcje związane 
z dynamiką wizualizacji mogą być zatrzymane  - w ten sposób wizualizacja staje się nieruchoma.  

\item{\bf Rendering and Display}\\
Ze względu na specyfikę obrazowanych elementów konieczne było stworzenie własnych klas renderujących. Klasy biblioteki Prefuse pozwalają tylko na rysowanie
wierzchołków o~kształcie prostokąta lub prostokąta z zaokrąglonymi rogami i prostych krawędzi lub krawędzi w kształcie strzałki. Właśnie dlatego powstała klasa obrazująca 
wierzchołki o~różnych kształtach i~kolorach oraz klasa pozwalająca na obrazowanie krawędzi posiadających różne groty oraz kolory. Wszystkie obiekty obrazowane są 
na obiekcie OVDisplay. Klasa rozszerzająca prefuse'owy Display posiada metody pozwalające użytkownikowi w łatwy sposób zobrazować zadaną ontologię. Obiektowi 
OVDisplay przypisane zostały metody obsługi akcji na prawy i lewy przycisk myszy. 

\end{itemize}
\section{Konwersja obiektu OWLOntology na struktury danych biblioteki Prefuse}
\subsection{Pełna wizualizacja}
\label{konwersja}
Kluczowym elementem wizualizacji jest pobranie z~ontologii potrzebnych do wizualizacji elementów. Elementy, w~szczególności klasy anonimowe występują w~różnych
 aksjomatach. Dlatego ważne jest, aby nie dublować tych samych elementów w wizualizacji. Algorytm (listing nr \ref{lst:code:algorytm}) zapewnia konwersję bazowych,
jak i anonimowych elementów \cite{nasz}. 

\lstset{language=PASCAL, caption={Algorytm konwertowanie obiektu OWLOntology na obiekty kontenerowe biblioteki Prefuse},stepnumber=1 , captionpos=b,
label={lst:code:algorytm}}
\lstinputlisting{code/algorytm.java}


Algorytm zakłada inicjalne dodanie do wizualizacji wszystkich wierzchołków klas, właściwości oraz bytów zdefiniowanych w ontologii i następnie wstawianie łączących
 je relacji i~dodatkowych wierzchołków anonimowych oraz definiujących użycie właściwości (Some/all values from i hasValue). Po wstawieniu wierzchołków(2) przeglądana
 jest lista aksjomatów(3). Wśród nich wyróżnić można kilka przypadków. Dla  aksjomatów definiujących właściwości (4): functional, inverseFunctional, symmetric 
i~transitive w grafie umieszczane są odpowiednie anonimowe wierzchołki (5) i~krawędzie łączące je ze stosowanym włąściwościami (6). Związki pomiędzy właściwościami takie jak 
inverseProperty, equivalent properties i~subproperty są również umieszczane zgodnie z~definicją ze specyfikacji wizualizacji. Podobnie postępuje się z~aksjomatami 
definiującymi związki między bytami (8-11) (diffrent, allDiffrent i same).
\par
W przypadku aksjomatów uwzględniających klasy (12) algorytm musi uwzględnić fakt, że w~każdym przypadku klasa może być zdefiniowana poprzez bardziej skomplikowany
 opis (description) (13-15). Do takich aksjomatów należą te prezentujące związki pomiędzy klasami : equivalentClasses, subclass,  disjoint classses.
  Ponadto zależności takie mogą wystąpić w przypadku definicji poprzez klasę range i~domain dla property oraz w definicjach typu classAsserction, określających 
klasę, do jakiej należy dany byt. Oczywiście jeśli jedna z tych zależności dotyczy samej, jawnie zdefiniowanej klasy, w grafie umieszcza się stosowna 
krawędź łączącą klasę z innych wierzchołkiem (16). 
\par
Opisy takie tworzone są poprzez różne klasy anonimowe  zdefiniowane w wizualizacji. W przypadku opisów reprezentującym relację hasValue, someValuesFrom 
i~allValuesFrom (23) (has Value jest relację allValuesFrom, jeśli  wierzchołkiem wynikowym jest byt, nie klasa)  najpierw umieszczana jest w grafie klasa anonimowa.
 Jeśli aksjomat uwzględnia definicją kardynalności, to klasa ta zamieniana jest na symbol kardynalności (28) (N) wraz z odpowiednim wierzchołkiem 
z~ograniczeniem liczbowym. Następnie do grafu wstawiana jest krawędź i~stosowny wierzchołek reprezentujący użycie właściwości, krawędź łącząca to „użycie” 
z~definicją i krawędź z~grotem do wartości zdefiniowanej przez aksjomat. Jako że wartością ta może być zarówno individual jaki i~klasa lub jej opis (description),
 należy w~przypadku opisu raz jeszcze dokonać stosowanej analizy (algorytm zachowuje się tu rekurencyjnie). Dla description opisujących zbiory klas lub ich opisów:
 unionOf, intersectionOf, complementOf (28) należy dla każdej składowej zbioru (29) umieścić w grafie odpowiednią krawędź i, o ile klasa jest zdefiniowana opisem,
 przeprowadzić jego analizę, także rekurencyjnie. Jednym z typów opisu może być także klasa zdefiniowana przez aksjomat oneOf – w grafie umieszcza się stosowny 
wierzchołek anonimowy i powiązania z bytami.
\par
Wszystkie te zasady stosuje się analogicznie dla typów danych (datatypes), gdzie zasady odnoszące się do klas obowiązują również dla typów danych. 
Większość ontologii nie definiuje jawnie, że klasa jest podklasą Thing, dlatego, aby zapewnić spójność grafu, dla klas nieposiadających zdefiniowanej
 nadklasy zakłada się, że jest nią Thing i umieszcza się stosowne krawędzie subclass (19-20). Jest to jedyne odstępstwo od zasady wizualizacji dokładnie tego, 
co zostało podane w ontologii, jakie algorytm musi poczynić. Dzięki temu grafy ontologii wizualizowane przy pomocy tego algorytmu można czytać niemal dokładnie 
tak samo jak samą ontologię zapisaną za pomocą języka OWL.

\subsection{Wywnioskowana hierarchia klas i bytów}
\label{konwersja-tree}

Wywnioskowana hierarchia klas i bytów prezentowana jest jak drzewo. Aby zobrazować ontologię w taki sposób, jest wymagany mechanizm wnioskowania (ang. reasoner).
W tym celu w projekcie został użyty pellet. Pellet \cite{pellet} jest otwartą biblioteką napisaną w javie, posiadającą mechanizmy wnioskowania. Obsługuje ona
zarówno OWL API, jak i bibliotekę Jena. Pellet jest pierwszą biblioteką wnioskującą, która w pełni obsługuje język OWL~DL oraz częściowo obsługuje OWL~Full. Twórcy 
ontologii dążą do zapisania jej w języku OWL~DL, jednak duża liczba restrykcji, jakie nakłada ten dialekt powoduje, że w rezultacie uzyskują ontologię zapisaną w OWL~Full. 
Pellet posiada szereg heurystyk pozwalających na wykrywanie i przekształcanie ontologii z OWL~Full na OWL~DL. 
 Użycie pelleta jest doskonałym sposobem na sprawdzenie spójności i poprawności ontologii. Pozwala on na znalezienie błędu niespójności ontologii poprzez znalezienie 
aksjomatu,  który wprowadza ten błąd lub relacji dla niespójnych pojęć. 
\par
Na listingu \ref{lst:code:algorytm2} przedstawiono kod użyty do tworzenia drzewa wywnioskowanej hierarchii klas i bytów. Metoda buildTree() inicjuje reasoner 
(mechanizm wnioskujący pelleta) ontologią, którą będziemy klasyfikować(4). Korzeniem drzewa jest zawsze klasa Thing, która jest przekazywana jako parametr 
do rekurencyjnej metody budowania drzewa(6). Algorytm sprawdza wierzchołek przekazany jako parametr funkcji i jeśli jest on bytem, dodaje go do drzewa (55-59). Jeśli
 wierzchołek  jest klasą, to po dodaniu go do drzewa (26-36)  mechanizm wnioskujący wyszukuje wszystkie podklasy analizowanego wierzchołka (40-43)
oraz wszystkie instancje tej klasy - byty (47-49). Wywołuje na nich rekurencyjne budowanie drzewa. 



\lstset{language=JAVA, caption={Algorytm budowy wywnioskowanej hierarchii klas i bytów},stepnumber=1 ,  captionpos=b,
label={lst:code:algorytm2}}
\lstinputlisting{code/algorytm2.java}

\par 
Algorytm budujący drzewo wywnioskowanej hierarchii klas i bytów mechanizmu wnioskowania używa do wyszukania bytów powiązanych z danym elementem oraz wszystkich 
jego podklas. Podklasy te mogą niejawnie zdefiniowane. 

\section {Plik właściwości}

Zaproponowane przez autora kolory wierzchołków i krawędzi wizualizacji zostały dobrane w taki sposób, aby wizualizacja była czytelna i~``miła dla oka''.
Niektórzy użytkownicy mogą posiadać inną percepcję wzrokową niż autor. Dlatego należy dać możliwość ustawienia przez użytkownika własnej palety
 kolorów. Aby zmienić kolory wizualizacji należy dostarczyć plik konfiguracyjny z własnymi ustawieniami kolorów. Plik ten składa się z~kluczy przypisanych 
zadanym elementom wizualizacji oraz wartości nowo nadanego koloru. Kompletny spis dopuszczanych wartości (kluczy) został opisany w dodatku B. 
\par Przykładowy wpis w~pliku właściwości wygląda następująco:
\begin{quote}
  \verb+ node.color.thingNodeColor=#00FF00+
\end{quote}
W powyższym przykładzie jest ustawiany kolor dla wierzchołka thing. Kolor zapisywany jest jako RGB reprezentowany szesnastkowo. 

\section{Plugin wizualizujący ontologie do \protege}
Plugin do \proteges powstał, aby ``osadzić`` bibliotekę SOVA w wymagającym środowisku. Ponieważ \proteges jest bardzo popularnym edytorem ontologii, to powstały 
i~upubliczniony plugin trafi do specjalistów zajmujących się ontologiami na co dzień i to oni będą mogli wyrazić opinie na temat tego rozwiązania. Rozwiązanie zostało
upowszechnione i opisane na stronach wiki należących od autorów \protege. Obecnie wydany na licencji LGPL plugin w wersji 0.6.0 można odnaleźć pod adresem:
 http://protegewiki.stanford.edu/wiki/SOVA.

 
\par Wykorzystanie w~najnowszej wersji \proteges biblioteki OWL API znacznie ułatwiło jego integrację z powstałą biblioteką wizualizującą ontologie. Napisanie
  pluginu pozwoliło również na sprawdzenie, czy API powstałej biblioteki posiada wszystkie wymagane funkcje oraz czy rozwiązanie może być łatwo modyfikowalne
 w używającym go środowisku.  Czyli zostało sprawdzone spełnienie jednego z wymagań stawianych bibliotece. 

Plugin w wersji binarnej został spakowany do pliku jar. Plik ten, poza kodem biblioteki, załącza  użytą bibliotekę prefuse oraz 3 pliki konfiguracyjne: plugin.xml, 
viewconfig.xml  oraz MANIFEST.MF. Biblioteka OWL API nie jest załączona do pluginu, ponieważ posiada już ją edytor \protege. 

\proteges 4.0 umożliwia dodanie kilku typów pluginów. Najczęściej używanym jest ViewComponent - implementuje on interfejs View, który pozwala na umieszczenie okna 
pluginu w dowolnej z zakładek. Sama zakładka też jest rodzajem pluginu - WorkspaceTab. Implementacja pluginu będącego zakładką nie wymaga tworzenia dodatkowych 
klas, jeśli plugin nie wymaga konfiguracji rozmieszczenia i wyrównania okien znajdujących się w zakładce. W przeciwnym razie jest wymagana klasa obsługi wyrównań 
oraz plik viewconfig.xml zawierający informacje o rozmieszczeniu okienek. 

Stworzony plugin, wizualizujący ontologię z wykorzystaniem biblioteki SOVA, zawiera 2 typy pluginów: ViewComponent oraz WorkspaceTab. Rodzaje użytych pluginów i ich opis 
zostały zawarte w pliku konfiguracyjnym plugin.xml, który został  przedstawiony na listingu nr \ref{lst:code:pluginxml}. Plugin zakładkowy (WorkspaceTab) (17-24) wyświetla 
2 rodzaje wizualizacji (zwykłą (2-8) i wywnioskowaną hierarchię(9-15)), które są pluginami typu ViewComponent i są umieszczone w nim jako 2 zakładki. Sposób rozmieszczenia 
elementów zakładki został skonfigurowany w pliku viewconfig.xml. 

\lstset{language=xml, caption={Zawartość pliku konfiguracyjnego plugin.xml},stepnumber=1 ,  captionpos=b,
label={lst:code:pluginxml}}
\lstinputlisting{code/plugin.xml}




 
