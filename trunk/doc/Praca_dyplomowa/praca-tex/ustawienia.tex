% ------------------------------------------------------------------------
%   Kropki po numerach sekcji, podsekcji, itd.
%   Np. 1.2. Tytuł podrozdziału
% ------------------------------------------------------------------------
\makeatletter
    \def\numberline#1{\hb@xt@\@tempdima{#1.\hfil}}                      %kropki w spisie treści
    \renewcommand*\@seccntformat[1]{\csname the#1\endcsname.\enspace}   %kropki w treści dokumentu
\makeatother

% ------------------------------------------------------------------------
%   Numeracja równań, rysunków i tabel
%   Np.: (1.2), gdzie:
%   1 - numer rozdzialu, 2 - numer równania, rysunku, tabeli
%   Uwaga ogólna: o otoczeniu figure ma byæ najpierw \caption{}, potem \label{}, inaczej odnośnik nie działa!
% ------------------------------------------------------------------------
\makeatletter

% stara wersja, gdzie : 1 - numer sekcji, 2 - numer równania, rysunku, tabeli
   % \@addtoreset{equation}{section} %resetuje licznik po rozpoczęciu nowej sekcji
    %\renewcommand{\theequation}{{\thesection}.\@arabic\c@equation} %dodaje kropkę
    \@addtoreset{equation}{chapter} %resetuje licznik po rozpoczęciu nowej sekcji
    \renewcommand{\theequation}{{\thechapter}.\@arabic\c@equation} %dodaje kropkę

% stara wersja, gdzie : 1 - numer sekcji, 2 - numer równania, rysunku, tabeli
%    \@addtoreset{figure}{section}
%    \renewcommand{\thefigure}{{\thesection}.\@arabic\c@figure}

    \@addtoreset{figure}{chapter}
    \renewcommand{\thefigure}{{\thechapter}.\@arabic\c@figure}

% stara wersja, gdzie : 1 - numer sekcji, 2 - numer równania, rysunku, tabeli
%    \@addtoreset{table}{section}
%    \renewcommand{\thetable}{{\thesection}.\@arabic\c@table}

    \@addtoreset{table}{chapter}
    \renewcommand{\thetable}{{\thechapter}.\@arabic\c@table}

% stara wersja, gdzie : 1 - numer sekcji, 2 - numer równania, rysunku, tabeli
%    \@addtoreset{lstlisting}{section}
%    \renewcommand{\thelstlisting}{\thesection .\arabic{lstlisting}}

    \@addtoreset{lstlisting}{chapter}
 
%   \renewcommand{\thelstlisting}{\thechapter .\arabic{lstlisting}}


\makeatother

% Tablica
% ------------------------------------------------------------------------
\newenvironment{tablica}[3]
{
    \begin{table}[!tb]
    \centering
    \caption[#1]{#2}
    \vskip 9pt
    #3
}{
    \end{table}
}

%
%OBRAZEK
%

\newcommand{\insertimage}[3]
{
	\begin{figure}[ht]
		\begin{center}
			\includegraphics[width=\textwidth]{#1}
		\end{center}
		\caption{#2}
		\label{#3}
	\end{figure}
}


\newcommand{\insertscaledimage}[4]
{
	\begin{figure}[ht]
		\begin{center}
			\includegraphics[scale=#1]{#2}
		\end{center}
		\caption{#3}
		\label{#4}
	\end{figure}
}


% ------------------------------------------------------------------------
% Definicje
% ------------------------------------------------------------------------
\def\nonumsection#1{%
    \section*{#1}%
    \addcontentsline{toc}{section}{#1}%
    }
\def\nonumsubsection#1{%
    \subsection*{#1}%
    \addcontentsline{toc}{subsection}{#1}%
    }
\reversemarginpar %umieszcza notki po lewej stronie, czyli tam gdzie jest więcej miejsca
\def\notka#1{%
    \marginpar{\footnotesize{#1}}%
    }
\def\mathcal#1{%
    \mathscr{#1}%
    }
\newcommand{\atp}{ATP/EMTP} % Inaczej: \def\atp{ATP/EMTP}
\newcommand{\myemptypage}{ \newpage  \thispagestyle{empty}~\newpage}

%-------------------------------------------------------------------------
% listingi
%-------------------------------------------------------------------------
\lstdefinestyle{praca}{basicstyle=\scriptsize, keywordstyle=\color{black}\bfseries, 
	numbers=left, stepnumber=5, numberstyle=\tiny, numbersep=6pt,
	extendedchars=true}
\lstset{style=praca}

% ------------------------------------------------------------------------
% Inne
% ------------------------------------------------------------------------
\frenchspacing          
\setlength{\parskip}{3pt}           %odstęp pomiędzy akapitami
% \linespread{1.25}                    %odstęp pomiędzy liniami (interlinia)
\setcounter{tocdepth}{3}            
\setcounter{secnumdepth}{3}         
\titleformat{\paragraph}[hang]      %wygląd nagłówków
{\normalfont\sffamily\bfseries}{\theparagraph}{1em}{}



\author{Piotr Kunowski}
\title{Wizualizacja ontologii zapisanych w języku OWL}
%ze strona_tytulowa.sty
\uczelnia{Politechnika Gdańska}
\instytut{Katedra Architektury Systemów Komputerowych}
\praca{Praca dyplomowa}
\promotor{dr inż. Piotr Szpryngier}
\miasto{GDAŃSK}
\miesiac{czerwiec}
\rok{2010}

  %polskie podpisy
  \renewcommand{\figurename}{Rys.}
  \renewcommand{\tablename}{Tab.}
  
    %bibliografia
%   \bibliographystyle{plain}