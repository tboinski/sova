\chapter{Analiza i projekt systemu}
\section{Wstęp}
W tej części pracy przedstawiona zostanie analiza systemowa i  projekt techniczny tworzonej aplikacji do wizualizacji ontologii. Część prac projektowych oraz implementacyjnych 
została wykonana w czteroosobowym zespole w ramach projektu grupowego.  Cele, wymagania i założenia zostały przedyskutowana z opiekunem pracy oraz innymi pracownikami 
Katedry Architektury Systemów Komputerowych. W ramach tych dyskusji ustalono, że tworzona aplikacja zostanie wydana jako biblioteka, którą programiści będą mogli 
wdrożyć w swoich rozwiązaniach. 
\par
W pierwszej części tego rozdziału zaprezentowane zostaną cele i wymagania stawiane tworzonej aplikacji. Następnie przedstawione zostaną aspekty techniczne dotyczące 
bibliotek pomocniczych, które zostaną wykorzystane w rozwiązaniu. Na podstawie tej wiedzy zastanie przeprowadzona analiza SWOT oraz określony zostanie harmonogram prac. 
W ostatniej części rozdziału zostanie zaprezentowany diagram pakietów. Każdy z pakietów zostanie krótko omówiony. Ze względy na zbyt dużą liczbę klas nie zostanie tutaj 
zaprezentowana pełna analiza obiektowa, która została przeprowadzona podczas realizacji projektu.
  
\section{Cele systemu}
Głównym celem systemu jest stworzenie biblioteki pozwalającej na wizualizację ontologii zapisanych w języku OWL. Istnieje zapotrzebowanie na bibliotekę, która 
tłumaczyłaby OWL bezpośrednio na elementy graficzne.  Programiści aplikacji związanych z ontologiami korzystając z gotowej biblioteki do wizualizacja, 
będą mogli więcej czasu poświęcić innym zagadnieniom tworzonej przez nich aplikacji.  Aby przetestować rozwiązanie zostanie ono wdrożone do 
rozwijanego przez Katedrę Architektury Systemów Komputerowych systemu OCS \cite{jankowski, boinski}. Na podstawie przeprowadzonych testów wiadomo, że moduł wizualizujący 
ontologię w systemie OCS wymaga modernizacji i rozbudowy funkcjonalności. Biblioteka wizualizująca ontologie ułatwi i przyspieszy zakończenie projektu OCS. Pozwoli 
również na zwiększenie atrakcyjności portalu OCS i przyciągnie użytkowników. 
\par
Zostanie również stworzony plugin do aplikacji \protege, dzięki niemu bibliotek trafi do większego grona użytkowników. Ponieważ autorzy \proteges dokonali zmian w architekturze 
ich edytora, wcześniej stworzone pluginy przestały działać w nowej wersji aplikacji. Dlatego istnieje zapotrzebowanie na plugin wizualizujący, który pozwoli na zwiększenie 
efektywności tworzenia ontologii. 
\section{Użytkownicy systemu}
Ze względu na specyfikę projektu można wyróżnić dwa rodzaje użytkowników: programistę i twórcę ontologii. 
\begin{enumerate}
 \item {\bf Programista} jest użytkownikiem, który wykorzystuje stworzoną bibliotekę w swoim rozwiązaniu. Użytkownik ten będzie wymagał od biblioteki łatwości użycia, 
intuicyjności oraz dobrej dokumentacji. Programista jest użytkownikiem, który może nie znać języka OWL. Jego zadaniem jest stworzenie aplikacji, która pozwoli m.in.
 obrazować ontologie. 
 \item {\bf Twórca ontologii} jest użytkownikiem, który zajmuję się tworzeniem i edycją ontologii. Będzie on korzystał z systemu OCS lub pluginu do \protege. 
Twórca ontologii najczęściej jest specjalistą w dziedzinie ontologii oraz dobrze zna zagadnienia związane z językiem OWL. 
\end{enumerate}

\section{Wymagane funkcje systemu}
Poniżej zostały przedstawione najważniejsze wymagania stawiane tworzonej bibliotece. Na liście znalazły się m.in. wymagania funkcjonalne i wymagania jakościowe.

\begin{enumerate}
 \item Biblioteka powinna udostępniać kilka trybów prezentacji grafów (np. w~formie drzewa, w~formie gwiazdy i~innych). 
 \item Biblioteka powinna dać możliwość wizualizacji wywnioskowanej hierarchii klas i~bytów. 
 \item  Domyślne parametry w trybach wizualizacji (takie jak długość krawędzi grafu, automatyczne układanie) powinny zostać dobrane w~taki sposób,
 by obraz był przejrzysty, stabilny i~czytelny.

\item Rozróżnianie podstawowych symboli. Class, Individual, Property powinny mieć wyróżniające je symbole. 
\item Rozróżnianie szczególnych typów Class. Klasa anonimowa, datatype, Thing i~Nothing powinny być łatwo rozpoznawalne. 
\item Rozróżnianie związków między klasami (Class), bytami (Individual) oraz właściwościami (Property). Różne symbole dla equivalentClass, disjointWith, 
subClassOf, sameAs, differentFrom, allDifferent, oneOf, unionOf, intersectionOf,  complementOf, subProperty, equivalentProperty, hasProperty.
\item Rozróżnianie ograniczeń predykatów (Restrictions). Wyróżnić należy kardynalność (cardinality), domeny (domains) predykatów, inverseOf, właściwości predykatów 
(transitive, symmetric, functional, inverseFunctional).
\item Powinna istnieć możliwość wyszukiwania elementów ontologii na wyświetlanym grafie. Elementy spełniające kryterium wyszukiwania powinny zostać wyróżnione.
\item Każdy z obrazowanych elementów powinien mieć możliwość zmiany jego koloru, tak aby użytkownik mógł dostosować wizualizację do swoich potrzeb. 
\item  Wszystkie wizualizowane elementy powinny pochodzić z ontologii otrzymanej na wejściu programu. Program nie powinien dodawać własnych elementów (np. wywnioskowanych). 
Wyjątkowo dla klas, które nie mają zdefiniowanych nadklas zostanie utworzony związek z klasą Thing. 
\item Biblioteka będzie udostępniać strumień danych, w którym znajdą się komunikaty o błędach. Strumień ten będzie mógł zostać wykorzystany przez użytkownika.
\item Jeżeli biblioteka nie wizualizuje danej funkcji OWL API, informacja o tym powinna znaleźć się w strumieniu błędów.
\end{enumerate}



\section{Nazwa i logo tworzonej biblioteki}

Po chwili burzliwych przemyśleń w obrębie zespołu postanowiono, iż biblioteka będzie posiadać nazwę: SOVA, z ang. Simple Ontology Visualization API (co można przetłumaczyć 
na polski jako: proste API do wizualizacji ontologii). Fonetyczna wymowa nazwy biblioteki oznacza ptaka - sowę. Sowa w języku angielskim to owl, co oczywiście kojarzy się 
z~językiem OWL i~tu koło znaczeń się zamyka. 
\par Jako logo biblioteki wybrano właśnie sowę. Logo zostało zaprezentowane na rysunku \ref{fig:sova}. Rysunek został pobrany z portalu Clker.com [www.clker.com], gdzie został 
upubliczniony na licencji: creative common public domain license. Licencja umożliwia darmowe kopiowanie, modyfikowanie i upublicznianie pobranej grafiki.

  \insertscaledimage{0.3}{images/SOVA.png}{Logo tworzonej biblioteki}{fig:sova}


\section{Wybór technologii}

Tworzona aplikacja zostanie napisana w Javie 1.6. O wyborze języka zadecydował fakt, iż zarówno OCS, jak i \proteges są napisane w tym języku. 
\subsection{Wybór biblioteki graficznej}
Napisanie całego kodu pozwalającego na wizualizację byłoby czasochłonne i wymagało dużego nakładu pracy. Dlatego aplikacja będzie się opierać na istniejącym już rozwiązaniu, 
 pozwalającym na obrazowania danych za pomocą grafów. W sieci można odnaleźć dużo takich rozwiązań. Jednak użyta biblioteka obrazująca grafy musi być nieodpłatna i~najlepiej 
posiadać otwarty kod. Poniżej przedstawiono najciekawsze rozwiązania, które były brane pod uwagę przy wyborze biblioteki. 

\begin{description}
\item[Prefuse \cite{prefuse,prefuse_sdj}]
 jest elastycznym pakietem dostarczającym programiście narzędzia do przechowywania danych, manipulowania nimi oraz ich interaktywnej wizualizacji. Biblioteka jest 
rozwijana w całości w języku Java. Może być wykorzystana do budowania niezależnych aplikacji, wizualnych komponentów rozbudowanych aplikacji oraz tworzenia apletów.

\pagebreak[3]
Podstawowe cechy i elementy biblioteki Prefuse:
\nopagebreak[4]
\begin{itemize}
\item różne algorytmy i metody wizualizacji danych m.in.: ForceDirectedLayout, RadialTreeLayout, NodeLinkTreeLayout, SquarifiedTreeMapLayout,
\item dynamiczne rozmieszczanie i animacje,
\item transformacje, przekształcenia geometryczne oraz przybliżanie/oddalanie obrazu,
\item podstawowym elementem struktury danych jest krotka,
\item krotki mogą być tworzone bezpośrednio w~aplikacji lub na~podstawie zewnętrznych danych,
\item wbudowany język zapytań do filtrowania danych,
\item tworzenie struktur danych na podstawie zewnętrznych plików (CSV, XML) oraz bazy danych,
\item klasy wspomagające synchronizację danych pomiędzy tabelami Prefuse, a~bazą danych,
\item Prefuse posiada licencję BSD.
\end{itemize}

 \item[Piccolo]  
 jest zastawem narzędzi używanych przy tworzeniu graficznych aplikacji, często wykorzystywanym do tworzenie interfejsów użytkownika, w których elementy 
są przybliżane i~oddalane. Istnieją trzy wersje tej biblioteki: Piccolo.Java, Piccolo.NET oraz PocketPiccolo.NET. Posiada Licencje BSD.

\item[JUNG (Java Universal Network/Graph Framework)]
 Biblioteka przeznaczona do wizualizacji danych za pomocą grafów oraz sieci. Umożliwia wizualizację nie tylko grafów prostych, ale m.in. multigrafów, 
digrafów oraz grafów posiadających wagi i etykiety na wierzchołkach i krawędziach. Biblioteka posiada podstawowe algorytmy grafowe. Została napisana w~całości w~Javie 
i~wydana na licencji BSD.
\item[JGraph]

 Napisana w pełni w Javie biblioteka do wizualizacji grafów kompatybilna ze Swingiem. Posiada wiele ciekawych opcji wizualizacji zarówno wierzchołków,
 jak i~krawędzi grafów. Poza algorytmami wizualizacji w jej skład wchodzą podstawowe algorytmy grafowe. Została wydana na licencji LGPL.
\end{description}

Po uważnym przejrzeniu bibliotek, najbardziej użyteczne wydają się Prefuse oraz Piccolo. Ze względu na dostępność dużej liczby przykładowego kodu wykorzystującego Prefuse 
w~edytorze OCS, wykorzystana zostanie biblioteka Prefuse. Ponadto opinie wyrażone w pracy magisterskiej Andrzeja Jakowskiego\cite{jankowski} silnie przemawiają 
na korzyść Prefuse.


\subsection{Format danych wejściowych}
Celem aplikacji jest wizualizowanie ontologii zapisanych w języku OWL. Za język danych, które mają zostać zobrazowane, przyjęto OWL DL, który jest wykorzystywany w~systemie OCS. 
 Nie oznacza to jednak, że pozostałe dialekty nie będą obsługiwane, jednak w OWL Full możemy napotkać na pewne niejasności (szczególnie pod względem rozróżniania typów).
\par 
W celu wczytania ontologii z pliku, zostanie wykorzystana biblioteka OWL API \cite{owlapi,owlapi1}. Ontologie, zapisane jako obiekty OWL API, będą danymi wejściowymi dla biblioteki SOVA. 
Co oznacza, że dla biblioteki wizualizującej ontologie nie jest istotne źródło danych (plik, baza danych). Aby zachować kompatybilność z systemem OCS zostanie użyta biblioteka 
OWL API w wersji 2.1.1. 

\section{Planowanie projektu}
Planowanie jest częścią projektu, którego zadaniem jest osiągnięcie celu projektu z uwzględnieniem jego ograniczeń. Poniżej zostanie przedstawiony harmonogram prac oraz
 analiza SWOT, które pozwolą lepiej zaplanować i zrozumieć tworzony projekt.

\subsection{Harmonogram projektu}
 \insertscaledimage{0.55}{images/projekt/harmonogram_tabela.png}{Harmonogram projektu}{projekt:harmonogram2}
Napisanie pracy dyplomowej wiąże się z napisaniem projektu informatycznego (aplikacji) oraz formalnym opisem pracy w postaci tego dokumentu. 
Na rysunkach \ref{projekt:harmonogram2} i~\ref{projekt:harmonogram}  przedstawiony został harmonogram, w którym uwzględniono prace związane z formalnym opisem projektu,
 jak i jego implementacją. 




\subsection{Analiza SWOT}

W tabeli \ref{t:swot} przedstawiono analizę SWOT przeprowadzona dla tworzonego projektu. Pozwoliła  ona na wyróżnienie słabych  i~mocnych stron projektu 
oraz dała możliwość poznania szans, dzięki którym projekt może zakończyć się powodzeniem, i~zagrożeń, które mogą spowodować, iż projekt nie zostanie zaakceptowany 
przez rynek i~użytkowników. 


\begin{longtable}{|m{7cm}|m{7cm}|} 
\caption{Analiza SWOT}
\label{t:swot} \\
\hline
\bf{Mocne strony} 	&  \bf{Słabe strony}  \\ 
\begin{enumerate}
\item Dobra znajomość języka Java wśród członków zespołu.
 \item Znajomość zagadnień związanych z ontologiami.
 \item Współpraca z pracownikami Katedry Architektury Systemów Komputerowych.
\end{enumerate}
 & 
\begin{enumerate}
 \item Brak doświadczenia w pracy z biblioteką Prefuse.
 \item Brak doświadczenia w pracy z biblioteką OWL API.
 \item Problem z uzyskaniem zgody Politechniki Gdańskiej na upublicznienie aplikacji.
\end{enumerate}
\\ \hline
\bf{Szanse} 	&  \bf{Zagrożenia}  \\ 
\begin{enumerate}
 \item Wzrost zainteresowania sieciami semantycznymi i ontologiami.
 \item Brak dobrych rozwiązań do wizualizacji ontologii, co sprawia, że istnieje zapotrzebowanie na bibliotekę do wizualizacji ontologii.
\end{enumerate}
 & 
\begin{enumerate}
 \item Zmniejszenie zainteresowania językiem OWL na korzyść OWL 2.
 \item Wprowadzenie nowego, lepszego sposobu zapisu wiedzy.
 \item Konkurencja ze strony istniejących rozwiązań, które umożliwiają wizualizacje ontologii. 
%  \item 
\end{enumerate}
\\ \hline
\end{longtable}


\section{Projekt Wizualizacji}

Specyfikacja języka OWL pozwala na dużą dowolność definiowania ograniczeń czy opisów elementów ontologii. Dlatego wizualizacja powinna 
odzwierciedlać elementy ontologii w sposób opisany przez autora w~języku OWL. Ważne jest, aby graficzna reprezentacja ontologii 
była jednoznaczna i~zarazem łatwa do zrozumienia. Chcąc spełnić te kryteria należy zdefiniować sposób obrazowania każdego z~elementów języka OWL~DL.
Źle zaprojektowana wizualizacja może być przyczyną niepowodzenia projektu. 
\par Najważniejszymi elementami ontologii są klasy, właściwości, typy danych oraz bytu. Trzy pierwsze elementy biorą udział w podobnych związkach, dlatego
 będą obrazowane w postaci zaokrąglonego prostokąta o~kolorze zależnym od rodzaju elementu. Reprezentacją bytów, aby wyróżnić je od pozostałych 
elementów, będzie prostokąt. Byty będę posiadały też inny kolor niż pozostałe elementy. (\figurename \space \ref{fig:viz:projekt1}).

\insertimage{images/projekt_wiz/wiz_elementy_1.png}{Symbole reprezentujące klasę (a), właściwość (b), typ danych (c) oraz byt (d)}{fig:viz:projekt1}

\par
Największym wyzwaniem wizualizacji jest zrozumiałe przedstawienie klas anonimowych. Zaproponowano symbole przedstawiające złożone relacje, 
w postaci kółek z wpisanymi w nie znaczącymi symbolami. Przykładowe symbole zaprezentowano na \figurename \space \ref{fig:viz:projekt2}. Wykorzystano symbolikę matematyczną  
w~przypadku intersekcji, komplementarności, unii oraz kardynalności. Matematyczną symbolikę otrzymały także relacje zachodzące 
pomiędzy bytami  - „sameAs” i~„allDifferent”, przy czym symbol relacji „sameAs” jest  nadmiarowy, aby zachować spójność wizualizacji
 (relacja ta jest przeciwna do differentFrom/allDifferent). Kardynalność również reprezentowana jest za pomocą anonimowego wierzchołka, 
przy czym dodatkowy wierzchołek z~ograniczeniem liczby jest wyróżniony kolorem w~zależności od typu ograniczenia (min, max, equal)

\insertimage{images/projekt_wiz/wiz_elementy_2.png}{Przykładowe symbole reprezentujące klasę anonimową (a), intersekcję (b), kardynalność typu max i min (c), 
relację sameAss (d) oraz relację allDifferent (e)}{fig:viz:projekt2}
\par


Zaprezentowanie relacji „allValuesFrom” i~ „someValuesFrom”  odbywa się poprzez wprowadzenie klasy anonimowej reprezentującej wynik podanego
 ograniczenia (\figurename \space \ref{fig:viz:projekt3}). Jest ona połączona z~symbolem właściwości, przy czym nazwa właściwości, występująca w~aksjomacie, 
 poprzedzona jest kwantyfikatorem
 ogólnym dla  „allValuesFrom” oraz szczególnym dla „someValuesFrom”. Następnie strzałka wskazuje klasę (warto zwrócić uwagę, że może to być także 
dowolnie złożona klasa anonimowa), określającą przeciwdziedzinę przedstawionej relacji. Dodatkowo użyty wierzchołek property połączony jest 
z~wierzchołkiem przedstawiającym jego definicję.

 \insertscaledimage{0.35}{images/projekt_wiz/wiz_elementy_3.png}{Reprezentacja relacji someValuesFrom oraz allValuesFrom}{fig:viz:projekt3}

Różne proste relacje reprezentowane są przez strzałki o~różnych grotach. Relację „SubClass” i~„SubProperty” prezentuje zaczerpnięta ze specyfikacji 
UML - strzałka o~pustym grocie.  Krawędzie reprezentujące związki „equivalent” i~„disjoint” mają odwrotne groty, które podkreślają odwrotność tych relacji.

\insertimage{images/projekt_wiz/wiz_elementy_4.png}{ Przykładowe symbole reprezentujące relacje proste: rdfs:subclassOf (a), instanceOf (b), owl:equivalentClass (c), owl:disjointWith (d), rdfs:domain (e) oraz rdfs:range (f)}{fig:viz:projekt4}

 W~przypadku definicji właściwości (Property), inwersja jest wyróżniona kolorem czerwonym i~oznaczana dwojako, ze względu na asymetryczność tej relacji. Z~kolei 
równoważność właściwości odróżniona jest od równoważności klas kolorem. Zastosowane zostały dodatkowe rodzaje grotów, umożliwiające łatwe rozróżnienie 
range i~domain dla property. Do oznaczenia konkretnych property wykorzystane zostały tradycyjne strzałki, które wskazują kierunek czytania  aksjomatu.
 Aksjomaty, które tworzą kolejne klasy anonimowe, są czytelne dzięki specjalnemu grotowi, który (podobnie jak przy właściwościach) był konieczny, aby wskazać 
kierunek ich czytania. Tam, gdzie nie jest to potrzebne, krawędzie nie posiadają grotów, co zwiększa czytelność wizualizacji. Przykładowe relacje 
zaprezentowano na \figurename \space \ref{fig:viz:projekt4}.

\section{Analiza obiektowa}

\subsection{Diagram klas i pakietów}
\insertimage{images/projekt/PackageDiagram.png}{Diagram pakietów}{fig:viz:package}
Na rysunku \figurename \space \ref{fig:viz:package} przedstawiono diagram pakietów i relacje zachodzące pomiędzy nimi. Wyróżniono sześć głównych
 pakietów, które zostaną opisane niżej.  Klasy zostały umieszczone w odpowiednich pakietach zgodnie z zachowaniem wzorca MVC (ang. Model View Controller).
Do nazw pakietów został dodany przedrostek "org.pg.eti.kask.sova". \\
\pagebreak[3]
\noindent 
{\bf Symbol pakietu :} P001 \newline
{\bf Nazwa pakietu :} options \newline
{\bf Opis :}  Pakiet zawierający klasy z polami opisującymi różne (modyfikowalne) ustawienia wizualizacji takie jak: kolory, grubość linii itp.  \newline

\noindent 
{\bf Symbol pakietu :} P002 \newline
{\bf Nazwa pakietu :} nodes \newline
{\bf Opis :} Pakiet z klasami odpowiedzialnymi za wizualizację i przechowywanie danych o wierzchołkach. Każdy rodzaj wierzchołka jest odzwierciedlany przez inną klasę \newline

\noindent 
{\bf Symbol pakietu :} P003 \newline
{\bf Nazwa pakietu :} edges \newline
{\bf Opis :} Pakiet z klasami odpowiedzialnymi za wizualizację i przechowywanie danych o krawędziach. Każdy typ krawędzie jest instancją innej klasy. \newline

\noindent 
{\bf Symbol pakietu :} P004 \newline
{\bf Nazwa pakietu :} visualization \newline
{\bf Opis :} Zawiera klasy obsługi wizualizacji m.in. klasę zwracającą display, klasy różnych trybów wizualizacji oraz klasy filtrów. Jest najważniejszym pakietem 
w~aplikacji.  \newline
\newline
\noindent 
{\bf Symbol pakietu :} P005 \newline
{\bf Nazwa pakietu :} graph \newline
{\bf Opis :} Pakiet zawiera klasy konwertujące obiekty OWL API na odpowiednie struktury danych, które pozwalają na wizualizację. \newline
\newline
\noindent 
{\bf Symbol pakietu :} P006 \newline
{\bf Nazwa pakietu :} utils \newline
{\bf Opis :}  Pakiet zawiera klasy pomocnicze i dodatkowa narzędzia pozwalające m.in. na wczytanie pliku właściwości czy dostarczenie narzędzi debugiera.  \newline



\subsection{Zalecenie tworzenia biblioteki}
Nie istnieją żadne formalne zalecenia dotyczące tworzenia bibliotek JAVA. Są jednak pewne zalecenia co do stosowanych praktyk \cite{biblioteka_standard}:
\begin{enumerate}
\item \textbf{Hermetyzacja kodu.} Publiczne powinny być jedynie te klasy i~metody, które są istotne dla użytkownika i~z~których będzie on~bezpośrednio korzystał.
\item \textbf{Możliwość debugowania.} Użytkownik powinien mieć możliwość debugowania kodu biblioteki, bez konieczności znajomości każdego jej szczegółu.
\item  \textbf{Przejrzystość.} Kod biblioteki powinien być odpowiednio udokumentowany za pomocą javadoc. W~szczególności, bardzo dokładnie należy opisać klasy oraz metody publiczne.
\item \textbf{Łatwość użycia.} Biblioteka powinna zawierać klasy, pokazujące przykłady wykorzystania jej klas i~metod.
\item \textbf{ Rozszerzalność.} Struktura wewnętrzna biblioteki powinna być odpowiednio podzielona na klasy (wykorzystując klasy abstrakcyjne i~interfejsy). Dzięki temu użytkownik będzie miał możliwość stworzenia własnych klas, rozszerzających funkcjonalność biblioteki.
\item \textbf{Uniwersalność.} Biblioteka powinna mieć jasno określony problem, który rozwiązuje. Wyniki powinny być podane użytkownikowi w~wygodny dla niego sposób (lub na kilka sposobów), 
który będzie umożliwiał wykorzystanie biblioteki w~różnych aplikacjach. Innymi słowy, biblioteka powinna udostępniać łatwy i~przejrzysty dla użytkownika interfejs.
\end{enumerate}


 \insertscaledimage{0.55}{images/projekt/harmonogram.png}{Harmonogram projektu}{projekt:harmonogram}