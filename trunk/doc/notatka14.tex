\documentclass[a4paper,10pt]{article}
\usepackage{polski}
\usepackage[utf8]{inputenc}
\usepackage{array}
%custom margins
\usepackage[left=2.5cm, top=2.5cm, bottom=3cm, right=2cm, foot=2cm, head=0.5cm]{geometry}
\usepackage{fancyhdr}
\usepackage[bookmarks=true, pdftex]{hyperref}

%styl nagłowków
\pagestyle{fancy}
\parindent 2cm

%opening
\title{Notatka ze spotkania}
\author{3@KASK}

\begin{document}
\bibliographystyle{plain}


\maketitle


\begin{center}
%budowanie tabeli
\begin{tabular}{|p{7cm}|p{7cm}|}
\hline
Symbol projektu: & Opiekun projektu:   \tabularnewline
3@KASK & mgr inż. Tomasz Boiński    \tabularnewline \hline
\multicolumn{2}{|l|}{Nazwa Projektu: } \tabularnewline
\multicolumn{2}{|l|}{Wizualizacja grafów za pomocą biblioteki Prefuse } \tabularnewline
\hline
\multicolumn{2}{l}{ } \tabularnewline %pusta linijka
\hline
Nazwa Dokumentu: & Data spotkania:   \tabularnewline
Notatka ze spotkania & 02.06.09 \tabularnewline \hline
Odpowiedzialny za dokument: & Obecni na spotkaniu:   \tabularnewline
Piotr Orłowski & Grupa projektowa \tabularnewline \hline
Przeznaczenie: & Data ostatniej aktualizacji:   \tabularnewline
WEWNĘTRZNE & \today \tabularnewline \hline
\end{tabular}
\end{center}



\section{Temat spotkania}
Prototypowanie, porządkowanie SVN,.

\section{Poruszone zagadnienia}
\begin{enumerate}
\item Sprzątanie SVN projektu,
\item Dokumentowanie napisanych wcześniej klas (javadoc).
\item Strumień błedów.
\item Metoda render - przykładowe strzałki i wzór klasy.
\end{enumerate}


\section{Podjęte ustalenia}
\begin{enumerate}
\item Stworzono pakiet util, a w nim klasę Debug z metodą send message. Klasa ta jest singletonem i ma z założenia udostępniać innym klasom strumień błędów.
\item Wykryto błędy w automatycznie wygenerowanych getterach i setterach - do poprawienia.
\item Część danych będziemy przechowywać we własnych tabelach.
\end{enumerate}



%\clearpage
%\phantomsection
%\addcontentsline{toc}{section}{Literatura}
%\bibliography{biblio}

\end{document}

