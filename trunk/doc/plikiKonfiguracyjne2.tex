\section{Spis plików konfiguracyjnych}

\begin{center}
%budowanie tabeli
\begin{tabular}{|p{7cm}|p{7cm}|}
\hline
Symbol projektu: & Opiekun projektu:   \tabularnewline
3@KASK & mgr inż. Tomasz Boiński    \tabularnewline \hline
\multicolumn{2}{|l|}{Nazwa Projektu: } \tabularnewline
\multicolumn{2}{|l|}{Wizualizacja grafów za pomocą biblioteki Prefuse } \tabularnewline
\hline
\multicolumn{2}{l}{ } \tabularnewline %pusta linijka
\hline
Nazwa Dokumentu: & Nr wersji:   \tabularnewline
Spis plików konfiguracyjnych & 0.0 \tabularnewline \hline
Odpowiedzialny za dokument: & Data pierwszego sporządzenia:   \tabularnewline
Radosław Kleczkowski & 03.02.10 \tabularnewline \hline
Przeznaczenie: & Data ostatniej aktualizacji:   \tabularnewline
WEWNĘTRZNE & \today \tabularnewline \hline
\end{tabular}
\end{center}

\begin{center}
\begin{tabular}{|c|p{4cm}|c|p{3cm}|c|}
\multicolumn{5}{c}{\textbf{Historia dokumentu}} \tabularnewline \hline
\textbf{Wersja} & \textbf{Opis modyfikacji} & \textbf{Rozdział/strona} & \textbf{Autor modyfikacji} & \textbf{Data} \tabularnewline \hline
0.0 & Przygotowanie dokumentu i dodanie visualisation.properties & wszystkie & Radosław Kleczkowski & 03.02.10 \tabularnewline \hline
\end{tabular}


\end{center}
\newpage

\subsection{Visualization.properties}

\subsubsection{Opis pliku}
\paragraph{} Plik z ustawieniami dotyczącymi wizualizacji. Zawiera ustawienia kolorów dla węzłów oraz krawędzi.

\subsubsection{Przykładowa zawartość}
\begin{quote}

\#przykładowe properties \newline
\#zawiera wszystkie możliwe do zdefiniowania właściwości \newline
\#aby zmienić kolor, usuń znak '\#' z początku linii, a za kolejnym wpisz wartość \newline
\#w formacie RGB (szesnastkowo) \newline
 \newline
\#\#\#\#\#\#\#\#\#\#\#\#\#\#\#\#\#\# Kolory węzłów. \newline
\#Należy zwrócić uwagę, iż etykiety są koloru czarnego. \newline
 \newline
\#Kolor węzłów reprezentujących definicje klas \newline
\#node.color.classNodeColor=\#00FF00 \newline
 \newline
\#Kolor węzła reprezentującego klasę "Thing" \newline
\#node.color.thingNodeColor=\#00FF00 \newline
 \newline
\#\#Kolor węzła reprezentującego klasę "Nothing" \newline
\#node.color.nothingNodeColor=\#00FF00 \newline
 \newline
\#Kolor węzłów reprezentujących instancje klas (OWL Individual) \newline
\#node.color.individualNodeColor=\#00FF00 \newline
 \newline
\#Kolor węzłów oznaczających relację DifferentFrom \newline
\#lub AllDifferent pomiędzy wystąpieniami klas (OWL Individual) \newline
\#node.color.differentNodeColor=\#00FF00 \newline
 \newline
\#Kolor węzłów oznaczających relację OWL SameAs \newline
\#pomiędzy wystąpieniami klas (OWL Individual) \newline
\#node.color.sameAsNodeColor=\#00FF00 \newline
 \newline
\#Kolor węzłów reprezentujących definicje predykatów (OWL Property) \newline
\#node.color.propertyNodeColor=\#00FF00 \newline
 \newline
\#Kolor węzłów Property typu SomeValuesFrom \newline
\#node.color.someValuesFromNodeColor=\#EE2222 \newline
 \newline
\#Kolor węzłów Property typu AllValuesFrom \newline
\#node.color.allValuesFromNodeColor=\#00FF00 \newline
 \newline
\#Kolor węzłów OWL DataType \newline
\#node.color.dataTypeNodeColor=\#00FF00 \newline
 \newline
\#Kolor węzłów reprezentujących różnorakie klasy anonimowe \newline
\#node.color.anonymousClassNodeColor=\#00FF00 \newline
 \newline
\#Kolor węzłów reprezentujących dokładne ograniczenie kardynalności \newline
\#node.color.cardinalityValueNodeColor=\#00FF00 \newline
 \newline
\#Kolor węzłów reprezentujących minimalne ograniczenie kardynalności \newline
\#node.color.minCardinalityValueNodeColor=\#00FF00 \newline
 \newline
\#Kolor węzłów reprezentujących maksymalne ograniczenie kardynalności \newline
\#node.color.maxCardinalityValueNodeColor=\#00FF00 \newline
 \newline
\#Kolor węzłów oznaczających właściwości predykatów (OWL Property) \newline
\#node.color.informationNodeColor=\#00FF00 \newline
 \newline
\#\#\#\#\#\#\#\#\#\#\#\#\#\#\#\#\#\# Kolory krawędzi. \newline
 \newline
\#Kolor zwykłych krawędzi (bez grotów) \newline
\#edge.color.edgeColor=\#888888 \newline
 \newline
\#Kolor krawędzi oznaczających relacje między Property a klasą \newline
\#edge.color.propertyEdgeColor=\#FF0000 \newline
 \newline
\#Kolor krawędzi łączących definicję property z jego domeną \newline
\#edge.color.domainEdgeColor=\#FF0000 \newline
 \newline
\#Kolor krawędzi łączących definicję property z jego przestrzenią (OWL Range) \newline
\#edge.color.rangeEdgeColor=\#FF0000 \newline
 \newline
\#Kolor krawędzi łączących klasy rozłączne \newline
\#edge.color.disjointEdgeColor=\#FF0000 \newline
 \newline
\#Kolor krawędzi łączących klasy równoważne (OWL Equivalent) \newline
\#edge.color.equivalentEdgeColor=\#FF0000 \newline
 \newline
\#Kolor krawędzi łączących predykaty (OWL Property) równoważne (OWL Equivalent) \newline
\#edge.color.equivalentPropertyEdgeColor=\#FF0000 \newline
 \newline
\#Kolor krawędzi łączących definicję Property z jego właściwościami \newline
\#np. functional, symmetric \newline
\#edge.color.functionalEdgeColor=\#FF0000 \newline
 \newline
\#Kolor krawędzi łączących predykat (OWL Property) odwrotny (OWL InverseOf) do zadanego \newline
\#edge.color.inverseOfEdgeColor=\#FF0000 \newline
 \newline
\#Kolor krawędzi łączących predykaty (OWL Property) wzajemnie odwrotne (OWL InverseOf) \newline
\#edge.color.inverseOfMutualEdgeColor=\#FF0000 \newline
 \newline
\#Kolor krawędzi oznaczających operacje, w wyniku których powstają klasy anonimowe \newline
\#np. unia, przecięcie \newline
\#edge.color.operationEdgeColor=\#005555 \newline
\end{quote}


